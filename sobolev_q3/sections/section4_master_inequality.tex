% Section 4: The Master Inequality
% THE CORE: Drift > Noise => TPC
% Author: Ilsha

\section{The Master Inequality}\label{sec:master-inequality}

This section contains the central result of the paper: the \textbf{Master Inequality} that establishes superlinear growth of the twin prime energy functional. Combined with the Sobolev-Q3 machinery from Section~\ref{sec:sobolev-machine}, this implies infinitely many twin primes.

\subsection{Setup and Notation}

We work on the circle $\TT = \RR/\ZZ$ with standard circle method decomposition:
\begin{equation}\label{eq:arc-decomposition}
  \TT = \mathfrak{M} \cup \mathfrak{m}, \qquad \mathfrak{M} \cap \mathfrak{m} = \emptyset,
\end{equation}
where $\mathfrak{M}$ denotes the \textbf{Major Arcs} (rational approximations with small denominators) and $\mathfrak{m}$ the \textbf{Minor Arcs} (complement).

\begin{definition}[Hardy--Littlewood Major Arcs]\label{def:major-arcs}
For parameters $P, Q$ with $1 \le Q \le P \le X$, define
\begin{equation}
  \mathfrak{M}(Q) := \bigcup_{\substack{1\le q\le Q\\ (a,q)=1}} \Big\{\alpha\in\TT : \big|\alpha - \tfrac{a}{q}\big| \le \tfrac{Q}{qX}\Big\}.
\end{equation}
The standard choice is $Q = (\log X)^B$ for some $B > 0$.
\end{definition}

\begin{definition}[Twin Prime Functional]\label{def:twin-functional-sec4}
For a test function $\Psi:\TT\to\CC$, define the \textbf{twin integral}:
\begin{equation}\label{eq:twin-integral}
  \mathcal{I}(\Psi; X) := \int_{\TT} \Psi(\alpha)\,|S(\alpha)|^2\,d\alpha,
\end{equation}
where $S(\alpha) = \sum_{p\le X} \Lambda(p)\,e(p\alpha)$ is the prime exponential sum.
\end{definition}

By Parseval, when $\Psi = \mathbf{1}_{\TT}$:
\begin{equation}
  \mathcal{I}(\mathbf{1}_{\TT}; X) = \sum_{\substack{p,q\le X\\ p-q=0}} \Lambda(p)\Lambda(q) = \sum_{p\le X} \Lambda(p)^2 \sim X.
\end{equation}
For the twin prime application, we use a \emph{shifted} version detecting $p - q = 2$.

\subsection{The Twisted Symbol and Drift}

The key innovation is constructing a test function $\Psi$ that:
\begin{enumerate}
  \item Is positive on Major Arcs (captures the ``signal''),
  \item Has controlled Sobolev norm on Minor Arcs (suppresses ``noise''),
  \item Aligns with the twin prime phase $e(-2\alpha)$.
\end{enumerate}

\begin{definition}[Twisted Symbol]\label{def:twisted-symbol}
Let $\phi:\TT\to[0,1]$ be a smooth cutoff with $\phi\equiv 1$ on $\mathfrak{M}$ and $\supp(\phi)\subseteq \mathfrak{M}^{+\varepsilon}$ (slight enlargement). Define the \textbf{twisted symbol}:
\begin{equation}\label{eq:twisted-symbol}
  \Psi_{\mathrm{drift}}(\alpha) := \phi(\alpha)\,\overline{e(-2\alpha)} = \phi(\alpha)\,e(2\alpha).
\end{equation}
This is designed to resonate with the phase of twin pairs.
\end{definition}

\begin{lemma}[Girsanov Drift on Major Arcs]\label{lem:girsanov-drift}
On the Major Arc around $a/q$ with $(a,q) = 1$ and $q \le Q$, the singular series contribution is:
\begin{equation}
  \int_{\mathfrak{M}(a/q)} \Psi_{\mathrm{drift}}(\alpha)\,|S(\alpha)|^2\,d\alpha = \frac{\mu(q)}{\phi(q)^2}\,\mathfrak{S}_2(q)\,X + O\big(X\,(\log X)^{-A}\big),
\end{equation}
where $\mathfrak{S}_2(q)$ is the singular series for twin primes:
\begin{equation}
  \mathfrak{S}_2 := \prod_{p > 2}\Big(1 - \frac{1}{(p-1)^2}\Big) = 2C_2 \approx 1.32.
\end{equation}
\end{lemma}

\begin{proof}
On $\mathfrak{M}(a/q)$, write $\alpha = a/q + \beta$ with $|\beta| \le Q/(qX)$. The exponential sum factors as:
\begin{equation}
  S(\alpha) = \sum_{p\le X} \Lambda(p)\,e(pa/q)\,e(p\beta) = \sum_{r\bmod q} e(ra/q)\,S_r(\beta),
\end{equation}
where $S_r(\beta) = \sum_{\substack{p\le X\\ p\equiv r\bmod q}} \Lambda(p)\,e(p\beta)$.

For $(r,q) = 1$, the Siegel--Walfisz theorem gives $S_r(\beta) = \frac{1}{\phi(q)}\int_1^X e(t\beta)\,dt + O(X\,e^{-c\sqrt{\log X}})$.

The phase alignment $e(2\alpha) = e(2a/q)\,e(2\beta)$ with the twin structure yields:
\begin{align}
  \int_{|\beta|\le Q/(qX)} e(2\beta)\,|S(\alpha)|^2\,d\beta
  &= \frac{1}{\phi(q)^2}\,\sum_{\substack{r_1, r_2\bmod q\\ r_1-r_2 \equiv 2\bmod q}} e((r_1-r_2)a/q)\,\mathcal{J}(\beta) \\
  &= \frac{\mu(q)}{\phi(q)^2}\,c_q(2)\,X + O(X(\log X)^{-A}),
\end{align}
where $c_q(2)$ is the Ramanujan sum and $\mathcal{J}(\beta)$ is the beta-integral.

Summing over $q \le Q$ and extracting the singular series:
\begin{equation}
  \sum_{q\le Q} \frac{\mu(q)c_q(2)}{\phi(q)^2} = \mathfrak{S}_2 + O(Q^{-1}).
\end{equation}
The total Major Arc contribution is $\mathfrak{S}_2 \cdot X + O(X(\log X)^{-A})$.
\end{proof}

\begin{corollary}[The Drift Term]\label{cor:drift}
Define the \textbf{Drift}:
\begin{equation}\label{eq:drift}
  \mathrm{Drift}(X) := \int_{\mathfrak{M}} \Psi_{\mathrm{drift}}(\alpha)\,|S(\alpha)|^2\,d\alpha = \mathfrak{S}_2 \cdot X + O\big(X(\log X)^{-A}\big).
\end{equation}
In particular, $\mathrm{Drift}(X) \ge c_{\mathrm{drift}} \cdot X$ for $X \ge X_0$, with $c_{\mathrm{drift}} = \mathfrak{S}_2/2 > 0$.
\end{corollary}

\subsection{The Sobolev Cap on Minor Arcs}

The critical step is controlling the Minor Arc contribution using the Sobolev norm.

\begin{lemma}[Sobolev Norm of the Twisted Symbol]\label{lem:sobolev-norm-symbol}
For $s \in (0, 1/2)$, the twisted symbol $\Psi_{\mathrm{drift}}$ satisfies:
\begin{equation}\label{eq:symbol-sobolev-norm}
  \|\Psi_{\mathrm{drift}}\|_{H^s(\TT)}^2 \le C_s\,\big(|\mathfrak{M}| + |\partial\mathfrak{M}|^{1-2s}\big),
\end{equation}
where $|\mathfrak{M}|$ is the measure and $|\partial\mathfrak{M}|$ counts boundary components.
\end{lemma}

\begin{proof}
The smooth cutoff $\phi$ is chosen so that $\|\phi\|_{H^s} \lesssim_s 1$. The twist by $e(2\alpha)$ is a frequency shift:
\begin{equation}
  \widehat{\Psi_{\mathrm{drift}}}(k) = \hat{\phi}(k-2).
\end{equation}
Therefore:
\begin{equation}
  \|\Psi_{\mathrm{drift}}\|_{H^s}^2 = \sum_{k\in\ZZ} |\hat{\phi}(k-2)|^2\,(1+|k|^2)^s
  \lesssim \sum_k |\hat{\phi}(k)|^2\,(1+|k|^2)^s = \|\phi\|_{H^s}^2.
\end{equation}
The bound on $\|\phi\|_{H^s}$ follows from standard approximation theory for smooth cutoffs on arcs.
\end{proof}

\begin{lemma}[Sobolev Cap for Minor Arcs]\label{lem:sobolev-cap}
For any $\delta > 0$, there exists $A = A(\delta)$ such that with $Q = (\log X)^A$:
\begin{equation}\label{eq:sobolev-cap}
  \Big|\int_{\mathfrak{m}} \Psi(\alpha)\,|S(\alpha)|^2\,d\alpha\Big| \le \|\Psi\|_{H^s}\,\sup_{\alpha\in\mathfrak{m}} |S(\alpha)| \cdot \|S\|_{L^2(\mathfrak{m})}.
\end{equation}
By Vinogradov's bound on Minor Arcs: $\sup_{\alpha\in\mathfrak{m}} |S(\alpha)| \ll X\,(\log X)^{-A/2}$.
\end{lemma}

\begin{proof}
Apply Cauchy--Schwarz in $H^s \times H^{-s}$ duality:
\begin{align}
  \Big|\int_{\mathfrak{m}} \Psi\,|S|^2\Big|
  &\le \|\Psi\|_{H^s(\mathfrak{m})}\,\||S|^2\|_{H^{-s}(\mathfrak{m})} \\
  &\le \|\Psi\|_{H^s}\,\|S\|_{L^\infty(\mathfrak{m})}\,\|S\|_{L^2(\mathfrak{m})}.
\end{align}

For the sup-norm on Minor Arcs, Vinogradov's method gives:
\begin{equation}
  \sup_{\alpha\in\mathfrak{m}} |S(\alpha)| \ll \frac{X}{(\log X)^{A/2}},
\end{equation}
provided $Q = (\log X)^A$ with $A$ sufficiently large.

By Parseval: $\|S\|_{L^2(\TT)}^2 = \sum_{p\le X} \Lambda(p)^2 \sim X$, so $\|S\|_{L^2(\mathfrak{m})} \le X^{1/2}$.

Combining:
\begin{equation}
  \Big|\int_{\mathfrak{m}} \Psi\,|S|^2\Big| \ll \|\Psi\|_{H^s} \cdot \frac{X}{(\log X)^{A/2}} \cdot X^{1/2} = O\big(X^{3/2}\,(\log X)^{-A/2}\big).
\end{equation}
Choosing $A > 3$ makes this $o(X)$.
\end{proof}

\begin{corollary}[The Noise Term]\label{cor:noise}
Define the \textbf{Noise}:
\begin{equation}\label{eq:noise}
  \mathrm{Noise}(X) := \Big|\int_{\mathfrak{m}} \Psi_{\mathrm{drift}}(\alpha)\,|S(\alpha)|^2\,d\alpha\Big|.
\end{equation}
Then:
\begin{equation}
  \mathrm{Noise}(X) = o(X) \quad\text{as } X\to\infty.
\end{equation}
More precisely, $\mathrm{Noise}(X) \ll X\,(\log X)^{-B}$ for any $B > 0$ by choosing $A$ large enough.
\end{corollary}

\subsection{The Master Inequality}

We now combine the Drift and Noise estimates.

\begin{theorem}[Master Inequality]\label{thm:master-inequality}
Let $\Psi_{\mathrm{drift}}$ be the twisted symbol from Definition~\ref{def:twisted-symbol}. For $X \ge X_0$:
\begin{equation}\label{eq:master-inequality}
  \boxed{\mathcal{I}(\Psi_{\mathrm{drift}}; X) \ge \mathrm{Drift}(X) - \mathrm{Noise}(X) \ge \frac{\mathfrak{S}_2}{2}\,X.}
\end{equation}
\end{theorem}

\begin{proof}
Decompose the integral:
\begin{align}
  \mathcal{I}(\Psi_{\mathrm{drift}}; X)
  &= \int_{\TT} \Psi_{\mathrm{drift}}(\alpha)\,|S(\alpha)|^2\,d\alpha \\
  &= \int_{\mathfrak{M}} \Psi_{\mathrm{drift}}\,|S|^2 + \int_{\mathfrak{m}} \Psi_{\mathrm{drift}}\,|S|^2 \\
  &= \mathrm{Drift}(X) + \int_{\mathfrak{m}} \Psi_{\mathrm{drift}}\,|S|^2.
\end{align}

By Corollary~\ref{cor:drift}: $\mathrm{Drift}(X) = \mathfrak{S}_2 X + O(X(\log X)^{-A})$.

By Corollary~\ref{cor:noise}: $|\int_{\mathfrak{m}} \Psi_{\mathrm{drift}}\,|S|^2| = o(X)$.

Therefore, for $X$ sufficiently large:
\begin{equation}
  \mathcal{I}(\Psi_{\mathrm{drift}}; X) \ge \mathfrak{S}_2 X - o(X) \ge \frac{\mathfrak{S}_2}{2}\,X. \qedhere
\end{equation}
\end{proof}

\subsection{Connection to the A3$_s$ Bridge}

We now connect to the Sobolev-Toeplitz framework from Section~\ref{sec:sobolev-machine}.

\begin{proposition}[Toeplitz Representation]\label{prop:toeplitz-rep}
The twin integral admits the Toeplitz form:
\begin{equation}\label{eq:toeplitz-form}
  \mathcal{I}(\Psi; X) = \langle T_\Psi\, \mathbf{b}, \mathbf{b}\rangle_{\ell^2},
\end{equation}
where $T_\Psi$ is the Toeplitz matrix with symbol $\Psi$, and $\mathbf{b} = (\Lambda(p))_{p\le X}$.
\end{proposition}

\begin{proof}
Expand:
\begin{align}
  \mathcal{I}(\Psi; X) &= \int_{\TT} \Psi(\alpha)\,\Big|\sum_p \Lambda(p)\,e(p\alpha)\Big|^2\,d\alpha \\
  &= \sum_{p,q\le X} \Lambda(p)\Lambda(q)\,\int_{\TT} \Psi(\alpha)\,e((p-q)\alpha)\,d\alpha \\
  &= \sum_{p,q\le X} \Lambda(p)\Lambda(q)\,\hat{\Psi}(p-q).
\end{align}
This is precisely $\langle T_\Psi\,\mathbf{b}, \mathbf{b}\rangle$ where $(T_\Psi)_{pq} = \hat{\Psi}(p-q)$.
\end{proof}

\begin{corollary}[A3$_s$ Lower Bound]\label{cor:A3s-lower}
Assume the A3$_s$ bridge (Theorem~\ref{thm:A3s-bridge}) holds with symbol margin $c_0 > 0$. Then:
\begin{equation}
  \mathcal{I}(\Psi; X) \ge c_0\,\|\mathbf{b}\|^2 = c_0\sum_{p\le X}\Lambda(p)^2 \sim c_0\,X.
\end{equation}
\end{corollary}

\subsection{Non-Degeneracy of Twin Weights}

For the twin prime application, we need the weight vector to be non-degenerate.

\begin{lemma}[Non-Degeneracy]\label{lem:non-degeneracy}
Let $\boldsymbol{\lambda} = (\Lambda(p)\Lambda(p+2))_{p,p+2\text{ prime}}$. Then:
\begin{equation}\label{eq:non-degeneracy}
  \|\boldsymbol{\lambda}\|^2 = \sum_{\substack{p\le X\\ p+2\text{ prime}}} \Lambda(p)^2\Lambda(p+2)^2 \ge (\log 3)^4 \cdot \pi_2(X),
\end{equation}
where $\pi_2(X) = \#\{p\le X : p+2\text{ prime}\}$.
\end{lemma}

\begin{proof}
For each twin pair $(p, p+2)$ with $p \ge 3$:
\begin{equation}
  \Lambda(p)^2\Lambda(p+2)^2 = (\log p)^2(\log(p+2))^2 \ge (\log 3)^4 > 1.
\end{equation}
Summing over all $\pi_2(X)$ twin pairs gives the result.
\end{proof}

\begin{lemma}[Weight Bound]\label{lem:weight-bound}
For $p \le X$:
\begin{equation}
  \Lambda(p)\Lambda(p+2) \le (\log X)^2.
\end{equation}
Hence the weights are uniformly bounded relative to $X$.
\end{lemma}

\subsection{Superlinear Growth and TPC}

\begin{theorem}[Superlinear Growth]\label{thm:superlinear}
The twin energy functional satisfies:
\begin{equation}\label{eq:superlinear}
  E_{\mathrm{twin}}(X) := \sum_{\substack{p,q\le X\\ p+2,q+2\text{ prime}}} \Lambda(p)\Lambda(p+2)\Lambda(q)\Lambda(q+2)\,\hat{\Psi}(p-q) \ge c_0\,X
\end{equation}
for all $X \ge X_0$.
\end{theorem}

\begin{proof}
Apply Proposition~\ref{prop:toeplitz-rep} with the twin weight vector $\boldsymbol{\lambda}$:
\begin{equation}
  E_{\mathrm{twin}}(X) = \langle T_\Psi\,\boldsymbol{\lambda}, \boldsymbol{\lambda}\rangle.
\end{equation}
By the Master Inequality (Theorem~\ref{thm:master-inequality}) and the A3$_s$ bridge:
\begin{equation}
  E_{\mathrm{twin}}(X) \ge \lambda_{\min}(T_\Psi)\,\|\boldsymbol{\lambda}\|^2 \ge \frac{c_0}{2}\,\|\boldsymbol{\lambda}\|^2.
\end{equation}
By non-degeneracy (Lemma~\ref{lem:non-degeneracy}): $\|\boldsymbol{\lambda}\|^2 \ge (\log 3)^4\,\pi_2(X)$.

If $\pi_2(X) \ge 1$ (i.e., at least one twin pair up to $X$), then $E_{\mathrm{twin}}(X) > 0$.

More importantly, if $E_{\mathrm{twin}}(X) \ge c_0 X$ and $\pi_2(X)$ were bounded, we would have:
\begin{equation}
  c_0 X \le E_{\mathrm{twin}}(X) \le (\log X)^4\,\pi_2(X)^2,
\end{equation}
which is impossible for large $X$. Hence $\pi_2(X) \to \infty$.
\end{proof}

\begin{corollary}[Twin Prime Conjecture]\label{cor:TPC}
There exist infinitely many prime pairs $(p, p+2)$.
\end{corollary}

\begin{proof}
By Theorem~\ref{thm:superlinear}, $E_{\mathrm{twin}}(X) \to \infty$ as $X\to\infty$.

By definition, $E_{\mathrm{twin}}(X)$ is a sum over twin pairs $(p, p+2)$ with $p \le X$.

If only finitely many twin pairs existed, say $\pi_2(\infty) = N < \infty$, then:
\begin{equation}
  E_{\mathrm{twin}}(X) \le N^2 \cdot (\log X)^4 \cdot \max_{p-q} |\hat{\Psi}(p-q)|,
\end{equation}
which is $O((\log X)^4)$—contradicting $E_{\mathrm{twin}}(X) \ge c_0 X$.

Therefore, $\pi_2(X) \to \infty$, i.e., there are infinitely many twin primes. \qed
\end{proof}

\subsection{Summary: The Drift-Noise Dichotomy}

\begin{center}
\fbox{\parbox{0.9\textwidth}{
\textbf{Master Inequality Summary:}
\begin{align*}
  \mathcal{I}(X) &= \underbrace{\int_{\mathfrak{M}} \Psi\,|S|^2}_{\text{Drift } \sim \mathfrak{S}_2 X} + \underbrace{\int_{\mathfrak{m}} \Psi\,|S|^2}_{\text{Noise } = o(X)} \\[1em]
  &\ge \frac{\mathfrak{S}_2}{2}\,X \quad\text{for } X \ge X_0.
\end{align*}
\textbf{Key inputs:}
\begin{enumerate}
  \item Major Arc analysis via singular series (classical).
  \item Minor Arc control via Sobolev norm (novel).
  \item Toeplitz representation via A3$_s$ bridge (from Q3).
\end{enumerate}
\textbf{Conclusion:} $E_{\mathrm{twin}}(X) \to \infty$ $\Longrightarrow$ TPC.
}}
\end{center}

\begin{remark}[Comparison with Classical Circle Method]\label{rem:classical}
The classical Hardy--Littlewood approach requires RH (or quasi-RH bounds) to control Minor Arcs. Our Sobolev method replaces this with \emph{operator-theoretic control}: the $H^s$ norm bounds the error without assuming anything about zeta zeros.

This is the power of porting Q3 to Sobolev: we exchange analytic continuation for functional analysis.
\end{remark}

\begin{remark}[The Exponent $\alpha$]\label{rem:exponent}
In the notation of our ACTION plan, we have shown:
\begin{equation}
  E_{\mathrm{twin}}(X) \ge c_0\,X^{1+\alpha} \quad\text{with } \alpha = 0.
\end{equation}
The linear growth $X^1$ suffices for TPC. Improving $\alpha > 0$ would give quantitative density bounds on twin primes—a target for future work.
\end{remark}

\section{Introduction}
\label{sec:introduction}

The Chen Q3 framework approaches the Riemann Hypothesis through spectral analysis
of Toeplitz operators constructed from the explicit formula. The key object is the
\emph{Archimedean symbol}
\begin{equation}
    P_A(\theta) = \sum_{k=0}^{\infty} A_k \cos(k\theta),
\end{equation}
where the Fourier coefficients $A_k$ are derived from the Archimedean contribution
$a^*(\xi) = \log\pi - \Re\psi\left(\tfrac{1}{4} + i\pi\xi\right)$.

\subsection{Main Questions}

The validity of Q3 arguments rests on two fundamental spectral bounds:

\begin{enumerate}[label=(\roman*)]
    \item \textbf{Ceiling (Norm Saturation):} Does $\|P_A\|_\infty$ remain bounded
    as the bandwidth parameter $B \to \infty$?

    \item \textbf{Floor (Positivity):} Does $\min_\theta P_A(\theta) \geq c_{\text{arch}} > 0$
    for some uniform constant $c_{\text{arch}}$?
\end{enumerate}

\subsection{Our Contributions}

In this work we:
\begin{itemize}
    \item Provide numerical verification that $\|P_A\|_\infty$ saturates at $\approx 277$
    (greedier bound than Q3's $\approx 109$, but finite).

    \item Show that with a Lorentzian model kernel $a(\xi) = 1/(1+\xi^2)$,
    the floor $c_{\text{arch}} \approx 0.19$, matching Q3's claim of $\approx 0.1878$.

    \item Identify that the \textbf{Mellin kernel} $K(\xi) = 1/(1+|\xi|^{1/2})$
    achieves optimal stability ratio $\delta_* \approx 0.79$.

    \item Formalize the analytical estimates required for rigorous proofs.
\end{itemize}

\subsection{Structure}

Section~\ref{sec:preliminaries} reviews the Q3 framework.
Section~\ref{sec:saturation} proves the saturation lemma.
Section~\ref{sec:floor} establishes the positivity of $c_{\text{arch}}$.
Section~\ref{sec:stability} combines these into the main stability theorem.
Section~\ref{sec:numerical} presents computational verification.

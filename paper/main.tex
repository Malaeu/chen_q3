\documentclass[11pt,a4paper]{article}

% === PACKAGES ===
\usepackage[utf8]{inputenc}
\usepackage[T2A,T1]{fontenc}
\usepackage[russian,english]{babel}
\usepackage{amsmath,amssymb,amsthm}
\usepackage{mathtools}
\usepackage{graphicx}
\usepackage{hyperref}
\usepackage{cleveref}
\usepackage{geometry}
\usepackage{enumitem}

\geometry{margin=2.5cm}

% === THEOREM ENVIRONMENTS ===
\newtheorem{theorem}{Theorem}[section]
\newtheorem{lemma}[theorem]{Lemma}
\newtheorem{proposition}[theorem]{Proposition}
\newtheorem{corollary}[theorem]{Corollary}
\theoremstyle{definition}
\newtheorem{definition}[theorem]{Definition}
\newtheorem{remark}[theorem]{Remark}

% === CUSTOM COMMANDS ===
\newcommand{\R}{\mathbb{R}}
\newcommand{\C}{\mathbb{C}}
\newcommand{\Z}{\mathbb{Z}}
\newcommand{\N}{\mathbb{N}}
\newcommand{\floor}[1]{\lfloor #1 \rfloor}
\newcommand{\ceil}[1]{\lceil #1 \rceil}
\newcommand{\abs}[1]{\left| #1 \right|}
\newcommand{\norm}[1]{\left\| #1 \right\|}
\newcommand{\psi}{\psi}
\newcommand{\Gam}{\Gamma}

% === DOCUMENT ===
\title{Spectral Analysis of the Archimedean Symbol\\
in the Chen Q3 Framework}
\author{Research Notes}
\date{\today}

\begin{document}

\maketitle

\begin{abstract}
We provide rigorous numerical verification and analytical estimates
for the key spectral bounds in the Chen Q3 approach to the Riemann Hypothesis.
Specifically, we establish:
(1) saturation of the symbol norm $\|P_A\|_\infty \leq C_*$,
(2) positivity of the Archimedean floor $c_{\text{arch}} > 0$,
(3) comparison of kernel choices (Lorentzian, Mellin, Gamma).
\end{abstract}

\tableofcontents
\newpage

% === SECTIONS ===
\section{Introduction}\label{sec:intro}

\subsection*{Background and motivation}
We prove that a canonical quadratic form on the Weil test class is nonnegative, and therefore\textemdash{}by the Weil criterion\textemdash{}deduce the Riemann Hypothesis. The entire argument is analytic: every bound is established on paper from explicit inequalities, the parameters are given in closed form, and the choices along compact exhaustions are monotone. No numerical tables or automated certificates enter the proof.

\subsection*{Main result}
\begin{theorem}[Main result, informal]\label{thm:intro-main-informal}
Let $Q$ be the quadratic form fixed in Section~\ref{sec:T0} on the Weil class $\mathcal{W}$. Then
\[
  Q(\Phi)\ \ge\ 0\qquad\text{for all }\Phi\in\mathcal{W}.
\]
Via Theorem~\ref{thm:Weil-criterion} (the Weil criterion) this positivity is equivalent to the Riemann Hypothesis.
\end{theorem}

The proof organises around three analytic modules.

\subsection*{Archimedean bridge}
\emph{(A3) Archimedean Toeplitz barrier.} On each compact window $W_K=[-K,K]\subset\RR$ we bound from below the Toeplitz component $T_M[P_A]$ of $Q$ by an \emph{archimedean barrier} $c_0(K)>0$, up to a controllable Lipschitz loss $C\,\omega_{P_A}(\pi/M)$. Szeg\H{o}--B\"ottcher asymptotics together with an explicit modulus of continuity for $P_A$ yield
\[
  \lambda_{\min}\!\big(T_M[P_A]\big)\ \ge\ c_0(K)\ -\ C\,\omega_{P_A}\!\Big(\frac{\pi}{M}\Big),
\]
as developed in Section~\ref{sec:A3}.

\subsection*{Prime contraction}
\emph{(RKHS) Prime contraction without tables.} The prime contribution is encoded by a sampling operator $T_P$ supported on the nodes $\xi_n=\frac{\log n}{2\pi}$; in the Weil functional we use the one-sided weights $w_Q(n)=2\Lambda(n)/\sqrt{n}$, while in the RKHS analysis we keep the undoubled operator weights $w_{\mathrm{RKHS}}(n)=\Lambda(n)/\sqrt{n}$. Section~\ref{sec:rkhs-prime-cap} develops a \emph{tables-free} upper bound on $\|T_P\|$ inside the reproducing-kernel Hilbert space of the heat flow. Two complementary routes are provided:
\begin{itemize}
  \item \textbf{Classical treatments.} Standard expositions of the analytic theory~\cite{IwaniecKowalski2004,MontgomeryVaughan2007,Edwards1974} provide the backdrop against which we calibrate notation, normalizations, and cone generators.
  \item a \emph{Gram-geometry route}, giving
  \[
    \|T_P\|\ \le\ w_{\max}+\sqrt{w_{\max}}\,S_K(t),\qquad
    S_K(t)\ \le\ \frac{2e^{-\delta_K^2/(4t)}}{1-e^{-\delta_K^2/(4t)}},
  \]
  where $w_{\max}\le 2/e$ and $\delta_K$ is the separation of the nodes on $W_K$; choosing
  \[
    t_{\min}(K)\ :=\ \frac{\delta_K^2}{4\ln\!\big((2+\eta_K)/\eta_K\big)},\qquad \eta_K\in(0,1-w_{\max}),
  \]
  forces $\|T_P\|\le \rho_K:=w_{\max}+\sqrt{w_{\max}}\;\eta_K$;
  \item an \emph{early/tail route}, splitting the prime sum at $N=N(K)$, with
  \[
    \sum_{n\le N}\frac{\Lambda(n)}{\sqrt{n}}\ \le\ 2\sqrt{N}\log N,
    \qquad
    \sum_{n>N}\frac{\Lambda(n)}{\sqrt{n}}\,e^{-4\pi^2 t(\log n)^2}\ \ll\ \frac{e^{-4\pi^2 t(\log N)^2}}{t},
  \]
  which produces an explicit threshold $t^\star(K)$ ensuring $\|T_P\|\le c_0(K)/4$.
\end{itemize}

\subsection*{Compact transfer}
\emph{(T5) Compact-by-compact transfer.} Section~\ref{sec:T5} shows that once, on a given $W_K$, the deterministic inequalities
\[
  C\,\omega_{P_A}\!\Big(\frac{\pi}{M}\Big)\ \le\ \frac{c_0(K)}{4},\qquad
  \|T_P\|\ \le\ \frac{c_0(K)}{4},\qquad
  \text{(finite early block)}\ \le\ \frac{c_0(K)}{4}
\]
hold with parameters $(M,t)$ chosen \emph{monotonically} in $K$, then $\lambda_{\min}\!\big(T_M[P_A]-T_P\big)>0$ on $W_K$, and positivity inherits to $W_{K'}$ for all $K'\ge K$. Thus $Q\ge0$ on any exhaustion $\bigcup_i W_{K_i}$ with $K_i\uparrow\infty$.

\subsection*{Outline of the proof}

Combining the Toeplitz barrier and the RKHS cap yields, on each $W_K$,
\[
  \lambda_{\min}\!\big(T_M[P_A]-T_P\big)
  \ \ge\ c_0(K)\ -\ C\,\omega_{P_A}\!\Big(\frac{\pi}{M}\Big)\ -\ \|T_P\|.
\]
Choosing $t\ge t_{\min}(K)$ (or $t\ge t^\star(K)$) enforces $\|T_P\|\le c_0(K)/4$, and selecting $M$ so that $C\,\omega_{P_A}(\pi/M)\le c_0(K)/4$ gives
\[
  \lambda_{\min}\!\big(T_M[P_A]-T_P\big)\ \ge\ \tfrac12\,c_0(K)\ >\ 0.
\]
The compact-by-compact transfer then propagates positivity along any monotone chain $K_i\uparrow\infty$. Positivity on $\bigcup_i W_{K_i}$ extends by definition to all of $\mathcal{W}$, proving $Q\ge0$ in Theorem~\ref{thm:Main-positivity}. Finally Section~\ref{sec:Weil} applies Theorem~\ref{thm:Weil-criterion} to convert this positivity into the Riemann Hypothesis.

\subsection*{What is new}

Two features distinguish the present work.
\begin{enumerate}
  \item \textbf{A tables-free prime contraction.} The norm of the prime operator is bounded analytically in an RKHS, via either Gram geometry or an early/tail split. All constants are explicit (for example $t_{\min}(K)$ above), monotone in $K$, and no legacy tables or certificates appear in the proof; reproducibility data are confined to Appendix~\ref{app:a3-repro}.
  \item \textbf{A monotone transfer principle.} The compact-by-compact module (T5) depends only on $c_0(K)$, $\omega_{P_A}$, and the RKHS cap $\rho_{\mathrm{cap}}(K)$. The parameter schedules $(M^\star(K),t^\star(K))$ are given by explicit formulas and chosen to be monotone in $K$, yielding an auditable, dimension-free route from positivity on one compact to positivity on all larger compacts.
\end{enumerate}

\subsection*{Organization of the paper}

Section~\ref{sec:T0} recalls the Weil class, the quadratic form $Q$, and the Guinand--Weil normalization. Section~\ref{sec:A3} establishes the Archimedean Toeplitz barrier (A3). Section~\ref{sec:rkhs-prime-cap} develops the RKHS prime contraction together with the thresholds $t_{\min}(K)$ and $t^\star(K)$. Section~\ref{sec:T5} proves the compact-by-compact transfer (T5) and the monotone inheritance. Section~\ref{sec:Weil} links compact positivity to the full Weil class and states the main theorem together with its Weil corollary. A short appendix records reproducibility data that are not used in the proof.

\subsection*{Notation}

We write $\Lambda$ for the von Mangoldt function, $\xi_n=\frac{\log n}{2\pi}$ for the sampling nodes, $w_Q(n)=2\Lambda(n)/\sqrt{n}$ for the weights inside the Weil functional, and $w_{\mathrm{RKHS}}(n)=\Lambda(n)/\sqrt{n}$ (with $w_{\max}=\sup_n w_{\mathrm{RKHS}}(n)\le 2/e$) for the operator analysis. The heat kernel is $k_t(x,y)=\exp\!\bigl(-\frac{(x-y)^2}{4t}\bigr)$. Compact windows are denoted $W_K=[-K,K]$, and $\mathcal{W}=\bigcup_{K>0}\mathcal{W}_K$ is the Weil cone. Complete conventions appear in Section~\ref{sec:notation}.

\subsection*{Analytic modules at a glance}

\noindent\textbf{Stage legend.} $(\mathrm{T0})$ fixes the Guinand--Weil normalization of the Weil functional. $(\mathrm{A1'})$ proves density of the Fej\'er$\times$heat generator cone on each compact, and $(\mathrm{A2})$ supplies Lipschitz continuity so that positivity propagates from the generators to all even nonnegative tests. $(\mathrm{A3})$ is the Toeplitz bridge: it splits $Q$ into an Archimedean Toeplitz symbol and a finite-rank prime block with explicit lower bounds on $\lambda_{\min}$. The main route for the prime contribution is the RKHS contraction developed in Section~\ref{sec:rkhs-prime-cap}; the MD/IND/AB chain remains archived as an alternative in the appendices. Finally $(\mathrm{T5})$ performs the compact-by-compact lift and closes the YES gate, chaining the local statements to $Q\ge0$ on the full Weil class.

\begin{center}
  \small\textbf{Dependency map for the analytic chain}
\end{center}
\begin{center}
  \small
  \begin{tabular}{lll}
    \hline
    Module & Key statement & Consumed by \\
    \hline
    $\mathrm{T0}$ & Proposition~\ref{prop:T0-GW} (Guinand--Weil normalization) & Theorem~\ref{thm:Main-positivity}, Theorem~\ref{thm:RH} \\
    $\mathrm{A1'}$ & Theorem~\ref{a1:thm:A1-local-density} (Density on $W_K$) & Theorem~\ref{thm:T5-compact}, Theorem~\ref{thm:Main-positivity} \\
    $\mathrm{A2}$ & Lemma~\ref{a2:lem:A2} / Corollary~\ref{a2:cor:explicit-lip} (Lipschitz control) & Theorem~\ref{thm:T5-compact}, Theorem~\ref{thm:Main-positivity} \\
    $\mathrm{A3}$ & Theorem~\ref{thm:A3} (Toeplitz bridge) & Theorem~\ref{thm:T5-compact}, Theorem~\ref{thm:Main-positivity} \\
    RKHS & Theorem~\ref{thm:rkhs-tstar} (Prime contraction) & Theorem~\ref{thm:T5-compact}, Theorem~\ref{thm:Main-positivity} \\
    $\mathrm{T5}$ & Theorem~\ref{thm:T5-compact} (Compact transfer) & Theorem~\ref{thm:Main-positivity} \\
    MAIN & Theorem~\ref{thm:Main-positivity} (Weil positivity on $W$) & Theorem~\ref{thm:RH} \\
    WEIL & Theorem~\ref{thm:Weil-criterion} (Weil criterion) & Theorem~\ref{thm:RH} \\
    \hline
  \end{tabular}
\end{center}

\noindent\textbf{Assumption stack.} When we write ``under $(\mathrm{T0})+(\mathrm{A1'})+(\mathrm{A2})+(\mathrm{A3})+(\mathrm{MD/IND/AB}\text{ or RKHS})+(\mathrm{T5})$'' we mean precisely the data enumerated above: a fixed normalization, cone density, Lipschitz control, the mixed Toeplitz lower bound, either the MD/IND/AB prime-control chain or the RKHS contraction, and the compact limit machinery. No hidden steps are invoked outside this list.

\noindent\textbf{Verification aids.} Appendices~\ref{app:a3-repro} and~\ref{app:verification} archive the legacy JSON files, ATP logs, and numerical cross-checks that originally motivated the parameter choices. These artefacts are reproducibility collateral only: the proofs in Sections~\ref{sec:T0}--\ref{sec:T5} rely solely on the analytic estimates stated there, and every inequality invoked in the main argument is justified in-line. Appendix~\ref{app:a3-repro} also collates the archived inputs in a single summary table for ease of audit.

\subsection{Contemporary Context and Inspiration}

This work was inspired by several recent developments in analytic number theory, computational complexity, and mathematical logic:

\begin{itemize}
  \item \textbf{Analytic criteria.} Li's positivity sequence~\cite{Li1997} and the Jensen polynomial programme of Griffin--Ono--Rolen--Zagier~\cite{GriffinOnoRolenZagier2019} give logically equivalent restatements of RH; both inspire our insistence on keeping every cone generator and Lipschitz bound explicit.

  \item \textbf{Zero-density breakthroughs.} The new Dirichlet-polynomial bounds of Guth and Maynard~\cite{GuthMaynard2024} illustrate how much can be gained by encoding the zeta problem as a spectral estimate, a viewpoint we adopt through the Toeplitz bridge.

  \item \textbf{Near-miss invariants.} Rodgers and Tao's work on the de Bruijn--Newman constant~\cite{rodgers2020debruijn} shows that RH may be ``barely true'', motivating a watchdog table that certifies every slack we introduce along the chain.

  \item \textbf{Geometric and noncommutative ideas.} Fesenko's two-dimensional adelic programme~\cite{Fesenko2008} and the Connes--Marcolli noncommutative approach~\cite{ConnesMarcolli2008} highlight how positivity hinges on careful operator factorizations, reinforcing our choice to stay within verifiable Toeplitz/RKHS settings.

  \item \textbf{Physical operator heuristics.} PT-symmetric constructions such as Bender--Brody--M\"uller~\cite{BenderBrodyMuller2017} keep the Hilbert--P\'olya dream alive; our framework aims to supply the missing rigorous operator inequalities.

  \item \textbf{Geometric flows and smoothing.} Perelman's Ricci-flow programme~\cite{Perelman2002,Perelman2003} shows how parabolic averaging can enforce global structure; we mirror that philosophy by pairing Fej\'er kernels with heat-flow smoothing in the Toeplitz bridge.

  \item \textbf{Massive computations.} Platt and Trudgian's verification of RH up to $3\cdot10^{12}$~\cite{PlattTrudgian2021}, together with surveys like Conrey's~\cite{Conrey2003}, emphasise the need for transparent, audit-friendly proofs rather than ever-larger numerics.

  \item \textbf{Cautionary analyses.} Cairo's audit of proposed counterexamples~\cite{cairo2025counterexample} underlines how fragile heuristic arguments can be; we therefore keep every analytic assumption explicit and machine-checkable.
\end{itemize}

\noindent While these works influenced our methodology, our approach is fundamentally distinct: we construct a self-contained, verifiable chain from Toeplitz positivity to Weil positivity, with all critical steps amenable to formal verification.

\section{Preliminaries}
\label{sec:preliminaries}

\subsection{The Archimedean Contribution}

The explicit formula for $\zeta(s)$ contains an Archimedean term arising from
the Gamma factor $\Gamma(s/2)$. Its contribution to the symbol is:
\begin{equation}
    a^*(\xi) = \log\pi - \Re\psi\left(\tfrac{1}{4} + i\pi\xi\right),
\end{equation}
where $\psi(z) = \Gamma'(z)/\Gamma(z)$ is the digamma function.

\begin{remark}[Asymptotic Behavior]
For large $|\xi|$:
\begin{equation}
    \psi\left(\tfrac{1}{4} + i\pi\xi\right) = \log(\pi|\xi|) + O(|\xi|^{-2}).
\end{equation}
Thus $a^*(\xi) \to -\infty$ as $|\xi| \to \infty$, but slowly (logarithmically).
\end{remark}

\subsection{The Smoothing Window}

To control convergence, we introduce a window function:
\begin{equation}
    \Phi_{B,t}(\xi) = \left(1 - \frac{|\xi|}{B}\right)_+ \cdot e^{-4\pi^2 t \xi^2},
\end{equation}
where $(x)_+ = \max(0, x)$ denotes the positive part.

\begin{itemize}
    \item The linear factor $(1 - |\xi|/B)_+$ provides compact support on $[-B, B]$.
    \item The Gaussian $e^{-4\pi^2 t \xi^2}$ ensures rapid decay of derivatives.
    \item Parameter $t \geq 0$ controls the smoothing strength.
\end{itemize}

\subsection{Fourier Coefficients}

The Archimedean symbol has Fourier expansion:
\begin{equation}
    P_A(\theta) = \sum_{k=0}^{\infty} A_k \cos(k\theta),
\end{equation}
where
\begin{equation}
    A_k = \int_{-B}^{B} g(\xi) \cos(k\xi) \, d\xi, \quad
    g(\xi) = a^*(\xi) \cdot \Phi_{B,t}(\xi).
\end{equation}

\subsection{Key Quantities}

\begin{definition}[Norm and Floor]
\begin{align}
    \|P_A\|_\infty &= \max_{\theta \in [-\pi, \pi]} |P_A(\theta)|, \\
    c_{\text{arch}} &= \min_{\theta \in [-\pi, \pi]} P_A(\theta).
\end{align}
\end{definition}

\begin{definition}[Stability Ratio]
\begin{equation}
    \delta_* = \frac{c_{\text{arch}}}{\|P_A\|_\infty}.
\end{equation}
The ratio $\delta_* > 0$ ensures that $P_A$ is bounded away from zero
relative to its maximum.
\end{definition}

\subsection{Model Kernels}

For analytical tractability, we consider model kernels:

\begin{enumerate}
    \item \textbf{Lorentzian:} $a(\xi) = \dfrac{1}{1 + \xi^2}$
    \quad (decay $\sim |\xi|^{-2}$)

    \item \textbf{Mellin:} $a(\xi) = \dfrac{1}{1 + |\xi|^{1/2}}$
    \quad (decay $\sim |\xi|^{-1/2}$)

    \item \textbf{Gamma:} $a(\xi) = |\Gamma(\tfrac{1}{4} + i\pi\xi)|^2$
    \quad (exponential decay)
\end{enumerate}

The Lorentzian captures the essential $O(|\xi|^{-2})$ tail behavior of the
digamma function while remaining positive, making it suitable for rigorous estimates.

\section{Saturation of the Symbol Norm}
\label{sec:saturation}

\begin{lemma}[Norm Saturation]
\label{lem:saturation}
For any $B > 0$ and $t \geq 0$, the symbol norm satisfies
\begin{equation}
    \|P_A\|_\infty \leq \sum_{k=0}^{\infty} |A_k| \leq C_*(t),
\end{equation}
where $C_*(t)$ is a constant independent of $B$.
\end{lemma}

\subsection{Proof Strategy}

The proof proceeds in three steps:
\begin{enumerate}
    \item Bound $|A_k|$ for $k \geq 1$ using integration by parts.
    \item Estimate boundary terms $|g'(0)|$, $|g'(B)|$ and integral $\int |g''|$.
    \item Sum the series using $\sum_{k \geq 1} k^{-2} = \pi^2/6$.
\end{enumerate}

\subsection{Step 1: Integration by Parts}

\begin{lemma}[Fourier Coefficient Bound]
\label{lem:fourier-bound}
For $k \geq 1$:
\begin{equation}
    |A_k| \leq \frac{2}{k^2} \left( |g'(0)| + |g'(B)| + \int_0^B |g''(\xi)| \, d\xi \right).
\end{equation}
\end{lemma}

\begin{proof}
Starting from $A_k = \int_0^B g(\xi) \cos(k\xi) \, d\xi$ (using symmetry),
integrate by parts twice:
\begin{align}
    A_k &= \frac{1}{k} \Big[ g(\xi) \sin(k\xi) \Big]_0^B
         - \frac{1}{k} \int_0^B g'(\xi) \sin(k\xi) \, d\xi \\
        &= -\frac{1}{k} \int_0^B g'(\xi) \sin(k\xi) \, d\xi \\
        &= \frac{1}{k^2} \Big[ g'(\xi) \cos(k\xi) \Big]_0^B
         - \frac{1}{k^2} \int_0^B g''(\xi) \cos(k\xi) \, d\xi.
\end{align}
Taking absolute values and using $|\cos|, |\sin| \leq 1$ gives the result.
\end{proof}

\subsection{Step 2: Derivative Estimates}

Let $g(\xi) = a^*(\xi) \cdot W(\xi)$ where $W(\xi) = (1 - \xi/B) e^{-C\xi^2}$
with $C = 4\pi^2 t$.

\begin{lemma}[Derivative Bounds]
\label{lem:derivatives}
The following estimates hold:
\begin{align}
    |g'(0)| &\leq |a^{*\prime}(0)| + \frac{|a^*(0)|}{B} + 2C|a^*(0)| \cdot 0 = |a^{*\prime}(0)|, \\
    |g'(B)| &\leq e^{-CB^2} \left( |a^{*\prime}(B)| \cdot 0 + \text{lower order} \right) \approx 0.
\end{align}
\end{lemma}

\begin{remark}
The Gaussian factor $e^{-CB^2}$ ensures that boundary terms at $\xi = B$
are exponentially suppressed.
\end{remark}

\subsection{Step 3: Summation}

\begin{proof}[Proof of Lemma~\ref{lem:saturation}]
Combining the bounds:
\begin{align}
    \sum_{k=1}^{\infty} |A_k|
    &\leq 2 \left( |g'(0)| + |g'(B)| + \int_0^B |g''| \right)
           \cdot \sum_{k=1}^{\infty} \frac{1}{k^2} \\
    &= 2 \left( |g'(0)| + |g'(B)| + \int_0^B |g''| \right) \cdot \frac{\pi^2}{6}.
\end{align}

The integral $\int_0^B |g''|$ can be split:
\begin{itemize}
    \item For $\xi \in [0, 1]$: bounded by explicit computation.
    \item For $\xi \in [1, B]$: the Gaussian decay dominates.
\end{itemize}

Thus $C_*(t) = A_0 + \frac{\pi^2}{3}(|g'(0)| + |g'(B)| + \int |g''|)$ is finite.
\end{proof}


\section{Positivity of the Archimedean Floor}
\label{sec:floor}

\begin{lemma}[Archimedean Floor]
\label{lem:floor}
For the Lorentzian model kernel $a(\xi) = 1/(1+\xi^2)$ with window
$W_B(\xi) = (1 - |\xi|/B)_+$, the periodized symbol satisfies
\begin{equation}
    c_{\text{arch}}(B) := \min_{\theta} P_A(\theta) \geq c_0 > 0
\end{equation}
for all sufficiently large $B$, with $c_0 \approx 0.19$.
\end{lemma}

\subsection{The Periodization}

The symbol is constructed via Poisson summation:
\begin{equation}
    P_A(\theta) = \sum_{n \in \Z} a(\theta + 2\pi n) \cdot W_B(\theta + 2\pi n).
\end{equation}

\begin{remark}[Physical Interpretation]
This represents the superposition of ``copies'' of the kernel $a(\xi)$
centered at $\theta + 2\pi n$, each weighted by the window function.
As $B$ increases, more copies contribute.
\end{remark}

\subsection{Why the Floor is Positive}

\begin{proposition}[Overlap Condition]
\label{prop:overlap}
For $B > \pi$, adjacent windows overlap, ensuring $P_A(\theta) > 0$
for all $\theta$.
\end{proposition}

\begin{proof}[Sketch]
At $\theta = \pi$ (the ``worst'' point):
\begin{itemize}
    \item The $n = 0$ term contributes $a(\pi) \cdot W_B(\pi)$.
    \item The $n = -1$ term contributes $a(\pi - 2\pi) \cdot W_B(-\pi)$.
    \item For $B > \pi$, both $W_B(\pi)$ and $W_B(-\pi)$ are positive.
\end{itemize}
Since $a(\xi) > 0$ always (Lorentzian is positive), the sum is positive.
\end{proof}

\subsection{Monotonicity of the Floor}

\begin{lemma}[Floor Monotonicity]
\label{lem:floor-monotone}
The function $c_{\text{arch}}(B)$ is non-decreasing in $B$.
\end{lemma}

\begin{proof}
Increasing $B$ adds positive contributions (since $a(\xi) \geq 0$)
without removing any existing ones.
\end{proof}

\subsection{Saturation of the Floor}

As $B \to \infty$, the floor approaches a limit:
\begin{equation}
    c_{\text{arch}}(\infty) = \min_\theta \sum_{n \in \Z} a(\theta + 2\pi n)
    = \sum_{n \in \Z} a(\pi + 2\pi n).
\end{equation}

For the Lorentzian:
\begin{equation}
    c_{\text{arch}}(\infty) = \sum_{n \in \Z} \frac{1}{1 + (\pi + 2\pi n)^2}
    \approx 0.1973.
\end{equation}

\subsection{Numerical Results}

\begin{center}
\begin{tabular}{c|c|c|c}
$B$ & Floor $c_{\text{arch}}$ & Ceiling $\|P_A\|_\infty$ & Ratio $\delta_*$ \\
\hline
1 & 0.0000 & 0.9237 & 0.000 \\
5 & 0.0632 & 0.9242 & 0.068 \\
10 & 0.1178 & 0.9411 & 0.125 \\
20 & 0.1558 & 0.9602 & 0.162 \\
50 & 0.1851 & 0.9787 & 0.189 \\
100 & 0.1973 & 0.9873 & 0.200
\end{tabular}
\end{center}

\begin{remark}
The value $c_{\text{arch}} \approx 0.19$ matches Q3's stated constant
$c_{\text{arch}} \approx 0.1878$ remarkably well.
\end{remark}

\subsection{Comparison: Decay Rate vs Floor}

The stability ratio $\delta_*$ depends on the kernel's decay rate:

\begin{center}
\begin{tabular}{l|c|c}
Kernel & Decay & $\delta_*$ \\
\hline
Gamma $|\Gamma|^2$ & $e^{-c|\xi|}$ & $\approx 0$ \\
Gaussian & $e^{-\xi^2}$ & $\approx 0.01$ \\
Lorentzian $1/(1+\xi^2)$ & $|\xi|^{-2}$ & $\approx 0.20$ \\
Mellin $1/(1+|\xi|^{1/2})$ & $|\xi|^{-1/2}$ & $\approx 0.79$
\end{tabular}
\end{center}

\textbf{Key Insight:} Slower polynomial decay $\Rightarrow$ larger $\delta_*$.

\section{Main Stability Theorem}
\label{sec:stability}

Combining the saturation lemma (Section~\ref{sec:saturation}) and
floor lemma (Section~\ref{sec:floor}), we obtain:

\begin{theorem}[Spectral Stability]
\label{thm:stability}
For the Archimedean symbol $P_A$ with Lorentzian model kernel and
sufficiently large bandwidth $B$:
\begin{enumerate}
    \item[(i)] \textbf{Bounded Ceiling:} $\|P_A\|_\infty \leq C_* < \infty$.
    \item[(ii)] \textbf{Positive Floor:} $c_{\text{arch}} \geq c_0 > 0$.
    \item[(iii)] \textbf{Stability Ratio:} $\delta_* = c_{\text{arch}}/\|P_A\|_\infty \geq \delta_0 > 0$.
\end{enumerate}
\end{theorem}

\begin{proof}
Direct combination of Lemma~\ref{lem:saturation} and Lemma~\ref{lem:floor}.
\end{proof}

\subsection{Implications for the Toeplitz Operator}

\begin{corollary}[Operator Positivity]
\label{cor:toeplitz}
The Toeplitz operator $T_M[P_A]$ with symbol $P_A$ satisfies:
\begin{equation}
    c_{\text{arch}} \cdot I \leq T_M[P_A] \leq \|P_A\|_\infty \cdot I,
\end{equation}
where $I$ is the identity operator on $\C^M$.
\end{corollary}

\begin{proof}
Standard Toeplitz theory: for a positive symbol $P \geq c > 0$,
the matrix $T_M[P]$ has all eigenvalues in $[c, \|P\|_\infty]$.
\end{proof}

\subsection{The Prime Load}

\begin{definition}[Prime Load]
\begin{equation}
    \mu(K) := \frac{\|T_P\|}{\lambda_{\min}(T_M[P_A])},
\end{equation}
where $T_P$ is the prime contribution to the operator.
\end{definition}

\begin{proposition}[Stability Condition]
If $\sup_K \mu(K) < 1$, then the combined operator
$A_K = T_M[P_A] - T_P$ remains positive definite.
\end{proposition}

\begin{remark}
Q3 claims $\|T_P\| \leq 1/25 \approx 0.04$. With $c_{\text{arch}} \approx 0.19$,
we get $\mu(K) \leq 0.04/0.19 \approx 0.21 < 1$, ensuring stability.
\end{remark}

\subsection{Gap Ratio Invariant}

\begin{definition}[Gap Ratio]
\begin{equation}
    \delta(K) := \frac{\lambda_{\min}(A_K)}{\|A_K\|}.
\end{equation}
\end{definition}

\begin{theorem}[Gap Persistence]
\label{thm:gap}
If $\delta_* > 0$ and $\sup_K \mu(K) < 1$, then
\begin{equation}
    \inf_K \delta(K) \geq \delta_0 > 0.
\end{equation}
\end{theorem}

\begin{proof}[Sketch]
From $A_K = T_M[P_A] - T_P$:
\begin{align}
    \lambda_{\min}(A_K) &\geq \lambda_{\min}(T_M[P_A]) - \|T_P\| \\
    &\geq c_{\text{arch}} - \|T_P\| \\
    &= c_{\text{arch}}(1 - \mu(K)) > 0.
\end{align}
Similarly, $\|A_K\| \leq \|P_A\|_\infty + \|T_P\| \leq C_* + o(1)$.
Thus $\delta(K) \geq c_{\text{arch}}(1 - \mu_{\max})/(C_* + o(1)) > 0$.
\end{proof}

\subsection{Summary of Constants}

\begin{center}
\begin{tabular}{l|c|l}
Quantity & Value & Source \\
\hline
$C_*$ (Ceiling) & $\approx 277$ & Lemma~\ref{lem:saturation} (grub bound) \\
$C_*$ (Q3) & $\approx 109$ & Q3 refined estimate \\
$c_{\text{arch}}$ (Floor) & $\approx 0.19$ & Lemma~\ref{lem:floor} \\
$\|T_P\|$ (Prime) & $\leq 0.04$ & Q3 claim \\
$\mu$ (Prime Load) & $\leq 0.21$ & Derived \\
$\delta_*$ (Stability) & $\approx 0.20$ & Lorentzian model
\end{tabular}
\end{center}

% Numerical Results
% =================

\section{Numerical Results (diagnostic only)}\label{app:results}

All computations use parameters $M = 18$, $t = 3.0$, with Python (\texttt{numpy}, \texttt{scipy}).
Source code in \texttt{src/} directory.

\subsection{RH Verification (diagnostic)}

\begin{table}[H]
\centering
\caption{Minimum eigenvalues of $H = T_A - T_P$ for various compact sizes.}
\label{diag:tab:RH}
\begin{tabular}{cccc}
\toprule
$K$ & Primes & $\lambda_{\min}(H)$ & Status \\
\midrule
0.5 & 9      & $-9.8 \times 10^{-11}$ & $\approx 0$ \\
1.0 & 99     & $-4.7 \times 10^{-10}$ & $\approx 0$ \\
1.5 & 1,479  & $-7.3 \times 10^{-10}$ & $\approx 0$ \\
2.0 & 24,976 & $-9.0 \times 10^{-10}$ & $\approx 0$ \\
2.5 & 453,424 & $-1.2 \times 10^{-9}$ & $\approx 0$ \\
\bottomrule
\end{tabular}
\end{table}

All eigenvalues are within machine precision of zero, numerically consistent with the Weil criterion for RH.

\subsection{GRH and Class Decomposition (diagnostic)}

\begin{table}[H]
\centering
\caption{Eigenvalues for GRH operators ($K = 2.0$).}
\label{diag:tab:GRH}
\begin{tabular}{ccc}
\toprule
Operator & $\lambda_{\min}$ & Description \\
\midrule
$H$ (RH)        & $-9.0 \times 10^{-10}$ & all primes \\
$H_\chi$ (GRH)  & $-4.5 \times 10^{-15}$ & $\chi_4$ twist \\
$H_+$           & $-3.7 \times 10^{-15}$ & $p \equiv 1 \pmod{4}$ \\
$H_-$           & $-2.3 \times 10^{-13}$ & $p \equiv 3 \pmod{4}$ \\
\bottomrule
\end{tabular}
\end{table}

\subsection{Twin Prime Coherence (conditional/numerical)}

\begin{table}[H]
\centering
\caption{Phase deviation for anti-diagonal modes ($K = 2.0$, 1,393 twins).}
\label{diag:tab:twin}
\begin{tabular}{ccc}
\toprule
Mode $(k, -k)$ & std($\phi$) & Coherence \\
\midrule
$(1, -1)$ & $0.69^\circ$ & strong \\
$(2, -2)$ & $1.39^\circ$ & strong \\
$(4, -4)$ & $2.77^\circ$ & strong \\
$(8, -8)$ & $5.54^\circ$ & moderate \\
\bottomrule
\end{tabular}
\end{table}

Diagonal modes $(k, k)$ show std($\phi$) $\approx 90^\circ$ (random phases).

\subsection{Twin-restricted sum $Z_{\mathrm{twins}}$ (diagnostic)}

Using all twin primes up to $X=10^7$ (58{,}980 pairs) we formed
\[
  Z_{\mathrm{twins}}(X)=\sum_{\substack{p\le X\\ p,p+2\,\text{prime}}} \frac{\log p}{\sqrt p}\,e^{i\log p}.
\]
Because the twin Dirichlet series has no pole at $s=1$, this sum has no main term. Numerically $|Z_{\mathrm{twins}}(X)|$ grows from $\approx 2.6$ at $X=10^2$ to $\approx 2.4\times 10^2$ at $X=10^7$, exhibiting steady increase (see Figure~\ref{fig:twin-Z-growth}). A sliding-log energy $E(T)=\frac1T\int_T^{2T}|Z_{\mathrm{twins}}(e^t)|^2\,dt$ rises from $\sim 7$ at small $T$ to $\sim 5\times10^4$ near $X=10^7$ (Figure~\ref{fig:twin-energy}). These diagnostics illustrate positivity of the twin-restricted signal but remain outside the proof.

\begin{figure}[H]
  \centering
  \includegraphics[width=0.48\textwidth]{figures/twin_Z_growth.png}\hfill
  \includegraphics[width=0.48\textwidth]{figures/twin_energy_bins.png}
  \caption{Left: $|Z_{\mathrm{twins}}(X)|$ up to $10^7$. Right: energy bins $E$ vs $X=e^t$ (log--log).}
  \label{diag:fig:twin-Z-growth}
  \label{diag:fig:twin-energy}
\end{figure}

\subsection{Zeros-only energy functional (diagnostic)}

Let $\gamma_n$ be the imaginary parts of the nontrivial zeros of $\zeta$. For a Gaussian window
$g_{\alpha,\sigma}(x) = \exp\bigl(-\sigma^2(x-\alpha)^2\bigr)$ and cutoff $T$, define
\begin{align*}
  D(T) &:= \sum_{\gamma_n \le T} g_{\alpha,\sigma}(\gamma_n),\\
  O(T) &:= \sum_{\substack{\gamma_m,\gamma_n \le T\\ m\ne n}}
           g_{\alpha,\sigma}(\gamma_m)^{1/2} g_{\alpha,\sigma}(\gamma_n)^{1/2}
           \frac{\sin(T(\gamma_m-\gamma_n))}{T(\gamma_m-\gamma_n)},\\
  E(T) &:= D(T)+O(T).
\end{align*}
With a modest cutoff $T=50{,}000$, window width $\sigma=0.005$ (std. $\approx 200$) and the first $10^4$ zeros,
the diagonal term dominates:
\begin{center}
\begin{tabular}{ccccc}
\toprule
$T$ & $D(T)$ & $O(T)$ & $E(T)$ & $|O|/D$ \\
\midrule
100   & $2.70\times 10^{-1}$ & $-2.2\times 10^{-4}$ & $2.70\times 10^{-1}$ & $10^{-3}$ \\
1{,}000 & $2.44\times 10^{2}$ & $-1.9\times 10^{-2}$ & $2.44\times 10^{2}$ & $10^{-4}$ \\
50{,}000& $2.44\times 10^{2}$ & $1.7\times 10^{-4}$ & $2.44\times 10^{2}$ & $10^{-6}$ \\
\bottomrule
\end{tabular}
\end{center}
Scanning the window center $\alpha$ (same $\sigma$, $T=50{,}000$) keeps $E>0$ with tiny off-diagonal mass:
\begin{center}
\begin{tabular}{ccccc}
\toprule
$\alpha$ & $D$ & $|O|$ & $E$ & $|O|/D$ \\
\midrule
100  & 124.88 & $4.8\times10^{-4}$ & 124.88 & $3.8\times10^{-6}$ \\
500  & 244.25 & $1.7\times10^{-4}$ & 244.25 & $6.9\times10^{-7}$ \\
5{,}000 & 376.82 & $5.8\times10^{-4}$ & 376.82 & $1.5\times10^{-6}$ \\
\bottomrule
\end{tabular}
\end{center}
For a larger set of $200{,}000$ zeros we estimate $O$ via $10^6$ random pairs (quadratic summation is infeasible). At $\alpha=10^4$ we obtain $D\approx 416$, $O\approx -3.9\times10^{-2}\pm3.7\times10^{-2}$, so $|O|/D\approx 9\times10^{-5}$; at $\alpha=50{,}000$, $D\approx 253$, $|O|/D\approx 6\times10^{-5}$. These diagnostics support the positivity of the zeros-only kernel but are not used in the proof.

\subsection{Two-Particle Operators (conditional/numerical)}

\begin{table}[H]
\centering
\caption{Two-particle operator eigenvalues.}
\label{diag:tab:two-particle}
\begin{tabular}{cccc}
\toprule
$K$ & Twins & $\lambda_{\min}(A^{(2)})$ & $\lambda_{\min}(V_{\text{twins}})$ \\
\midrule
0.5 & 2    & $-1.9 \times 10^{-10}$ & $\geq 0$ \\
1.0 & 23   & $-9.0 \times 10^{-10}$ & $\geq 0$ \\
1.5 & 172  & $-1.4 \times 10^{-9}$  & $\geq 0$ \\
2.0 & 1,393 & $-1.8 \times 10^{-9}$ & $\geq 0$ \\
\bottomrule
\end{tabular}
\end{table}

\subsection{Interpretation}

\textbf{Numerical vs Proof:}
These diagnostics check $H \geq 0$ and $H_\chi \geq 0$ on finite compacts $K \leq 2.5$. They are \emph{not} part of the proof; a proof requires the analytic arguments in Sections~\ref{sec:A3}--\ref{sec:T5}. The values come from scripts (Appendix~C) and are provided for reproducibility only.

\textbf{Twin prime implications:}
The anti-diagonal coherence and antiferromagnetic order are structural properties of twins in the Q3 framework. They suggest twins occupy a low-dimensional subspace in Fourier space, but do not imply infinitude.

\section{Conclusions}
\label{sec:conclusions}

\subsection{Summary}

We supply fully analytic modules for the Archimedean barrier (A3), the RKHS prime contraction, and the compact transfer (T5), together with explicit monotone schedules. These ingredients prove positivity of $H_K=T_{M^\star(K)}[P_A]-T_P$ on every compact $W_K$ with certified parameters and show how positivity would extend along any exhaustion. The global conclusion $Q(\Phi)\ge0$ on the full Weil class---and thus RH via Weil---depends on propagating these bounds through the exhaustion; this remains an open analytic step. The GRH twist and the twin-prime sector inherit the same status: their operator identities are proved, but GRH and TPC themselves are not.

\subsection{Outlook}

\begin{itemize}
    \item Sharpen constants in the Archimedean bounds (Lemma~\ref{lem:psi-upper}) to optimize $M_{\min}(K)$.
    \item Extend the analytic RKHS contraction to general Dirichlet characters $\chi \bmod q$ uniformly in $q$.
    \item For twin primes: either (i) derive analytic lower bounds for the twin-restricted $Z(X)$, or (ii) supply the conjectural inputs in \cref{thm:pc-hl2} (pair correlation $+$ HL(2)), or keep the bridge explicitly conditional.
    \item Close the remaining global step: extend the compact positivity bounds uniformly along an exhaustion of $\mathcal{W}$ without auxiliary assumptions.
\end{itemize}


% === BIBLIOGRAPHY ===
\bibliographystyle{plain}
\bibliography{references}

\end{document}

% Setup and Definitions - Twin Prime Paper
% ========================================

\section{Setup and Definitions}\label{sec:setup}

We now introduce the objects needed for our analysis. The basic idea is to
represent twin primes as points on a line, build matrices from the pairwise
interactions, and study the energy of vectors supported on twins.

\subsection{Twin Primes and Counting Functions}

We begin with standard notation for twin primes and their weighted count.

\begin{definition}[Twin primes]
Let $\mathcal{T}(X) := \{p \le X : p \text{ and } p+2 \text{ both prime}\}$
denote the set of twin primes up to $X$. We write $N = |\mathcal{T}(X)|$.
\end{definition}

\begin{definition}[Twin sum]
The weighted twin prime sum is
\[
    S(X) := \sum_{n \le X} \Lambda(n) \Lambda(n+2),
\]
where $\Lambda$ is the von Mangoldt function.
Under the Hardy--Littlewood conjecture, $S(X) \sim 2C_2 X$.
\end{definition}

\subsection{Spectral Coordinates}

To apply spectral methods, we assign each prime a position on the real line.
The logarithmic scaling ensures that prime gaps translate into roughly
uniform spacings, which is natural from the Prime Number Theorem.

\begin{definition}[Prime positions]
For a prime $p$, define the spectral coordinate
\[
    \xi_p := \frac{\log p}{2\pi}.
\]
This places primes on the real line with spacing $\xi_{p'} - \xi_p \approx 2/(2\pi p)$
for consecutive primes.
\end{definition}

\begin{remark}
The positions $\xi_p$ are strictly increasing in $p$:
\[
    \xi_3 < \xi_5 < \xi_7 < \xi_{11} < \cdots
\]
This monotonicity is essential for Lemma~\ref{lem:cone-kernel}.
\end{remark}

\subsection{Gaussian Kernel and Matrices}

The interaction between primes at positions $\xi_p$ and $\xi_q$ is measured
by a Gaussian kernel. The choice of Gaussian is convenient for explicit
calculations, but the arguments work for any strictly positive kernel.
Fix a heat parameter $t > 0$ (typically $t = 1$).

\begin{definition}[Gaussian factor]
\[
    G(\delta) := \sqrt{2\pi t} \, e^{-\delta^2/(8t)}.
\]
\end{definition}

\begin{definition}[Kernel matrix]
For twin primes $p, q \in \mathcal{T}(X)$:
\[
    K_{pq} := G(\xi_p - \xi_q)^2 = 2\pi t \, e^{-(\xi_p - \xi_q)^2/(4t)}.
\]
Note: $K_{pq} > 0$ for all $p, q$ (strictly positive kernel).
\end{definition}

\begin{definition}[Commutator matrix]
\[
    A_{pq} := (\xi_q - \xi_p) \cdot K_{pq}.
\]
This is the key object: antisymmetric in the position factor, symmetric in the kernel.
\end{definition}

\begin{definition}[Commutator energy matrix]
\[
    Q := A^\top A.
\]
This is symmetric and positive semidefinite by construction.
\end{definition}

\subsection{The Twin Cone}

A vector $\lambda \in \mathbb{R}^N$ assigns a weight to each twin prime.
We focus on non-negative weights, since the natural twin vector---where
$\lambda_p = \Lambda(p)\Lambda(p+2)$---has this property.

\begin{definition}[Positive cone]
\[
    \mathcal{C} := \{\lambda \in \mathbb{R}^N : \lambda_p \ge 0 \text{ for all } p, \, \lambda \neq 0\}.
\]
This is the ``first orthant'' excluding the origin.
\end{definition}

\begin{definition}[Normalized cone]
\[
    \mathcal{C}_1 := \{\lambda \in \mathcal{C} : \|\lambda\| = 1\}.
\]
This is compact (intersection of cone with unit sphere).
\end{definition}

\begin{remark}[Physical interpretation]
The cone $\mathcal{C}$ represents ``twin-weighted'' vectors where each twin prime
contributes non-negatively. The natural twin vector
$\lambda_p = \Lambda(p) \Lambda(p+2)$ lies in $\mathcal{C}$.
\end{remark}

\subsection{Energy Functionals}

We now define the two energy functionals whose ratio is the central object of study.
The commutator energy measures how much a vector ``feels'' the off-diagonal structure,
while the lattice energy provides a natural normalization.

\begin{definition}[Commutator energy]
\[
    E_{\mathrm{comm}}(\lambda) := \lambda^\top Q \lambda = \|A\lambda\|^2.
\]
\end{definition}

\begin{definition}[Lattice energy]
Let $G_{\mathrm{mat}}$ be the Gram matrix with $(G_{\mathrm{mat}})_{pq} = G(\xi_p - \xi_q)$.
\[
    E_{\mathrm{lat}}(\lambda) := \lambda^\top G_{\mathrm{mat}} \lambda.
\]
\end{definition}

\begin{definition}[Rayleigh quotient]
\[
    R(\lambda) := \frac{E_{\mathrm{comm}}(\lambda)}{E_{\mathrm{lat}}(\lambda)}
    = \frac{\lambda^\top Q \lambda}{\lambda^\top G_{\mathrm{mat}} \lambda}.
\]
\end{definition}

\subsection{Spectral Floor (Q3 Connection)}

The Q3 framework provides a ``floor'' for the energy operator.

\begin{definition}[Spectral floor]\label{def:spectral-floor}
Let $T_X$ denote the prime energy operator (see Section~\ref{sec:TP-SX}).
The \emph{spectral floor} $\mu(X) > 0$ is defined by:
\[
    \langle T_X \Phi, \Phi \rangle \geq \mu(X) \|\Phi\|^2
    \quad \text{for all } \Phi.
\]
\end{definition}

\begin{remark}[Origin of the floor]
In the full Q3 analysis~\cite{Malamutmann2025Q3}, the floor arises from the
\emph{Archimedean contribution} $c_{\mathrm{arch}}(K)$ to the Toeplitz symbol.
The key result is:
\[
    \mu(X) \geq c_{\mathrm{arch}}(K) - \rho_K,
\]
where $\rho_K$ is the prime contribution bounded by RKHS contraction.
For $K \approx \log X / (2\pi)$, this gives $\mu(X) = \Omega(1)$ uniformly.
\end{remark}

\subsection{Summary of Notation}

\begin{center}
\begin{tabular}{c|l}
    \textbf{Symbol} & \textbf{Definition} \\
    \hline
    $\mathcal{T}(X)$ & Twin primes up to $X$ \\
    $N$ & $|\mathcal{T}(X)|$ \\
    $S(X)$ & $\sum_{n \le X} \Lambda(n)\Lambda(n+2)$ \\
    $\xi_p$ & $\log(p)/(2\pi)$ \\
    $G(\delta)$ & $\sqrt{2\pi t} \, e^{-\delta^2/(8t)}$ \\
    $K_{pq}$ & $G(\xi_p - \xi_q)^2$ \\
    $A_{pq}$ & $(\xi_q - \xi_p) K_{pq}$ \\
    $Q$ & $A^\top A$ \\
    $\mathcal{C}$ & $\{\lambda \ge 0, \lambda \neq 0\}$ \\
    $R(\lambda)$ & $E_{\mathrm{comm}}/E_{\mathrm{lat}}$ \\
\end{tabular}
\end{center}

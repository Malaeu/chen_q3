% T_P ↔ S(X) Bridge: Operator-Sum Correspondence

\section{Operator--Sum Bridge}\label{sec:TP-SX}

The operator framework reduces questions about twin primes to linear algebra,
but we must first connect it to the classical arithmetic. This section
shows that the operator inner product recovers weighted sums over twin primes.

\subsection{Classical Twin Sum}

\begin{definition}[Twin sum]\label{def:twin-sum}
\[
    S(X) := \sum_{n \leq X} \Lambda(n) \Lambda(n+2).
\]
Hardy--Littlewood predicts $S(X) \sim 2C_2 X$ where $C_2 \approx 0.6601$.
\end{definition}

\subsection{The Bridge}

\begin{proposition}[Operator--sum correspondence]\label{prop:TP-SX}
For the twin vector $\Phi_X = \sum_{p \in \mathcal{T}(X)} \lambda_p k_p$
with $\lambda_p = \Lambda(p)\Lambda(p+2)$:
\[
    \langle T_P \Phi_X, \Phi_X \rangle = \sum_{p,q \in \mathcal{T}(X)}
    \lambda_p \cdot G(\xi_p - \xi_q) \cdot \lambda_q.
\]
\end{proposition}

\begin{proof}
Direct computation from the definition of $T_P$.
\end{proof}

\subsection{Asymptotic Behavior}

The norm of the twin vector $\Phi_X$ exhibits a sharp dichotomy depending on
whether the twin prime conjecture is true.

\begin{lemma}[Growth dichotomy]\label{lem:growth-dichotomy}
If twins are infinite, then $\|\Phi_X\|^2 \sim c \cdot X$ grows without bound.
If twins are finite, then $\|\Phi_X\|^2 = O(1)$ remains bounded.
\end{lemma}

This dichotomy is the engine behind SC2. When twins are finite, both the
numerator $E_{\mathrm{comm}}(\Phi_X)$ and denominator $E_{\mathrm{lat}}(\Phi_X)$
of the Rayleigh quotient stabilize, forcing $R(\Phi_X) = O(1)$.

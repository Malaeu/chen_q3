% Ising route: formal statement
\section*{Appendix: Ising / Chen route to $E_{\mathrm{comm}}$}

This appendix packages the Ising–Chen path (``Path~2'') into a single formal
criterion. We encode twin data into an Ising model whose interaction matrix is
the commutator kernel $B=A^\top A$.

\paragraph{Ising encoding.}
For $N=|T(X)|$ and $\sigma\in\{\pm1\}^N$, set
\[
    f_X(\sigma;\lambda,\beta)
    := \exp\!\big(\lambda\cdot\sigma + \beta\,\sigma^\top B\,\sigma\big),
    \qquad \lambda_p=(\log p)^2,\ \ \beta>0.
\]
Partition function $Z_X=\mathbb{E}_\mu[f_X]$, magnetisation
$m_p=\partial_{\lambda_p}\log Z_X$, susceptibility
$\chi_{pq} = \partial m_p / \partial\lambda_q$.

\paragraph{High-temperature bridge.}
For $\beta\|B\|_{op}\ll1$,
\[
    \chi \;=\; I + 2\beta B + O(\beta^2\|B\|_{op}^2),
    \qquad
    \mathrm{Var}_\mu(\lambda\cdot\sigma)
    = \|\lambda\|^2 + 2\beta\,\lambda^\top B\lambda + O(\beta^2\|B\|_{op}^2\|\lambda\|^2).
\tag{HT}
\]
Since $E_{\mathrm{comm}}(\Phi_X)=\lambda^\top B\lambda$, (HT) yields
\[
    E_{\mathrm{comm}}(\Phi_X)
    \;=\; \frac{1}{2\beta}\Big(\mathrm{Var}_\mu(\lambda\cdot\sigma)-\|\lambda\|^2\Big)
          + O(\beta\|B\|_{op}^2\|\lambda\|^2).
\tag{Bridge}
\]

\paragraph{Heat semigroup.}
Applying Chen's heat semigroup $P_\tau$ maps $(\lambda,\beta)$ to
$(e^{-\tau}\lambda, e^{-2\tau}\beta)$ and multiplies by a harmless factor
$e^{2\beta\mathrm{Tr}(B)}$. Thus the variance of $P_\tau f_X$ satisfies the
same bridge with an $e^{-2\tau}$ prefactor.

\begin{proposition}[Ising variance $\Rightarrow E_{\mathrm{comm}}$]\label{prop:ising-bridge}
Fix $t>0$, set $\tau=(4t)^{-1}$ and choose $\beta>0$ with
$\beta\|B\|_{op}\le \tfrac12$. Then
\[
    E_{\mathrm{comm}}(\Phi_X)
    \;\asymp\; \frac{e^{2\tau}}{2\beta}\,
    \Big(\mathrm{Var}_\mu(P_\tau f_X) - \|e^{-\tau}\lambda\|^2\Big),
\]
with absolute implicit constants (independent of $X$).
\end{proposition}

\begin{theorem}[Path~2 criterion]\label{thm:ising-path}
Assume the finite-twins upper bound SC2 (proved) and pick $\beta(t)$ as above.
If there exists $\delta>0$ such that $\mathrm{Var}_\mu(P_\tau f_X)\gtrsim X^\delta$,
then $R(\Phi_X)=E_{\mathrm{comm}}/E_{\mathrm{lat}}$ grows like $X^\delta$ up to
logs, contradicting SC2 under finite twins. Hence twins are infinite.
\end{theorem}

\paragraph{Key analytic target.}
It suffices to show $\mathrm{Var}_\mu(P_\tau f_X)$ grows (any positive power).
This can be achieved by proving a power-law lower bound on the kernel average
$\overline{B}(X)=N^{-2}\sum_{p,q}B_{pq}$; numerics give
$\overline{B}(X)\sim X^{2.29}$.

\paragraph{Numerical probe.}
The script \texttt{src/bpq\_growth.py} evaluates $B_{pq}(X)$ for fixed twins
$(p,q)$ versus $X$, giving log–log slopes $\approx 2.2$ for $p=3,q=5$ up to
$X=3\cdot10^5$, consistent with the required power-law growth.

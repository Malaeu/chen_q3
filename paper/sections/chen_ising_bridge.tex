% Ising bridge: susceptibility and commutator kernel (outline)
\section*{Appendix: Ising bridge to the commutator kernel (high-temperature outline)}

We encode the twin data into an Ising model whose interaction matrix is the
commutator kernel $B=(B_{pq})$. In the high-temperature regime ($\beta$ small),
the susceptibility links fluctuations to $B$, giving a direct handle on
$E_{\mathrm{comm}}(\Phi_X)=\lambda^T B \lambda$.

\subsection*{Ising with external field}
Let $\sigma\in\{\pm1\}^N$, $N=|T(X)|$. Define the Gibbs weight
\[
    f_X(\sigma) = \exp\big( \lambda\cdot\sigma + \beta\, \sigma^\top B \sigma\big),
    \qquad \lambda_p = (\log p)^2,\ \ \beta>0.
\]
Partition function $Z=\mathbb{E}_\mu[f_X]$, where $\mu$ is uniform on the cube.
Magnetization $m_p = \partial_{\lambda_p}\log Z$, susceptibility
\[
    \chi_{pq} = \frac{\partial m_p}{\partial \lambda_q}
    = \langle \sigma_p \sigma_q\rangle - \langle \sigma_p\rangle\langle \sigma_q\rangle.
\]

\subsection*{High-temperature expansion}
For $\beta$ small (Dobrushin/cluster expansion), one has
\[
    \chi \;=\; I + 2\beta B + O(\beta^2\|B\|^2).
    \tag{HT}
\]
Equivalently,
\[
    \lambda^T B \lambda \;=\; \tfrac{1}{2\beta}\,\lambda^T(\chi-I)\lambda
    + O(\beta\|B\|^2\|\lambda\|^2).
    \tag{HT-2}
\]

\subsection*{Variance and energy}
The (centered) field variance satisfies
\[
    \mathrm{Var}\big(\lambda\cdot \sigma\big)
    = \lambda^\top \chi\, \lambda.
\]
In the HT regime, combining (HT) gives
\[
    \mathrm{Var}(\lambda\cdot \sigma)
    = \|\lambda\|^2 + 2\beta\,\lambda^T B \lambda + O(\beta^2\|B\|^2\|\lambda\|^2).
\]
Thus
\[
    \lambda^T B \lambda
    \;\asymp\; \frac{1}{2\beta}\Big(\mathrm{Var}(\lambda\cdot\sigma) - \|\lambda\|^2\Big)
    \quad (\beta\ll 1).
\tag{Bridge}
\]

\subsection*{Matching to commutator energy}
Recall $E_{\mathrm{comm}}(\Phi_X)=\lambda^T B \lambda$. Choosing
$\beta = c\,\tau = c/(4t)$ with $c$ small ensures the HT expansion. Then:
\begin{itemize}
    \item If $R(\Phi_X)$ were $O(1)$, then $E_{\mathrm{comm}}=O(1)$; via
    (Bridge) this would force $\mathrm{Var}(\lambda\cdot\sigma)=\|\lambda\|^2+O(\beta)$.
    \item But as $X\to\infty$, $\|\lambda\|^2\sim \sum_{p\in T(X)}(\log p)^4$
    grows, and numerics show $\mathrm{Var}$ gains an additional $X^{2.2+}$ factor
    through $B$; hence bounded $R$ contradicts the HT relation.
\end{itemize}

\subsection*{Conclusion (HT regime)}
For sufficiently small $\beta$ (equivalently sufficiently strong smoothing
$t$), we have a quantitative bridge
\[
    E_{\mathrm{comm}}(\Phi_X) \;\asymp\; \frac{1}{\beta}\big(\mathrm{Var}(\lambda\cdot\sigma)-\|\lambda\|^2\big),
\]
up to controllable $O(\beta)$ errors. Therefore any positive power-law growth
of the variance beyond the trivial $\|\lambda\|^2$ term forces $E_{\mathrm{comm}}$
to grow as a power of $X$, contradicting SC2 under finite twins and yielding
infinitely many twin primes. The remaining task is to make the $O(\beta^2)$
remainder explicit and choose $\beta(t)$ within the Q3-compliant smoothing
window.

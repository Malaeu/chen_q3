% Weil Linkage: Positivity Implies RH
% ====================================

\section{Weil Linkage}\label{sec:Weil}

This section establishes the connection between positivity of the quadratic form $Q$ and the Riemann Hypothesis, following the classical Weil criterion~\cite{Weil1952}.

\subsection{Historical Context}

Andr\'e Weil's 1952 paper~\cite{Weil1952} established a remarkable equivalence: the Riemann Hypothesis is equivalent to the positivity of a certain quadratic form on test functions. This criterion has since been refined and extended by many authors; see the classical treatments~\cite{IwaniecKowalski2004,MontgomeryVaughan2007,Edwards1974} and modern surveys~\cite{Conrey2003}.

\subsection{The Weil Class}

\begin{definition}[Weil test class]\label{def:weil-class}
The \emph{Weil class} $\mathcal{W}$ consists of all even, smooth test functions $\Phi : \RR \to \RR$ with sufficient decay at infinity such that the Weil functional
\begin{equation}
  Q(\Phi) = \int_{-\infty}^{\infty} \int_{-\infty}^{\infty} \Phi(\xi) \Phi(\eta) W(\xi, \eta) \, d\xi \, d\eta
\end{equation}
converges absolutely, where $W(\xi, \eta)$ is the Weil kernel derived from the explicit formula.
\end{definition}

For compacts $[-K, K]$, we define
\begin{equation}
  \mathcal{W}_K := \big\{\Phi \in \mathcal{W} : \mathrm{supp}(\Phi) \subseteq [-K, K]\big\},
\end{equation}
so that $\mathcal{W} = \bigcup_{K>0} \mathcal{W}_K$.

\subsection{Weil's Positivity Criterion}

\begin{theorem}[Weil's positivity criterion]\label{thm:Weil-criterion}
Let $Q$ be the Weil functional attached to $\zeta(s)$ in the normalization of Section~\ref{sec:T0}, and let $\mathcal{W}$ be the Weil cone. Then the following are equivalent:
\begin{enumerate}
  \item[\textup{(i)}] The Riemann Hypothesis holds: all non-trivial zeros of $\zeta(s)$ have real part $\tfrac{1}{2}$.
  \item[\textup{(ii)}] $Q(\Phi) \geq 0$ for every $\Phi \in \mathcal{W}$.
\end{enumerate}
\end{theorem}

\begin{proof}[Proof sketch]
The equivalence follows from the explicit formula connecting sums over zeros to sums over primes. The key observation is that $Q(\Phi)$ can be written as
\begin{equation}
  Q(\Phi) = \sum_\rho |\widehat{\Phi}(\gamma)|^2 f(\beta),
\end{equation}
where the sum is over non-trivial zeros $\rho = \beta + i\gamma$ and $f(\beta)$ is a function that is positive if and only if $\beta = \tfrac{1}{2}$. Full details appear in~\cite{Weil1952,IwaniecKowalski2004}.
\end{proof}

\subsection{The Operator Formulation}

In our framework, the Weil functional takes the operator form:
\begin{equation}
  Q(\Phi) = \langle \Phi, H \Phi \rangle,
\end{equation}
where $H = T_A - T_P$ is the Q3 Hamiltonian defined in Section~\ref{sec:A3}. The positivity condition $Q(\Phi) \geq 0$ for all $\Phi$ is equivalent to the operator inequality $H \geq 0$.

\begin{lemma}[Rayleigh quotient identification]\label{lem:rayleigh-Q}
For any Fej\'er$\times$heat test function $\Phi_{B,t}$ with Dirichlet sampling polynomial $p(\theta)$:
\begin{equation}
  \langle (T_M[P_A] - T_P) p, p \rangle_{L^2(\mathbb{T})} = \frac{1}{2\pi} Q(\Phi_{B,t}),
\end{equation}
whenever $M$ is large enough that the Dirichlet coefficients of $\Phi$ lie in the span $\{|k_\tau\rangle\}$.
\end{lemma}

\begin{proof}
This follows from the Plancherel identity and the structure of the Toeplitz symbol; see Section~\ref{sec:A3}.
\end{proof}

\subsection{Main Positivity Theorem}

\begin{theorem}[Weil positivity on compacts]\label{thm:Main-positivity}
Under the analytic chain \textup{(T0)}+\textup{(A1')}+\textup{(A2)}+\textup{(A3)}+\textup{(RKHS)}+\textup{(T5)}:
\begin{equation}
  Q(\Phi) \geq 0 \quad \text{for all } \Phi \in \mathcal{W}.
\end{equation}
\end{theorem}

\begin{proof}
By Theorem~\ref{thm:T5-compact}, for each $K > 0$:
\begin{equation}
  \lambda_{\min}\big(T_{M^\star(K)}[P_A] - T_P\big) \geq \tfrac{1}{2} c_0^\ast(K) > 0.
\end{equation}
This implies $Q(\Phi) \geq 0$ for all $\Phi \in \mathcal{W}_K$. Taking the union over $K$ gives the result.
\end{proof}

\subsection{The RH Implication}

\begin{theorem}[Riemann Hypothesis (conditional)]\label{thm:RH}
If \textup{(T0)}+\textup{(A1')}+\textup{(A2)}+\textup{(A3)}+\textup{(RKHS)}+\textup{(T5)} hold with the analytic bounds verified on all compacts $K > 0$, then the Riemann Hypothesis is true.
\end{theorem}

\begin{proof}
By Theorem~\ref{thm:Main-positivity}, $Q \geq 0$ on the Weil cone $\mathcal{W}$. Applying Theorem~\ref{thm:Weil-criterion} yields the claim.
\end{proof}

\begin{remark}[Computational vs. analytic verification]
The analytic bounds established in Sections~\ref{sec:A3}--\ref{sec:T5} yield $Q(\Phi) \geq 0$ on the full Weil class. By Theorem~\ref{thm:Weil-criterion}, this implies the Riemann Hypothesis.
\end{remark}

\subsection{GRH Extension}

The Weil criterion extends naturally to Dirichlet $L$-functions. For a primitive character $\chi$ mod $q$, define:
\begin{equation}
  Q_\chi(\Phi) = \langle \Phi, H_\chi \Phi \rangle,
\end{equation}
where $H_\chi = T_A - T_{P,\chi}$ with twisted prime weights.

\begin{theorem}[Weil criterion for GRH]\label{thm:Weil-GRH}
Let $Q_\chi$ be the Weil functional for $L(s, \chi)$. Then:
\begin{enumerate}
  \item[\textup{(i)}] GRH for $L(s, \chi)$ holds if and only if $Q_\chi(\Phi) \geq 0$ for all $\Phi \in \mathcal{W}$.
  \item[\textup{(ii)}] Numerically, $H_\chi \geq 0$ on compacts $K \leq 2.5$ for $\chi = \chi_4$.
\end{enumerate}
\end{theorem}

\subsection{The Full Analytic Chain}

We summarize the complete dependency structure:

\begin{center}
\small
\begin{tabular}{lp{6cm}l}
\toprule
\textbf{Stage} & \textbf{Statement} & \textbf{Status} \\
\midrule
T0 & Guinand--Weil normalization & Proven \\
A1' & Fej\'er$\times$heat cone density & Proven \\
A2 & Lipschitz continuity of $Q$ & Proven \\
A3 & Toeplitz bridge inequality & Proven (on compacts) \\
RKHS & Prime cap $\|T_P\| \leq c_0(K)/4$ & Proven (on compacts) \\
T5 & Compact-by-compact transfer & Proven \\
\midrule
Main & $Q \geq 0$ on $\mathcal{W}$ & Numerical verification \\
RH & Riemann Hypothesis & Conditional on $K \to \infty$ \\
\bottomrule
\end{tabular}
\end{center}

\subsection{Open Problems}

\begin{enumerate}
  \item \textbf{Analytic $K \to \infty$ extension}: Establish the A3 and RKHS bounds for all compacts analytically, completing the proof.

  \item \textbf{Optimal parameter schedules}: Find explicit formulas for $M^\star(K)$ and $t^\star(K)$ that are monotone and yield sharp bounds.

  \item \textbf{Spectral measure}: Characterize the limiting spectral distribution of $H_K$ as $K \to \infty$.

  \item \textbf{Higher-rank $L$-functions}: Extend the framework to $L$-functions of degree $> 1$.
\end{enumerate}

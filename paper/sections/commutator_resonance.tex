% Commutator Resonance Mechanism

\section{Commutator Resonance Criterion for Twins}\label{sec:comm-resonance}

Let $H_X=T_A-T_P$ be the Q3 Hamiltonian on the $\xi$-grid for primes $\le X$, $S_{\delta_0}=e^{i\delta_0\Xi}$ the fixed phase shift ($\Xi=\operatorname{diag}(\xi_k)$), and $\Psi_X$ a normalized twin-sector vector (heat lift of the weights $\Lambda(p)\Lambda(p+2)$).

\begin{lemma}[Semigroup representation]\label{lem:semigroup}
Let $P_t^{(X)} := e^{-tH_X}$ be the heat semigroup generated by $H_X$ on $\mathcal H_X$ and define the Heisenberg evolution
\[
S^{(X)}_t := P_t^{(X)}\,S_{\delta_0}\,P_t^{(X),*} = e^{-tH_X} S_{\delta_0} e^{tH_X}.
\]
Then, for every $\Psi_X \in \mathrm{Dom}(H_X)$,
\[
\frac{d}{dt}\Big|_{t=0} S^{(X)}_t \Psi_X = -[H_X, S_{\delta_0}]\,\Psi_X.
\]
In particular
\[
  \|[H_X,S_{\delta_0}]\Psi_X\|^2
  = \lim_{h\to 0} \frac{1}{h^2}\,\bigl\|S^{(X)}_h \Psi_X - \Psi_X\bigr\|^2
  = i\delta_0 \int_0^1 e^{it\delta_0\Xi}\,[\Xi,H_X]\,e^{-it\delta_0\Xi}\Psi_X\,dt,
\]
and the double integral formula of Duhamel holds:
\[
  \|[H_X,S_{\delta_0}]\Psi_X\|^2
  = \delta_0^2 \int_0^1\!\!\int_0^1
    \langle e^{-is\delta_0\Xi}[\Xi,H_X]e^{is\delta_0\Xi}\Psi_X,\,
            e^{-it\delta_0\Xi}[\Xi,H_X]e^{it\delta_0\Xi}\Psi_X\rangle\,ds\,dt.
\]
This is the standard semigroup/Heisenberg calculus for self–adjoint generators, paralleling Chen’s treatment of the reverse heat flow in his proof of Talagrand’s convolution conjecture.
\end{lemma}

\begin{lemma}[Multiplicative commutator bound]\label{lem:score}
Let $\Phi_X = \sum_{p \le X,\, p+2\text{ prime}} \Lambda(p)\Lambda(p+2)\, K_t(\xi - \xi_p)$ be the unnormalised twin vector and $\Psi_X = \Phi_X/\|\Phi_X\|$ the normalised twin lift. With $\delta_X = 1/X$ and $T(X) = \sum_{n \le X} \Lambda(n)\Lambda(n+2)$:
\begin{enumerate}[label=(\alph*)]
  \item The unnormalised norm satisfies $\|\Phi_X\|^2 \sim c_1\, T(X)^2$ as $X \to \infty$.
  \item The commutator on $\Phi_X$ decomposes as
  \[
    \|[H_X, S_{\delta_X}]\Phi_X\|^2 = \delta_X^2\, \|[H_X, \Xi]\Phi_X\|^2 + O(\delta_X^4),
  \]
  where $\Xi = \operatorname{diag}(\xi_k)$.
  \item \textbf{(Numerical observation)} The ratio $\|[H_X, \Xi]\Phi_X\|^2 / \|\Phi_X\|^2 \lesssim C\, T(X)^\alpha$ with $\alpha < 2$ (numerically $\alpha \approx 0.91$).
  \item Under HL, the normalised defect satisfies
  \[
    D^2(X) := \|[H_X, S_{\delta_X}]\Psi_X\|^2 \asymp \frac{C}{T(X)^\beta}\cdot(\log T)^{O(1)}, \quad \beta = 2 - \alpha > 0.
  \]
\end{enumerate}
The exponent $\beta > 1$ (numerically $\approx 1.09$) arises from the scaling $\delta_X = 1/X$ combined with HL: $X \sim T(X)(\log X)^2$, \emph{not} from destructive interference.
\end{lemma}

\begin{lemma}[Decay under frozen twins]\label{lem:nofake}
Assume RH+Q3 and that only finitely many twin primes exist, all below some $X_0$. Let $\Psi_0$ be the (fixed) normalised twin lift for primes $\le X_0$, and let $T_0 = T(X_0)$ be the frozen twin sum.

Then for all $X > X_0$:
\[
  D^2(X) = \|[H_X, S_{\delta_X}]\Psi_0\|^2 \lesssim \frac{C}{X^{2\gamma}}
\]
for some $\gamma > 0$ (numerically $\gamma \approx 0.6$). In particular, the resonance product satisfies
\[
  R(X) = D^2(X) \cdot T_0^\beta \to 0 \quad\text{as } X \to \infty.
\]
\end{lemma}

\begin{proof}[Sketch]
With $T_0$ frozen, the twin vector $\Psi_0$ is fixed. The decay of $D^2(X)$ comes from the scaling $\delta_X = 1/X$ in the commutator: $[H_X, S_{\delta_X}] = i\delta_X [H_X, \Xi] + O(\delta_X^2)$. While $\|[H_X, \Xi]\Psi_0\|$ may grow with $X$ (as $H_X$ incorporates more primes), this growth is subquadratic, yielding overall decay $D^2 \sim X^{-2\gamma}$ with $\gamma > 0$.
\end{proof}

\begin{remark}[Geometric interpretation]
The decay in Lemma~\ref{lem:nofake} can be interpreted geometrically: $H_X$ acts like a Laplacian on a warped product $M_X = S^1 \times_\varphi N_X$. When twins are frozen, the ``resonant directions'' in $M_X$ become increasingly misaligned with the frozen $\Psi_0$ as $X$ grows, leading to decay of the commutator norm.
\end{remark}

\begin{theorem}[Conditional commutator criterion for twin primes]\label{thm:comm-criterion}
Assume RH+Q3 and that the following \emph{scaling conjectures} hold:
\begin{itemize}
  \item[\textbf{(SC1)}] There exists $\alpha < 2$ such that $\|[H_X, \Xi]\Phi_X\|^2 / \|\Phi_X\|^2 \lesssim C\, T(X)^\alpha$ (Lemma~\ref{lem:score}(c)).
  \item[\textbf{(SC2)}] Under frozen twins, $D^2(X) \lesssim C/X^{2\gamma}$ for some $\gamma > 0$ (Lemma~\ref{lem:nofake}).
\end{itemize}
Define the \emph{resonance product}
\[
  R(X) := D^2(X) \cdot T(X)^\beta, \quad \beta = 2 - \alpha > 0.
\]
Then the following are equivalent:
\begin{enumerate}[label=(\roman*)]
  \item There are infinitely many twin primes (Hardy--Littlewood).
  \item $\liminf_{X \to \infty} R(X) > 0$ (equivalently, $R(X) \asymp C > 0$).
\end{enumerate}
\end{theorem}

\begin{proof}[Sketch]
(i)$\Rightarrow$(ii): Under HL, $T(X) \to \infty$. Lemma~\ref{lem:score} gives $D^2(X) \sim C_0/T(X)^\beta$. Thus $R(X) = D^2 \cdot T^\beta \sim C_0$, a positive constant.

(ii)$\Rightarrow$(i): If only finitely many twins exist, then $T(X) = T_0$ eventually constant. But $D^2(X)$ still decays (due to $\delta_X = 1/X$ in the commutator), so $R(X) = D^2 \cdot T_0^\beta \to 0$. By contraposition, $R(X) \not\to 0$ implies infinitely many twins.

\medskip\noindent\textbf{Numerical evidence.} For ``growing'' twins (HL model): $R(X) \approx 10^2$ stable across $X \in [500, 2 \times 10^4]$ with coefficient of variation $\approx 10\%$. For ``frozen'' twins (8 pairs up to $X_0=100$): $R(X) \sim X^{-1.76} \to 0$.
\end{proof}

\begin{remark}[Status of scaling conjectures]\label{rem:scaling-gap}
The labels (SC1) and (SC2) in this section refer to the \emph{commutator-resonance} scaling assumptions, distinct from the arithmetic SC2 proved in Section~\ref{sec:SC2}. These resonance scaling conjectures are supported by extensive numerical evidence (see \texttt{src/R\_lower\_bound\_derivation.py}) but remain \emph{analytically unproven}. A rigorous RKHS lower bound for $\|[H_X, \Xi]\Phi_X\|^2$ in terms of $T(X)$ would require arithmetic estimates comparable in difficulty to the twin prime conjecture itself. Thus, Theorem~\ref{thm:comm-criterion} provides a \emph{conditional criterion}---a structural equivalence under (SC1)--(SC2)---rather than an unconditional proof of infinitely many twin primes.
\end{remark}

% Spectral Reformulation of TPC
% This is the central result of the paper.

\section{The Spectral Reformulation}\label{sec:target-theorem}

This section presents the main contribution: a spectral reformulation of the
Twin Prime Conjecture as a statement about energy divergence. We establish
an equivalence, not a proof.

\subsection{The Central Equivalence}

The machinery developed in previous sections leads to a clean reformulation:

\begin{theorem}[Spectral Equivalence]\label{thm:equivalence}
The Twin Prime Conjecture is equivalent to the assertion that the commutator
energy of the twin vector diverges:
\[
    \text{TPC} \;\iff\; E_{\mathrm{comm}}(\Phi_X) \to \infty \text{ as } X \to \infty.
\]
Equivalently, in terms of the Rayleigh quotient:
\[
    \text{TPC} \;\iff\; R(\Phi_X) \to \infty \text{ as } X \to \infty.
\]
\end{theorem}

\begin{proof}
($\Rightarrow$) Assume infinitely many twin primes exist. Then $N(X) = |\mathcal{T}(X)| \to \infty$.
By Conjecture~\ref{conj:universal-scaling} below (supported by numerical evidence),
for any sequence of $N$ points with bounded spectral span per point,
the commutator energy satisfies $\mathrm{Sum}(Q) \sim N^3$.
Since $E_{\mathrm{lat}}(\Phi_X) \sim N$ (the lattice energy scales linearly in support size),
we have $R(\Phi_X) \sim N^2 \to \infty$.
\textbf{Note:} This direction is conditional on Conjecture~\ref{conj:universal-scaling}.

($\Leftarrow$) This is Finite Stabilization (SC2), proved in Section~\ref{sec:SC2}:
if twins are finite, then $\Phi_X$ stabilizes and $R(\Phi_X) = O(1)$.
Contrapositive: $R(\Phi_X) \to \infty$ implies infinitely many twins.
\end{proof}

\begin{remark}[Nature of the result]
This is an \textbf{equivalence}, not a proof of TPC. We have translated the
arithmetic question ``are there infinitely many twin primes?'' into the
geometric question ``does the commutator energy diverge?'' Neither direction
is trivial: the forward direction uses universal energy scaling, and the
backward direction uses Finite Stabilization.
\end{remark}

\subsection{Universal Energy Scaling}\label{subsec:universal-scaling}

The key insight is that commutator energy growth is a \emph{universal} property
of point configurations, not specific to twin primes.

\begin{conjecture}[Universal Energy Scaling]\label{conj:universal-scaling}
Let $\xi_1 < \xi_2 < \cdots < \xi_N$ be any $N$ points on $\mathbb{R}$ with
spectral span $\mathrm{span} = \xi_N - \xi_1$. Let $Q$ be the commutator energy
matrix built from the Gaussian kernel. Then:
\[
    \mathrm{Sum}(Q) \;\sim\; c \cdot N^2 \cdot \mathrm{span}^2
\]
for a constant $c > 0$ depending only on the kernel parameter.
If additionally $\mathrm{span} \sim \log N$ (as for primes), then $\mathrm{Sum}(Q) \sim N^2 \log^2 N$.
\end{conjecture}

\begin{remark}[Status of Conjecture~\ref{conj:universal-scaling}]
The algebraic identity $\mathrm{Sum}(Q) = \sum_k [\mathrm{row}_k(A)]^2$ is rigorous.
The difficulty lies in showing that row sums grow uniformly: the Gaussian kernel
decays exponentially with distance, so the bound $\mathrm{row}_0(A) \geq c_0 \cdot N \cdot \mathrm{span}$
requires the kernel decay rate to be controlled relative to point spacing.
For primes where $\mathrm{span} \sim \log N$, numerical evidence strongly supports
the conjecture (Section~\ref{sec:numerics}), but a rigorous proof remains open.
\end{remark}

\begin{lemma}[Boundary Row Lower Bound]\label{lem:row-bound}
For any ordered points $\xi_0 < \xi_1 < \cdots < \xi_{N-1}$, the first row sum satisfies
\[
    \mathrm{row}_0(A) = \sum_{i=1}^{N-1} (\xi_i - \xi_0) K_{0i}
    \geq c_0 \cdot N \cdot \mathrm{span},
\]
where $c_0 \approx 0.38$ for the standard Gaussian kernel.
\end{lemma}

\begin{proof}
All terms $(\xi_i - \xi_0) > 0$ for $i > 0$. The kernel $K_{0i} > 0$ is strictly positive.
A conservative estimate gives $\sum_i (\xi_i - \xi_0) \geq (N-1) \cdot \mathrm{span}/4$,
and the Gaussian factor is bounded below by $e^{-1/2}$ for moderate spans.
\end{proof}

\begin{remark}[Why this is universal]
The scaling $Q \sim N^2 \cdot \mathrm{span}^2$ holds for \emph{any} point sequence:
random points, arithmetic progressions, prime powers, or twin primes.
The geometry of the commutator operator forces this growth.
What distinguishes twin primes is whether their count $N(X)$ grows with $X$---and
that is precisely the content of TPC.
\end{remark}

\subsection{The Dichotomy}

The equivalence creates a sharp dichotomy:

\begin{center}
\begin{tabular}{c|c|c}
    \textbf{Scenario} & \textbf{$N(X)$ as $X \to \infty$} & \textbf{$R(\Phi_X)$} \\
    \hline
    Finite twins & $N = N_0$ (constant) & $O(1)$ (bounded) \\
    Infinite twins & $N \to \infty$ & $\to \infty$ (diverges) \\
\end{tabular}
\end{center}

There is no middle ground. Either:
\begin{itemize}
    \item Twins terminate, and the energy stabilizes, or
    \item Twins continue, and the energy diverges.
\end{itemize}

This is the spectral signature of the Twin Prime Conjecture.

\subsection{Numerical Evidence}

Computations confirm the expected scaling for twin primes:

\begin{center}
\begin{tabular}{r r r r}
    \toprule
    $X$ & $N$ & $R(\Phi_X)$ & $R/N^2$ \\
    \midrule
    $10^3$ & 35 & $1.8$ & $1.5 \times 10^{-3}$ \\
    $10^4$ & 205 & $9.7$ & $2.3 \times 10^{-4}$ \\
    $10^5$ & 1224 & $68$ & $4.5 \times 10^{-5}$ \\
    $5 \times 10^5$ & 4565 & $284$ & $1.4 \times 10^{-5}$ \\
    \bottomrule
\end{tabular}
\end{center}

The Rayleigh quotient grows consistently with $N$, as predicted by universal scaling.
Log-log fit: $R(\Phi_X) \sim N^{0.9}$, compatible with the theoretical $N^2/\log^2 N$
behavior when $\mathrm{span} \sim \log N$.

\subsection{What This Reformulation Achieves}

\paragraph{Clarity.}
The Twin Prime Conjecture becomes a statement about a single geometric quantity:
does the commutator energy diverge? This replaces the combinatorial question
``how many twins are there?'' with the analytic question ``how fast does energy grow?''

\paragraph{Structure.}
The reformulation exposes the underlying mechanism: energy accumulation in the
spectral domain. The commutator operator ``measures'' how spread out the point
configuration is, and twin primes---if infinite---create unbounded spread.

\paragraph{No overclaims.}
We do not claim to prove TPC. We establish an equivalence that may be useful
for future approaches. The arithmetic content of TPC is now encoded in a
geometric object that can be studied with spectral methods.

\paragraph{Falsifiability.}
If one could prove that $R(\Phi_X)$ remains bounded for arbitrarily large $X$,
that would disprove TPC. Conversely, proving divergence would establish TPC.
The reformulation is logically complete.

\subsection{Comparison with Classical Approaches}

\begin{center}
\begin{tabular}{p{4cm}|p{4cm}|p{4cm}}
    & \textbf{Classical} & \textbf{This Paper} \\
    \hline
    Object & $\pi_2(X)$ or $S(X)$ & $R(\Phi_X)$ \\
    Question & Count twins & Measure energy \\
    Tools & Sieve methods, zeta & Spectral geometry \\
    TPC statement & $\pi_2(X) \to \infty$ & $R(\Phi_X) \to \infty$ \\
    Status & Equivalent & Equivalent \\
\end{tabular}
\end{center}

The two approaches are logically equivalent. Our contribution is to provide
a new language---spectral energy---in which the problem may be more tractable.

% Twin Prime Coherence Analysis
% =============================

\section{Twin Prime Coherence Phenomenon}\label{sec:twin-coherence}

\subsection{Anti-diagonal Fourier Modes}

Twin prime vectors exhibit constructive interference along specific Fourier modes.

\begin{definition}[Anti-diagonal modes]\label{def:antidiag-modes}
In the complex Fourier basis $\{e^{ik\xi}\}$, the anti-diagonal modes are those with:
\begin{equation}
  k_1 + k_2 = 0, \quad \text{i.e., } k_2 = -k_1.
\end{equation}
\end{definition}

\begin{lemma}[Phase difference for twins]\label{lem:phase-diff}
For a twin pair $(p, p+2)$ with log-coordinates $\xi_p = \log p/(2\pi)$ and $\xi_{p+2} = \log(p+2)/(2\pi)$:
\begin{equation}
  \delta := \xi_{p+2} - \xi_p = \frac{\log(1 + 2/p)}{2\pi}.
\end{equation}
As $p \to \infty$, we have $\delta \to 0$ with $\delta \approx 1/(\pi p)$.
\end{lemma}

\begin{theorem}[Anti-diagonal coherence]\label{thm:antidiag-coherence}
For the anti-diagonal mode $(k, -k)$, the combined phase contribution from a twin pair is:
\begin{equation}
  e^{2\pi i k \xi_p} \cdot e^{-2\pi i k \xi_{p+2}} = e^{-2\pi i k \delta}.
\end{equation}
Since $\delta \to 0$ as $p \to \infty$, all large twins contribute with phase $\approx 0$.
\end{theorem}

\begin{proof}
Direct calculation:
\begin{equation}
  e^{2\pi i k (\xi_p - \xi_{p+2})} = e^{-2\pi i k \delta} = e^{-i k \log(1 + 2/p)} \to 1
\end{equation}
as $p \to \infty$.
\end{proof}

\begin{corollary}[Phase deviation bound]\label{cor:phase-deviation}
For mode $k$ and twin pair $(p, p+2)$, the phase deviation from 0 is:
\begin{equation}
  |\arg(e^{-2\pi i k \delta})| = 2\pi |k| \delta \approx \frac{2|k|}{p}.
\end{equation}
For $k = 1$ and $p \geq 100$, this is under $1.15^{\circ}$.
\end{corollary}

\subsection{Coherent vs Incoherent Scaling}

\begin{definition}[Coherence measures]\label{def:coherence-measures}
For $N$ twin pairs with phases $\phi_j$:
\begin{align}
  S_{\mathrm{coh}} &:= \left| \sum_{j=1}^N e^{i\phi_j} \right| \quad \text{(coherent sum)}, \\
  S_{\mathrm{inc}} &:= N \quad \text{(incoherent/random walk baseline)}.
\end{align}
\end{definition}

\begin{theorem}[Numerical scaling law]\label{thm:scaling}
This statement summarizes \emph{empirical log--log fits} from the code (complex Fourier basis, weights $w(p)=\log p/\sqrt{p}$) on the range $N \leq 1.4\times10^3$ twin pairs. It is not a rigorous asymptotic claim.
\begin{itemize}
  \item Coherent sum: $S_{\mathrm{coh}} \approx N^{0.96\text{--}1.01}$ on anti-diagonal modes.
  \item Baseline (diagonal / incoherent): $S_{\mathrm{inc}} \approx N^{0.10\text{--}0.20}$.
  \item Gain $G := S_{\mathrm{coh}} / S_{\mathrm{inc}}^{1/2} \approx N^{0.70\text{--}0.80}$.
\end{itemize}
\end{theorem}

\begin{remark}[Weight and window dependence]
The exponents vary with the weighting scheme and the compact window $K$; we do not attempt to identify a canonical value (e.g., $3/4$). Appendix tables and code logs record the precise fit ranges used.
\end{remark}

\subsection{Antiferromagnetic Order}

\begin{theorem}[Antiferromagnetic structure]\label{thm:antiferro}
Every twin pair $(p, p+2)$ with $p > 3$ satisfies:
\begin{equation}
  \chifour(p) \times \chifour(p+2) = -1.
\end{equation}
This defines an antiferromagnetic ordering in residue-class space.
\end{theorem}

\begin{proof}
For $p > 3$ prime, $p$ is odd. If $p \equiv 1 \pmod{4}$, then $p + 2 \equiv 3 \pmod{4}$, giving $\chifour(p) = 1$ and $\chifour(p+2) = -1$. The converse case is symmetric.
\end{proof}

\begin{remark}[Physical interpretation]
The antiferromagnetic order means twin primes always connect opposite residue classes (class 1 to class 3). This structural constraint underlies the coherence in the anti-diagonal Fourier basis, where the $\pm$ sectors couple constructively.
\end{remark}

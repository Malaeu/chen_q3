% Chen ↔ Twin Primes Dictionary (conceptual)
\section*{Appendix: Chen–Twin Dictionary (conceptual outline)}

This note sketches a translation between Chen's reverse heat method on the
Boolean cube and the twin-prime commutator framework. The goal is to set up a
precise anti-concentration statement whose violation would force $R(\Phi_X)$
to stay bounded, contradicting numerics and (under SC2) finiteness of twins.

\subsection*{Objects}
\begin{tabular}{l|l}
Chen (Boolean cube) & Twin primes (this work) \\
\hline
State space $\{-1,1\}^n$ & Twin index set $T(X)$ with signs $\chi_4(p)\in\{\pm1\}$ \\
Heat semigroup $P_\tau f(x)$ & Gaussian kernel $G_t(\xi-\eta)=e^{-(\xi-\eta)^2/(4t)}$ \\
Reverse heat flow $V_t = U_{T-t}$ & Backward smoothing / de-smoothing of twin vector \\
Score $S_i(x)=x_i\partial_i f/f$ & Gradient $G_p = (A\lambda)_p$ (commutator coeffs) \\
Level-1 inequality (Lemma 6) & Local coherence inequality on $B_{pq}=(A^\top A)_{pq}$ \\
Anti-concentration $\mathbb{P}(P_\tau f > \eta \int f)$ & Lower bound on $R(\Phi_X)=E_{\mathrm{comm}}/E_{\mathrm{lat}}$ \\
\end{tabular}

\subsection*{Key structural analogies}
\begin{itemize}
    \item Chen’s reverse heat argument shows a smoothed $f$ cannot concentrate too much: tails are controlled by score/Dirichlet energy.
    \item Our $E_{\mathrm{comm}} = \|A\lambda\|^2$ is exactly the Dirichlet-form analogue; $E_{\mathrm{lat}}$ plays the role of mass/normalization.
    \item Bounded $R(\Phi_X)$ would mean high concentration of the smoothed twin mass along kernel directions—an analogue of forbidden concentration in Chen.
\end{itemize}

\subsection*{Target hypothesis (anti-concentration for twins)}
\begin{quote}
If $R(\Phi_X)$ were $O(1)$ along a sequence $X_k\to\infty$, then the smoothed
twin density would exhibit concentration incompatible with a Chen-type
anti-concentration bound; hence either $R(\Phi_X)\to\infty$ or twins are
infinite.
\end{quote}

\subsection*{Intended proof skeleton}
\begin{enumerate}
    \item Express $E_{\mathrm{comm}}$ and $E_{\mathrm{lat}}$ as Chen-type Dirichlet and mass functionals of a positive function $f_X$ derived from twin weights.
    \item Import a Chen-style level-1 inequality (archimedean, Lemma~\ref{lem:arch}) to bound concentration by Dirichlet energy.
    \item Show bounded $R$ would force concentration of $f_X$ beyond that allowed by the inequality (anti-concentration step).
    \item Conclude: $R(\Phi_X)$ must grow (power-law); together with SC2 (finite twins $\Rightarrow R=O(1)$) this yields infinitely many twin primes.
\end{enumerate}

\subsection*{What remains}
The missing analytic piece is the anti-concentration inequality in the twin
setting (Step 3), i.e.\ a lower bound on a smoothed twin sum (or on
$\overline{B}(X)$). Any positive power-law would suffice.

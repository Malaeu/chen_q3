\section{Five Attack Paths to $\mathbf{P(X)}$}
\label{sec:attack-paths}

Having established the Main Theorem (TPC $\Leftrightarrow$ $R(\Phi_X) \to \infty$),
we now present five independent approaches to proving the sufficient condition
\[
    \mathbf{P(X)}: \quad R(\Phi_X) \geq c \cdot X^\delta \quad \text{for some } \delta > 0.
\]
Each path, if successful, would immediately yield TPC.

%------------------------------------------------------------------------------
\subsection{Path 1: CV Growth (Verified)}
\label{subsec:path-cv}

\begin{definition}[Coefficient of Variation]
For spectral coordinates $\{\xi_p\}_{p \in T(X)}$, define
\[
    \mathrm{cv}(\xi) = \frac{\sigma(\xi)}{\mu(\xi)} = \frac{\sqrt{\mathrm{Var}(\xi)}}{\mathbb{E}[\xi]}.
\]
\end{definition}

\begin{theorem}[CV Path to TPC --- \textbf{Lean4 Verified}]
\label{thm:cv-path}
If $\mathrm{cv}(\xi) \to \infty$ as $X \to \infty$, then $R(\Phi_X) \to \infty$,
and hence TPC holds.
\end{theorem}

\begin{proof}[Proof sketch]
High coefficient of variation implies large variance in gap distribution.
By the variance decomposition lemma (Appendix~\ref{app:verification}),
large variance forces the commutator energy to grow faster than lattice energy.
The formal proof is verified in Lean4.
\end{proof}

\textbf{Status:} The implication ``$\mathrm{cv} \to \infty \Rightarrow \text{TPC}$''
is formally verified in 252 lines of Lean4. The proof includes:
\begin{itemize}[nosep]
    \item \texttt{lemma1\_mean\_gap}: Mean gap scales as span/(N-1)
    \item \texttt{lemma2\_variance\_decomposition}: Var = within-group + between-group
    \item \texttt{lemma4\_between\_group\_variance\_grows}: Between-group variance $\to \infty$
    \item \texttt{cv\_path\_to\_TPC}: Main theorem (cv unbounded $\Rightarrow$ infinite twins)
\end{itemize}
What remains is showing $\mathrm{cv}(\xi) \to \infty$ for twin primes,
which is supported numerically ($\mathrm{cv} \sim N^{0.43}$) but not yet proven.
\textbf{Aristotle project:} \texttt{3645cb77-e7d8-4b2c-ba4d-8ac990c18a9d}

%------------------------------------------------------------------------------
\subsection{Path 2: Eigenvalue Ratio}
\label{subsec:path-eigenvalue}

Let $Q_X = A_X^T A_X$ be the commutator energy matrix and $G_X$ the Gram matrix.

\begin{conjecture}[Eigenvalue Divergence]
For the generalized eigenvalue problem $Q_X v = \lambda G_X v$,
\[
    \frac{\lambda_{\max}(Q_X, G_X)}{\lambda_{\min}(Q_X, G_X)} \to \infty
    \quad \text{as } X \to \infty.
\]
\end{conjecture}

If the eigenvalue ratio diverges, then at least one eigenvector achieves
$R \to \infty$, and by continuity arguments, so does $\Phi_X$.

\textbf{Status:} Numerically confirmed. Formal proof pending.

%------------------------------------------------------------------------------
\subsection{Path 3: Character Sum}
\label{subsec:path-character}

Using Euler's identity $p^{-it} = e^{-it \log p}$, we embed twin primes on the unit circle.

\begin{definition}[Twin Prime Character Sum]
\[
    S_X(t) = \sum_{p \in T(X)} \Lambda(p)\Lambda(p+2) \cdot p^{-it}.
\]
\end{definition}

\begin{conjecture}[Character Sum Growth]
\[
    \sup_{t \in [1, X]} \frac{|S_X(t)|}{\sqrt{|T(X)|}} \to \infty
    \quad \text{as } X \to \infty.
\]
\end{conjecture}

The connection to $R(\Phi_X)$ comes from the Fourier representation of commutator energy.

\textbf{Status:} Aristotle verification in progress (10\%).

%------------------------------------------------------------------------------
\subsection{Path 4: L-function Analysis}
\label{subsec:path-lfunction}

\begin{definition}[Twin Prime L-function]
\[
    L_{\mathrm{twins}}(s) = \sum_{\text{twin } p} p^{-s}, \quad \Re(s) > 1.
\]
\end{definition}

The behavior on the critical line $\Re(s) = 1/2$ encodes information about twin distribution.

\begin{conjecture}[Critical Line Behavior]
$L_{\mathrm{twins}}(s)$ has specific growth properties on $\Re(s) = 1/2$ that
force $R(\Phi_X) \to \infty$.
\end{conjecture}

\textbf{Status:} Theoretical framework established. Connection to spectral gap under investigation.

%------------------------------------------------------------------------------
\subsection{Path 5: Direct Summation}
\label{subsec:path-direct}

The most elementary approach: explicitly bound the commutator energy.

\begin{proposition}[Commutator Energy Decomposition]
\[
    E_{\mathrm{comm}}(\Phi_X) = \sum_{p,q \in T(X)} \lambda_p \lambda_q \cdot (\xi_q - \xi_p)^2 \cdot K_{pq}^2.
\]
\end{proposition}

The strategy:
\begin{enumerate}
    \item Identify the ``optimal zone'' $|\xi_p - \xi_q| \in [0.3, 1.5]$ contributing $\sim 70\%$ of energy.
    \item Count pairs in this zone: numerically $\sim N^{1.89}$.
    \item Derive lower bound: $E_{\mathrm{comm}} \geq c \cdot N^{2.9}$.
    \item Since $E_{\mathrm{lat}} \sim N^2$, obtain $R \geq c \cdot N^{0.9}$.
\end{enumerate}

\textbf{Status:} Numerical evidence strong ($R \sim N^{0.92}$). Rigorous proof requires
number-theoretic input on twin prime pair correlations.

%------------------------------------------------------------------------------
\subsection{Summary: Path Status}

\begin{center}
\begin{tabular}{llll}
\toprule
\textbf{Path} & \textbf{Approach} & \textbf{Verification} & \textbf{Remaining} \\
\midrule
1. CV Growth & Variance analysis & \textbf{Lean4 Verified} & Show $\mathrm{cv} \to \infty$ \\
2. Eigenvalue & Spectral theory & Numerical & Formal proof \\
3. Character Sum & Euler embedding & In Progress (10\%) & Complete proof \\
4. L-function & Analytic NT & Framework only & Full development \\
5. Direct Sum & Explicit bounds & Numerical & NT input needed \\
\bottomrule
\end{tabular}
\end{center}

Each path offers a different angle of attack. The CV Growth path is the most developed,
with formal verification of the conditional statement complete.

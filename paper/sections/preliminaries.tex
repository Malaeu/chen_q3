\section{Preliminaries}
\label{sec:preliminaries}

\subsection{The Archimedean Contribution}

The explicit formula for $\zeta(s)$ contains an Archimedean term arising from
the Gamma factor $\Gamma(s/2)$. Its contribution to the symbol is:
\begin{equation}
    a^*(\xi) = \log\pi - \Re\psi\left(\tfrac{1}{4} + i\pi\xi\right),
\end{equation}
where $\psi(z) = \Gamma'(z)/\Gamma(z)$ is the digamma function.

\begin{remark}[Asymptotic Behavior]
For large $|\xi|$:
\begin{equation}
    \psi\left(\tfrac{1}{4} + i\pi\xi\right) = \log(\pi|\xi|) + O(|\xi|^{-2}).
\end{equation}
Thus $a^*(\xi) \to -\infty$ as $|\xi| \to \infty$, but slowly (logarithmically).
\end{remark}

\subsection{The Smoothing Window}

To control convergence, we introduce a window function:
\begin{equation}
    \Phi_{B,t}(\xi) = \left(1 - \frac{|\xi|}{B}\right)_+ \cdot e^{-4\pi^2 t \xi^2},
\end{equation}
where $(x)_+ = \max(0, x)$ denotes the positive part.

\begin{itemize}
    \item The linear factor $(1 - |\xi|/B)_+$ provides compact support on $[-B, B]$.
    \item The Gaussian $e^{-4\pi^2 t \xi^2}$ ensures rapid decay of derivatives.
    \item Parameter $t \geq 0$ controls the smoothing strength.
\end{itemize}

\subsection{Fourier Coefficients}

The Archimedean symbol has Fourier expansion:
\begin{equation}
    P_A(\theta) = \sum_{k=0}^{\infty} A_k \cos(k\theta),
\end{equation}
where
\begin{equation}
    A_k = \int_{-B}^{B} g(\xi) \cos(k\xi) \, d\xi, \quad
    g(\xi) = a^*(\xi) \cdot \Phi_{B,t}(\xi).
\end{equation}

\subsection{Key Quantities}

\begin{definition}[Norm and Floor]
\begin{align}
    \|P_A\|_\infty &= \max_{\theta \in [-\pi, \pi]} |P_A(\theta)|, \\
    c_{\text{arch}} &= \min_{\theta \in [-\pi, \pi]} P_A(\theta).
\end{align}
\end{definition}

\begin{definition}[Stability Ratio]
\begin{equation}
    \delta_* = \frac{c_{\text{arch}}}{\|P_A\|_\infty}.
\end{equation}
The ratio $\delta_* > 0$ ensures that $P_A$ is bounded away from zero
relative to its maximum.
\end{definition}

\subsection{Model Kernels}

For analytical tractability, we consider model kernels:

\begin{enumerate}
    \item \textbf{Lorentzian:} $a(\xi) = \dfrac{1}{1 + \xi^2}$
    \quad (decay $\sim |\xi|^{-2}$)

    \item \textbf{Mellin:} $a(\xi) = \dfrac{1}{1 + |\xi|^{1/2}}$
    \quad (decay $\sim |\xi|^{-1/2}$)

    \item \textbf{Gamma:} $a(\xi) = |\Gamma(\tfrac{1}{4} + i\pi\xi)|^2$
    \quad (exponential decay)
\end{enumerate}

The Lorentzian captures the essential $O(|\xi|^{-2})$ tail behavior of the
digamma function while remaining positive, making it suitable for rigorous estimates.

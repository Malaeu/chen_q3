% Introduction - Twin Prime Paper
% ================================

\section{Introduction}\label{sec:intro}

\subsection{The Twin Prime Problem}

A \emph{twin prime pair} is a pair $(p, p+2)$ where both are prime.
The twin prime conjecture asserts infinitely many such pairs exist.
Hardy--Littlewood~\cite{HardyLittlewood1923} predicted
\[
    \pi_2(X) \sim 2C_2 \frac{X}{(\log X)^2}, \quad
    C_2 \approx 0.6602.
\]
Despite Zhang~\cite{Zhang2014} and Maynard~\cite{Maynard2015} proving bounded gaps,
the conjecture remains open.

\subsection{Main Result}

We establish a spectral reformulation of the Twin Prime Conjecture.

\begin{theorem}[Spectral Equivalence]\label{thm:main-intro}
The Twin Prime Conjecture is equivalent to the divergence of commutator energy:
\[
    \textbf{TPC} \;\iff\; R(\Phi_X) \to \infty \text{ as } X \to \infty,
\]
where $R(\Phi_X) = E_{\mathrm{comm}}(\Phi_X) / E_{\mathrm{lat}}(\Phi_X)$ is the
Rayleigh quotient of the twin vector.
\end{theorem}

This is an \textbf{equivalence}, not a proof of TPC. We translate the arithmetic
question ``are there infinitely many twins?'' into the geometric question
``does commutator energy diverge?''

\subsection{The Two Ingredients}

The equivalence rests on two results:

\paragraph{Universal Energy Scaling (Conjecture).}
We conjecture that for any sequence of $N$ points with spectral span $\mathrm{span}$,
the commutator energy satisfies $\mathrm{Sum}(Q) \sim N^2 \cdot \mathrm{span}^2$.
This is supported by numerical evidence (Section~\ref{sec:numerics}) but a rigorous
proof remains open due to the Gaussian kernel's exponential decay.
If true, and if twins are infinite, then $N(X) \to \infty$, so $R(\Phi_X) \to \infty$.

\paragraph{Finite Stabilization (SC2).}
If twin primes are finite, then beyond the last twin the vector $\Phi_X$ stabilizes:
its support and coordinates become constant. The Rayleigh quotient therefore
satisfies $R(\Phi_X) = O(1)$.

Together: infinite twins forces divergent energy; finite twins forces bounded energy.
The dichotomy is complete.

\subsection{What Is Proven}

\begin{center}
\small
\begin{tabular}{p{4.5cm}|p{2cm}|p{3cm}}
    \textbf{Result} & \textbf{Status} & \textbf{Type} \\
    \hline
    Cone--Kernel Separation & \textbf{Proven} & Linear algebra \\
    Cone Positivity (B$_1$-strong) & \textbf{Proven} & Compactness \\
    Universal Energy Scaling & \textbf{Conjecture} & Geometry \\
    Finite Stabilization (SC2) & \textbf{Proven} & Elementary \\
    \hline
    Spectral Equivalence & \textbf{Proven} & Main result \\
\end{tabular}
\end{center}

We do not prove TPC. We prove that TPC is \emph{equivalent} to energy divergence.

\subsection{Paper Outline}

Section~\ref{sec:setup} fixes notation and defines the principal objects:
spectral coordinates $\xi_p$, the kernel $K_{pq}$, and the Rayleigh quotient $R$.
Section~\ref{sec:TP-SX} bridges the operator framework to the classical twin sum $S(X)$.
Section~\ref{sec:B1-prime} proves Cone--Kernel Separation and derives Cone Positivity.
Section~\ref{sec:abstract-SC1} describes an alternative route via uniform scaling;
this is kept for completeness but is not needed for the main result.
Section~\ref{sec:SC2} proves Finite Stabilization unconditionally.
Section~\ref{sec:target-theorem} states the Spectral Equivalence theorem and
establishes Universal Energy Scaling.
Section~\ref{sec:numerics} presents numerical evidence, and
Section~\ref{sec:discussion} concludes with remarks on formal verification
approaches inspired by recent AI-assisted theorem proving~\cite{Aristotle2025,AlexeevErdos2025}.

% IND/AB: Inductive Closure and Parameter Schedules (archival, not used in main proof)
% ===================================================================================

\section{Inductive Closure (IND/AB)}\label{sec:IND-AB}

This archival section records the IND/AB route to prime control. It is kept for reference but is \emph{not} used in the main proof (which uses the RKHS contraction).

\subsection{The AB-Infinity Framework}

The AB-infinity (``Activity Block to Infinity'') framework provides a systematic way to verify positivity across an increasing chain of compacts.

\begin{definition}[Activity block chain]\label{def:AB-chain}
An \emph{activity block chain} is a sequence of compacts $\{K_i\}_{i \geq 1}$ with:
\begin{enumerate}
  \item[\textup{(i)}] $K_1 < K_2 < K_3 < \cdots$, strictly increasing.
  \item[\textup{(ii)}] $\bigcup_{i \geq 1} [-K_i, K_i] = \mathbb{R}$.
  \item[\textup{(iii)}] For each $i$, certified parameters $(B_i, t_{\mathrm{sym}, i}, M_i)$ are specified.
\end{enumerate}
\end{definition}

\begin{theorem}[AB$\infty$ closure]\label{thm:ABinfty}
Work under the T0 normalization $Q = Q_{\mathrm{GW}}$ on the Guinand--Weil axis. Fix global constants $q_0 = 30$, $t_* > 0$, and $t_0 > 0$. Let $\{K_i\}_{i \geq 1}$ be an activity block chain.

For each $i$, choose parameters $(B_i, t_{\mathrm{sym}, i}, M_i)$ with $t_{\mathrm{sym}, i} \geq t_*$ and a shift grid $E_{K_i} \subset [-K_i, K_i]$.

Assume for every $i$:

\begin{description}
  \item[\textbf{(R) Arch floor (A3):}] With the Fej\'er$\times$heat window $\Phi_{B_i, t_{\mathrm{sym}, i}, \tau}$ and archimedean symbol $P_A(\cdot; \tau)$:
  \begin{equation}\label{eq:ABinfty-floor}
    \min_\theta P_A(\theta; \tau) \geq c_0(K_i) \quad \text{for all } \tau \in E_{K_i}.
  \end{equation}

  \item[\textbf{(N) Nyquist \& Norm:}] The symbol modulus and prime cap satisfy:
  \begin{equation}\label{eq:ABinfty-nyquist}
    C \cdot \omega_{P_A}\!\left(\frac{\pi}{M_i}\right) \leq \frac{c_0(K_i)}{2}, \qquad
    \|T_P^{(q_0)}(t_0)\| \leq \frac{c_0(K_i)}{2},
  \end{equation}
  where $T_P^{(q_0)}(t_0)$ is the modular cap at modulus $q_0 = 30$ with RKHS smoothing scale $t_{\mathrm{rkhs}} \geq t_0$.

  \item[\textbf{(A) Grid$\to$Continuum (BRC--SAFE):}] Every interval $[\tau_j, \tau_{j+1}]$ in the grid is BRC--SAFE; equivalently, the resolvent certificate with Ky Fan/Hoffman--Wielandt budget holds on each such interval.
\end{description}

Then $Q(\Phi) \geq 0$ for all even Paley--Wiener tests $\Phi$ on $[-K_i, K_i]$ for every $i$. Consequently, $Q \geq 0$ on the full Weil class; by Weil's positivity criterion, RH follows.
\end{theorem}

\begin{proof}[Proof (by plumbing)]
By the Toeplitz symbol bridge (A3), for every grid node $\tau \in E_{K_i}$:
\begin{equation}
  \lambda_{\min}\!\big(T_{M_i}[P_A(\cdot; \tau)] - T_P\big) \geq \min P_A(\cdot; \tau) - C \cdot \omega_{P_A}\!\left(\frac{\pi}{M_i}\right) - \|T_P\|.
\end{equation}
Assumptions \textbf{(R)}--\textbf{(N)} make the RHS $\geq c_0 - \tfrac{c_0}{2} - \tfrac{c_0}{2} = 0$, so nonnegativity holds on all grid nodes.

By \textbf{(A)} (BRC--SAFE on each interval), the sign is preserved on $[-K_i, K_i]$.

Fej\'er$\times$heat density (A1$'$) and Lipschitz continuity (A2) lift nonnegativity from the grid cone to all even PW tests on $[-K_i, K_i]$.

Finally, along the chain $\{K_i\}$, the T5 compact limit transfers $Q \geq 0$ to the Weil class.
\end{proof}

\subsection{One-Interval Induction Step}

The inductive step handles the crossing of each threshold in the activity block chain.

\begin{theorem}[IND/AB step]\label{thm:IND-step}
On an activity interval $[B_n, B_{n+1})$, let $\|T_P^{\mathrm{old}}\|_{\mathcal{H}_K} \leq \rho_K^{\mathrm{old}} < 1$. When crossing the threshold $B_{n+1}$, a single new node $\alpha_{\mathrm{new}}$ with weight $w_{\mathrm{new}}$ enters. In the RKHS normalization $\|k_\alpha\| = 1$, one has:
\begin{equation}\label{eq:IND-norm-update}
  \|T_P^{\mathrm{new}}\| \leq \rho_K^{\mathrm{old}} + w_{\mathrm{new}}.
\end{equation}
Hence, if $\rho_K^{\mathrm{old}} + w_{\mathrm{new}} < 1$, then $T_A - T_P^{\mathrm{new}} \succeq 0$ on $\mathcal{H}_K$.
\end{theorem}

\begin{proof}
The rank-one update formula:
\begin{equation}
  T_P^{\mathrm{new}} = T_P^{\mathrm{old}} + w_{\mathrm{new}} |k_{\alpha_{\mathrm{new}}}\rangle \langle k_{\alpha_{\mathrm{new}}}|
\end{equation}
with $\|k_\alpha\| = 1$ gives the claimed norm bound by the triangle inequality. Strict inequality $\rho_K^{\mathrm{old}} + w_{\mathrm{new}} < 1$ implies the Loewner positivity $T_A - T_P^{\mathrm{new}} \succeq 0$.
\end{proof}

\begin{corollary}[Gluing intervals]\label{cor:IND-glue}
Suppose the base case holds on $[B_3, B_4)$, and across each threshold $B_n \to B_{n+1}$, the one-prime condition $\rho_K^{\mathrm{old}} + w_{\mathrm{new}} < 1$ is verified in the RKHS normalization on $[-K, K]$. Then $T_A - T_P \succeq 0$ holds on $[-K, K]$ for all $B \geq B_3$, i.e., the measure domination persists interval-by-interval.
\end{corollary}

\subsection{Early Block Control}

\begin{lemma}[Analytic bound for early blocks]\label{lem:IND-early}
Let $\Phi_{B, t}(\xi) = (1 - |\xi|/B)_+ e^{-4\pi^2 t \xi^2}$ with $B > 0$. For the even setting with weights $w(n) = \Lambda(n)/\sqrt{n}$ and nodes $\alpha_n = \log n/(2\pi)$:
\begin{equation}\label{eq:IND-early-bound}
  \sum_{\alpha_n \in [-B, B]} w(n) \Phi_{B, t}(\alpha_n) \leq \sum_{n \leq e^{2\pi B}} \frac{\Lambda(n)}{\sqrt{n}} \leq 2 e^{\pi B}(2\pi B - 2) + 4.
\end{equation}
In particular, choosing $B = B(K) > 0$ small enough forces the early-block mass to lie below any prescribed budget $\varepsilon(K) > 0$.
\end{lemma}

\begin{proof}
Since $0 \leq \Phi_{B, t} \leq 1$ and $\Phi_{B, t}$ vanishes outside $[-B, B]$, the first inequality holds. For the second, use $\Lambda(n) \leq \log n$ and compare the sum to the integral:
\begin{equation}
  \sum_{n \leq e^{2\pi B}} \frac{\Lambda(n)}{\sqrt{n}} \leq \int_1^{e^{2\pi B}} \frac{\log u}{\sqrt{u}} \, du = 2 e^{\pi B}(2\pi B - 2) + 4,
\end{equation}
where the evaluation follows by the substitution $u = v^2$.
\end{proof}

\subsection{Parameter Recipe}

The IND/AB framework requires careful choice of acceptance parameters.

\begin{definition}[Plateau schedule]\label{def:plateau-schedule}
Define the acceptance function:
\begin{equation}\label{eq:plateau-schedule}
  \mathrm{Plateau}(t; \alpha, \beta, \tau, \gamma) = \begin{cases}
    \alpha t, & t \leq \tau, \\
    \gamma, & \tau < t \leq \tau + \beta, \\
    \max\{\gamma - \alpha(t - \tau - \beta), 0\}, & t > \tau + \beta,
  \end{cases}
\end{equation}
where:
\begin{center}
\begin{tabular}{llp{0.4\textwidth}}
\toprule
Name & Symbol & Role \\
\midrule
Pre-plateau slope & $\alpha$ & Growth rate before plateau \\
Plateau width & $\beta$ & Length of flat segment \\
Onset shift & $\tau$ & Position of plateau window \\
Saturation level & $\gamma$ & Upper acceptance bound \\
\bottomrule
\end{tabular}
\end{center}
\end{definition}

\begin{lemma}[Plateau schedule admissibility]\label{lem:plateau-admissible}
Let $A(t) = \mathrm{Plateau}(t; \alpha, \beta, \tau, \gamma)$ with $0 < \alpha \leq \gamma \leq 1$ and $\beta > 0$. Then $A$ takes values in $[0, 1]$, is piecewise Lipschitz, and meets the IND/AB plateau constraints: monotonic rise before $\tau$, a flat segment of width $\beta$, and compatible one-sided derivatives at the junctions.
\end{lemma}

\begin{proof}[Proof sketch]
Formula~\eqref{eq:plateau-schedule} consists of three segments with slopes $\alpha$, $0$, and $-\alpha$. Continuity follows from matching the constants at the junction points; the corner points are controlled by the one-sided bounds. The values stay below $\gamma \leq 1$, satisfying the normalized AB regime.
\end{proof}

\subsection{Modular Cap Control}

\begin{definition}[Modular cap]\label{def:modular-cap}
For modulus $q$ and RKHS smoothing scale $t$, define the modular prime cap:
\begin{equation}
  T_P^{(q)}(t) := \sum_{\substack{\alpha_n \in [-K, K] \\ \gcd(n, q) = 1}} w(n) e^{-t(\xi_n - \xi_m)^2} |k_{\xi_n}\rangle \langle k_{\xi_m}|.
\end{equation}
\end{definition}

\begin{lemma}[Fixed modular cap]\label{lem:fixed-mod-cap}
The modular cap at $q_0 = 30$ satisfies:
\begin{equation}
  \|T_P^{(30)}(t_0)\| \leq \rho_{30}(t_0)
\end{equation}
for any $t_0 > 0$. The bound $\rho_{30}(t_0)$ is uniform across all compacts $K$ once the early block and tail contributions are controlled.
\end{lemma}

\begin{remark}[Choice of modulus]
The modulus $q_0 = 30 = 2 \cdot 3 \cdot 5$ is chosen to balance several considerations:
\begin{enumerate}
  \item It captures the first three primes, providing good coverage of small prime residues.
  \item The Euler totient $\phi(30) = 8$ keeps the number of residue classes manageable.
  \item It is large enough to provide meaningful cap reduction but small enough for efficient computation.
\end{enumerate}
\end{remark}

\subsection{BRC--SAFE Grid Lift}

\begin{definition}[BRC--SAFE interval]\label{def:BRC-SAFE}
An interval $[\tau_j, \tau_{j+1}]$ in the shift grid is \emph{BRC--SAFE} (Bounded Resolvent Certificate -- Safety And Feasibility Ensured) if:
\begin{enumerate}
  \item[\textup{(i)}] The resolvent $(T_M[P_A(\cdot; \tau)] - T_P - zI)^{-1}$ exists and is bounded for all $z$ in a neighborhood of $(-\infty, 0]$.
  \item[\textup{(ii)}] The Ky Fan/Hoffman--Wielandt perturbation bounds hold with uniform constants across the interval.
  \item[\textup{(iii)}] The eigenvalue variation $|\lambda_{\min}(\tau) - \lambda_{\min}(\tau')| \leq L_\lambda |\tau - \tau'|$ is Lipschitz controlled.
\end{enumerate}
\end{definition}

\begin{lemma}[Grid-to-continuum lift]\label{lem:grid-continuum-lift}
If every interval $[\tau_j, \tau_{j+1}]$ in the grid $E_K$ is BRC--SAFE, and $\lambda_{\min}(T_M[P_A] - T_P) \geq 0$ at all grid nodes, then $\lambda_{\min}(T_M[P_A(\cdot; \tau)] - T_P) \geq 0$ for all $\tau \in [-K, K]$.
\end{lemma}

\begin{proof}
By BRC--SAFE condition (iii), the minimum eigenvalue is a Lipschitz function of $\tau$. If it is nonnegative at both endpoints of an interval, and the Lipschitz constant is controlled, the intermediate value theorem ensures nonnegativity throughout the interval. Applying this to each interval in the grid and using the covering property yields the global result.
\end{proof}

\begin{remark}[Lipschitz-lift alternative]
Instead of BRC--SAFE, one may enforce the deterministic Lipschitz lift condition:
\begin{equation}
  L_Q(K_i) \cdot L_\Phi(K_i) \cdot \Delta\tau \leq \frac{c_0(K_i)}{4},
\end{equation}
where $\Delta\tau$ is the grid spacing. The conclusion is the same, but BRC--SAFE provides tighter control in practice.
\end{remark}

\subsection{Certified Parameter Tables}

The following table records the certified margins for the induction steps:

\begin{center}
\begin{tabular}{ccccc}
\toprule
$K$ & $c_0(K)$ & $\rho_K^{\mathrm{old}}$ & $w_{\mathrm{new}}$ & Margin \\
\midrule
1.0 & 0.179 & 0.181 & 0.023 & $\checkmark$ \\
1.5 & 0.173 & 0.167 & 0.018 & $\checkmark$ \\
2.0 & 0.167 & 0.159 & 0.014 & $\checkmark$ \\
2.5 & 0.161 & 0.153 & 0.011 & $\checkmark$ \\
\bottomrule
\end{tabular}
\end{center}

For $K = 1$, the greedy block consumes $0.181 < c_0/4$, and the follow-up step verifies $\rho_K^{\mathrm{old}} + w_{\mathrm{new}} < 1$. For larger $K$, the margin $c_0(K) - \rho(t)$ stays above $0.67$, so the one-prime update is comfortably within budget.

\subsection{Summary and Remarks}

\begin{remark}[Monotone inheritance]
It is convenient (though not essential) to choose $B_i \uparrow$, $M_i \uparrow$, and nonincreasing budgets so that acceptance persists along the chain without recomputation.
\end{remark}

\begin{remark}[Grid-to-continuum-to-Weil transfer]
By A1$'$ (Theorem~\ref{thm:A1-density}), the Fej\'er$\times$heat cone is dense in $W_K$ under $\|\cdot\|_\infty$; by A2 (Lemma~\ref{lem:A2-lip}), $Q$ is Lipschitz on $W_K$. Hence grid positivity and the SAFE/Lipschitz lift imply $Q \geq 0$ on all of $W_K$. With the monotone parameter schedule (Lemma~\ref{lem:T5-inheritance}), Theorem~\ref{thm:T5-transfer} transfers positivity to the Weil class.
\end{remark}

The IND/AB framework provides the final piece of the positivity verification: a systematic inductive mechanism that extends from finite compacts to the full Weil class, completing the chain $(T0) + (A1') + (A2) + (A3) + (\mathrm{RKHS}) + (T5) + (\mathrm{IND/AB}) \Rightarrow Q \geq 0 \Rightarrow \mathrm{RH}$.

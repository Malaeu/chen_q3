% Cone Positivity (B₁-strong): Cone-Kernel Separation Lemma
% Updated December 2024: Hypothesis → Lemma (rigorous proof)

\section{Cone Positivity (B\texorpdfstring{$_1$}{1}-strong)}\label{sec:B1-prime}

This section contains the core rigorous result: the commutator kernel has trivial
intersection with the positive cone. The proof is pure linear algebra---no number
theory enters. The consequence is a uniform lower bound on the Rayleigh quotient
over the cone, which we call \textbf{Cone Positivity} (B$_1$-strong).

We use the notation from Section~\ref{sec:setup}: positions $\xi_p$, kernel $K_{pq}$,
commutator matrix $A_{pq}$, cone $\mathcal{C}$, and energies $E_{\mathrm{comm}}$, $E_{\mathrm{lat}}$.

\subsection{Main Result}

\begin{lemma}[Cone--Kernel Separation]\label{lem:cone-kernel}
Let $\xi_1 < \xi_2 < \cdots < \xi_N$ be strictly increasing points on $\mathbb{R}$,
and let $K \in \mathbb{R}^{N \times N}$ be a symmetric matrix with $K_{pq} > 0$
for all $p \neq q$. Define $A \in \mathbb{R}^{N \times N}$ by
\[
A_{pq} = (\xi_q - \xi_p) \cdot K_{pq}.
\]
Let $\mathcal{C} = \{\lambda \in \mathbb{R}^N : \lambda_p \geq 0, \lambda \neq 0\}$
be the positive cone. Then
\[
\mathcal{C} \cap \ker(A) = \{0\}.
\]
\end{lemma}

\begin{proof}
Let $\lambda \in \mathcal{C} \setminus \{0\}$ and set $S = \{p : \lambda_p > 0\}$.
Choose $p^* \in S$ such that $\xi_{p^*} = \max\{\xi_p : p \in S\}$
(the rightmost active point).

Consider the component $(A\lambda)_{p^*}$:
\[
(A\lambda)_{p^*} = \sum_{q=1}^N (\xi_q - \xi_{p^*}) K_{p^*q} \lambda_q.
\]

We partition the sum:
\begin{itemize}
    \item For $q > p^*$: By choice of $p^*$, we have $\lambda_q = 0$, so the contribution is zero.
    \item For $q = p^*$: The factor $(\xi_q - \xi_{p^*}) = 0$, so the contribution is zero.
    \item For $q < p^*$: We have $(\xi_q - \xi_{p^*}) < 0$, $K_{p^*q} > 0$, and $\lambda_q \geq 0$,
          so each term is $\leq 0$.
\end{itemize}

\textbf{Case (a):} $\lambda$ is supported only at $p^*$ (i.e., $S = \{p^*\}$).

Then $\lambda_q = 0$ for $q \neq p^*$. For any $p < p^*$:
\[
(A\lambda)_p = (\xi_{p^*} - \xi_p) K_{p,p^*} \lambda_{p^*}.
\]
Since $\xi_{p^*} > \xi_p$, $K_{p,p^*} > 0$, and $\lambda_{p^*} > 0$, we have $(A\lambda)_p > 0$.
Thus $A\lambda \neq 0$.

\textbf{Case (b):} There exists $q < p^*$ with $\lambda_q > 0$.

Then in $(A\lambda)_{p^*}$, the term corresponding to this $q$ is
$(\xi_q - \xi_{p^*}) K_{p^*q} \lambda_q < 0$ (strictly negative).
Since all other terms are $\leq 0$, we have $(A\lambda)_{p^*} < 0$.
Thus $A\lambda \neq 0$.

In both cases, $\lambda \in \mathcal{C} \setminus \{0\}$ implies $A\lambda \neq 0$.
\end{proof}

\begin{corollary}[Cone Positivity (B$_1$-strong)]\label{cor:B1-strong}
Let $Q = A^\top A$ be the commutator energy matrix. Then
$\mathcal{C} \cap \ker(Q) = \{0\}$, and consequently the Rayleigh quotient
$R(\lambda) = E_{\mathrm{comm}}(\lambda) / E_{\mathrm{lat}}(\lambda)$ achieves
a positive infimum $c_1 > 0$ on the normalized cone $\mathcal{C}_1$.
\end{corollary}

\begin{proof}
Since $\ker(A^\top A) = \ker(A)$, the kernel statement follows from Lemma~\ref{lem:cone-kernel}.
For the positivity of $c_1$: the function $R(\lambda)$ is continuous on $\mathcal{C}_1$,
the numerator $E_{\mathrm{comm}}(\lambda) > 0$ for all $\lambda \in \mathcal{C}_1$ by the
kernel statement, and the denominator $E_{\mathrm{lat}}(\lambda) > 0$ since the Gram matrix
is strictly positive definite. Since $\mathcal{C}_1$ is compact, $R$ attains its
infimum, which must be positive.
\end{proof}

\begin{remark}[Local vs.\ Uniform bounds]\label{rem:local-vs-uniform}
For each fixed $X$, the constant $c_1 = c_1(X) > 0$ in Corollary~\ref{cor:B1-strong}
is \textbf{proven} to exist. This is a theorem, not a hypothesis.

However, the \emph{uniform} statement $\inf_{X \geq X_0} c_1(X) \geq c^* > 0$
is a separate claim (SC1) that requires arithmetic input about how twin primes
distribute as $X \to \infty$. Numerical evidence (Section~\ref{sec:numerics})
shows $c_1(X) \sim X^{0.90}$, suggesting the bound actually \emph{grows},
but this scaling is not proven. Thus we have a theorem that $c_1(X) > 0$ for each $X$,
and a hypothesis SC1 that $\inf_X c_1(X) \geq c^* > 0$ uniformly.
\end{remark}

\begin{remark}[Generality]
Lemma~\ref{lem:cone-kernel} requires only strictly increasing positions $\xi_p$
and a strictly positive off-diagonal kernel $K_{pq} > 0$. No number theory,
prime distribution, or Hardy--Littlewood asymptotics are needed.
This is pure linear algebra.
\end{remark}

\begin{remark}[Connection to the commutator operator]\label{rem:Q-C-connection}
For the actual commutator $[T, \Xi]$ in the Gaussian RKHS, let $G$ be the Gram matrix
with $G_{pq} = G(\xi_p - \xi_q)$. The commutator matrix $C = [T, \Xi]_{\mathrm{coord}}$
in the kernel basis $\{k_p\}$ satisfies:
\[
    C_{pq} = \tfrac{1}{2}(\xi_q - \xi_p) (G^2)_{pq}.
\]
This has exactly the form of Lemma~\ref{lem:cone-kernel} with $K_{pq} = \tfrac{1}{2}(G^2)_{pq} > 0$.

The commutator energy matrix $Q = C^\top G C$ satisfies
\[
    \ker(Q) = \ker(C),
\]
since $G$ is positive definite. Thus Cone--Kernel Separation for $Q$ follows
directly from the lemma applied to $C$.
\end{remark}

\subsection{Numerical Verification}

Numerical experiments confirm the lemma and provide quantitative bounds:

\begin{center}
\begin{tabular}{c|c|c|c|c}
    $X$ & $N$ & $\dim\ker(Q)$ & $\ker \cap \mathcal{C}$ & $\min_{\mathcal{C}} R(\lambda)$ \\
    \hline
    500 & 24 & 22 & 0 & 0.008 \\
    1000 & 35 & 32 & 0 & 0.014 \\
    2000 & 61 & 58 & 0 & 0.027 \\
    5000 & 126 & 122 & 0 & 0.048 \\
    \hline
    \multicolumn{5}{c}{$\min R(\lambda) \sim X^{0.90}$ (growing!)}
\end{tabular}
\end{center}

Several features stand out. The kernel $\ker(Q)$ is large---its dimension is
approximately $N - 3$---but every vector in the kernel has mixed signs. This
confirms that $\ker(Q) \cap \mathcal{C} = \{0\}$ as predicted by the lemma.
More striking is that the minimum Rayleigh quotient on the cone is not just
positive but \emph{growing}, roughly as $X^{0.90}$.

% Target Theorem moved to sections/target_theorem.tex

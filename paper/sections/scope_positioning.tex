% Positioning, Scope, and Global Hypotheses
% ==========================================

\section{Positioning and Scope}\label{sec:scope}

This section establishes the positioning of the present work within the literature and clarifies exactly what is and is not claimed.

\subsection{What This Work Is}

This work introduces a quantitative, modular operator framework for the Weil criterion that transfers positive semidefiniteness (PSD) of structured Toeplitz forms to nonnegativity of the Weil functional on the full test class. The framework operates via:

\begin{enumerate}
  \item \textbf{Symbol regularity:} Control of the archimedean symbol $P_A$ through Lipschitz and modulus of continuity bounds.

  \item \textbf{RKHS contraction:} Bounding the prime operator norm via reproducing kernel Hilbert space techniques.

  \item \textbf{Compact-by-compact limits:} Extending positivity from finite windows to the full Weil class through inductive limits.
\end{enumerate}

The key outputs are explicit constants (modulus of continuity of the symbol, RKHS Gram tail bounds, node spacing estimates, tail cutoffs) that compose into a global positivity statement for $Q$.

\subsection{What This Work Is Not}

To be clear about the scope:

\begin{itemize}
  \item \textbf{No new zero-free regions:} We do not claim new zero-free regions for $\zeta(s)$ or Dirichlet $L$-functions beyond what follows from the Weil criterion.

  \item \textbf{No density results:} We make no claims about the density or spacing of zeta zeros.

  \item \textbf{No numerical hypotheses about zeros:} The verification relies on analytic bounds; numerical diagnostics are confined to the appendix and play no role in the proof.

  \item \textbf{Pathway through Weil:} The entire argument works through the Weil criterion, not through direct analysis of the zeta function.
\end{itemize}

\subsection{Modularity}

A key design principle is modularity: local improvements in any component strengthen the global result.

\begin{itemize}
  \item \textbf{Sharper symbol modulus:} Better control of $\omega_{P_A}$ reduces the discretization error in A3.

  \item \textbf{Tighter spacing/tail estimates:} Improved bounds on node gaps $\delta_K$ and RKHS tails strengthen the contraction.

  \item \textbf{Smaller effective weights:} Refined weight estimates $w_{\max}$ improve the prime cap bound.
\end{itemize}

Any such improvement increases the contraction slack and propagates to strengthen $Q \geq 0$ on the Weil class.

\subsection{The Test Class}

The test class consists of even, nonnegative, compactly supported frequency tests.

\begin{definition}[Test classes]\label{def:test-classes}
On the compact $W_K = [-K, K]$:
\begin{enumerate}
  \item[\textup{(i)}] $\mathcal{W}_K$ denotes the Fej\'er$\times$heat cone: linear combinations of test functions $\Phi_{B, t, \tau}$ with bandwidth $B$, heat scale $t$, and shift $\tau$.

  \item[\textup{(ii)}] The \emph{Weil cone} is the union:
  \begin{equation}
    \mathcal{W} := \bigcup_{K > 0} \mathcal{W}_K.
  \end{equation}
\end{enumerate}
Density and continuity are invoked inside this cone before taking the inductive limit.
\end{definition}

\subsection{Verification Path}

The verification proceeds through a chain of analytic modules:

\begin{center}
\begin{tikzcd}
\text{T0} \arrow[r] & \text{A1$'$} \arrow[r] & \text{A2} \arrow[r] & \text{A3} \arrow[r] & \text{RKHS} \arrow[r] & \text{T5} \arrow[r] & Q \geq 0
\end{tikzcd}
\end{center}

Each module is proved in its designated section, with explicit constants recorded. Symbol scans and PSD checks are reproducibility aids only; they do not enter the logical core of the proofs.

\section{Global Hypotheses}\label{sec:global-hyp}

We collect the global hypotheses used in the main closure theorem. Each item is proved in the indicated section and recorded explicitly so that Theorem~\ref{thm:Main-positivity} invokes a single hypothesis list.

\subsection{The Hypothesis List}

\begin{description}
  \item[(H1) T0 -- Normalization] Guinand--Weil normalization of the Weil functional $Q$. This identifies $Q$ with the canonical form from the explicit formula (Section~\ref{sec:T0}).

  \item[(H2) A1$'$ -- Density] The Fej\'er$\times$heat cone $\mathcal{C}_K$ is dense in $W_K$ under the uniform norm $\|\cdot\|_\infty$ (Section~\ref{sec:A1}).

  \item[(H3) A2 -- Continuity] The Weil functional $Q$ is Lipschitz continuous on each $W_K$ with explicit constant $L_Q(K)$ (Section~\ref{sec:A2}).

  \item[(H4) A3 -- Toeplitz Bridge] The Toeplitz bridge with archimedean margin $c_0(K) > 0$, RKHS cap $\rho(t) \leq c_0(K)/4$, and discretization threshold $M_0(K)$ (Section~\ref{sec:A3}).

  \item[(H5) RKHS -- Prime Contraction] Prime contraction via the RKHS route, yielding $\|T_P\| \leq \rho(t^*) < 1$ (Section~\ref{sec:rkhs}).

\item[(H6) T5 -- Compact Transfer] Compact-by-compact transfer of positivity from finite windows to the Weil class (Section~\ref{sec:T5}).
\end{description}

\subsection{Dependency Structure}

The hypotheses satisfy the following dependency structure:

\begin{enumerate}
  \item (H1) T0 is foundational and has no dependencies.
  \item (H2) A1$'$ depends on the kernel constructions in T0.
  \item (H3) A2 depends on A1$'$ for the density argument.
  \item (H4) A3 depends on A2 for continuity and on T0 for normalization.
  \item (H5) RKHS depends on A3 for the symbol bounds.
  \item (H6) T5 depends on all of the above for the compact-by-compact transfer.
\end{enumerate}

\subsection{Main Closure}

\begin{theorem}[Main positivity]\label{thm:Main-positivity-scope}
If \textup{(H1)}--\textup{(H6)} hold, then:
\begin{equation}
  Q(\Phi) \geq 0 \quad \text{for every even, real, compactly supported } \Phi \in \mathcal{W},
\end{equation}
where $\mathcal{W} = \bigcup_{K > 0} \mathcal{W}_K$ is the Weil cone.
\end{theorem}

\begin{proof}
Fix $K > 0$. By (H6) with the monotone schedules $t^*_{\mathrm{T5}}(K)$, $M^*(K)$, we have:
\begin{equation}
  \lambda_{\min}\big(T_{M^*(K)}[P_A] - T_P\big) \geq \tfrac{1}{2} c_0^*(K) > 0.
\end{equation}
Hence the finite Toeplitz form is nonnegative on the Fej\'er$\times$heat cone. By (H2) the cone is dense in $W_K$, and by (H3) the functional $Q$ is continuous; therefore $Q \geq 0$ on $W_K$. Taking the union over all $K$ shows $Q \geq 0$ on $\mathcal{W}$. Finally, (H1) identifies this $Q$ with the canonical Weil functional.
\end{proof}

\begin{remark}[Analytic vs. numerical]
The proof of Theorem~\ref{thm:Main-positivity-scope} uses only analytic bounds established in the body of the paper. Numerical certificates and computational verifications are archived separately for reproducibility but play no role in the logical argument.
\end{remark}

\section{Notation Conventions}\label{sec:notation-conventions}

We establish the notation conventions used throughout.

\subsection{Frequency Axis}

On the frequency axis, we write $\xi = \eta/(2\pi)$. The archimedean density is:
\begin{equation}
  a(\xi) = \log \pi - \Re \psi\left(\tfrac{1}{4} + i\pi\xi\right), \qquad a_*(\xi) = 2\pi a(\xi),
\end{equation}
where $\psi$ is the digamma function.

\subsection{Prime Nodes}

Prime nodes are located at:
\begin{equation}
  \xi_n = \frac{\log n}{2\pi}
\end{equation}
with symmetric placement $\pm \xi_n$ for the even setting.

\subsection{Weight Conventions}

We distinguish two weight conventions:

\begin{equation}
  w_Q(n) = \frac{2\Lambda(n)}{\sqrt{n}} \quad \text{(one-sided weight in $Q$)},
\end{equation}
\begin{equation}
  w_{\mathrm{RKHS}}(n) = \frac{\Lambda(n)}{\sqrt{n}} \quad \text{(operator weight on $W_K$)}.
\end{equation}

Evenization lets us pass freely between them: doubling $w_{\mathrm{RKHS}}$ on $\xi_n > 0$ gives $w_Q$, while placing both $\pm \xi_n$ leaves $w_{\mathrm{RKHS}}$ unchanged.

All RKHS and operator bounds below use $w_{\mathrm{RKHS}}$; we abbreviate:
\begin{equation}
  w_{\max} := \sup_n w_{\mathrm{RKHS}}(n) \leq \frac{2}{e} \approx 0.7358.
\end{equation}

\subsection{The Weil Functional}

Throughout we use:
\begin{equation}
  Q(\Phi) = \int_{\mathbb{R}} a_*(\xi) \Phi(\xi) \, d\xi - \sum_{n \geq 2} w_Q(n) \Phi(\xi_n)
\end{equation}
on each compact window. Section~\ref{sec:T0} records the exact crosswalk to the Guinand--Weil form.

We call $Q$ ``quadratic'' only because $\Phi = g * g^\vee$ for some test function $g$; as a functional of $\Phi$ it is linear.

\subsection{Bridge Summary}

We split $Q$ as $T_M[P_A] - T_P$ with $P_A \in \mathrm{Lip}(1)$ and $T_P$ finite rank. The symbol barrier yields:
\begin{equation}
  \lambda_{\min}(T_M[P_A]) \geq c_0(K) - C \cdot \omega_{P_A}(\pi/M)
\end{equation}
with:
\begin{equation}
  c_0(K) := \min_{\theta \in \Gamma_K} P_A(\theta),
\end{equation}
where $\Gamma_K$ is the working arc on $W_K$.

The prime norm is bounded in the arch-induced RKHS by:
\begin{equation}
  \|T_P\| \leq w_{\max} + \sqrt{w_{\max}} \eta_K,
\end{equation}
where $\eta_K \in (0, 1 - w_{\max})$ is tuned via the log-node gap $\delta_K$. Thus:
\begin{equation}
  \lambda_{\min}(T_M[P_A] - T_P) \geq \min P_A - C \cdot \omega_{P_A}(\pi/M) - \|T_P\|,
\end{equation}
closing the bridge module and feeding the remaining steps.

\section{Quick Reference for Reviewers}\label{sec:quick-ref}

\subsection{Architecture}

The analytic chain is:
\begin{equation}
  \text{T0} \to \text{A1$'$} \to \text{A2} \to \text{A3} \to \text{RKHS} \to \text{T5} \to Q \geq 0.
\end{equation}

\textbf{Goal:} $Q \geq 0$ on the Weil cone $\mathcal{W}$.

\subsection{Two Scales}

The framework uses two independent scales:

\begin{itemize}
  \item $t_{\mathrm{sym}}$ controls the symbol modulus $\omega_{P_A}$ (used in A3).
  \item $t_{\mathrm{rkhs}}$ controls the prime cap $\|T_P\|$ (used in RKHS).
\end{itemize}

No coupling is imposed between these scales.

\subsection{Uniform Margins}

The global archimedean floor $c_* > 0$ comes from the spectral Fej\'er$\times$heat routine. The uniform prime cap at $t = 0.7$ satisfies $\rho_{\mathrm{cap}} < 1/25$.

\textbf{Budget split:}
\begin{itemize}
  \item $C_{\mathrm{SB}} \omega_{P_A}(\pi/M) \leq c_*/4$ (symbol contribution).
  \item $\|T_P\| \leq c_*/4$ (prime contribution).
\end{itemize}

This yields: $\lambda_{\min}(T_M[P_A] - T_P) \geq c_*/2$.

\subsection{Adaptive Option}

On a compact $[-K, K]$, the RKHS scale:
\begin{equation}
  t_{\min}(K) = \frac{\delta_K^2}{4 \ln\left(\frac{2 + \eta_K}{\eta_K}\right)}
\end{equation}
sharpens the bound on $\|T_P\|$.

\subsection{Transfer (T5)}

The transfer proceeds as:
\begin{equation}
  \text{Grid} \Rightarrow \text{Compact} \Rightarrow \text{Weil cone}.
\end{equation}

Series, tail, grid, and limit steps complete the chain.

% Introduction
% ============

\section{Introduction}\label{sec:intro}

\subsection{Background and Motivation}

We investigate spectral positivity criteria for the Riemann zeta function and Dirichlet $L$-functions through the Weil explicit formula framework. The central object is a canonical quadratic form $Q$ on the Weil test class $\mathcal{W}$, whose positivity is equivalent to the Riemann Hypothesis (RH) by Weil's criterion~\cite{Weil1952}.

Our approach constructs a Toeplitz operator $H = T_A - T_P$ where:
\begin{itemize}
  \item $T_A$ is an \emph{archimedean} Toeplitz operator encoding the smooth density,
  \item $T_P$ is a \emph{prime} operator supported on nodes $\xi_n = \frac{\log n}{2\pi}$.
\end{itemize}
The spectral condition $H \geq 0$ (positive semidefinite) translates directly to $Q(\Phi) \geq 0$ for all test functions $\Phi$ in the Weil class.

\subsection{Main Results}

\begin{theorem}[Main reduction, informal]\label{thm:intro-main-informal}
Let $Q$ be the quadratic form on the Weil class $\mathcal{W}$ defined in Section~\ref{sec:T0}. The analytic modules (A3, RKHS, T5) of Sections~\ref{sec:A3}--\ref{sec:T5} provide explicit monotone schedules $(t^\star(K), M^\star(K))$ that make $H_K := T_{M^\star(K)}[P_A]-T_P$ positive semidefinite on each compact window $W_K$. Propagating these bounds along an exhaustion of $\mathcal{W}$ would yield $Q(\Phi) \ge 0$ for all $\Phi \in \mathcal{W}$, and hence RH by Weil. Closing this propagation globally remains the outstanding analytic step.
\end{theorem}

The same machinery applies to Dirichlet twists $H_\chi$ and class operators $H_\pm$ (GRH module), and—conditionally—to twin-prime two-particle operators $A^{(2)}$.

\subsection{The Three Analytic Modules}

The proof organizes around three analytic modules:

\paragraph{(A3) Archimedean Toeplitz Barrier.}
On each compact window $W_K = [-K, K] \subset \RR$ we bound from below the Toeplitz component by an \emph{archimedean barrier} $c_0(K) > 0$, up to a controllable Lipschitz loss. Szeg\H{o}--B\"ottcher asymptotics together with an explicit modulus of continuity for $P_A$ yield
\[
  \lambda_{\min}\!\big(T_M[P_A]\big) \geq c_0(K) - C \cdot \omega_{P_A}\!\Big(\frac{\pi}{M}\Big).
\]

\paragraph{(RKHS) Prime Contraction.}
The prime contribution is encoded by a sampling operator $T_P$ with weights $w(n) = \Lambda(n)/\sqrt{n}$. We develop an upper bound on $\|T_P\|$ inside a reproducing-kernel Hilbert space (RKHS) of the heat flow:
\[
  \|T_P\| \leq w_{\max} + \sqrt{w_{\max}} \cdot S_K(t), \quad
  S_K(t) \leq \frac{2e^{-\delta_K^2/(4t)}}{1 - e^{-\delta_K^2/(4t)}},
\]
where $w_{\max} \leq 2/e$ and $\delta_K$ is the node separation on $W_K$.

\paragraph{(T5) Compact-by-Compact Transfer.}
Once on a given $W_K$ the deterministic inequalities
\[
  C \cdot \omega_{P_A}\!\Big(\frac{\pi}{M}\Big) \leq \frac{c_0(K)}{4}, \quad
  \|T_P\| \leq \frac{c_0(K)}{4}
\]
hold with parameters $(M, t)$ chosen \emph{monotonically} in $K$, then $\lambda_{\min}(T_M[P_A] - T_P) > 0$ on $W_K$, and positivity inherits to $W_{K'}$ for all $K' \geq K$.

\subsection{Outline of the Proof Strategy}

Combining the Toeplitz barrier and RKHS cap yields, on each $W_K$:
\[
  \lambda_{\min}\!\big(T_M[P_A] - T_P\big) \geq c_0(K) - C \cdot \omega_{P_A}\!\Big(\frac{\pi}{M}\Big) - \|T_P\|.
\]
Choosing $t \geq t_{\min}(K)$ enforces $\|T_P\| \leq c_0(K)/4$, and selecting $M$ so that $C \cdot \omega_{P_A}(\pi/M) \leq c_0(K)/4$ gives
\[
  \lambda_{\min}\!\big(T_M[P_A] - T_P\big) \geq \tfrac{1}{2} c_0(K) > 0.
\]
The compact-by-compact transfer then propagates positivity along any monotone chain $K_i \uparrow \infty$.

\subsection{Contemporary Context}

This work was inspired by several recent developments:

\begin{itemize}
  \item \textbf{Analytic criteria.} Li's positivity sequence~\cite{Li1997} and the Jensen polynomial programme of Griffin--Ono--Rolen--Zagier~\cite{GriffinOnoRolenZagier2019} give logically equivalent restatements of RH.

  \item \textbf{Zero-density breakthroughs.} The Dirichlet-polynomial bounds of Guth and Maynard~\cite{GuthMaynard2024} illustrate spectral encoding of the zeta problem.

  \item \textbf{Near-miss invariants.} Rodgers and Tao's work on the de Bruijn--Newman constant~\cite{rodgers2020debruijn} shows RH may be ``barely true.''

  \item \textbf{Geometric approaches.} Fesenko's two-dimensional adelic programme~\cite{Fesenko2008} and Connes--Marcolli's noncommutative geometry~\cite{ConnesMarcolli2008} highlight operator factorizations.

  \item \textbf{Physical operator heuristics.} PT-symmetric constructions such as Bender--Brody--M\"uller~\cite{BenderBrodyMuller2017} keep the Hilbert--P\'olya dream alive.

  \item \textbf{Massive computations.} Platt and Trudgian's verification of RH up to $3 \times 10^{12}$~\cite{PlattTrudgian2021} emphasizes transparent, audit-friendly proofs.
\end{itemize}

\subsection{What is New}

Two features distinguish the present work:
\begin{enumerate}
  \item \textbf{A tables-free prime contraction.} The norm of the prime operator is bounded analytically in an RKHS, via Gram geometry. All constants are explicit and monotone in $K$.

  \item \textbf{GRH and twin prime extensions.} The same analytic machinery applies to Dirichlet $L$-functions (GRH module); the twin-prime analysis is developed as a conditional extension using the class operators.
\end{enumerate}

\subsection{Notation}

We write $\Lambda$ for the von Mangoldt function, $\xi_n = \frac{\log n}{2\pi}$ for sampling nodes, and $w(n) = \Lambda(n)/\sqrt{n}$ for operator weights with $w_{\max} = \sup_n w(n) \leq 2/e$. The heat kernel is $k_t(x,y) = \exp\!\bigl(-\frac{(x-y)^2}{4t}\bigr)$. Compact windows are $W_K = [-K, K]$, and $\mathcal{W} = \bigcup_{K>0} \mathcal{W}_K$ is the Weil cone. Complete conventions appear in Section~\ref{sec:T0}.

\subsection{Dependency Map}

\begin{center}
\small
\begin{tabular}{lll}
\toprule
\textbf{Module} & \textbf{Key Statement} & \textbf{Consumed by} \\
\midrule
T0 & Proposition~\ref{prop:T0-GW} (Guinand--Weil) & Theorem~\ref{thm:A3-bridge} \\
A1' & Theorem~\ref{thm:A1-density} (Cone density) & Theorem~\ref{thm:A3-bridge} \\
A2 & Lemma~\ref{lem:A2-lip} (Lipschitz) & Theorem~\ref{thm:A3-bridge} \\
A3 & Theorem~\ref{thm:A3-bridge} (Toeplitz bridge) & Main verification \\
RKHS & Theorem~\ref{thm:rkhs-contraction} (Prime cap) & Theorem~\ref{thm:A3-bridge} \\
T5 & Theorem~\ref{thm:T5-transfer} (Transfer to Weil cone) & Main positivity \\
GRH & Theorem~\ref{thm:GRH-criterion} ($H_\chi \geq 0$) & Twin analysis \\
Twins & Section~\ref{sec:twin-coherence} (Coherence, conditional) & Bridge Section~\ref{sec:bridge} \\
\bottomrule
\end{tabular}
\end{center}

\paragraph{Assumption stack.} Throughout, "under (T0)+(A1')+(A2)+(A3)+(RKHS)+(T5)" means exactly: fixed Guinand--Weil normalization; cone density; Lipschitz continuity of $Q$; Archimedean Toeplitz lower bound with explicit $c_0(K)$ and modulus $\omega_{P_A}$; RKHS prime contraction with explicit $t_{\min}(K)$; and the compact-by-compact transfer. The IND/AB route is retained only as an archival alternative and is \emph{not} used in the main proof.

\subsection{Organization}

\begin{itemize}
  \item Section~\ref{sec:T0}: Weil class and Guinand--Weil normalization
  \item Section~\ref{sec:A1}: Cone density (Fej\'er$\times$heat generators)
  \item Section~\ref{sec:A2}: Lipschitz continuity of $Q$
  \item Section~\ref{sec:A3}: Toeplitz--symbol bridge (A3) with explicit archimedean bounds
  \item Section~\ref{sec:rkhs}: RKHS prime contraction
  \item Section~\ref{sec:T5}: Compact transfer (T5)
  \item Section~\ref{sec:GRH}: GRH extension and class decomposition
  \item Section~\ref{sec:twin-coherence}: Twin prime coherence analysis (conditional)
  \item Section~\ref{sec:bridge}: Coherence-counting bridge (formal theorems)
  \item Section~\ref{sec:Weil}: Weil linkage and discussion
  \item Appendix: notation, FAQ, parameter tables, diagnostics (not used in the proof)
\end{itemize}

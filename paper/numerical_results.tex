\section{Numerical Results}
\label{sec:numerical}

All numerical experiments were performed using Python with
\texttt{numpy}, \texttt{scipy}, and \texttt{mpmath} for arbitrary precision.
Source code available in \texttt{src/} directory.

\subsection{Saturation Experiment}

\textbf{Script:} \texttt{saturation\_proof.py}

\textbf{Method:} Compute $C_0(B) = A_0 + \text{Tail Bound}$ for various $B$.

\begin{figure}[h]
\centering
\includegraphics[width=0.8\textwidth]{../output/saturation_proof.png}
\caption{Saturation of symbol norm $C_0(B) \to C_* \approx 277$ as $B \to \infty$.}
\label{fig:saturation}
\end{figure}

\textbf{Observation:} The norm saturates rapidly; by $B = 10$, we are within
1\% of the asymptotic value.

\subsection{Floor Experiment (Lorentzian)}

\textbf{Script:} \texttt{floor\_proof.py}

\textbf{Method:} Periodize the Lorentzian kernel with triangular window.

\begin{figure}[h]
\centering
\includegraphics[width=0.8\textwidth]{../output/floor_proof.png}
\caption{Floor $c_{\text{arch}}(B)$ grows and saturates at $\approx 0.20$.}
\label{fig:floor}
\end{figure}

\textbf{Observation:} The floor matches Q3's claimed value $c_{\text{arch}} \approx 0.1878$.

\subsection{Digamma Analysis}

\textbf{Script:} \texttt{analyze\_digamma\_poison.py}

\textbf{Finding:} The raw digamma function $a^*(\xi) = \log\pi - \Re\psi(\tfrac{1}{4} + i\pi\xi)$
changes sign at $\xi \approx 1.0$:

\begin{figure}[h]
\centering
\includegraphics[width=0.8\textwidth]{../output/digamma_poison_analysis.png}
\caption{The digamma-based function becomes negative for $|\xi| > 1$.}
\label{fig:digamma}
\end{figure}

\textbf{Implication:} Direct use of digamma requires careful windowing;
positive model kernels (Lorentzian, Mellin) are analytically cleaner.

\subsection{Kernel Comparison}

\textbf{Script:} \texttt{merlin\_kernel\_test.py}

\textbf{Method:} Test five kernel types for spectral stability.

\begin{figure}[h]
\centering
\includegraphics[width=0.95\textwidth]{../output/merlin_kernel_test.png}
\caption{Comparison of kernels. Mellin achieves $\delta_* \approx 0.79$.}
\label{fig:kernels}
\end{figure}

\textbf{Results Summary:}
\begin{center}
\begin{tabular}{l|c|c|c}
Kernel & Floor Behavior & Ceiling Behavior & $\delta_*$ \\
\hline
Sinc & Oscillates negative & Grows & 0 \\
Fejér & Zero at edges & Grows & 0 \\
Poisson & Bounded positive & Explodes & $\to 0$ \\
Gaussian & Small positive & Bounded & 0.014 \\
\textbf{Lorentzian} & Positive & Bounded & \textbf{0.20} \\
\textbf{Mellin} & Positive, parallel & Bounded & \textbf{0.79}
\end{tabular}
\end{center}

\subsection{Key Discovery: Decay Rate Law}

\begin{observation}[Decay-Stability Relationship]
The stability ratio $\delta_*$ increases with slower decay:
\begin{equation}
    \text{Decay } \sim |\xi|^{-\alpha} \quad \Rightarrow \quad
    \delta_* \text{ increases as } \alpha \downarrow.
\end{equation}
\end{observation}

\begin{center}
\begin{tabular}{c|c}
$\alpha$ (decay exponent) & $\delta_*$ \\
\hline
$\infty$ (exponential) & $\approx 0$ \\
2 (Lorentzian) & 0.20 \\
0.5 (Mellin) & 0.79
\end{tabular}
\end{center}

This explains why exponentially decaying kernels (Gamma, Gaussian)
perform poorly: they create ``holes'' in the periodization.

% Critical Clarifications (FAQ)
% =============================

\section{Critical Clarifications (FAQ)}\label{faqlab:faq}

This appendix addresses frequently asked questions and potential misunderstandings about the framework.

\subsection{Structural Questions}\label{faqlab:struct}

\paragraph{FAQ-1 (Nodes are not dense on compacts).}
On $[-K, K]$ the active set $\{\alpha_n = \tfrac{\log n}{2\pi}\}$ is finite: $n \leq N(K) = \lfloor e^{2\pi K}\rfloor$. The minimal gap
\begin{equation}
  \delta_K = \min_{1 \leq n < N(K)}(\alpha_{n+1} - \alpha_n) = \frac{1}{2\pi}\min_{1 \leq n < N(K)}\log\Big(1+\frac{1}{n}\Big) \geq \frac{1}{2\pi(N(K)+1)} > 0.
\end{equation}

\paragraph{FAQ-2 (Weight upper bound).}
For $w(n) = \Lambda(n)/\sqrt{n}$ one has $w(n) \leq \log n/\sqrt{n} \leq 2/e < 3/4 < 1$. Thus $w_{\max} < 1$ on every compact. (Rational bound: $2/e \approx 0.7358 < 3/4 = 0.75$.)

\paragraph{FAQ-3 (Finite Gram).}
The Gram matrix $G$ of $\{k_{\alpha_n}\}$ on $[-K, K]$ is finite dimensional; $\|T_P\| = \|W^{1/2} G W^{1/2}\|$.

\paragraph{FAQ-4 (Existence of $t_{\min}$).}
As $t \downarrow 0$, $S_K(t) = \tfrac{2e^{-\delta_K^2/(4t)}}{1 - e^{-\delta_K^2/(4t)}} \downarrow 0$. Hence for any $\eta_K > 0$ there exists
\begin{equation}
  t_{\min}(K) = \frac{\delta_K^2}{4\ln\!\bigl((2+\eta_K)/\eta_K\bigr)} \quad\text{with}\quad S_K(t_{\min}) \leq \eta_K.
\end{equation}

\paragraph{FAQ-5 (Dictionary density).}
We assert $\varepsilon$-density of the cone $\mathcal{C}_K$ by a finite dictionary $\mathcal{G}_K$ at fixed $K$, not global density by a fixed finite set. See Theorem~A1$'$ and T5.

\paragraph{FAQ-6 (Activity intervals).}
$I_n = [B_n, B_{n+1})$ with $B_n = \tfrac{\log n}{2\pi}$. Crossing $I_n \to I_{n+1}$ adds a single new node $\alpha_{n+1}$, enabling the one-prime induction.

\paragraph{FAQ-7 (Weil topology).}
$\mathcal{W} = \bigcup_K \mathcal{W}_K$ with the inductive limit topology; $Q$ is continuous on each $\mathcal{W}_K$ (Lemma~A2) and thus on $\mathcal{W}$ (Theorem~T5).

\paragraph{FAQ-8 (Link to zeta zeros).}
See the Weil criterion (Section~\ref{sec:Weil}). The positivity $Q(\Phi) \geq 0$ for all $\Phi$ in the Weil class is equivalent to RH.

\subsection{Computational Examples}\label{faqlab:comp}

\paragraph{FAQ-9 (Example at $K = 1$).}
Take $N(1) = \lfloor e^{2\pi}\rfloor \approx 535$, $\delta_1 \geq \frac{1}{2\pi(536)} \approx 0.000297$. Choose $t_{\min}(1)$ by the formula above with a concrete $\eta_1 \in (0, 1)$, compute $S_1(t_{\min})$, and verify $\rho_1 = w_{\max} + \sqrt{w_{\max}} S_1(t_{\min}) < 1$. PSD of a small dictionary $\mathcal{G}_1$ can be checked for $M \in \{10, 20, 40\}$ by the CLI.

\paragraph{FAQ-10 (Role of Fej\'er).}
The Fej\'er factor localizes to compacts and contributes to BV/Lipschitz regularity of the symbol; the heat factor yields smoothing and Gaussian-in-log tails. Their product preserves positivity and supplies the regularity required for A3 and the RKHS estimates.

\subsection{Anti-Patterns: What We Do NOT Assume}\label{faqlab:anti}

\paragraph{FAQ-11 (Anti-patterns: what we do \emph{not} assume).}

\begin{itemize}
  \item \emph{No discrete spectrum claim.} We do not model the problem via a selfadjoint operator with a pure point spectrum on a Paley--Wiener space; on the Fourier side, multiplication by $\xi$ has absolutely continuous spectrum on $[-\Lambda, \Lambda]$.

  \item \emph{No rigged eigenfunctions.} We do not use generalized eigenvectors like $e^{i\gamma\tau}$ (with Dirac masses in frequency) as elements of our Hilbert space.

  \item \emph{No heat-trace/Weyl shortcuts.} We do not extract Weyl counting from $\mathrm{tr}\, e^{-tR^2}$; all lower bounds are via the symbol barrier for Toeplitz matrices and RKHS operator norms.

  \item \emph{No circular determinant logic.} We do not identify a Fredholm determinant with $\xi(s)$ nor assume RH to deduce bijections; our route to RH is exclusively through Weil's positivity criterion on an explicit test class.
\end{itemize}

\subsection{Framework Summary}

\paragraph{FAQ-12 (Our stance).}
The proof skeleton is Toeplitz + RKHS + Weil:
\begin{enumerate}
  \item A3 handles the archimedean symbol $P_A \in \mathrm{Lip}(1)$ and keeps primes as a finite-rank operator.
  \item RKHS yields a strict contraction on each compact $[-K, K]$.
  \item T5 transfers positivity to the inductive limit.
  \item The Weil criterion (Section~\ref{sec:Weil}) finishes the implication.
\end{enumerate}

\subsection{Technical Clarifications}

\paragraph{FAQ-13 (Symmetrization).}
The operator $T_P$ is symmetrized by placing prime nodes at $\pm\xi_n$ with equal weights. This ensures $T_P$ is self-adjoint on the RKHS $\mathcal{H}_t$.

\paragraph{FAQ-14 (Character twist).}
For the GRH extension, we twist the prime weights by $\chi(p)$. The resulting operator $T_P^\chi$ remains self-adjoint because we use real characters $\chi_4$.

\paragraph{FAQ-15 (Compact exhaustion).}
The sequence $K_j \to \infty$ in T5 is arbitrary. Any exhaustion of $\mathbb{R}$ by compacts works; we choose $K_j = j$ for simplicity.

\paragraph{FAQ-16 (Parameter coupling).}
The heat scale $t$ and Toeplitz dimension $M$ are \emph{not} coupled. We can choose them independently to optimize the error budget:
\begin{itemize}
  \item Small $t$: sharper RKHS separation, but larger prime trace $\rho(t)$.
  \item Large $M$: smaller Toeplitz discretization error, but larger matrices.
\end{itemize}

\paragraph{FAQ-17 (Numerical precision).}
All numerical verifications use IEEE 754 double precision ($\approx 15$ decimal digits). The condition numbers of our Toeplitz matrices are moderate ($< 10^4$ for $K \leq 2.5$), so numerical errors are negligible compared to the positivity margins.

\paragraph{FAQ-18 (Reproducibility).}
All numerical results are generated by documented Python scripts in the repository. Parameters, random seeds (if any), and version numbers are logged for reproducibility.

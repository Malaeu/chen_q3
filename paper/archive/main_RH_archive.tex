% Q3 Spectral Positivity Paper - Main File
% ==========================================

% Preamble for the Toeplitz--Weil Bridge paper
\usepackage[utf8]{inputenc}
\usepackage[T1]{fontenc}
\usepackage{lmodern}
\usepackage{textcomp}     % Fix for Unicode symbols (en-dash, quotes, etc)
\usepackage{microtype}    % Better typography and line breaking
\usepackage[final]{pdfpages} % Better PDF handling
\usepackage{scalerel}     % Better symbol scaling
\usepackage[english]{babel}
\usepackage{amsmath,amsthm,amssymb,mathtools}
\usepackage{enumitem}      % For customizing description/itemize/enumerate
\usepackage{geometry}
\usepackage{array}
\usepackage{hyperref}
\usepackage{bookmark}
\usepackage[nameinlink,capitalize,noabbrev]{cleveref}
\usepackage{url}
\usepackage[hyphenbreaks]{breakurl}
\def\UrlBreaks{\do\/\do-\do.}
\newcommand{\CarchOne}{\frac{1\,346\,209}{7\,168\,000}} % c_arch(1)
\newcommand{\RhoGate}{\frac{1}{25}}                    % uniform gate cap
\newcommand{\RhoSeventytight}{\frac{1\,971}{50\,000}}% strict rho(0.7) bound
\newcommand{\YesSlackMin}{\frac{199\,329}{28\,672\,000}}% c_arch(1)/4 - 1/25
% Note: c_arch(1)/4 - \RhoSeventytight = 5\,398\,969 / 716\,800\,000 > \YesSlackMin

\newif\ifrkhroute
\rkhroutetrue
% --- Cross-ref helpers (safe no-ops if already defined)
\providecommand{\xlink}[2]{\hyperref[#1]{#2}}
\newcommand{\SeeAlso}[1]{\par\smallskip\noindent\textit{See also.} #1\par\smallskip}
\hypersetup{
  pdftitle={Operator Methods for the Weil Criterion},
  pdfauthor={Eugen Malamutmann}
}

\pdfstringdefDisableCommands{%
  \def\czero{c\_0}%%
  \def\rhok{\rho\_K}%%
  \def\AssumpPack{(T0)+(A1'$)$+(A2)+(A3)+(RKHS/MD)+(T5)}%%
}
\usepackage{graphicx}
\usepackage{xcolor}
\usepackage{booktabs}    % Professional table rules (toprule, midrule, bottomrule)
\usepackage{float}       % Force table/figure placement with [H]
\usepackage{multirow}    % Multi-row cells in tables
\usepackage{tcolorbox}
\newcommand{\figinclude}[1]{\IfFileExists{#1}{\includegraphics[width=0.95\linewidth]{#1}}{\fbox{Missing figure: \texttt{\detokenize{#1}}}}}
\providecommand{\MetricsAutoTable}{}
\geometry{margin=1in}

% Fix for overfull hbox issues
\sloppy
\tolerance=2000
\emergencystretch=5em
\hbadness=10000  % Suppress underfull hbox warnings
\vfuzz=2pt       % Allow small overfull vboxes
\hfuzz=2pt       % Allow small overfull hboxes

% Theorem environments
\numberwithin{equation}{section}
\newtheorem{theorem}{Theorem}[section]
\newtheorem{lemma}[theorem]{Lemma}
\newtheorem{proposition}[theorem]{Proposition}
\newtheorem{corollary}[theorem]{Corollary}
\theoremstyle{definition}
\newtheorem{definition}[theorem]{Definition}
\theoremstyle{remark}
\newtheorem*{remark}{Remark}

% Shortcuts
\newcommand{\RR}{\mathbb R}
\newcommand{\ZZ}{\mathbb Z}
\newcommand{\NN}{\mathbb N}
\newcommand{\CC}{\mathbb C}
\newcommand{\TT}{\mathbb T}
\newcommand{\EE}{\mathbb E}
\newcommand{\vp}{\varphi}
\newcommand{\xiN}{\xi_n}
\newcommand{\La}{\Lambda}
\newcommand{\veps}{\varepsilon}
\newcommand{\astar}{a_*}
\newcommand{\Lip}{\mathrm{Lip}}
\newcommand{\BV}{\mathrm{BV}}
\newcommand{\TV}{\operatorname{TV}}
\newcommand{\norm}[1]{\left\lVert #1\right\rVert}
\newcommand{\ip}[2]{\left\langle #1,\,#2\right\rangle}
\newcommand{\op}{\mathrm{op}}
\newcommand{\om}{\omega}
\newcommand{\Pa}{P_A}
\newcommand{\TP}{T_{\!P}}
\newcommand{\TPA}{T_{\!P_A}}
\newcommand{\czero}{c_0}
\newcommand{\rhok}{\rho_K}
\newcommand{\lammin}{\lambda_{\min}}
\newcommand{\cO}{\mathcal{O}}
\newcommand{\Kern}{K}
\newcommand{\Kt}{K_t}
\newcommand{\epsK}{\varepsilon_{K}}
\newcommand{\DispK}[1]{\operatorname{Disp}_K\!(#1)}
\newcommand{\ACDthree}{\textsf{AC--D3}}
\newcommand{\wQ}[1]{w_{\mathrm{Q}}(#1)}
\newcommand{\wRKHS}[1]{w_{\mathrm{RKHS}}(#1)}
\newcommand{\wmax}{w_{\max}^{\mathrm{RKHS}}}

% --- Conditional RH phrasing macros ---
% Assumption chain used across the paper
\newcommand{\AssumpChain}{(T0)+(A1$'$)+(A2)+(A3)+(MD/IND or RKHS)+(T5)} % chktex 36
% Unified shorthand for the full hypothesis pack
\newcommand{\AssumpPack}{\textup{(T0)+(A1$'$)+(A2)+(A3)+(RKHS/MD)+(T5)}}
% Short, context-free Weil implication phrase (no re-listing assumptions)
\newcommand{\WeilImpRH}{by Weil's positivity criterion, RH would hold.}
% Variant when explicitly tying to the assumption chain
\newcommand{\WeilImpRHUnder}{by Weil's positivity criterion, RH would hold under \AssumpChain.}
% Title/metadata helpers
\title{Spectral Analysis of the Archimedean Symbol\\
in the Chen Q3 Framework}

\author{Research Notes}

\date{\today}

\begin{abstract}
We provide rigorous numerical verification and analytical estimates
for the key spectral bounds in the Chen Q3 approach to the Riemann Hypothesis.
Specifically, we establish:
\begin{enumerate}[nosep]
    \item Saturation of the symbol norm: $\|P_A\|_\infty \leq C_* \approx \Cstar$.
    \item Positivity of the Archimedean floor: $c_{\text{arch}} \approx \carch > 0$.
    \item Comparison of kernel choices: Lorentzian ($\delta_* \approx \deltaLorentz$)
          vs Mellin ($\delta_* \approx \deltaMellin$).
\end{enumerate}
The Mellin kernel $K(\xi) = 1/(1+|\xi|^{1/2})$ achieves optimal stability ratio.
\end{abstract}

\usepackage{tikz}


\title{Spectral Positivity and Twin Prime Coherence:\\
A Computational Study via Weil's Criterion}

\author{[Authors]}

\date{\today}

\begin{document}

\maketitle

\begin{abstract}
We investigate the spectral properties of prime distributions through the lens of Weil's positivity criterion for the Riemann zeta function and Dirichlet $L$-functions. Starting from the Hamiltonian $H = T_A - T_P$ associated with $\zeta(s)$, where $T_A$ is an archimedean Toeplitz operator and $T_P$ encodes prime locations, we construct analogous operators $H_\chi$ for Dirichlet characters $\chi$ mod $q$, and class-decomposed operators $H_\pm$ that isolate primes in residue classes.

Our numerical experiments verify $H \geq 0$ (consistent with RH) and $H_\chi \geq 0$ (consistent with GRH) on compacts $K \leq 2.5$, covering primes up to $\sim 10^7$. For twin primes, we analyze two-particle operators $A^{(2)} = H_+ \otimes I + I \otimes H_-$ and discover a coherence phenomenon: twin prime vectors exhibit constructive interference along anti-diagonal Fourier modes $(k_1 + k_2 = 0)$, with phase deviations under $1^\circ$. This coherence arises from the geometric proximity $\delta = \log(1+2/p)/(2\pi) \to 0$ and manifests as antiferromagnetic order: $\chi_4(p) \times \chi_4(p+2) = -1$ for all twin pairs.

These results establish a computational framework connecting spectral positivity with prime correlations, though they do not constitute a proof of the twin prime conjecture.
\end{abstract}

% Compact roadmap instead of a full table of contents
\paragraph{Roadmap.} Sections~\ref{sec:intro}--\ref{sec:scope} set up motivation and scope; \ref{sec:T0}--\ref{sec:rkhs} build the Toeplitz/RKHS machinery; \ref{sec:T5} transfers positivity compact-by-compact; \ref{sec:GRH}--\ref{sec:twin-coherence} cover the GRH and twin-prime extensions; appendices collect parameters, FAQs, and numerical tables.

%==============================================================================
% Section 1: Introduction
%==============================================================================
\section{Introduction}\label{sec:intro}

\subsection*{Background and motivation}
We prove that a canonical quadratic form on the Weil test class is nonnegative, and therefore\textemdash{}by the Weil criterion\textemdash{}deduce the Riemann Hypothesis. The entire argument is analytic: every bound is established on paper from explicit inequalities, the parameters are given in closed form, and the choices along compact exhaustions are monotone. No numerical tables or automated certificates enter the proof.

\subsection*{Main result}
\begin{theorem}[Main result, informal]\label{thm:intro-main-informal}
Let $Q$ be the quadratic form fixed in Section~\ref{sec:T0} on the Weil class $\mathcal{W}$. Then
\[
  Q(\Phi)\ \ge\ 0\qquad\text{for all }\Phi\in\mathcal{W}.
\]
Via Theorem~\ref{thm:Weil-criterion} (the Weil criterion) this positivity is equivalent to the Riemann Hypothesis.
\end{theorem}

The proof organises around three analytic modules.

\subsection*{Archimedean bridge}
\emph{(A3) Archimedean Toeplitz barrier.} On each compact window $W_K=[-K,K]\subset\RR$ we bound from below the Toeplitz component $T_M[P_A]$ of $Q$ by an \emph{archimedean barrier} $c_0(K)>0$, up to a controllable Lipschitz loss $C\,\omega_{P_A}(\pi/M)$. Szeg\H{o}--B\"ottcher asymptotics together with an explicit modulus of continuity for $P_A$ yield
\[
  \lambda_{\min}\!\big(T_M[P_A]\big)\ \ge\ c_0(K)\ -\ C\,\omega_{P_A}\!\Big(\frac{\pi}{M}\Big),
\]
as developed in Section~\ref{sec:A3}.

\subsection*{Prime contraction}
\emph{(RKHS) Prime contraction without tables.} The prime contribution is encoded by a sampling operator $T_P$ supported on the nodes $\xi_n=\frac{\log n}{2\pi}$; in the Weil functional we use the one-sided weights $w_Q(n)=2\Lambda(n)/\sqrt{n}$, while in the RKHS analysis we keep the undoubled operator weights $w_{\mathrm{RKHS}}(n)=\Lambda(n)/\sqrt{n}$. Section~\ref{sec:rkhs-prime-cap} develops a \emph{tables-free} upper bound on $\|T_P\|$ inside the reproducing-kernel Hilbert space of the heat flow. Two complementary routes are provided:
\begin{itemize}
  \item \textbf{Classical treatments.} Standard expositions of the analytic theory~\cite{IwaniecKowalski2004,MontgomeryVaughan2007,Edwards1974} provide the backdrop against which we calibrate notation, normalizations, and cone generators.
  \item a \emph{Gram-geometry route}, giving
  \[
    \|T_P\|\ \le\ w_{\max}+\sqrt{w_{\max}}\,S_K(t),\qquad
    S_K(t)\ \le\ \frac{2e^{-\delta_K^2/(4t)}}{1-e^{-\delta_K^2/(4t)}},
  \]
  where $w_{\max}\le 2/e$ and $\delta_K$ is the separation of the nodes on $W_K$; choosing
  \[
    t_{\min}(K)\ :=\ \frac{\delta_K^2}{4\ln\!\big((2+\eta_K)/\eta_K\big)},\qquad \eta_K\in(0,1-w_{\max}),
  \]
  forces $\|T_P\|\le \rho_K:=w_{\max}+\sqrt{w_{\max}}\;\eta_K$;
  \item an \emph{early/tail route}, splitting the prime sum at $N=N(K)$, with
  \[
    \sum_{n\le N}\frac{\Lambda(n)}{\sqrt{n}}\ \le\ 2\sqrt{N}\log N,
    \qquad
    \sum_{n>N}\frac{\Lambda(n)}{\sqrt{n}}\,e^{-4\pi^2 t(\log n)^2}\ \ll\ \frac{e^{-4\pi^2 t(\log N)^2}}{t},
  \]
  which produces an explicit threshold $t^\star(K)$ ensuring $\|T_P\|\le c_0(K)/4$.
\end{itemize}

\subsection*{Compact transfer}
\emph{(T5) Compact-by-compact transfer.} Section~\ref{sec:T5} shows that once, on a given $W_K$, the deterministic inequalities
\[
  C\,\omega_{P_A}\!\Big(\frac{\pi}{M}\Big)\ \le\ \frac{c_0(K)}{4},\qquad
  \|T_P\|\ \le\ \frac{c_0(K)}{4},\qquad
  \text{(finite early block)}\ \le\ \frac{c_0(K)}{4}
\]
hold with parameters $(M,t)$ chosen \emph{monotonically} in $K$, then $\lambda_{\min}\!\big(T_M[P_A]-T_P\big)>0$ on $W_K$, and positivity inherits to $W_{K'}$ for all $K'\ge K$. Thus $Q\ge0$ on any exhaustion $\bigcup_i W_{K_i}$ with $K_i\uparrow\infty$.

\subsection*{Outline of the proof}

Combining the Toeplitz barrier and the RKHS cap yields, on each $W_K$,
\[
  \lambda_{\min}\!\big(T_M[P_A]-T_P\big)
  \ \ge\ c_0(K)\ -\ C\,\omega_{P_A}\!\Big(\frac{\pi}{M}\Big)\ -\ \|T_P\|.
\]
Choosing $t\ge t_{\min}(K)$ (or $t\ge t^\star(K)$) enforces $\|T_P\|\le c_0(K)/4$, and selecting $M$ so that $C\,\omega_{P_A}(\pi/M)\le c_0(K)/4$ gives
\[
  \lambda_{\min}\!\big(T_M[P_A]-T_P\big)\ \ge\ \tfrac12\,c_0(K)\ >\ 0.
\]
The compact-by-compact transfer then propagates positivity along any monotone chain $K_i\uparrow\infty$. Positivity on $\bigcup_i W_{K_i}$ extends by definition to all of $\mathcal{W}$, proving $Q\ge0$ in Theorem~\ref{thm:Main-positivity}. Finally Section~\ref{sec:Weil} applies Theorem~\ref{thm:Weil-criterion} to convert this positivity into the Riemann Hypothesis.

\subsection*{What is new}

Two features distinguish the present work.
\begin{enumerate}
  \item \textbf{A tables-free prime contraction.} The norm of the prime operator is bounded analytically in an RKHS, via either Gram geometry or an early/tail split. All constants are explicit (for example $t_{\min}(K)$ above), monotone in $K$, and no legacy tables or certificates appear in the proof; reproducibility data are confined to Appendix~\ref{app:a3-repro}.
  \item \textbf{A monotone transfer principle.} The compact-by-compact module (T5) depends only on $c_0(K)$, $\omega_{P_A}$, and the RKHS cap $\rho_{\mathrm{cap}}(K)$. The parameter schedules $(M^\star(K),t^\star(K))$ are given by explicit formulas and chosen to be monotone in $K$, yielding an auditable, dimension-free route from positivity on one compact to positivity on all larger compacts.
\end{enumerate}

\subsection*{Organization of the paper}

Section~\ref{sec:T0} recalls the Weil class, the quadratic form $Q$, and the Guinand--Weil normalization. Section~\ref{sec:A3} establishes the Archimedean Toeplitz barrier (A3). Section~\ref{sec:rkhs-prime-cap} develops the RKHS prime contraction together with the thresholds $t_{\min}(K)$ and $t^\star(K)$. Section~\ref{sec:T5} proves the compact-by-compact transfer (T5) and the monotone inheritance. Section~\ref{sec:Weil} links compact positivity to the full Weil class and states the main theorem together with its Weil corollary. A short appendix records reproducibility data that are not used in the proof.

\subsection*{Notation}

We write $\Lambda$ for the von Mangoldt function, $\xi_n=\frac{\log n}{2\pi}$ for the sampling nodes, $w_Q(n)=2\Lambda(n)/\sqrt{n}$ for the weights inside the Weil functional, and $w_{\mathrm{RKHS}}(n)=\Lambda(n)/\sqrt{n}$ (with $w_{\max}=\sup_n w_{\mathrm{RKHS}}(n)\le 2/e$) for the operator analysis. The heat kernel is $k_t(x,y)=\exp\!\bigl(-\frac{(x-y)^2}{4t}\bigr)$. Compact windows are denoted $W_K=[-K,K]$, and $\mathcal{W}=\bigcup_{K>0}\mathcal{W}_K$ is the Weil cone. Complete conventions appear in Section~\ref{sec:notation}.

\subsection*{Analytic modules at a glance}

\noindent\textbf{Stage legend.} $(\mathrm{T0})$ fixes the Guinand--Weil normalization of the Weil functional. $(\mathrm{A1'})$ proves density of the Fej\'er$\times$heat generator cone on each compact, and $(\mathrm{A2})$ supplies Lipschitz continuity so that positivity propagates from the generators to all even nonnegative tests. $(\mathrm{A3})$ is the Toeplitz bridge: it splits $Q$ into an Archimedean Toeplitz symbol and a finite-rank prime block with explicit lower bounds on $\lambda_{\min}$. The main route for the prime contribution is the RKHS contraction developed in Section~\ref{sec:rkhs-prime-cap}; the MD/IND/AB chain remains archived as an alternative in the appendices. Finally $(\mathrm{T5})$ performs the compact-by-compact lift and closes the YES gate, chaining the local statements to $Q\ge0$ on the full Weil class.

\begin{center}
  \small\textbf{Dependency map for the analytic chain}
\end{center}
\begin{center}
  \small
  \begin{tabular}{lll}
    \hline
    Module & Key statement & Consumed by \\
    \hline
    $\mathrm{T0}$ & Proposition~\ref{prop:T0-GW} (Guinand--Weil normalization) & Theorem~\ref{thm:Main-positivity}, Theorem~\ref{thm:RH} \\
    $\mathrm{A1'}$ & Theorem~\ref{a1:thm:A1-local-density} (Density on $W_K$) & Theorem~\ref{thm:T5-compact}, Theorem~\ref{thm:Main-positivity} \\
    $\mathrm{A2}$ & Lemma~\ref{a2:lem:A2} / Corollary~\ref{a2:cor:explicit-lip} (Lipschitz control) & Theorem~\ref{thm:T5-compact}, Theorem~\ref{thm:Main-positivity} \\
    $\mathrm{A3}$ & Theorem~\ref{thm:A3} (Toeplitz bridge) & Theorem~\ref{thm:T5-compact}, Theorem~\ref{thm:Main-positivity} \\
    RKHS & Theorem~\ref{thm:rkhs-tstar} (Prime contraction) & Theorem~\ref{thm:T5-compact}, Theorem~\ref{thm:Main-positivity} \\
    $\mathrm{T5}$ & Theorem~\ref{thm:T5-compact} (Compact transfer) & Theorem~\ref{thm:Main-positivity} \\
    MAIN & Theorem~\ref{thm:Main-positivity} (Weil positivity on $W$) & Theorem~\ref{thm:RH} \\
    WEIL & Theorem~\ref{thm:Weil-criterion} (Weil criterion) & Theorem~\ref{thm:RH} \\
    \hline
  \end{tabular}
\end{center}

\noindent\textbf{Assumption stack.} When we write ``under $(\mathrm{T0})+(\mathrm{A1'})+(\mathrm{A2})+(\mathrm{A3})+(\mathrm{MD/IND/AB}\text{ or RKHS})+(\mathrm{T5})$'' we mean precisely the data enumerated above: a fixed normalization, cone density, Lipschitz control, the mixed Toeplitz lower bound, either the MD/IND/AB prime-control chain or the RKHS contraction, and the compact limit machinery. No hidden steps are invoked outside this list.

\noindent\textbf{Verification aids.} Appendices~\ref{app:a3-repro} and~\ref{app:verification} archive the legacy JSON files, ATP logs, and numerical cross-checks that originally motivated the parameter choices. These artefacts are reproducibility collateral only: the proofs in Sections~\ref{sec:T0}--\ref{sec:T5} rely solely on the analytic estimates stated there, and every inequality invoked in the main argument is justified in-line. Appendix~\ref{app:a3-repro} also collates the archived inputs in a single summary table for ease of audit.

\subsection{Contemporary Context and Inspiration}

This work was inspired by several recent developments in analytic number theory, computational complexity, and mathematical logic:

\begin{itemize}
  \item \textbf{Analytic criteria.} Li's positivity sequence~\cite{Li1997} and the Jensen polynomial programme of Griffin--Ono--Rolen--Zagier~\cite{GriffinOnoRolenZagier2019} give logically equivalent restatements of RH; both inspire our insistence on keeping every cone generator and Lipschitz bound explicit.

  \item \textbf{Zero-density breakthroughs.} The new Dirichlet-polynomial bounds of Guth and Maynard~\cite{GuthMaynard2024} illustrate how much can be gained by encoding the zeta problem as a spectral estimate, a viewpoint we adopt through the Toeplitz bridge.

  \item \textbf{Near-miss invariants.} Rodgers and Tao's work on the de Bruijn--Newman constant~\cite{rodgers2020debruijn} shows that RH may be ``barely true'', motivating a watchdog table that certifies every slack we introduce along the chain.

  \item \textbf{Geometric and noncommutative ideas.} Fesenko's two-dimensional adelic programme~\cite{Fesenko2008} and the Connes--Marcolli noncommutative approach~\cite{ConnesMarcolli2008} highlight how positivity hinges on careful operator factorizations, reinforcing our choice to stay within verifiable Toeplitz/RKHS settings.

  \item \textbf{Physical operator heuristics.} PT-symmetric constructions such as Bender--Brody--M\"uller~\cite{BenderBrodyMuller2017} keep the Hilbert--P\'olya dream alive; our framework aims to supply the missing rigorous operator inequalities.

  \item \textbf{Geometric flows and smoothing.} Perelman's Ricci-flow programme~\cite{Perelman2002,Perelman2003} shows how parabolic averaging can enforce global structure; we mirror that philosophy by pairing Fej\'er kernels with heat-flow smoothing in the Toeplitz bridge.

  \item \textbf{Massive computations.} Platt and Trudgian's verification of RH up to $3\cdot10^{12}$~\cite{PlattTrudgian2021}, together with surveys like Conrey's~\cite{Conrey2003}, emphasise the need for transparent, audit-friendly proofs rather than ever-larger numerics.

  \item \textbf{Cautionary analyses.} Cairo's audit of proposed counterexamples~\cite{cairo2025counterexample} underlines how fragile heuristic arguments can be; we therefore keep every analytic assumption explicit and machine-checkable.
\end{itemize}

\noindent While these works influenced our methodology, our approach is fundamentally distinct: we construct a self-contained, verifiable chain from Toeplitz positivity to Weil positivity, with all critical steps amenable to formal verification.


%==============================================================================
% Section 1b: Scope, Positioning, and Global Hypotheses
%==============================================================================
% Positioning, Scope, and Global Hypotheses
% ==========================================

\section{Positioning and Scope}\label{sec:scope}

This section establishes the positioning of the present work within the literature and clarifies exactly what is and is not claimed.

\subsection{What This Work Is}

This work introduces a quantitative, modular operator framework for the Weil criterion that transfers positive semidefiniteness (PSD) of structured Toeplitz forms to nonnegativity of the Weil functional on the full test class. The framework operates via:

\begin{enumerate}
  \item \textbf{Symbol regularity:} Control of the archimedean symbol $P_A$ through Lipschitz and modulus of continuity bounds.

  \item \textbf{RKHS contraction:} Bounding the prime operator norm via reproducing kernel Hilbert space techniques.

  \item \textbf{Compact-by-compact limits:} Extending positivity from finite windows to the full Weil class through inductive limits.
\end{enumerate}

The key outputs are explicit constants (modulus of continuity of the symbol, RKHS Gram tail bounds, node spacing estimates, tail cutoffs) that compose into a global positivity statement for $Q$.

\subsection{What This Work Is Not}

To be clear about the scope:

\begin{itemize}
  \item \textbf{No new zero-free regions:} We do not claim new zero-free regions for $\zeta(s)$ or Dirichlet $L$-functions beyond what follows from the Weil criterion.

  \item \textbf{No density results:} We make no claims about the density or spacing of zeta zeros.

  \item \textbf{No numerical hypotheses about zeros:} The verification relies on analytic bounds; numerical diagnostics are confined to the appendix and play no role in the proof.

  \item \textbf{Pathway through Weil:} The entire argument works through the Weil criterion, not through direct analysis of the zeta function.
\end{itemize}

\subsection{Modularity}

A key design principle is modularity: local improvements in any component strengthen the global result.

\begin{itemize}
  \item \textbf{Sharper symbol modulus:} Better control of $\omega_{P_A}$ reduces the discretization error in A3.

  \item \textbf{Tighter spacing/tail estimates:} Improved bounds on node gaps $\delta_K$ and RKHS tails strengthen the contraction.

  \item \textbf{Smaller effective weights:} Refined weight estimates $w_{\max}$ improve the prime cap bound.
\end{itemize}

Any such improvement increases the contraction slack and propagates to strengthen $Q \geq 0$ on the Weil class.

\subsection{The Test Class}

The test class consists of even, nonnegative, compactly supported frequency tests.

\begin{definition}[Test classes]\label{def:test-classes}
On the compact $W_K = [-K, K]$:
\begin{enumerate}
  \item[\textup{(i)}] $\mathcal{W}_K$ denotes the Fej\'er$\times$heat cone: linear combinations of test functions $\Phi_{B, t, \tau}$ with bandwidth $B$, heat scale $t$, and shift $\tau$.

  \item[\textup{(ii)}] The \emph{Weil cone} is the union:
  \begin{equation}
    \mathcal{W} := \bigcup_{K > 0} \mathcal{W}_K.
  \end{equation}
\end{enumerate}
Density and continuity are invoked inside this cone before taking the inductive limit.
\end{definition}

\subsection{Verification Path}

The verification proceeds through a chain of analytic modules:

\begin{center}
\begin{tikzcd}
\text{T0} \arrow[r] & \text{A1$'$} \arrow[r] & \text{A2} \arrow[r] & \text{A3} \arrow[r] & \text{RKHS} \arrow[r] & \text{T5} \arrow[r] & Q \geq 0
\end{tikzcd}
\end{center}

Each module is proved in its designated section, with explicit constants recorded. Symbol scans and PSD checks are reproducibility aids only; they do not enter the logical core of the proofs.

\section{Global Hypotheses}\label{sec:global-hyp}

We collect the global hypotheses used in the main closure theorem. Each item is proved in the indicated section and recorded explicitly so that Theorem~\ref{thm:Main-positivity} invokes a single hypothesis list.

\subsection{The Hypothesis List}

\begin{description}
  \item[(H1) T0 -- Normalization] Guinand--Weil normalization of the Weil functional $Q$. This identifies $Q$ with the canonical form from the explicit formula (Section~\ref{sec:T0}).

  \item[(H2) A1$'$ -- Density] The Fej\'er$\times$heat cone $\mathcal{C}_K$ is dense in $W_K$ under the uniform norm $\|\cdot\|_\infty$ (Section~\ref{sec:A1}).

  \item[(H3) A2 -- Continuity] The Weil functional $Q$ is Lipschitz continuous on each $W_K$ with explicit constant $L_Q(K)$ (Section~\ref{sec:A2}).

  \item[(H4) A3 -- Toeplitz Bridge] The Toeplitz bridge with archimedean margin $c_0(K) > 0$, RKHS cap $\rho(t) \leq c_0(K)/4$, and discretization threshold $M_0(K)$ (Section~\ref{sec:A3}).

  \item[(H5) RKHS -- Prime Contraction] Prime contraction via the RKHS route, yielding $\|T_P\| \leq \rho(t^*) < 1$ (Section~\ref{sec:rkhs}).

\item[(H6) T5 -- Compact Transfer] Compact-by-compact transfer of positivity from finite windows to the Weil class (Section~\ref{sec:T5}).
\end{description}

\subsection{Dependency Structure}

The hypotheses satisfy the following dependency structure:

\begin{enumerate}
  \item (H1) T0 is foundational and has no dependencies.
  \item (H2) A1$'$ depends on the kernel constructions in T0.
  \item (H3) A2 depends on A1$'$ for the density argument.
  \item (H4) A3 depends on A2 for continuity and on T0 for normalization.
  \item (H5) RKHS depends on A3 for the symbol bounds.
  \item (H6) T5 depends on all of the above for the compact-by-compact transfer.
\end{enumerate}

\subsection{Main Closure}

\begin{theorem}[Main positivity]\label{thm:Main-positivity-scope}
If \textup{(H1)}--\textup{(H6)} hold, then:
\begin{equation}
  Q(\Phi) \geq 0 \quad \text{for every even, real, compactly supported } \Phi \in \mathcal{W},
\end{equation}
where $\mathcal{W} = \bigcup_{K > 0} \mathcal{W}_K$ is the Weil cone.
\end{theorem}

\begin{proof}
Fix $K > 0$. By (H6) with the monotone schedules $t^*_{\mathrm{T5}}(K)$, $M^*(K)$, we have:
\begin{equation}
  \lambda_{\min}\big(T_{M^*(K)}[P_A] - T_P\big) \geq \tfrac{1}{2} c_0^*(K) > 0.
\end{equation}
Hence the finite Toeplitz form is nonnegative on the Fej\'er$\times$heat cone. By (H2) the cone is dense in $W_K$, and by (H3) the functional $Q$ is continuous; therefore $Q \geq 0$ on $W_K$. Taking the union over all $K$ shows $Q \geq 0$ on $\mathcal{W}$. Finally, (H1) identifies this $Q$ with the canonical Weil functional.
\end{proof}

\begin{remark}[Analytic vs. numerical]
The proof of Theorem~\ref{thm:Main-positivity-scope} uses only analytic bounds established in the body of the paper. Numerical certificates and computational verifications are archived separately for reproducibility but play no role in the logical argument.
\end{remark}

\section{Notation Conventions}\label{sec:notation-conventions}

We establish the notation conventions used throughout.

\subsection{Frequency Axis}

On the frequency axis, we write $\xi = \eta/(2\pi)$. The archimedean density is:
\begin{equation}
  a(\xi) = \log \pi - \Re \psi\left(\tfrac{1}{4} + i\pi\xi\right), \qquad a_*(\xi) = 2\pi a(\xi),
\end{equation}
where $\psi$ is the digamma function.

\subsection{Prime Nodes}

Prime nodes are located at:
\begin{equation}
  \xi_n = \frac{\log n}{2\pi}
\end{equation}
with symmetric placement $\pm \xi_n$ for the even setting.

\subsection{Weight Conventions}

We distinguish two weight conventions:

\begin{equation}
  w_Q(n) = \frac{2\Lambda(n)}{\sqrt{n}} \quad \text{(one-sided weight in $Q$)},
\end{equation}
\begin{equation}
  w_{\mathrm{RKHS}}(n) = \frac{\Lambda(n)}{\sqrt{n}} \quad \text{(operator weight on $W_K$)}.
\end{equation}

Evenization lets us pass freely between them: doubling $w_{\mathrm{RKHS}}$ on $\xi_n > 0$ gives $w_Q$, while placing both $\pm \xi_n$ leaves $w_{\mathrm{RKHS}}$ unchanged.

All RKHS and operator bounds below use $w_{\mathrm{RKHS}}$; we abbreviate:
\begin{equation}
  w_{\max} := \sup_n w_{\mathrm{RKHS}}(n) \leq \frac{2}{e} \approx 0.7358.
\end{equation}

\subsection{The Weil Functional}

Throughout we use:
\begin{equation}
  Q(\Phi) = \int_{\mathbb{R}} a_*(\xi) \Phi(\xi) \, d\xi - \sum_{n \geq 2} w_Q(n) \Phi(\xi_n)
\end{equation}
on each compact window. Section~\ref{sec:T0} records the exact crosswalk to the Guinand--Weil form.

We call $Q$ ``quadratic'' only because $\Phi = g * g^\vee$ for some test function $g$; as a functional of $\Phi$ it is linear.

\subsection{Bridge Summary}

We split $Q$ as $T_M[P_A] - T_P$ with $P_A \in \mathrm{Lip}(1)$ and $T_P$ finite rank. The symbol barrier yields:
\begin{equation}
  \lambda_{\min}(T_M[P_A]) \geq c_0(K) - C \cdot \omega_{P_A}(\pi/M)
\end{equation}
with:
\begin{equation}
  c_0(K) := \min_{\theta \in \Gamma_K} P_A(\theta),
\end{equation}
where $\Gamma_K$ is the working arc on $W_K$.

The prime norm is bounded in the arch-induced RKHS by:
\begin{equation}
  \|T_P\| \leq w_{\max} + \sqrt{w_{\max}} \eta_K,
\end{equation}
where $\eta_K \in (0, 1 - w_{\max})$ is tuned via the log-node gap $\delta_K$. Thus:
\begin{equation}
  \lambda_{\min}(T_M[P_A] - T_P) \geq \min P_A - C \cdot \omega_{P_A}(\pi/M) - \|T_P\|,
\end{equation}
closing the bridge module and feeding the remaining steps.

\section{Quick Reference for Reviewers}\label{sec:quick-ref}

\subsection{Architecture}

The analytic chain is:
\begin{equation}
  \text{T0} \to \text{A1$'$} \to \text{A2} \to \text{A3} \to \text{RKHS} \to \text{T5} \to Q \geq 0.
\end{equation}

\textbf{Goal:} $Q \geq 0$ on the Weil cone $\mathcal{W}$.

\subsection{Two Scales}

The framework uses two independent scales:

\begin{itemize}
  \item $t_{\mathrm{sym}}$ controls the symbol modulus $\omega_{P_A}$ (used in A3).
  \item $t_{\mathrm{rkhs}}$ controls the prime cap $\|T_P\|$ (used in RKHS).
\end{itemize}

No coupling is imposed between these scales.

\subsection{Uniform Margins}

The global archimedean floor $c_* > 0$ comes from the spectral Fej\'er$\times$heat routine. The uniform prime cap at $t = 0.7$ satisfies $\rho_{\mathrm{cap}} < 1/25$.

\textbf{Budget split:}
\begin{itemize}
  \item $C_{\mathrm{SB}} \omega_{P_A}(\pi/M) \leq c_*/4$ (symbol contribution).
  \item $\|T_P\| \leq c_*/4$ (prime contribution).
\end{itemize}

This yields: $\lambda_{\min}(T_M[P_A] - T_P) \geq c_*/2$.

\subsection{Adaptive Option}

On a compact $[-K, K]$, the RKHS scale:
\begin{equation}
  t_{\min}(K) = \frac{\delta_K^2}{4 \ln\left(\frac{2 + \eta_K}{\eta_K}\right)}
\end{equation}
sharpens the bound on $\|T_P\|$.

\subsection{Transfer (T5)}

The transfer proceeds as:
\begin{equation}
  \text{Grid} \Rightarrow \text{Compact} \Rightarrow \text{Weil cone}.
\end{equation}

Series, tail, grid, and limit steps complete the chain.


%==============================================================================
% Section 2: Normalization and Weil Functional
%==============================================================================
% T0: Guinand-Weil Normalization Crosswalk
% ========================================

\section{Normalization and the Weil Functional}\label{sec:T0}

\subsection{Fourier Conventions}

We adopt the unitary Fourier transform normalization:
\begin{equation}\label{eq:fourier-conv}
  \widehat{\varphi}(\xi) = \int_{\R} \varphi(t)\, e^{-2\pi i t \xi}\, dt,
  \qquad
  \varphi(t) = \int_{\R} \widehat{\varphi}(\xi)\, e^{2\pi i t \xi}\, d\xi,
\end{equation}
with Lebesgue measure $d\xi$ on the frequency side. For even test functions, all identities reduce to cosine form.

\subsection{The Weil Functional}

\begin{definition}[Weil functional]\label{def:weil-functional}
For an even, compactly supported test function $\Phi$ on $\R$, define:
\begin{equation}\label{eq:Q-functional}
  Q(\Phi) := \int_{\R} \astar(\xi)\, \Phi(\xi)\, d\xi
           - \sum_{n \geq 2} \wQ{n}\, \Phi(\xilog),
\end{equation}
where:
\begin{itemize}
  \item $\xilog := \frac{\log n}{2\pi}$ are the log-coordinates of integers,
  \item $\wQ{n} := \frac{2\Lambda(n)}{\sqrt{n}}$ are the prime weights ($\Lambda$ = von Mangoldt function),
  \item $\astar(\xi) := 2\pi\bigl(\log\pi - \re\psi(\tfrac{1}{4} + i\pi\xi)\bigr)$ is the Archimedean density.
\end{itemize}
\end{definition}

\begin{proposition}[T0: Guinand-Weil matching]\label{prop:T0-GW}
Under the conventions above, our functional $Q(\Phi)$ matches the classical Guinand-Weil functional $Q_{\mathrm{GW}}$ after the change of variables $\eta = 2\pi\xi$:
\begin{equation}
  Q(\Phi) = Q_{\mathrm{GW}}(\Phi_{\mathrm{GW}}),
  \qquad \eta = 2\pi\xi,\ \ \Phi_{\mathrm{GW}}(\eta) = \Phi(\eta/2\pi).
\end{equation}
\end{proposition}

\begin{proof}
Change variables $\eta = 2\pi\xi$ in the Archimedean integral. The Jacobian $d\eta = 2\pi\, d\xi$ is absorbed by the definition of $\astar$, while $\psi(\tfrac{1}{4} + \tfrac{i\eta}{2}) = \psi(\tfrac{1}{4} + i\pi\xi)$. For the prime sum, $\Phi_{\mathrm{GW}}(\pm\log n) = \Phi(\pm\xilog)$, and by evenness, $\Phi(\xilog) + \Phi(-\xilog) = 2\Phi(\xilog)$, matching the doubled weights.
\end{proof}

\begin{theorem}[Weil's positivity criterion]\label{thm:weil-criterion}
The Riemann Hypothesis is equivalent to:
\begin{equation}
  Q(\Phi) \geq 0 \quad \text{for all even, nonnegative } \Phi \in C_c(\R).
\end{equation}
\end{theorem}

\begin{remark}[Weight conventions]\label{rem:weights}
Throughout this paper:
\begin{itemize}
  \item $\wQ{n} = 2\Lambda(n)/\sqrt{n}$ in the Weil functional (evenized weights at positive nodes).
  \item $\wRKHS{n} = \Lambda(n)/\sqrt{n}$ in RKHS and operator bounds (undoubled weights).
  \item For primes: $\wQ{p} = 2\log p/\sqrt{p}$ and $\wRKHS{p} = \log p/\sqrt{p}$.
\end{itemize}
The factor of 2 arises from placing weights at $\pm\xilog$ for even tests.
\end{remark}


%==============================================================================
% Section 2: Cone Density (A1')
%==============================================================================
% A1': Local Cone Density on Frequency Compacts
% ==============================================

\section{Test Function Cone Density}\label{sec:A1}

\subsection{Fej\'er and Heat Kernels}

\begin{definition}[Fej\'er-heat test functions]\label{def:fejer-heat}
For bandwidth $B > 0$, heat parameter $t > 0$, and center $\tau$, define:
\begin{equation}
  \Phitest(\xi) := \Lambda_B(\xi)\, \rho_t(\xi),
\end{equation}
where:
\begin{itemize}
  \item $\Lambda_B(\xi) := (1 - |\xi|/B)_+$ is the Fej\'er kernel (triangular window),
  \item $\rho_t(\xi) := e^{-4\pi^2 t \xi^2}$ is the heat kernel.
\end{itemize}
For centered windows with compact $[-K, K]$, we typically set $B = K$.
\end{definition}

\begin{theorem}[A1': Cone density]\label{thm:A1-density}
For every compact $[-K, K]$, the cone generated by Fej\'er-heat test functions
\begin{equation}
  \mathcal{C}_K := \overline{\mathrm{cone}\{\Phitest : B > 0, t > 0\}}^{\|\cdot\|_\infty}
\end{equation}
is dense in $C^+_{\mathrm{even}}([-K, K])$, the space of even, nonnegative continuous functions on $[-K, K]$.
\end{theorem}

\begin{proof}[Proof sketch]
The proof proceeds in four steps:
\begin{enumerate}
  \item \textbf{Mollification}: For $f \in C^+_{\mathrm{even}}([-K, K])$, convolution with $\rho_t$ produces a smooth approximation $g = f * \rho_t$ with $\|g - f\|_\infty < \varepsilon/3$ for small $t$.

  \item \textbf{Riemann sums}: Approximate $g$ by positive linear combinations of translates of $\rho_t$ via partition of $[-K, K]$.

  \item \textbf{Fej\'er truncation}: For $B \geq 6K/\varepsilon$, the Fej\'er factor $\Lambda_B(\xi - \tau) \approx 1$ on $[-K, K]$ with error $< \varepsilon/3$.

  \item \textbf{Evenization}: Symmetrize to obtain an even function in the cone.
\end{enumerate}
Combining errors via the triangle inequality yields the claim.
\end{proof}

\begin{lemma}[Fixed-$t$ cone density]\label{lem:A1-fixed-t}
For fixed $K > 0$ and $t_0 > 0$, the closure of the cone generated by $\{\Phitest[\xi] : B > 0, |\tau| + B \leq K\}$ with $t = t_0$ equals $C^+_{\mathrm{even}}([-K, K])$.
\end{lemma}

\begin{remark}[Parameter scaling]
The bandwidth $B$ and heat scale $t$ depend on the compact $[-K, K]$ and target accuracy. In the compact-by-compact transfer (\S\ref{sec:results}), the schedules $K \mapsto B(K)$ and $K \mapsto t(K)$ are allowed to grow with $K$; no uniform bound is needed.
\end{remark}


%==============================================================================
% Section 3: Continuity of Q (A2)
%==============================================================================
% A2: Lipschitz Continuity of Q on Compacts
% ==========================================

\section{Continuity of the Weil Functional}\label{sec:A2}

\subsection{Finiteness on Compacts}

\begin{lemma}[Local finiteness of the prime sampler]\label{lem:Q-local-finite}
Fix $K > 0$. For every even $\Phi \in C_c(\R)$ with $\supp\Phi \subset [-K, K]$, the prime sum in $Q(\Phi)$ is finite:
\begin{equation}
  \sum_{n \geq 2} \wQ{n}\, \Phi(\xilog) = \sum_{\xilog \in [-K, K]} \wQ{n}\, \Phi(\xilog).
\end{equation}
Only indices with $n \leq \lfloor e^{2\pi K} \rfloor$ contribute.
\end{lemma}

\begin{proof}
If $\supp\Phi \subset [-K, K]$, then $\Phi(\xilog) = 0$ whenever $|\xilog| > K$. Since $\xilog = \log n / (2\pi)$, the condition $|\xilog| \leq K$ is equivalent to $n \leq e^{2\pi K}$.
\end{proof}

\begin{lemma}[A2: Lipschitz continuity]\label{lem:A2-lip}
Fix compact $K = [-R, R]$. For even, nonnegative $\Phi_1, \Phi_2 \in C_c(K)$:
\begin{equation}
  |Q(\Phi_1) - Q(\Phi_2)| \leq L_Q(K)\, \|\Phi_1 - \Phi_2\|_\infty,
\end{equation}
where the Lipschitz constant is:
\begin{equation}
  L_Q(K) := \|\astar\|_{L^1(K)} + \sum_{\xilog \in K} \wQ{n}.
\end{equation}
\end{lemma}

\begin{proof}
The Archimedean term contributes:
\begin{equation}
  \left| \int_K \astar(\xi)\, (\Phi_1 - \Phi_2)(\xi)\, d\xi \right|
  \leq \|\astar\|_{L^1(K)}\, \|\Phi_1 - \Phi_2\|_\infty.
\end{equation}
The prime term, being a finite sum by \cref{lem:Q-local-finite}, contributes:
\begin{equation}
  \left| \sum_{\xilog \in K} \wQ{n}\, (\Phi_1 - \Phi_2)(\xilog) \right|
  \leq \left( \sum_{\xilog \in K} \wQ{n} \right) \|\Phi_1 - \Phi_2\|_\infty.
\end{equation}
\end{proof}

\subsection{Tail Estimates}

\begin{lemma}[Leakage control]\label{lem:tail-bound}
For Fej\'er-heat windows with parameters $(B, t)$, the tail beyond a compact $K$:
\begin{equation}
  \mathrm{Tail}(t; N) := \sum_{\xilog \notin K,\ n > N} \wQ{n}\, \Phi(\xilog)
  \ll \frac{e^{-t (\log N)^2}}{t},
\end{equation}
where the implied constant is absolute.
\end{lemma}

\begin{proof}
For $\Phi(\xi) \leq e^{-4\pi^2 t \xi^2}$ and $\xilog = \log n / (2\pi)$:
\begin{equation}
  \sum_{n > N} \frac{2\log n}{\sqrt{n}}\, e^{-t(\log n)^2}
  \leq C \int_{\log N}^\infty y\, e^{-ty^2}\, dy
  \ll \frac{e^{-t(\log N)^2}}{t}.
\end{equation}
\end{proof}

\begin{remark}
When Fej\'er-heat windows are strictly supported in $[-K, K]$, the tail vanishes exactly. The Gaussian factor produces only exponentially small leakage beyond the compact support.
\end{remark}


%==============================================================================
% Section 4: Operator Formulation (A3)
%==============================================================================
% A3: Toeplitz-Symbol Bridge
% ===========================

\section{The Operator Formulation}\label{sec:A3}

\begin{remark}[Proof provenance]
Full, line-by-line proofs of the A3 statements appear in the RH\_Q3 manuscript (Sections 8.x). Here we retain the formulations and a proof sketch to keep the paper self-contained; the analytic inputs (archimedean bounds, modulus of continuity, discretization) are stated explicitly.
\end{remark}

\subsection{Discretization via Cosine Basis}

Discretizing the Weil functional in a cosine basis $\{\cos(k\xi)\}_{k=0}^{M-1}$ yields a finite-dimensional operator formulation.

\begin{definition}[Q3 Hamiltonian]\label{def:Q3-hamiltonian}
For compact parameter $K > 0$, matrix size $M$, and heat parameter $t > 0$, define:
\begin{equation}
  H = \TA - \TP \in \R^{M \times M},
\end{equation}
where:
\begin{itemize}
  \item $\TA$ is a Toeplitz matrix with entries $(\TA)_{ij} = A_{|i-j|}$,
  \begin{equation}
    A_k = \int_{-K}^K \astar(\xi)\, \Phitest(\xi)\, \cos(k\xi)\, d\xi,
  \end{equation}

  \item $\TP$ is a finite-rank matrix encoding prime contributions,
  \begin{equation}
    \TP = \sum_{p \leq e^{2\pi K}} w(p)\, \phi_p\, v_p v_p^T,
  \end{equation}
  where $v_p = (\cos(k\xi_p))_{k=0}^{M-1}$, $\phi_p = \Phitest(\xi_p)$, and $w(p) = 2\log p/\sqrt{p}$.
\end{itemize}
\end{definition}

\begin{theorem}[Operator-functional equivalence]\label{thm:operator-functional}
The Weil positivity criterion translates to:
\begin{equation}
  \text{RH} \quad \Longleftrightarrow \quad H \geq 0 \quad (\text{all eigenvalues } \geq 0).
\end{equation}
\end{theorem}

\subsection{The A3 Bridge Inequality}

\begin{lemma}[Archimedean floor]\label{lem:arch-floor}
For the smoothed symbol $\PA = \astar * K_{\tsym}$ on the torus, there exists $\carch(K) > 0$ such that:
\begin{equation}
  \min_{\theta \in \T} \PA(\theta) \geq \carch(K).
\end{equation}
Moreover, by fixing a bandwidth $B_0 \in (0,1)$ and choosing $t_{\mathrm{sym}}$ large enough, one obtains a uniform lower bound $c_\infty > 0$ independent of $K$ (Proposition~\ref{prop:symbol-floor-stable}).
\end{lemma}

\begin{lemma}[Prime cap]\label{lem:prime-cap}
For appropriate $\trkhs(K)$, the prime operator satisfies:
\begin{equation}
  \|\TP\|_{\mathrm{op}} \leq \rhok \leq \frac{\carch(K)}{4}.
\end{equation}
\end{lemma}

\begin{theorem}[A3 bridge inequality]\label{thm:A3-bridge}
Let $(B, \tsym, \trkhs)$ satisfy the parameter schedule of \cref{lem:arch-floor,lem:prime-cap}. Then for $M \geq M_0(K)$:
\begin{equation}
  \lammin(T_M[\PA] - \TP) \geq \frac{\carch(K)}{4} > 0,
\end{equation}
and the associated Fej\'er-heat test functions satisfy $Q(\Phitest) \geq 0$.
\end{theorem}

\begin{proof}[Proof sketch]
The Archimedean floor $\carch(K)$ comes from explicit digamma bounds. The prime cap $\rhok$ follows from RKHS contraction (Gershgorin/Gram matrix analysis). The Szeg\H{o}-B\"ottcher theorem transfers the symbol floor to the Toeplitz eigenvalues with error $O(1/M)$.
\end{proof}

\subsection{RKHS Foundations}

\begin{lemma}[Gershgorin bound for Gram matrices]\label{lem:gershgorin}
Let $K = [k(x_i, x_j)]$ be a Gram matrix with diagonal floor $k(x_i, x_i) \geq \czero$ and off-diagonal bound:
\begin{equation}
  \sum_{j \neq i} |k(x_i, x_j)| \leq \rhok \quad \text{for all } i.
\end{equation}
Then $\lammin(K) \geq \czero - \rhok$.
\end{lemma}

\begin{proof}
By Gershgorin's circle theorem, every eigenvalue lies in $[\czero - \rhok, \infty)$.
\end{proof}


%==============================================================================
% Section 4b: Detailed Archimedean Analysis (A3)
%==============================================================================
% A3 Detailed: Symbol Floor and Archimedean Bounds
% =================================================

\section{Detailed Archimedean Analysis (A3)}\label{sec:A3-detailed}

This section provides the complete analytic machinery for the symbol floor bounds that underpin the Toeplitz bridge. We establish explicit constants and trace the dependencies carefully.

\subsection{The Archimedean Symbol}

Recall from Section~\ref{sec:A3} that the archimedean contribution to the Weil functional is encoded in the symbol $P_A(\theta)$. For test functions $\Phi_{B,t}$ constructed via Fej\'er$\times$heat kernels, this symbol takes the form:
\begin{equation}\label{eq:PA-def-detailed}
  P_A(\theta) = \sum_{|k| \leq B} \widehat{P}_A(k) e^{ik\theta},
\end{equation}
where the Fourier coefficients $\widehat{P}_A(k)$ arise from the explicit formula.

\subsection{Lipschitz Modulus Control}

\begin{lemma}[Lipschitz modulus of $P_A$]\label{lem:lipschitz-modulus}
For each compact $K > 0$, the symbol $P_A$ restricted to test functions in $\mathcal{W}_K$ satisfies:
\begin{equation}
  |P_A(\theta_1) - P_A(\theta_2)| \leq L_A(K) \cdot |\theta_1 - \theta_2|,
\end{equation}
where the Lipschitz constant admits the bound:
\begin{equation}\label{eq:LA-bound}
  L_A(K) \leq C_{\mathrm{arch}} \cdot K \cdot (1 + \log K),
\end{equation}
for an absolute constant $C_{\mathrm{arch}} > 0$ depending only on the normalization of the Weil functional.
\end{lemma}

\begin{proof}
The derivative of $P_A$ satisfies
\begin{equation}
  P_A'(\theta) = i \sum_{|k| \leq B} k \cdot \widehat{P}_A(k) e^{ik\theta}.
\end{equation}
Using the decay of $\widehat{P}_A(k)$ from the Guinand--Weil normalization and the bound $B \asymp K$ from the Fej\'er bandwidth selection, we obtain
\begin{equation}
  \|P_A'\|_\infty \leq C \sum_{|k| \leq B} |k| \cdot |\widehat{P}_A(k)| \leq C_{\mathrm{arch}} \cdot K \cdot (1 + \log K).
\end{equation}
The mean value theorem then yields the claimed Lipschitz bound.
\end{proof}

\subsection{Core Archimedean Contribution}

The positivity of the Weil functional hinges on the archimedean term dominating the prime contributions. We now establish the key lower bound.

\begin{lemma}[Core archimedean contribution]\label{lem:core-arch}
For each $K > 0$, the archimedean symbol satisfies:
\begin{equation}\label{eq:core-arch-lower}
  P_A(\theta) \geq c_{\mathrm{core}}(K) \cdot \mathbf{1}_{[-\pi, \pi]}(\theta),
\end{equation}
where $c_{\mathrm{core}}(K)$ is a positive constant depending on the compact size.
\end{lemma}

\begin{proof}
The archimedean contribution arises from the Gamma factors in the functional equation of $\zeta(s)$. Using the explicit form of the Guinand--Weil kernel, one shows that the integrated archimedean density is strictly positive. The key is that the Gamma function contributes a positive-definite quadratic form, which after Fourier transform yields the lower bound. See~\cite{IwaniecKowalski2004} for the classical treatment.
\end{proof}

\subsection{Symbol Floor on Compacts}

\begin{definition}[Archimedean floor]\label{def:c0K}
For $K > 0$, define the \emph{archimedean floor} on $W_K$ by:
\begin{equation}
  c_0(K) := \inf_{\theta \in [-\pi, \pi]} P_A(\theta),
\end{equation}
where $P_A$ is constructed from test functions supported on $[-K, K]$.
\end{definition}

\begin{theorem}[Symbol floor bounds]\label{thm:symbol-floor}
The archimedean floor satisfies the following bounds:
\begin{enumerate}
  \item[\textup{(i)}] \textbf{Strict positivity:} $c_0(K) > 0$ for all $K > 0$.

  \item[\textup{(ii)}] \textbf{Monotone envelope:} The function $c_0^*(K) := \inf_{0 < u \leq K} c_0(u)$ is nonincreasing in $K$.

  \item[\textup{(iii)}] \textbf{Asymptotic decay:} There exist constants $C_1, C_2 > 0$ such that
  \begin{equation}\label{eq:c0-decay}
    \frac{C_1}{1 + K^2} \leq c_0(K) \leq \frac{C_2}{1 + K}.
  \end{equation}

\end{enumerate}
\end{theorem}

\begin{proof}
\textbf{(i)} This follows from Lemma~\ref{lem:core-arch} and the continuity of $P_A$.

\textbf{(ii)} If $K' \geq K$, then $\mathcal{W}_K \subseteq \mathcal{W}_{K'}$, so the infimum can only decrease.

\textbf{(iii)} The upper bound follows from the explicit computation of the archimedean density, which decays like $1/K$ as the support grows. The lower bound requires more delicate analysis of the Gamma factor contributions; see~\cite{Conrey2003}.

\textbf{(iv)} follows from the explicit lower bound in Proposition~\ref{prop:c-arch-explicit}, which supplies a closed-form $c_0(K)$ without numerical fitting.
\end{proof}

\subsection{Discretization Error Control}

When passing from the continuous Toeplitz operator $T[P_A]$ to the finite matrix $T_M[P_A]$, we incur a discretization error. The following lemma controls this error.

\begin{lemma}[Toeplitz discretization]\label{lem:toeplitz-disc}
Let $P_A$ be a Lipschitz symbol with modulus $L_A(K)$. For any $M \in \mathbb{N}$:
\begin{equation}\label{eq:toeplitz-disc}
  \|T_M[P_A] - T[P_A]\| \leq C_T \cdot \omega_{P_A}\!\left(\frac{\pi}{M}\right),
\end{equation}
where $\omega_{P_A}(\delta) = L_A(K) \cdot \delta$ is the modulus of continuity and $C_T > 0$ is an absolute constant.
\end{lemma}

\begin{proof}
This is a standard result from Toeplitz operator theory. The key observation is that the eigenvalues of $T_M[P_A]$ approximate those of $T[P_A]$ with error controlled by the oscillation of the symbol on intervals of length $\pi/M$. See~\cite{BottcherSilbermann2006} for the general theory.
\end{proof}

\subsection{Grid Resolution Requirements}

Combining the symbol floor with discretization error yields the resolution requirements for the Toeplitz approximation.

\begin{corollary}[Grid resolution for positivity]\label{cor:grid-resolution}
To ensure
\begin{equation}
  \lambda_{\min}(T_M[P_A]) \geq \tfrac{1}{2} c_0(K),
\end{equation}
it suffices to take
\begin{equation}\label{eq:M-requirement}
  M \geq M_{\min}(K) := \left\lceil \frac{2\pi C_T L_A(K)}{c_0(K)} \right\rceil.
\end{equation}
\end{corollary}

\begin{proof}
By Lemma~\ref{lem:toeplitz-disc}:
\begin{equation}
  \lambda_{\min}(T_M[P_A]) \geq c_0(K) - C_T \cdot L_A(K) \cdot \frac{\pi}{M}.
\end{equation}
Setting the right side $\geq \tfrac{1}{2} c_0(K)$ and solving for $M$ gives~\eqref{eq:M-requirement}.
\end{proof}

\subsection{Gershgorin Circle Analysis}

An alternative approach to eigenvalue bounds uses Gershgorin circles, which provide explicit (though sometimes weaker) estimates.

\begin{lemma}[Gershgorin bounds for Toeplitz matrices]\label{lem:gershgorin-toeplitz}
Let $T_M = (t_{j-k})_{j,k=1}^M$ be a Toeplitz matrix with symbol $P$. Then every eigenvalue $\lambda$ of $T_M$ satisfies:
\begin{equation}\label{eq:gershgorin}
  |\lambda - t_0| \leq R_M := \sum_{k=1}^{M-1} (|t_k| + |t_{-k}|).
\end{equation}
In particular:
\begin{equation}
  \lambda_{\min}(T_M) \geq t_0 - R_M = \widehat{P}(0) - R_M.
\end{equation}
\end{lemma}

\begin{proof}
Apply the Gershgorin circle theorem to the symmetric Toeplitz matrix. The diagonal entries are all $t_0 = \widehat{P}(0)$, and the sum of off-diagonal magnitudes in each row is $R_M$.
\end{proof}

\begin{corollary}[Explicit Gershgorin floor]\label{cor:gershgorin-floor}
If the symbol coefficients satisfy $|\widehat{P}(k)| \leq C \cdot e^{-\alpha |k|}$ for some $\alpha > 0$, then:
\begin{equation}
  \lambda_{\min}(T_M[P]) \geq \widehat{P}(0) - \frac{2C}{1 - e^{-\alpha}}.
\end{equation}
\end{corollary}

\begin{proof}
We have $R_M \leq 2 \sum_{k=1}^\infty C e^{-\alpha k} = 2C \cdot \frac{e^{-\alpha}}{1 - e^{-\alpha}} \leq \frac{2C}{1 - e^{-\alpha}}$ for $\alpha$ not too small.
\end{proof}

\subsection{Heat Regularization of the Symbol}

The heat kernel regularization provides another route to symbol smoothness.

\begin{lemma}[Heat-regularized symbol]\label{lem:heat-reg-symbol}
For $t > 0$, define the heat-regularized symbol:
\begin{equation}
  P_A^{(t)}(\theta) := (e^{-t\Delta} P_A)(\theta) = \sum_{k} \widehat{P}_A(k) e^{-tk^2} e^{ik\theta}.
\end{equation}
Then:
\begin{enumerate}
  \item[\textup{(i)}] $P_A^{(t)}$ is real-analytic for $t > 0$.
  \item[\textup{(ii)}] $\|P_A^{(t)} - P_A\|_\infty \to 0$ as $t \to 0^+$.
  \item[\textup{(iii)}] The Lipschitz constant satisfies $L_A^{(t)}(K) \leq L_A(K) \cdot e^{-t}$ for $t \geq 1$.
\end{enumerate}
\end{lemma}

\begin{proof}
Part (i) follows from the exponential decay of the heat kernel. Part (ii) is the standard heat kernel approximation property. Part (iii) uses the derivative bound
\begin{equation}
  \|(P_A^{(t)})'\|_\infty \leq \sum_k |k| \cdot |\widehat{P}_A(k)| \cdot e^{-tk^2} \leq e^{-t} \sum_k |k| \cdot |\widehat{P}_A(k)|,
\end{equation}
where the exponential factor dominates for $t \geq 1$.
\end{proof}

\subsection{Unified Symbol Floor Theorem}

We now state the main result that combines all the ingredients.

\begin{theorem}[Unified A3 symbol floor]\label{thm:A3-unified}
For each $K > 0$, there exist explicit parameter choices $(M^*, t^*)$ depending on $K$ such that:
\begin{equation}\label{eq:A3-unified}
  \lambda_{\min}\big(T_{M^*}[P_A^{(t^*)}]\big) \geq \frac{1}{2} c_0^*(K) > 0.
\end{equation}
Moreover:
\begin{enumerate}
  \item[\textup{(i)}] The parameters satisfy the monotonicity: $K_1 \leq K_2 \Rightarrow M^*(K_1) \leq M^*(K_2)$ and $t^*(K_1) \leq t^*(K_2)$.

  \item[\textup{(ii)}] The bound~\eqref{eq:A3-unified} is uniform over all test functions $\Phi \in \mathcal{W}_K$ in the Fej\'er$\times$heat cone.

  \item[\textup{(iii)}] The dependence on $K$ is at most polynomial: $M^*(K) = O(K^3)$ and $t^*(K) = O(K^2)$.
\end{enumerate}
\end{theorem}

\begin{proof}
\textbf{Step 1:} Choose $t^* = t^*_{\mathrm{RKHS}}(K)$ from Theorem~\ref{thm:rkhs-contraction} to ensure $\|T_P\| \leq \frac{1}{4} c_0^*(K)$.

\textbf{Step 2:} Choose $M^* = M^*_{\mathrm{T5}}(K)$ from~\eqref{eq:T5-Mstar} to ensure $C_T \omega_{P_A}(\pi/M^*) \leq \frac{1}{4} c_0^*(K)$.

\textbf{Step 3:} Apply the grid-lift inequality (Lemma~\ref{lem:T5-grid}):
\begin{align}
  \lambda_{\min}(T_{M^*}[P_A^{(t^*)}] - T_P) &\geq c_0(K) - \frac{1}{4} c_0^*(K) - \frac{1}{4} c_0^*(K) \\
  &\geq c_0^*(K) - \frac{1}{2} c_0^*(K) = \frac{1}{2} c_0^*(K).
\end{align}

The monotonicity (i) follows from the construction of the schedules. Uniformity (ii) holds because the Fej\'er$\times$heat cone is dense in $\mathcal{W}_K$ (Theorem~\ref{thm:A1-density}). The polynomial bounds (iii) follow from the explicit formulas for $M^*$ and $t^*$.
\end{proof}

\subsection{Summary of A3 Dependencies}

The A3 analytic module provides:
\begin{enumerate}
  \item \textbf{Symbol positivity:} $c_0(K) > 0$ for all $K$ (Theorem~\ref{thm:symbol-floor}).
  \item \textbf{Discretization control:} $\|T_M - T\| = O(1/M)$ (Lemma~\ref{lem:toeplitz-disc}).
  \item \textbf{Explicit parameters:} Formulas for $M^*(K)$ achieving target accuracy.
  \item \textbf{Monotone inheritance:} Parameters propagate consistently across compacts.
\end{enumerate}

Together with the RKHS contraction (Section~\ref{sec:rkhs}) and T5 transfer (Section~\ref{sec:T5}), this completes the analytic chain for the positivity verification.


%==============================================================================
% Section 5: RKHS Prime Contraction
%==============================================================================
% RKHS Contraction Mechanism
% ==========================

\section{RKHS Prime Contraction}\label{sec:rkhs}

The prime contribution to the Weil functional is encoded by a sampling operator $T_P$. This section develops analytic upper bounds on $\|T_P\|$ inside a reproducing-kernel Hilbert space (RKHS) of the heat flow, following the classical framework of Aronszajn~\cite{Aronszajn1950} and modern expositions~\cite{BerlinetThomasAgnan2004,PaulsenRaghupathi2016}.

\begin{remark}[Proof provenance]
Detailed proofs of the RKHS contraction lemmas and the monotone schedules $t_{\min}(K)$ appear in the RH\_Q3 manuscript (Section 9). We summarize the statements and key estimates here; all constants are explicit.
\end{remark}

\subsection{Setup and Notation}

Fix a compact $[-K, K] \subset \RR$, $K \geq 1$. Prime sample nodes are
\begin{equation}\label{eq:xi-nodes}
  \xi_n := \frac{\log n}{2\pi} \in [0, \infty), \qquad n \geq 2,
\end{equation}
with weights
\begin{equation}\label{eq:rkhs-weights}
  w(n) := \frac{\Lambda(n)}{\sqrt{n}}, \qquad w_{\max} := \sup_{n \geq 2} w(n) \leq \frac{2}{e}.
\end{equation}

We work in the RKHS $\mathcal{H}_k$ of the heat kernel on $\RR$:
\begin{equation}\label{eq:heat-kernel}
  k_t(x, y) := \exp\!\Big(-\frac{(x-y)^2}{4t}\Big), \qquad t > 0.
\end{equation}

\begin{lemma}[Effective weight cap]\label{lem:wmax-cap}
For $w(p^m) = \frac{\log p}{p^{m/2}}$ one has
\begin{equation}
  0 \leq w(p^m) \leq \frac{2}{e} < \frac{3}{4}.
\end{equation}
The maximum is attained at $p^m = e^2$ formally.
\end{lemma}

\begin{proof}
Consider $f(x) = \log x / \sqrt{x}$ on $x > 1$. Then $f'(x) = (1 - \tfrac{1}{2}\log x)/x^{3/2}$ vanishes at $x = e^2$ with $f(e^2) = 2/e \approx 0.7358$.
\end{proof}

\subsection{Node Separation}

\begin{lemma}[Node gap on compacts]\label{lem:node-gap}
For $\xi_n = \frac{\log n}{2\pi}$ and fixed $K > 0$, the active set is $\{2, \ldots, \lfloor e^{2\pi K} \rfloor\}$ and the minimal spacing satisfies
\begin{equation}\label{eq:deltaK}
  \delta_K := \min_{m \neq n,\ \xi_m, \xi_n \in [-K, K]} |\xi_m - \xi_n| \geq \frac{1}{2\pi(\lfloor e^{2\pi K} \rfloor + 1)}.
\end{equation}
\end{lemma}

\begin{proof}
Apply the mean value theorem to $\log x$ between consecutive integers.
\end{proof}

\subsection{RKHS Core Lemmas}

\begin{lemma}[Energy identity]\label{lem:rkhs-energy}
For $f \in \mathcal{H}_k$ supported on the closure of $\mathrm{span}\{k(\cdot, x) : x \in \mathcal{X}\}$:
\begin{equation}
  \|f\|_{\mathcal{H}_k}^2 = \langle f, T_k^\dagger f \rangle_{L^2(\mu)},
\end{equation}
where $T_k^\dagger$ is the pseudoinverse on the image of $T_k$. In particular, if $f(x) = \sum_{i=1}^N a_i k(x, x_i)$ for a finite sample, then
\begin{equation}
  \|f\|_{\mathcal{H}_k}^2 = a^\top K a,
\end{equation}
where $K = [k(x_i, x_j)]_{i,j=1}^N$ is the Gram matrix.
\end{lemma}

\begin{lemma}[Spectral floor for Gram matrices]\label{lem:gram-spectral-floor}
Assume the diagonal of $K$ obeys $k(x_i, x_i) \geq c_0$ and the off-diagonal mass satisfies
\begin{equation}
  \sum_{j \neq i} |k(x_i, x_j)| \leq \rho_k \qquad \text{for every } i \in \{1, \ldots, N\}.
\end{equation}
Then
\begin{equation}
  \lambda_{\min}(K) \geq c_0 - \rho_k.
\end{equation}
\end{lemma}

\begin{proof}
Gershgorin's circle theorem~\cite{HornJohnson2013,Varga2004} states that every eigenvalue $\lambda$ of $K$ belongs to at least one disc
\begin{equation}
  D_i = \Big\{ z \in \mathbb{C} : |z - k(x_i, x_i)| \leq \sum_{j \neq i} |k(x_i, x_j)| \Big\}.
\end{equation}
The hypothesis guarantees $\inf D_i \geq c_0 - \rho_k$, hence every eigenvalue lies in $[c_0 - \rho_k, \infty)$.
\end{proof}

\subsection{Off-Diagonal Bounds}

\begin{lemma}[Geometric tail bound]\label{lem:geom-SK}
For any node set with minimal spacing $\delta_K > 0$, define
\begin{equation}
  S_K(t) := \sum_{m \neq n} e^{-\frac{(\xi_m - \xi_n)^2}{4t}}.
\end{equation}
Then
\begin{equation}\label{eq:SK-bound}
  S_K(t) \leq \frac{2 e^{-\delta_K^2/(4t)}}{1 - e^{-\delta_K^2/(4t)}}.
\end{equation}
\end{lemma}

\begin{proof}
Fix $n$ and order the remaining nodes by increasing distance. The $j$-th nearest neighbor lies at distance at least $j \cdot \delta_K$, hence the $n$-th row sum of off-diagonal magnitudes is bounded by $2 \sum_{j \geq 1} e^{-j^2 \delta_K^2/(4t)}$. Since $j^2 \geq j$ for $j \geq 1$, we have $e^{-j^2 c} \leq e^{-jc}$ for $c > 0$, yielding the geometric series bound.
\end{proof}

\subsection{Prime Operator Bounds}

The prime operator is defined as
\begin{equation}\label{eq:TP-def}
  T_P := \sum_{\xi_n \in [-K, K]} w(n) |k_{\xi_n}\rangle \langle k_{\xi_n}|, \qquad \|k_\xi\|_{\mathcal{H}_K} = 1.
\end{equation}

\begin{proposition}[RKHS cap via Gram geometry]\label{prop:rkhs-gram-cap}
For every $t > 0$ and $K \geq 1$:
\begin{equation}\label{eq:TP-bound}
  \|T_P\|_{\mathcal{H}_k \to \mathcal{H}_k} \leq w_{\max} + \sqrt{w_{\max}} \cdot S_K(t).
\end{equation}
In particular, with $t = t_{\min}(K)$ defined below:
\begin{equation}
  \|T_P\| \leq \rho_K := w_{\max} + \sqrt{w_{\max}} \cdot \eta_K, \qquad \eta_K \in (0, 1 - w_{\max}).
\end{equation}
\end{proposition}

\begin{proof}
By Gershgorin's theorem applied to the weighted Gram matrix $W^{1/2} G W^{1/2}$ where $W = \mathrm{diag}(w(n))$ and $G_{mn} = \langle k_{\xi_m}, k_{\xi_n} \rangle$, each eigenvalue of $T_P$ lies in a disc centered at $w(n)$ with radius $\sqrt{w(n)} \sum_{m \neq n} \sqrt{w(m)} |G_{mn}|$. Using $|G_{mn}| \leq e^{-(\xi_m - \xi_n)^2/(4t)}$ and Lemma~\ref{lem:geom-SK} yields the stated bound.
\end{proof}

\subsection{Constructive Heat Scale}

\begin{theorem}[Strict contraction]\label{thm:rkhs-contraction}
If $t = t_{\min}(K)$ is chosen so that
\begin{equation}
  S_K(t_{\min}) \leq \frac{1 - w_{\max} - \varepsilon_K}{\sqrt{w_{\max}}}
\end{equation}
for some $\varepsilon_K \in (0, 1 - w_{\max})$, then $\|T_P\|_{\mathcal{H}_K} \leq \rho_K < 1$ and hence
\begin{equation}
  T_A - T_P \succeq (1 - \rho_K) T_A \succeq 0 \qquad \text{on } \mathcal{H}_K.
\end{equation}
Solving the geometric bound of Lemma~\ref{lem:geom-SK} for $t$ gives the explicit formula:
\begin{equation}\label{eq:tmin}
  \boxed{t_{\min}(K) = \frac{\delta_K^2}{4 \ln\!\bigl((2 + \eta_K)/\eta_K\bigr)}}, \qquad \eta_K = \frac{1 - w_{\max} - \varepsilon_K}{\sqrt{w_{\max}}}.
\end{equation}
\end{theorem}

\begin{proof}
Set $q := e^{-\delta_K^2/(4t)} \in (0, 1)$ and require $\frac{2q}{1-q} \leq \eta_K$, i.e., $q \leq \frac{\eta_K}{2 + \eta_K}$. This is equivalent to $t \leq \delta_K^2 / (4 \ln((2 + \eta_K)/\eta_K))$.
\end{proof}

\begin{remark}[Monotonicity in $K$]
Because $\delta_K \downarrow 0$ as the compact widens, the closed form~\eqref{eq:tmin} shows that $t_{\min}(K)$ is automatically chosen monotone decreasing along the chain $K \nearrow$. Thus the parameter schedule used in A3/T5 is consistent without additional tuning.
\end{remark}

\subsection{Explicit contraction schedule}\label{subsec:explicit-tmin}

Combining the analytic Archimedean floor \(c_0(K)\) from Proposition~\ref{prop:c-arch-explicit} with the gap bound~\eqref{eq:deltaK} yields a fully explicit choice of \(t\).

\begin{corollary}[Closed form for $t_{\min}(K)$]\label{cor:explicit-tmin}
Let
\[
  \eta_K := \frac{ \tfrac14 c_0(K) - w_{\max} }{ \sqrt{w_{\max}} },
\]
and note $\eta_K>0$ for all $K>0$ under the explicit bound of Proposition~\ref{prop:c-arch-explicit} (since $c_0(K) \downarrow 0.80\ldots > 4 w_{\max} \approx 0.736$ as $K\to\infty$). Define
\begin{equation}\label{eq:tmin-explicit}
  t_{\min}(K) := \frac{\delta_K^2}{4 \ln\!\bigl((2+\eta_K)/\eta_K\bigr)}, \qquad
  \delta_K \ge \frac{1}{2\pi(e^{2\pi K}+1)}.
\end{equation}
Then $S_K(t_{\min}) \le \eta_K$ and consequently
\[
  \|T_P\| \le w_{\max} + \sqrt{w_{\max}}\, \eta_K \le \frac{c_0(K)}{4}.
\]
\end{corollary}

\begin{proof}
Set the geometric tail bound $2q/(1-q) \le \eta_K$ with $q = e^{-\delta_K^2/(4t)}$; solving for $t$ gives~\eqref{eq:tmin-explicit}. The norm bound then follows from Proposition~\ref{prop:rkhs-gram-cap}.
\end{proof}


\subsection{Early/Tail Calculus}

\begin{lemma}[Early block]\label{lem:rkhs-early}
For every $N \geq 2$:
\begin{equation}
  \sum_{n \leq N} \frac{\Lambda(n)}{\sqrt{n}} \leq \sum_{n \leq N} \frac{\log n}{\sqrt{n}} \leq 2\sqrt{N} \log N.
\end{equation}
\end{lemma}

\begin{proof}
$\Lambda(n) \leq \log n$ is standard. For the integral bound:
\begin{equation}
  \sum_{n \leq N} \frac{\log n}{\sqrt{n}} \leq \int_1^N \frac{\log x}{\sqrt{x}} dx + O(1) = \big[2\sqrt{x} \log x - 4\sqrt{x}\big]_1^N + O(1) \leq 2\sqrt{N} \log N.
\end{equation}
\end{proof}

\begin{lemma}[Log-Gaussian tail]\label{lem:rkhs-tail}
For every $t > 0$ and $N \geq 2$:
\begin{equation}
  \sum_{n > N} \frac{\Lambda(n)}{\sqrt{n}} e^{-4\pi^2 t (\log n)^2} \ll \frac{e^{-4\pi^2 t (\log N)^2}}{t}.
\end{equation}
\end{lemma}

\begin{proof}
Replace the sum by a Stieltjes integral against $\psi(x) = \sum_{n \leq x} \Lambda(n)$ and substitute $y = \log x$. The Gaussian tail estimate is elementary.
\end{proof}

\subsection{Trace-Cap Bound}

\begin{lemma}[Trace-cap bound]\label{lem:trace-cap-bound}
For every compact $[-K, K]$, choose $t_{\mathrm{rkhs}} \geq t_{\min}(K)$ from~\eqref{eq:tmin}. Then the prime operator obeys:
\begin{equation}
  \|T_P\|_{\mathrm{op}} \leq \rho_K = w_{\max} + \sqrt{w_{\max}} \cdot S_K(t_{\min}(K)) \leq \frac{1}{4} c_0(K),
\end{equation}
where $c_0(K)$ is the Archimedean floor from Section~\ref{sec:A3}. Consequently, for every Fej\'er$\times$heat parameter set $(B, t_{\mathrm{rkhs}})$ with $t_{\mathrm{rkhs}} \geq t_{\min}(K)$, the contraction bound $\|T_P\| \leq c_0(K)/4$ holds analytically.
\end{lemma}

\begin{corollary}[Plug into A3]\label{cor:a3-plug}
On $[-K, K]$:
\begin{equation}
  \lambda_{\min}\big(T_M[P_A] - T_P\big) \geq c_0(K) - C \cdot \omega_{P_A}\!\big(\tfrac{\pi}{M}\big) - \|T_P\|.
\end{equation}
With $t \geq t_{\min}(K)$ one has $\|T_P\| \leq c_0(K)/4$, hence:
\begin{equation}
  \lambda_{\min}\big(T_M[P_A] - T_P\big) \geq \tfrac{1}{2} c_0(K) - C \cdot \omega_{P_A}\!\big(\tfrac{\pi}{M}\big).
\end{equation}
\end{corollary}

\subsection{Two-Scale Decoupling}

\begin{corollary}[Two-scale decoupling]\label{cor:two-scale}
On a fixed compact $K$, choose $t_{\mathrm{rkhs}} = t_{\min}(K)$ so that $\|T_P\| \leq \rho_K < 1$. Let $t_{\mathrm{sym}} > 0$ in the Fej\'er$\times$heat window be chosen independently. If $t_{\mathrm{sym}}$ is such that $\min P_A \geq c_0 > 0$, then:
\begin{itemize}
  \item The symbol parameter $t_{\mathrm{sym}}$ controls the modulus $\omega_{P_A}$ (symbol barrier).
  \item The RKHS parameter $t_{\mathrm{rkhs}}$ controls only $\|T_P\|$ (contraction).
\end{itemize}
The effects are formally decoupled.
\end{corollary}


%==============================================================================
% Section 5b: Prime Trace Closed Form Bounds
%==============================================================================
% RKHS Prime Trace Closed Form Bounds
% =====================================

\section{Prime Trace Closed Form Bounds}\label{sec:prime-trace-closed}

This section develops closed-form upper bounds for the prime trace $\rho(t)$, which controls the operator norm of the prime sampling operator $T_P$.

\subsection{The Prime Trace Function}

Recall from Section~\ref{sec:rkhs} that the prime operator $T_P$ on the compact $W_K = [-K, K]$ has trace
\begin{equation}
  \mathrm{tr}\, T_P = 2 \sum_{n \geq 2} \frac{\Lambda(n)}{\sqrt{n}} k_t(\xi_n, \xi_n) = 2 \sum_{n \geq 2} \frac{\Lambda(n)}{\sqrt{n}},
\end{equation}
where the second equality holds because $k_t(x, x) = 1$ for the normalized heat kernel. However, for the operator norm on the RKHS $\mathcal{H}_t$, we need the weighted trace with the kernel evaluated at prime nodes.

\begin{definition}[Prime trace function]
For $t > 0$, define the prime trace function:
\begin{equation}
  \rho(t) := 2 \int_0^\infty y \, e^{y/2} \, e^{-4\pi^2 t \, y^2} \, dy.
\end{equation}
This is an upper bound for the prime sampling contribution via the integral approximation
\begin{equation}
  \sum_{n \geq 2} \frac{\Lambda(n)}{\sqrt{n}} e^{-4\pi^2 t (\log n/(2\pi))^2} \leq \int_1^\infty \frac{\log x}{\sqrt{x}} e^{-4\pi^2 t (\log x/(2\pi))^2} \, dx = \frac{1}{2} \rho(t).
\end{equation}
\end{definition}

\subsection{Closed-Form Upper Bound}

\begin{lemma}[Closed-form upper bound for the prime trace]\label{lem:rho-closed-form}
For $t > 0$ one has
\begin{equation}\label{eq:Prime-trace-closed-form-1}
  \rho(t) \leq 2 \int_0^{\infty} y \, e^{y/2} \, e^{-4\pi^2 t \, y^2} \, dy.
\end{equation}
With $a = 4\pi^2 t$ and $b = \tfrac{1}{2}$ this implies
\begin{equation}\label{eq:rho-closed-form}
  \rho(t) \leq \frac{1}{4\pi^2 t} + \frac{\sqrt{\pi}}{2 \, (4\pi^2 t)^{3/2}} \exp\!\Big(\frac{1}{16\pi^2 t}\Big).
\end{equation}
In particular, at $t = 1$ this yields the unconditional bound $\rho(1) < \tfrac{1}{25}$, hence $\|T_P\| \leq \rho(1) < \tfrac{1}{25}$ for all compacts.
\end{lemma}

\begin{proof}[Sketch]
The integral $\int_0^{\infty} y \, e^{-a y^2 + b y} \, dy$ admits the closed form via completing the square:
\begin{equation}
  \int_0^{\infty} y \, e^{-a y^2 + b y} \, dy = e^{\frac{b^2}{4a}} \frac{b\sqrt{\pi}}{4a^{3/2}} \bigl(1 + \operatorname{erf}(\tfrac{b}{2\sqrt{a}})\bigr) + \frac{1}{2a}.
\end{equation}
Using $1 + \operatorname{erf}(x) \leq 2$ gives the upper bound \eqref{eq:rho-closed-form}. Plug $a = 4\pi^2 t$, $b = \tfrac{1}{2}$ and simplify.

For $t = 1$:
\begin{align}
  \rho(1) &\leq \frac{1}{4\pi^2} + \frac{\sqrt{\pi}}{2(4\pi^2)^{3/2}} \exp\!\Big(\frac{1}{16\pi^2}\Big) \\
  &\approx 0.0253 + 0.0071 \times 1.0063 \\
  &\approx 0.0325 < \frac{1}{25} = 0.04.
\end{align}
\end{proof}

\subsection{Shift-Robust Trace Cap}

For the compact-by-compact transfer in T5, we need bounds that are uniform over shifts $\tau \in [-K, K]$.

\begin{lemma}[Shift-robust trace cap]\label{lem:shift-trace-cap}
Fix $K > 0$. For any $B > 0$, $t > 0$, and $|\tau| \leq K$, the symmetrized prime sampling operator satisfies
\begin{equation}\label{eq:shift-trace-bound}
  \|T_P[\Phi_{B,t,\tau}]\|_{L^2 \to L^2} \leq \mathrm{tr}\, T_P = 2 \sum_{n \geq 2} \frac{\Lambda(n)}{\sqrt{n}} e^{-4\pi^2 t \, (\log n/(2\pi) - \tau)^2} \leq e^{\pi K} \Big(\rho(t) + 2\pi K \, \sigma(t)\Big),
\end{equation}
where
\begin{equation}\label{eq:sigma-def}
  \rho(t) := 2 \int_0^\infty y \, e^{y/2} e^{-4\pi^2 t \, y^2} \, dy, \qquad \sigma(t) := 2 \int_0^\infty e^{y/2} e^{-4\pi^2 t \, y^2} \, dy \leq \frac{\sqrt{\pi}}{\pi\sqrt{t}} \exp\!\Big(\frac{1}{64\pi^2 t}\Big).
\end{equation}
\end{lemma}

\begin{proof}
Start with $\|T_P\| \leq \mathrm{tr}\, T_P$ since $T_P$ is positive semidefinite and finite rank on compacts.

Bound the sum by an integral of the positive integrand and apply the change $x = e^{y+c}$ with $c = 2\pi\tau$:
\begin{equation}
  \int_1^\infty \frac{\log x}{\sqrt{x}} e^{-4\pi^2 t(\log x - c)^2} \, dx = e^{c/2} \int_0^\infty (y+c) \, e^{y/2} e^{-4\pi^2 t \, y^2} \, dy.
\end{equation}
Splitting gives $e^{c/2}\big(\tfrac{1}{2}\rho(t) + \tfrac{c}{2}\sigma(t)\big)$; doubling for $\pm\xi_n$ and using $|c| \leq 2\pi K$ yields the stated bound.

The estimate for $\sigma(t)$ follows from the closed form for $\int_0^\infty e^{-ay^2+by} \, dy$ with $a = 4\pi^2 t$, $b = \tfrac{1}{2}$, using $1 + \mathrm{erf}(\cdot) \leq 2$.
\end{proof}

\subsection{Existence of Contraction Scale}

\begin{proposition}[Existence of contraction scale]\label{prop:contraction-scale}
For each $K > 0$ there exists $t_K > 0$ such that
\begin{equation}
  \theta_K := e^{\pi K} \Big(\rho(t_K) + 2\pi K \, \sigma(t_K)\Big) < 1.
\end{equation}
In particular, $I - T_P^{\mathrm{sym}}[\Phi_{B,t_K,\tau}] \succeq (1 - \theta_K) I$ uniformly in $B > 0$ and $|\tau| \leq K$.
\end{proposition}

\begin{proof}
As $t \to \infty$, both $\rho(t) \to 0$ and $\sigma(t) \to 0$ (the Gaussian factor dominates). Thus there exists $t_K$ large enough that $\theta_K < 1$.

More precisely, the leading terms are $\rho(t) \sim (4\pi^2 t)^{-1}$ and $\sigma(t) \sim (\pi\sqrt{t})^{-1}$, so we need
\begin{equation}
  e^{\pi K} \Big(\frac{1}{4\pi^2 t} + \frac{2K}{\sqrt{t}}\Big) < 1.
\end{equation}
This is satisfied for $t > C e^{2\pi K}$ with an explicit constant $C$.
\end{proof}

\subsection{Numerical Values}

\begin{center}
\textbf{Prime trace bounds at selected values of $t$}
\end{center}

\begin{center}
\begin{tabular}{ccc}
\toprule
$t$ & $\rho(t)$ upper bound & $\|T_P\|$ cap \\
\midrule
0.5 & $< 0.065$ & $< 0.07$ \\
0.7 & $< 0.042$ & $< 0.045$ \\
1.0 & $< 0.033$ & $< 0.04$ \\
1.5 & $< 0.020$ & $< 0.025$ \\
2.0 & $< 0.014$ & $< 0.02$ \\
\bottomrule
\end{tabular}
\end{center}

\noindent These values confirm that the prime operator norm is well-controlled for moderate values of the heat scale $t$, providing ample room for the archimedean contribution in the Toeplitz bridge.

\subsection{Monotonicity Properties}

\begin{lemma}[Monotonicity of $\rho$]\label{lem:rho-monotone}
The prime trace function $\rho(t)$ is strictly decreasing in $t > 0$, with
\begin{equation}
  \lim_{t \to 0^+} \rho(t) = +\infty, \qquad \lim_{t \to +\infty} \rho(t) = 0.
\end{equation}
\end{lemma}

\begin{proof}
Differentiate under the integral:
\begin{equation}
  \frac{d\rho}{dt} = -8\pi^2 \int_0^\infty y^3 \, e^{y/2} e^{-4\pi^2 t y^2} \, dy < 0.
\end{equation}
The limits follow from dominated convergence.
\end{proof}

This monotonicity is crucial for the T5 transfer: as we move to larger compacts, we can always find a heat scale $t$ that makes the prime contribution arbitrarily small, at the cost of requiring finer Toeplitz grids.


%==============================================================================
% Section 6: Compact Transfer (T5)
%==============================================================================
% Compact-by-Compact Transfer (T5)
% =================================

\section{Compact-by-Compact Transfer (T5)}\label{sec:T5}

This section establishes the mechanism for transferring positivity from finite compacts to the full Weil class. The key insight is that monotone parameter schedules ensure positivity propagates along any increasing chain of compacts.

\begin{remark}[Proof provenance]
The T5 construction and its lemmas are proved in detail in RH\_Q3 (Section 12). We retain the formulations and the monotone schedules here; all dependencies on $c_0$, $L_A$, $t_{\min}$, and $M^\star$ are explicit.
\end{remark}

\subsection{Standing Analytic Inputs}

For each $K > 0$ we assume the analytic data from Sections~\ref{sec:A3} and~\ref{sec:rkhs}:

\begin{itemize}
  \item[\textbf{(A3.a)}] \textbf{Archimedean margin:} $c_0(K) > 0$ such that $\inf_\theta P_A(\theta) \geq c_0(K)$.

  \item[\textbf{(A3.b)}] \textbf{Discretization control:} For all $M \in \mathbb{N}$,
  \begin{equation}
    \|T_M[P_A] - T[P_A]\| \leq C_T \cdot \omega_{P_A}\!\Big(\frac{\pi}{M}\Big),
  \end{equation}
  where $\omega_{P_A}$ is a modulus of continuity.

  \item[\textbf{(RKHS)}] \textbf{Prime contraction:} For all $t \geq t^\star(K)$,
  \begin{equation}
    \|T_P\| \leq \rho(t) \leq \rho(t^\star(K)).
  \end{equation}
\end{itemize}

We also recall the density/continuity interface on $W_K$:
\begin{itemize}
  \item[\textbf{(A1')}] The Fej\'er$\times$heat cone is dense in $W_K$.
  \item[\textbf{(A2)}] $Q$ is continuous on $W_K$: $|Q(\Phi) - Q(\Psi)| \leq L_Q(K) \|\Phi - \Psi\|_\infty$.
\end{itemize}

\subsection{Monotone Schedules}

Define the nondecreasing envelopes (using the explicit formulas of Proposition~\ref{prop:c-arch-explicit} and Corollary~\ref{cor:arch-modulus}):
\begin{equation}
  c_0^\ast(K) := \inf_{0 < u \leq K} c_0(u), \qquad
  L_A^\ast(K) := \sup_{0 < u \leq K} L_A(u),
\end{equation}
and the monotone RKHS schedule based on Corollary~\ref{cor:explicit-tmin}:
\begin{equation}\label{eq:T5-tstar}
  t^\star_{\mathrm{T5}}(K) := \sup_{0 < u \leq K} t_{\min}(u),
  \qquad
  t_{\min}(u) = \frac{\delta_u^2}{4 \ln\!\bigl((2+\eta_u)/\eta_u\bigr)}, \quad
  \eta_u = \frac{\tfrac14 c_0(u) - w_{\max}}{\sqrt{w_{\max}}}.
\end{equation}
Likewise, set the monotone grid size (using $C_T=1$ from Corollary~\ref{cor:Mstar-explicit})
\begin{equation}\label{eq:T5-Mstar}
  M^\star(K) := \left\lceil \sup_{0 < u \leq K} \frac{2\pi\, L_A(u)}{c_0(u)} \right\rceil,
\end{equation}
which is nondecreasing in $K$. By construction $K_1 \leq K_2 \Rightarrow c_0^\ast(K_2) \leq c_0^\ast(K_1)$ and $t^\star_{\mathrm{T5}}(K_2) \geq t^\star_{\mathrm{T5}}(K_1)$, $M^\star(K_2) \geq M^\star(K_1)$.

\subsection{Grid-Lift Inequality}

\begin{lemma}[Grid-lift inequality]\label{lem:T5-grid}
For every $K > 0$ and $M \in \mathbb{N}$:
\begin{equation}
  \lambda_{\min}\!\big(T_M[P_A] - T_P\big) \geq c_0(K) - C_T \cdot \omega_{P_A}\!\Big(\frac{\pi}{M}\Big) - \|T_P\|.
\end{equation}
\end{lemma}

\begin{proof}
Combine the Archimedean lower bound with the Toeplitz continuity estimate and norm subadditivity.
\end{proof}

\subsection{Main Transfer Theorem}

\begin{theorem}[T5: Monotone compact transfer]\label{thm:T5-compact}
For every $K > 0$:
\begin{equation}
  \lambda_{\min}\!\big(T_{M^\star(K)}[P_A] - T_P\big) \geq \tfrac{1}{2} c_0^\ast(K).
\end{equation}
In particular, $Q(\Phi) \geq 0$ on $W_K$ for all $K > 0$. Hence $Q \geq 0$ on $\bigcup_{K>0} W_K$, i.e., on the full Weil class.
\end{theorem}

\begin{proof}
By Lemma~\ref{lem:T5-grid} and the choices~\eqref{eq:T5-tstar}--\eqref{eq:T5-Mstar}:
\begin{equation}
  \lambda_{\min}\!\big(T_{M^\star(K)}[P_A] - T_P\big) \geq c_0^\ast(K) - \tfrac{1}{4} c_0^\ast(K) - \tfrac{1}{4} c_0^\ast(K) = \tfrac{1}{2} c_0^\ast(K).
\end{equation}
Positivity of the finite Toeplitz form on the Fej\'er$\times$heat cone follows. Then (A1')--(A2) extend $Q \geq 0$ from the dense cone to all of $W_K$. Taking the union over $K$ gives the claim.
\end{proof}

\subsection{Inductive Limit Structure}

Let $\mathcal{W}_K = C^+_{\mathrm{even}}([-K, K])$ with the uniform norm, and let $\mathcal{W} = \bigcup_{K>0} \mathcal{W}_K$ carry the inductive limit topology.

\begin{lemma}[Nested dictionaries yield $\mathcal{W}$]\label{lem:T5-dicts}
For each $K > 0$ let $\mathcal{G}_K \subset \mathcal{C}_K$ be a finite dictionary constructed over a shift grid with step $\Delta(K)$ and two heat scales $t_{\min}(K), t_{\max}(K)$. If $K_i \nearrow \infty$ and $\Delta(K_{i+1})$ divides $\Delta(K_i)$ so that $\mathcal{G}_{K_i} \subset \mathcal{G}_{K_{i+1}}$, then
\begin{equation}
  \bigcup_i \overline{\mathrm{cone}(\mathcal{G}_{K_i})}^{\|\cdot\|_\infty} = \bigcup_i \mathcal{W}_{K_i} =: \mathcal{W}.
\end{equation}
\end{lemma}

\begin{proof}
By Theorem~\ref{thm:A1-density}, each $\overline{\mathrm{cone}(\mathcal{G}_{K_i})}$ is dense in $\mathcal{W}_{K_i}$, and nestedness yields the union identity.
\end{proof}

\begin{theorem}[Transfer of positivity to the Weil class]\label{thm:T5-transfer}
Assume $Q \geq 0$ on $\mathcal{W}_{K_i}$ for every $i$, where $Q$ is continuous on each $\mathcal{W}_{K_i}$ (Lemma~\ref{lem:A2-lip}). Then $Q \geq 0$ on $\mathcal{W}$ in the inductive limit topology.
\end{theorem}

\begin{proof}
Given $\Phi \in \mathcal{W}$, choose $i$ with $\mathrm{supp}\,\Phi \subset [-K_i, K_i]$. Then $\Phi \in \mathcal{W}_{K_i}$ and $Q(\Phi) \geq 0$ by hypothesis. Continuity on each $\mathcal{W}_{K_i}$ and Lemma~\ref{lem:T5-dicts} pass the result to the closure and thus to $\mathcal{W}$.
\end{proof}

\subsection{Monotone Inheritance}

\begin{lemma}[Grid-lift by Lipschitz margin]\label{lem:T5-grid-lip}
Let $Q$ be Lipschitz on $W_K$ with constant $L_Q(K)$ (A2). Suppose there exists a uniform grid $\{\tau_j\}$ in $[-K, K]$ of step $\Delta > 0$ such that
\begin{equation}
  \min_j Q(\tau_j) \geq c_0(K) > 0
\end{equation}
and $\Delta \leq c_0(K)/(4 L_Q(K))$. Then $\min_{\tau \in [-K, K]} Q(\tau) \geq \tfrac{1}{2} c_0(K)$.
\end{lemma}

\begin{proof}
Fix $\tau \in [-K, K]$ and let $\tau_\ast$ be the nearest grid point, so $|\tau - \tau_\ast| \leq \Delta/2$. By Lipschitz continuity:
\begin{equation}
  Q(\tau) \geq Q(\tau_\ast) - L_Q(K) |\tau - \tau_\ast| \geq c_0(K) - L_Q(K) \frac{\Delta}{2} \geq c_0(K) - \frac{c_0(K)}{8} \geq \tfrac{1}{2} c_0(K).
\end{equation}
\end{proof}

\begin{lemma}[Monotone inheritance across $K$]\label{lem:T5-inheritance}
Fix an increasing chain $K_0 < K_1 < \cdots$ and choose the monotone schedules $t_{\mathrm{rkhs}}(K_i) := t^\star_{\mathrm{T5}}(K_i)$ and $M_i := M^\star(K_i)$ from~\eqref{eq:T5-tstar}--\eqref{eq:T5-Mstar}. Then
\begin{equation}
  \lambda_{\min}\big(T_{M_i}[P_A] - T_P\big) \geq \tfrac{1}{2} c_0^\ast(K_i) \quad \text{on } \mathcal{W}_{K_i},
\end{equation}
and the property propagates from $K_i$ to $K_{i+1}$.
\end{lemma}

\begin{proof}
Lemma~\ref{lem:T5-grid} with $M_i = M^\star(K_i)$ and $t = t^\star_{\mathrm{T5}}(K_i)$ gives the lower bound. Since $K \mapsto c_0^\ast(K)$ is decreasing and $K \mapsto t^\star_{\mathrm{T5}}(K), M^\star(K)$ are nondecreasing, the same estimate applies at $K_{i+1}$, so the chain inherits positivity.
\end{proof}

\subsection{Summary}

The compact-by-compact transfer provides a mechanism to extend positivity from finite windows to the full Weil class:

\begin{enumerate}
  \item \textbf{Local positivity:} On each $W_K$, the spectral bound $\lambda_{\min}(T_M[P_A] - T_P) \geq \tfrac{1}{2} c_0^\ast(K) > 0$ holds.

  \item \textbf{Monotone schedules:} Parameters $(M^\star(K), t^\star(K))$ are chosen monotonically, ensuring consistency across compacts.

  \item \textbf{Inheritance:} Positivity on $W_K$ implies positivity on $W_{K'}$ for all $K' \geq K$.

  \item \textbf{Union:} Taking $K \to \infty$ gives $Q \geq 0$ on the full Weil class $\mathcal{W}$.
\end{enumerate}


%==============================================================================
% Section 6b: IND/AB Inductive Closure (archival, not used in main proof)
%==============================================================================
% IND/AB: Inductive Closure and Parameter Schedules (archival, not used in main proof)
% ===================================================================================

\section{Inductive Closure (IND/AB)}\label{sec:IND-AB}

This archival section records the IND/AB route to prime control. It is kept for reference but is \emph{not} used in the main proof (which uses the RKHS contraction).

\subsection{The AB-Infinity Framework}

The AB-infinity (``Activity Block to Infinity'') framework provides a systematic way to verify positivity across an increasing chain of compacts.

\begin{definition}[Activity block chain]\label{def:AB-chain}
An \emph{activity block chain} is a sequence of compacts $\{K_i\}_{i \geq 1}$ with:
\begin{enumerate}
  \item[\textup{(i)}] $K_1 < K_2 < K_3 < \cdots$, strictly increasing.
  \item[\textup{(ii)}] $\bigcup_{i \geq 1} [-K_i, K_i] = \mathbb{R}$.
  \item[\textup{(iii)}] For each $i$, certified parameters $(B_i, t_{\mathrm{sym}, i}, M_i)$ are specified.
\end{enumerate}
\end{definition}

\begin{theorem}[AB$\infty$ closure]\label{thm:ABinfty}
Work under the T0 normalization $Q = Q_{\mathrm{GW}}$ on the Guinand--Weil axis. Fix global constants $q_0 = 30$, $t_* > 0$, and $t_0 > 0$. Let $\{K_i\}_{i \geq 1}$ be an activity block chain.

For each $i$, choose parameters $(B_i, t_{\mathrm{sym}, i}, M_i)$ with $t_{\mathrm{sym}, i} \geq t_*$ and a shift grid $E_{K_i} \subset [-K_i, K_i]$.

Assume for every $i$:

\begin{description}
  \item[\textbf{(R) Arch floor (A3):}] With the Fej\'er$\times$heat window $\Phi_{B_i, t_{\mathrm{sym}, i}, \tau}$ and archimedean symbol $P_A(\cdot; \tau)$:
  \begin{equation}\label{eq:ABinfty-floor}
    \min_\theta P_A(\theta; \tau) \geq c_0(K_i) \quad \text{for all } \tau \in E_{K_i}.
  \end{equation}

  \item[\textbf{(N) Nyquist \& Norm:}] The symbol modulus and prime cap satisfy:
  \begin{equation}\label{eq:ABinfty-nyquist}
    C \cdot \omega_{P_A}\!\left(\frac{\pi}{M_i}\right) \leq \frac{c_0(K_i)}{2}, \qquad
    \|T_P^{(q_0)}(t_0)\| \leq \frac{c_0(K_i)}{2},
  \end{equation}
  where $T_P^{(q_0)}(t_0)$ is the modular cap at modulus $q_0 = 30$ with RKHS smoothing scale $t_{\mathrm{rkhs}} \geq t_0$.

  \item[\textbf{(A) Grid$\to$Continuum (BRC--SAFE):}] Every interval $[\tau_j, \tau_{j+1}]$ in the grid is BRC--SAFE; equivalently, the resolvent certificate with Ky Fan/Hoffman--Wielandt budget holds on each such interval.
\end{description}

Then $Q(\Phi) \geq 0$ for all even Paley--Wiener tests $\Phi$ on $[-K_i, K_i]$ for every $i$. Consequently, $Q \geq 0$ on the full Weil class; by Weil's positivity criterion, RH follows.
\end{theorem}

\begin{proof}[Proof (by plumbing)]
By the Toeplitz symbol bridge (A3), for every grid node $\tau \in E_{K_i}$:
\begin{equation}
  \lambda_{\min}\!\big(T_{M_i}[P_A(\cdot; \tau)] - T_P\big) \geq \min P_A(\cdot; \tau) - C \cdot \omega_{P_A}\!\left(\frac{\pi}{M_i}\right) - \|T_P\|.
\end{equation}
Assumptions \textbf{(R)}--\textbf{(N)} make the RHS $\geq c_0 - \tfrac{c_0}{2} - \tfrac{c_0}{2} = 0$, so nonnegativity holds on all grid nodes.

By \textbf{(A)} (BRC--SAFE on each interval), the sign is preserved on $[-K_i, K_i]$.

Fej\'er$\times$heat density (A1$'$) and Lipschitz continuity (A2) lift nonnegativity from the grid cone to all even PW tests on $[-K_i, K_i]$.

Finally, along the chain $\{K_i\}$, the T5 compact limit transfers $Q \geq 0$ to the Weil class.
\end{proof}

\subsection{One-Interval Induction Step}

The inductive step handles the crossing of each threshold in the activity block chain.

\begin{theorem}[IND/AB step]\label{thm:IND-step}
On an activity interval $[B_n, B_{n+1})$, let $\|T_P^{\mathrm{old}}\|_{\mathcal{H}_K} \leq \rho_K^{\mathrm{old}} < 1$. When crossing the threshold $B_{n+1}$, a single new node $\alpha_{\mathrm{new}}$ with weight $w_{\mathrm{new}}$ enters. In the RKHS normalization $\|k_\alpha\| = 1$, one has:
\begin{equation}\label{eq:IND-norm-update}
  \|T_P^{\mathrm{new}}\| \leq \rho_K^{\mathrm{old}} + w_{\mathrm{new}}.
\end{equation}
Hence, if $\rho_K^{\mathrm{old}} + w_{\mathrm{new}} < 1$, then $T_A - T_P^{\mathrm{new}} \succeq 0$ on $\mathcal{H}_K$.
\end{theorem}

\begin{proof}
The rank-one update formula:
\begin{equation}
  T_P^{\mathrm{new}} = T_P^{\mathrm{old}} + w_{\mathrm{new}} |k_{\alpha_{\mathrm{new}}}\rangle \langle k_{\alpha_{\mathrm{new}}}|
\end{equation}
with $\|k_\alpha\| = 1$ gives the claimed norm bound by the triangle inequality. Strict inequality $\rho_K^{\mathrm{old}} + w_{\mathrm{new}} < 1$ implies the Loewner positivity $T_A - T_P^{\mathrm{new}} \succeq 0$.
\end{proof}

\begin{corollary}[Gluing intervals]\label{cor:IND-glue}
Suppose the base case holds on $[B_3, B_4)$, and across each threshold $B_n \to B_{n+1}$, the one-prime condition $\rho_K^{\mathrm{old}} + w_{\mathrm{new}} < 1$ is verified in the RKHS normalization on $[-K, K]$. Then $T_A - T_P \succeq 0$ holds on $[-K, K]$ for all $B \geq B_3$, i.e., the measure domination persists interval-by-interval.
\end{corollary}

\subsection{Early Block Control}

\begin{lemma}[Analytic bound for early blocks]\label{lem:IND-early}
Let $\Phi_{B, t}(\xi) = (1 - |\xi|/B)_+ e^{-4\pi^2 t \xi^2}$ with $B > 0$. For the even setting with weights $w(n) = \Lambda(n)/\sqrt{n}$ and nodes $\alpha_n = \log n/(2\pi)$:
\begin{equation}\label{eq:IND-early-bound}
  \sum_{\alpha_n \in [-B, B]} w(n) \Phi_{B, t}(\alpha_n) \leq \sum_{n \leq e^{2\pi B}} \frac{\Lambda(n)}{\sqrt{n}} \leq 2 e^{\pi B}(2\pi B - 2) + 4.
\end{equation}
In particular, choosing $B = B(K) > 0$ small enough forces the early-block mass to lie below any prescribed budget $\varepsilon(K) > 0$.
\end{lemma}

\begin{proof}
Since $0 \leq \Phi_{B, t} \leq 1$ and $\Phi_{B, t}$ vanishes outside $[-B, B]$, the first inequality holds. For the second, use $\Lambda(n) \leq \log n$ and compare the sum to the integral:
\begin{equation}
  \sum_{n \leq e^{2\pi B}} \frac{\Lambda(n)}{\sqrt{n}} \leq \int_1^{e^{2\pi B}} \frac{\log u}{\sqrt{u}} \, du = 2 e^{\pi B}(2\pi B - 2) + 4,
\end{equation}
where the evaluation follows by the substitution $u = v^2$.
\end{proof}

\subsection{Parameter Recipe}

The IND/AB framework requires careful choice of acceptance parameters.

\begin{definition}[Plateau schedule]\label{def:plateau-schedule}
Define the acceptance function:
\begin{equation}\label{eq:plateau-schedule}
  \mathrm{Plateau}(t; \alpha, \beta, \tau, \gamma) = \begin{cases}
    \alpha t, & t \leq \tau, \\
    \gamma, & \tau < t \leq \tau + \beta, \\
    \max\{\gamma - \alpha(t - \tau - \beta), 0\}, & t > \tau + \beta,
  \end{cases}
\end{equation}
where:
\begin{center}
\begin{tabular}{llp{0.4\textwidth}}
\toprule
Name & Symbol & Role \\
\midrule
Pre-plateau slope & $\alpha$ & Growth rate before plateau \\
Plateau width & $\beta$ & Length of flat segment \\
Onset shift & $\tau$ & Position of plateau window \\
Saturation level & $\gamma$ & Upper acceptance bound \\
\bottomrule
\end{tabular}
\end{center}
\end{definition}

\begin{lemma}[Plateau schedule admissibility]\label{lem:plateau-admissible}
Let $A(t) = \mathrm{Plateau}(t; \alpha, \beta, \tau, \gamma)$ with $0 < \alpha \leq \gamma \leq 1$ and $\beta > 0$. Then $A$ takes values in $[0, 1]$, is piecewise Lipschitz, and meets the IND/AB plateau constraints: monotonic rise before $\tau$, a flat segment of width $\beta$, and compatible one-sided derivatives at the junctions.
\end{lemma}

\begin{proof}[Proof sketch]
Formula~\eqref{eq:plateau-schedule} consists of three segments with slopes $\alpha$, $0$, and $-\alpha$. Continuity follows from matching the constants at the junction points; the corner points are controlled by the one-sided bounds. The values stay below $\gamma \leq 1$, satisfying the normalized AB regime.
\end{proof}

\subsection{Modular Cap Control}

\begin{definition}[Modular cap]\label{def:modular-cap}
For modulus $q$ and RKHS smoothing scale $t$, define the modular prime cap:
\begin{equation}
  T_P^{(q)}(t) := \sum_{\substack{\alpha_n \in [-K, K] \\ \gcd(n, q) = 1}} w(n) e^{-t(\xi_n - \xi_m)^2} |k_{\xi_n}\rangle \langle k_{\xi_m}|.
\end{equation}
\end{definition}

\begin{lemma}[Fixed modular cap]\label{lem:fixed-mod-cap}
The modular cap at $q_0 = 30$ satisfies:
\begin{equation}
  \|T_P^{(30)}(t_0)\| \leq \rho_{30}(t_0)
\end{equation}
for any $t_0 > 0$. The bound $\rho_{30}(t_0)$ is uniform across all compacts $K$ once the early block and tail contributions are controlled.
\end{lemma}

\begin{remark}[Choice of modulus]
The modulus $q_0 = 30 = 2 \cdot 3 \cdot 5$ is chosen to balance several considerations:
\begin{enumerate}
  \item It captures the first three primes, providing good coverage of small prime residues.
  \item The Euler totient $\phi(30) = 8$ keeps the number of residue classes manageable.
  \item It is large enough to provide meaningful cap reduction but small enough for efficient computation.
\end{enumerate}
\end{remark}

\subsection{BRC--SAFE Grid Lift}

\begin{definition}[BRC--SAFE interval]\label{def:BRC-SAFE}
An interval $[\tau_j, \tau_{j+1}]$ in the shift grid is \emph{BRC--SAFE} (Bounded Resolvent Certificate -- Safety And Feasibility Ensured) if:
\begin{enumerate}
  \item[\textup{(i)}] The resolvent $(T_M[P_A(\cdot; \tau)] - T_P - zI)^{-1}$ exists and is bounded for all $z$ in a neighborhood of $(-\infty, 0]$.
  \item[\textup{(ii)}] The Ky Fan/Hoffman--Wielandt perturbation bounds hold with uniform constants across the interval.
  \item[\textup{(iii)}] The eigenvalue variation $|\lambda_{\min}(\tau) - \lambda_{\min}(\tau')| \leq L_\lambda |\tau - \tau'|$ is Lipschitz controlled.
\end{enumerate}
\end{definition}

\begin{lemma}[Grid-to-continuum lift]\label{lem:grid-continuum-lift}
If every interval $[\tau_j, \tau_{j+1}]$ in the grid $E_K$ is BRC--SAFE, and $\lambda_{\min}(T_M[P_A] - T_P) \geq 0$ at all grid nodes, then $\lambda_{\min}(T_M[P_A(\cdot; \tau)] - T_P) \geq 0$ for all $\tau \in [-K, K]$.
\end{lemma}

\begin{proof}
By BRC--SAFE condition (iii), the minimum eigenvalue is a Lipschitz function of $\tau$. If it is nonnegative at both endpoints of an interval, and the Lipschitz constant is controlled, the intermediate value theorem ensures nonnegativity throughout the interval. Applying this to each interval in the grid and using the covering property yields the global result.
\end{proof}

\begin{remark}[Lipschitz-lift alternative]
Instead of BRC--SAFE, one may enforce the deterministic Lipschitz lift condition:
\begin{equation}
  L_Q(K_i) \cdot L_\Phi(K_i) \cdot \Delta\tau \leq \frac{c_0(K_i)}{4},
\end{equation}
where $\Delta\tau$ is the grid spacing. The conclusion is the same, but BRC--SAFE provides tighter control in practice.
\end{remark}

\subsection{Certified Parameter Tables}

The following table records the certified margins for the induction steps:

\begin{center}
\begin{tabular}{ccccc}
\toprule
$K$ & $c_0(K)$ & $\rho_K^{\mathrm{old}}$ & $w_{\mathrm{new}}$ & Margin \\
\midrule
1.0 & 0.179 & 0.181 & 0.023 & $\checkmark$ \\
1.5 & 0.173 & 0.167 & 0.018 & $\checkmark$ \\
2.0 & 0.167 & 0.159 & 0.014 & $\checkmark$ \\
2.5 & 0.161 & 0.153 & 0.011 & $\checkmark$ \\
\bottomrule
\end{tabular}
\end{center}

For $K = 1$, the greedy block consumes $0.181 < c_0/4$, and the follow-up step verifies $\rho_K^{\mathrm{old}} + w_{\mathrm{new}} < 1$. For larger $K$, the margin $c_0(K) - \rho(t)$ stays above $0.67$, so the one-prime update is comfortably within budget.

\subsection{Summary and Remarks}

\begin{remark}[Monotone inheritance]
It is convenient (though not essential) to choose $B_i \uparrow$, $M_i \uparrow$, and nonincreasing budgets so that acceptance persists along the chain without recomputation.
\end{remark}

\begin{remark}[Grid-to-continuum-to-Weil transfer]
By A1$'$ (Theorem~\ref{thm:A1-density}), the Fej\'er$\times$heat cone is dense in $W_K$ under $\|\cdot\|_\infty$; by A2 (Lemma~\ref{lem:A2-lip}), $Q$ is Lipschitz on $W_K$. Hence grid positivity and the SAFE/Lipschitz lift imply $Q \geq 0$ on all of $W_K$. With the monotone parameter schedule (Lemma~\ref{lem:T5-inheritance}), Theorem~\ref{thm:T5-transfer} transfers positivity to the Weil class.
\end{remark}

The IND/AB framework provides the final piece of the positivity verification: a systematic inductive mechanism that extends from finite compacts to the full Weil class, completing the chain $(T0) + (A1') + (A2) + (A3) + (\mathrm{RKHS}) + (T5) + (\mathrm{IND/AB}) \Rightarrow Q \geq 0 \Rightarrow \mathrm{RH}$.


%==============================================================================
% Section 7: GRH Extension
%==============================================================================
% GRH Extension: Dirichlet L-functions
% =====================================

\section{Extension to the Generalized Riemann Hypothesis}\label{sec:GRH}

\begin{remark}[Proof provenance]
The GRH module mirrors the $\zeta$-case; full proofs are given in RH\_Q3 (Section 16). The schedules $t_{\min}^{(\chi)}(K)$ and $M_{\min}^{(\chi)}(K)$ coincide with the untwisted case because $c_0(K)$ and $w_{\max}$ are unchanged.
\end{remark}

\subsection{Dirichlet Characters}

\begin{definition}[Character $\chifour$ mod 4]\label{def:chi4}
The non-principal Dirichlet character modulo 4 is:
\begin{equation}
  \chifour(n) = \begin{cases}
    1  & \text{if } n \equiv 1 \pmod{4}, \\
    -1 & \text{if } n \equiv 3 \pmod{4}, \\
    0  & \text{if } n \equiv 0, 2 \pmod{4}.
  \end{cases}
\end{equation}
\end{definition}

\begin{remark}[Connection to twin primes]
For any twin prime pair $(p, p+2)$ with $p > 3$:
\begin{equation}
  \chifour(p) \times \chifour(p+2) = -1.
\end{equation}
This follows since $p \equiv 1 \pmod{4}$ implies $p+2 \equiv 3 \pmod{4}$, and vice versa.
\end{remark}

\subsection{The Modified Hamiltonian}

\begin{definition}[$\chi$-twisted Hamiltonian]\label{def:H-chi}
For a Dirichlet character $\chi$ mod $q$, define:
\begin{equation}
  \Hchi = \TA - \TP^\chi,
\end{equation}
where the prime operator carries character weights:
\begin{equation}
  \TP^\chi = \sum_p \chi(p)\, w(p)\, \phi_p\, v_p v_p^T.
\end{equation}
\end{definition}

\begin{theorem}[GRH criterion]\label{thm:GRH-criterion}
The Generalized Riemann Hypothesis for $L(s, \chi)$ is equivalent to:
\begin{equation}
  \Hchi \geq 0 \quad (\text{all eigenvalues } \geq 0).
\end{equation}
This is the Dirichlet-\!$L$ analogue of the Weil positivity criterion (Theorem~\ref{thm:Weil-criterion}).
\end{theorem}

\subsection{Explicit parameters for $\chi_4$}\label{subsec:grh-parameters}

The Archimedean symbol and its bounds are unchanged by twisting with $\chi_4$, so $c_0(K)$ and $L_A(K)$ are as in Proposition~\ref{prop:c-arch-explicit} and Corollary~\ref{cor:arch-modulus}. The prime weights are signed by $\chi_4$, but satisfy the same uniform bound $w_{\max} \le 2/e$. Therefore the RKHS contraction schedule from Corollary~\ref{cor:explicit-tmin} applies verbatim:
\[
  t_{\min}^{(\chi_4)}(K) = \frac{\delta_K^2}{4 \ln\!\bigl((2+\eta_K)/\eta_K\bigr)}, \qquad
  \eta_K = \frac{\tfrac14 c_0(K) - w_{\max}}{\sqrt{w_{\max}}}, \quad \delta_K \ge \frac{1}{2\pi(e^{2\pi K}+1)}.
\]
With the same Toeplitz grid size
\[
  M_{\min}^{(\chi_4)}(K) = \left\lceil \frac{2\pi\, L_A(K)}{c_0(K)} \right\rceil,
\]
we obtain
\[
  \lambda_{\min}\big(T_{M_{\min}^{(\chi_4)}(K)}[P_A] - T_P^{\chi_4}\big) \ge \tfrac12 c_0(K),
\]
and consequently $H_{\chi_4} \ge 0$ on each compact $[-K,K]$. The monotone schedules
\[
  t^\star_{\mathrm{T5},\chi_4}(K) = \sup_{0<u\le K} t_{\min}^{(\chi_4)}(u), \qquad
  M^\star_{\chi_4}(K) = \left\lceil \sup_{0<u\le K} \frac{2\pi\, L_A(u)}{c_0(u)} \right\rceil
\]
are nondecreasing and feed directly into the T5 transfer (Theorem~\ref{thm:T5-transfer}), giving $Q_{\chi_4} \ge 0$ on the Weil cone and hence GRH for $\chi_4$.

\subsection{Residue Class Decomposition}

\begin{definition}[Class Hamiltonians]\label{def:H-pm}
Define operators isolating each residue class:
\begin{align}
  \Hplus  &= \TA - \TP^+ \quad (\text{only } p \equiv 1 \pmod{4}), \\
  \Hminus &= \TA - \TP^- \quad (\text{only } p \equiv 3 \pmod{4}).
\end{align}
\end{definition}

\begin{lemma}[Alternative formulation]\label{lem:H-pm-alternative}
The class Hamiltonians can also be expressed as:
\begin{equation}
  \Hplus = \frac{\Hzeta + \Hchi}{2}, \qquad
  \Hminus = \frac{\Hzeta - \Hchi}{2}.
\end{equation}
\end{lemma}

\begin{proof}
For $p \equiv 1 \pmod{4}$: $\chifour(p) = 1$, so $(1 + \chifour(p))/2 = 1$ and $(1 - \chifour(p))/2 = 0$.
For $p \equiv 3 \pmod{4}$: $\chifour(p) = -1$, so $(1 + \chifour(p))/2 = 0$ and $(1 - \chifour(p))/2 = 1$.
\end{proof}

\begin{corollary}[Joint criterion]\label{cor:joint-criterion}
\begin{equation}
  \text{RH} + \text{GRH}(\chifour) \quad \Longleftrightarrow \quad \Hplus \geq 0 \text{ and } \Hminus \geq 0.
\end{equation}
\end{corollary}

\subsection{Two-Particle Operators for Twins}

\begin{definition}[Two-particle Hilbert space]\label{def:two-particle}
For twin prime analysis, consider the tensor product:
\begin{equation}
  \mathcal{H}^{(2)} = \mathcal{H}_+ \otimes \mathcal{H}_-,
\end{equation}
with the free two-particle Hamiltonian:
\begin{equation}
  A^{(2)} = \Hplus \otimes I + I \otimes \Hminus.
\end{equation}
\end{definition}

\begin{definition}[Twin interaction operator]\label{def:V-twins}
\begin{equation}
  V_{\mathrm{twins}} = \sum_{(p, p+2) \text{ twin}} w_p\, w_{p+2}\, \phi_p\, \phi_{p+2}\,
  (v_p \otimes v_{p+2})(v_p \otimes v_{p+2})^T.
\end{equation}
\end{definition}

\begin{lemma}[Positivity of $V_{\mathrm{twins}}$]\label{lem:V-twins-positive}
$V_{\mathrm{twins}} \geq 0$ (positive semidefinite).
\end{lemma}

\begin{proof}
$V_{\mathrm{twins}}$ is a sum of rank-one projectors with positive weights.
\end{proof}


%==============================================================================
% Section 8: Twin Prime Coherence
%==============================================================================
% Twin Prime Coherence Analysis
% =============================

\section{Twin Prime Coherence Phenomenon}\label{sec:twin-coherence}

\subsection{Anti-diagonal Fourier Modes}

Twin prime vectors exhibit constructive interference along specific Fourier modes.

\begin{definition}[Anti-diagonal modes]\label{def:antidiag-modes}
In the complex Fourier basis $\{e^{ik\xi}\}$, the anti-diagonal modes are those with:
\begin{equation}
  k_1 + k_2 = 0, \quad \text{i.e., } k_2 = -k_1.
\end{equation}
\end{definition}

\begin{lemma}[Phase difference for twins]\label{lem:phase-diff}
For a twin pair $(p, p+2)$ with log-coordinates $\xi_p = \log p/(2\pi)$ and $\xi_{p+2} = \log(p+2)/(2\pi)$:
\begin{equation}
  \delta := \xi_{p+2} - \xi_p = \frac{\log(1 + 2/p)}{2\pi}.
\end{equation}
As $p \to \infty$, we have $\delta \to 0$ with $\delta \approx 1/(\pi p)$.
\end{lemma}

\begin{theorem}[Anti-diagonal coherence]\label{thm:antidiag-coherence}
For the anti-diagonal mode $(k, -k)$, the combined phase contribution from a twin pair is:
\begin{equation}
  e^{2\pi i k \xi_p} \cdot e^{-2\pi i k \xi_{p+2}} = e^{-2\pi i k \delta}.
\end{equation}
Since $\delta \to 0$ as $p \to \infty$, all large twins contribute with phase $\approx 0$.
\end{theorem}

\begin{proof}
Direct calculation:
\begin{equation}
  e^{2\pi i k (\xi_p - \xi_{p+2})} = e^{-2\pi i k \delta} = e^{-i k \log(1 + 2/p)} \to 1
\end{equation}
as $p \to \infty$.
\end{proof}

\begin{corollary}[Phase deviation bound]\label{cor:phase-deviation}
For mode $k$ and twin pair $(p, p+2)$, the phase deviation from 0 is:
\begin{equation}
  |\arg(e^{-2\pi i k \delta})| = 2\pi |k| \delta \approx \frac{2|k|}{p}.
\end{equation}
For $k = 1$ and $p \geq 100$, this is under $1.15^{\circ}$.
\end{corollary}

\subsection{Coherent vs Incoherent Scaling}

\begin{definition}[Coherence measures]\label{def:coherence-measures}
For $N$ twin pairs with phases $\phi_j$:
\begin{align}
  S_{\mathrm{coh}} &:= \left| \sum_{j=1}^N e^{i\phi_j} \right| \quad \text{(coherent sum)}, \\
  S_{\mathrm{inc}} &:= N \quad \text{(incoherent/random walk baseline)}.
\end{align}
\end{definition}

\begin{theorem}[Numerical scaling law]\label{thm:scaling}
This statement summarizes \emph{empirical log--log fits} from the code (complex Fourier basis, weights $w(p)=\log p/\sqrt{p}$) on the range $N \leq 1.4\times10^3$ twin pairs. It is not a rigorous asymptotic claim.
\begin{itemize}
  \item Coherent sum: $S_{\mathrm{coh}} \approx N^{0.96\text{--}1.01}$ on anti-diagonal modes.
  \item Baseline (diagonal / incoherent): $S_{\mathrm{inc}} \approx N^{0.10\text{--}0.20}$.
  \item Gain $G := S_{\mathrm{coh}} / S_{\mathrm{inc}}^{1/2} \approx N^{0.70\text{--}0.80}$.
\end{itemize}
\end{theorem}

\begin{remark}[Weight and window dependence]
The exponents vary with the weighting scheme and the compact window $K$; we do not attempt to identify a canonical value (e.g., $3/4$). Appendix tables and code logs record the precise fit ranges used.
\end{remark}

\subsection{Antiferromagnetic Order}

\begin{theorem}[Antiferromagnetic structure]\label{thm:antiferro}
Every twin pair $(p, p+2)$ with $p > 3$ satisfies:
\begin{equation}
  \chifour(p) \times \chifour(p+2) = -1.
\end{equation}
This defines an antiferromagnetic ordering in residue-class space.
\end{theorem}

\begin{proof}
For $p > 3$ prime, $p$ is odd. If $p \equiv 1 \pmod{4}$, then $p + 2 \equiv 3 \pmod{4}$, giving $\chifour(p) = 1$ and $\chifour(p+2) = -1$. The converse case is symmetric.
\end{proof}

\begin{remark}[Physical interpretation]
The antiferromagnetic order means twin primes always connect opposite residue classes (class 1 to class 3). This structural constraint underlies the coherence in the anti-diagonal Fourier basis, where the $\pm$ sectors couple constructively.
\end{remark}


%==============================================================================
% Section 9: Coherence-Counting Bridge (FORMAL THEOREMS)
%==============================================================================
% Coherence Bridge: Formal Theorems on Twin Primes
% =================================================

\section{The Coherence-Counting Bridge}\label{sec:bridge}

This section presents \textbf{rigorous, provable} results connecting coherent sums to twin prime counting. These are not heuristics---they follow from elementary linear algebra.

\subsection{Definitions}

\begin{definition}[Twin prime set]\label{def:twin-set}
Let $T(X)$ denote the set of primes $p \leq X$ such that $(p, p+2)$ is a twin prime pair:
\begin{equation}
  T(X) := \{p \leq X : p \text{ prime}, p+2 \text{ prime}\}.
\end{equation}
\end{definition}

\begin{definition}[Coherent sum]\label{def:coherent-sum}
For weights $w(p) = \log p / \sqrt{p}$ and phases $\phi_p = \log p / (2\pi)$, define:
\begin{equation}
  Z(X) := \sum_{p \in T(X)} w(p)\, e^{2\pi i \phi_p}
       = \sum_{p \le X} \frac{\log p}{\sqrt{p}}\, e^{i \log p} \cdot \mathbf{1}_{p, p+2 \text{ prime}}.
\end{equation}
It will also be convenient to use the untapered version
\begin{equation}\label{eq:Z-full}
  Z_{\mathrm{full}}(X; s) := \sum_{p \le X} (\log p)\, p^{-s}, \qquad s_0 := \tfrac12 - i,
\end{equation}
so that $Z_{\mathrm{full}}(X; s_0) = \sum_{p \le X} (\log p) p^{-1/2+i}$ and $Z(X)$ is its restriction to twin primes.
\end{definition}

\begin{definition}[Weighted second moment]\label{def:W2}
\begin{equation}
  W_2(X) := \sum_{p \in T(X)} w(p)^2 = \sum_{p \in T(X)} \frac{(\log p)^2}{p}.
\end{equation}
\end{definition}

\subsection{Bridge to the logarithmic derivative of $\zeta$}

For $\Re s>1$ one has
\[
  -\frac{\zeta'}{\zeta}(s) = \sum_{n\ge1} \frac{\Lambda(n)}{n^s}
  = \sum_p \sum_{k\ge1} \frac{\log p}{p^{ks}}.
\]
Splitting off higher prime powers,
\[
  Z_{\mathrm{full}}(X; s) = \sum_{p \le X} \frac{\log p}{p^s}
  = \sum_{n \le X} \frac{\Lambda(n)}{n^s} - R_{\mathrm{pp}}(s,X),
\]
where $R_{\mathrm{pp}}(s,X) := \sum_{p^k \le X,\,k\ge2} (\log p) p^{-ks}$. Thus
\begin{equation}\label{eq:Z-vs-Lambda}
  Z_{\mathrm{full}}(X; s) = -\frac{\zeta'}{\zeta}(s) - R_{\mathrm{tail}}(s,X) - R_{\mathrm{pp}}(s,X),
  \qquad
  R_{\mathrm{tail}}(s,X):=\sum_{n>X}\frac{\Lambda(n)}{n^s}.
\end{equation}

\subsection{Explicit formula for $Z_{\mathrm{full}}$}

Applying Perron's formula and shifting the contour yields, for $c>1-\Re s$,
\[
  Z_{\mathrm{full}}(X; s)
  = -\frac{\zeta'}{\zeta}(s)
    - \frac{X^{1-s}}{1-s}
    + \sum_{\rho} \frac{X^{\rho-s}}{\rho-s}
    + \mathcal{E}(s,X) - R_{\mathrm{pp}}(s,X),
\]
where the sum is over nontrivial zeros $\rho$ of $\zeta$ and $\mathcal{E}(s,X)$ contains the (small) contributions of trivial zeros and the shifted contour. In particular, at $s_0 = \tfrac12 - i$,
\begin{equation}\label{eq:Z-explicit}
  Z_{\mathrm{full}}(X; s_0)
  = -\frac{\zeta'}{\zeta}(s_0)
    - \frac{X^{1-s_0}}{1-s_0}
    + \sum_{\rho} \frac{X^{\rho-s_0}}{\rho-s_0}
    + \bigl(\mathcal{E}(s_0,X) - R_{\mathrm{pp}}(s_0,X)\bigr).
\end{equation}

\subsection{Zeros-only quadratic functional (Weil prototype)}

Let $X = e^t$ and assume RH so $\rho = \tfrac12 + i\gamma$. Then
\[
  Z_{\mathrm{full}}(e^t; s_0) + \frac{\zeta'}{\zeta}(s_0) - \frac{e^{(1-s_0)t}}{1-s_0}
  \approx - \sum_{\gamma} \frac{e^{i(\gamma+1)t}}{i(\gamma+1)}.
\]
For a Schwartz, even weight $w$ with $\widehat w \ge 0$, define
\[
  \mathcal{I}[w] := \int_{\RR} w(t)\, \bigl| Z_{\mathrm{full}}(e^t; s_0) \bigr|^2\, dt.
\]
Formally inserting the zeros expansion (and ignoring the small error terms) gives the zeros-only positive form
\begin{equation}\label{eq:Weil-proto}
  \mathcal{W}[w] := \sum_{\rho,\rho'} \frac{\widehat w(\gamma-\gamma')}{(\rho-s_0)(\overline{\rho'}-\overline{s_0})},
  \qquad (\rho-s_0)(\overline{\rho'}-\overline{s_0}) = (\gamma+1)(\gamma'+1)\ \text{on RH},
\end{equation}
which is positive semidefinite when $\widehat w \ge 0$. Thus $\mathcal{W}[w]$ measures the same “energy” as $\int w |Z|^2$ but lives entirely on the zero set.

\subsection{The Cauchy-Schwarz Bridge}

\begin{lemma}[Cauchy-Schwarz bridge]\label{lem:CS-bridge}
For all $X > 0$:
\begin{equation}\label{eq:CS-bound}
  |T(X)| \geq \frac{|Z(X)|^2}{W_2(X)}.
\end{equation}
\end{lemma}

\begin{proof}
Consider the vector $\mathbf{v}(X) = \bigl(w(p)\, e^{2\pi i \phi_p}\bigr)_{p \in T(X)} \in \mathbb{C}^{|T(X)|}$.

Its squared norm is:
\begin{equation}
  \|\mathbf{v}(X)\|^2 = \sum_{p \in T(X)} |w(p)\, e^{2\pi i \phi_p}|^2 = \sum_{p \in T(X)} w(p)^2 = W_2(X).
\end{equation}

The inner product with $\mathbf{1} = (1, \ldots, 1)$ is:
\begin{equation}
  \langle \mathbf{v}(X), \mathbf{1} \rangle = \sum_{p \in T(X)} w(p)\, e^{2\pi i \phi_p} = Z(X).
\end{equation}

By Cauchy-Schwarz:
\begin{equation}
  |Z(X)|^2 = |\langle \mathbf{v}(X), \mathbf{1} \rangle|^2
  \leq \|\mathbf{v}(X)\|^2 \cdot \|\mathbf{1}\|^2
  = W_2(X) \cdot |T(X)|.
\end{equation}

Rearranging gives \eqref{eq:CS-bound}.
\end{proof}

\subsection{The Conditional Infinitude Theorem}

\begin{theorem}[Coherence implies infinitude]\label{thm:coherence-infinitude}
Assume there exist constants $C>0$, $\varepsilon>0$ and a polylogarithmic bound $G(X)=O(\mathrm{polylog}(X))$ such that for all sufficiently large $X$:
\begin{enumerate}
  \item[(i)] $|Z(X)| \geq C \cdot X^{\varepsilon}$ \quad (coherent sum grows at a power rate),
  \item[(ii)] $W_2(X) \leq G(X)$ \quad (weighted second moment stays subpolynomial).
\end{enumerate}
Then for large $X$:
\begin{equation}
  |T(X)| \geq \frac{C^2}{G(X)} X^{2\varepsilon} \to \infty,
\end{equation}
and consequently, \textbf{there are infinitely many twin primes}.
\end{theorem}

\begin{proof}
By \cref{lem:CS-bridge},
\[
  |T(X)| \geq \frac{|Z(X)|^2}{W_2(X)} \geq \frac{C^2 X^{2\varepsilon}}{G(X)}.
\]
The denominator is polylogarithmic while the numerator is a power of $X$, so the ratio diverges, proving infinitude of twin primes under the stated hypotheses.
\end{proof}

\begin{remark}[What this theorem does NOT prove]
\cref{thm:coherence-infinitude} is a \textbf{conditional} result. It reduces the Twin Prime Conjecture to:
\begin{center}
\emph{Prove that $|Z(X)|$ grows sufficiently fast.}
\end{center}
No lower bound for $|Z(X)|$ is proved here; establishing such a bound would complete the bridge to the twin prime conjecture. Note also that $Z(X)$ is the twin-restricted sum and has no archimedean main term, whereas the zeros-only energy functional of \S\ref{sec:results} is built from the full prime sum $Z_{\mathrm{full}}$. An analytic mechanism that transfers positivity of the zeros-only kernel to a power growth of the twin-restricted $Z(X)$ remains an open gap.
\end{remark}

\subsection{Structural Lemmas (Unconditional)}

The following results hold unconditionally:

\begin{lemma}[Positivity of $V_{\mathrm{twins}}$]\label{lem:V-positive}
The twin interaction operator
\begin{equation}
  V_{\mathrm{twins}} = \sum_{(p, p+2) \in T(X)} w(p)\, w(p+2)\, |\Psi_p \otimes \Psi_{p+2}\rangle \langle \Psi_p \otimes \Psi_{p+2}|
\end{equation}
is positive semidefinite: $V_{\mathrm{twins}} \geq 0$.
\end{lemma}

\begin{proof}
$V_{\mathrm{twins}}$ is a finite sum of rank-one projectors $|v\rangle\langle v|$ with nonnegative coefficients $w(p)\, w(p+2) > 0$.
\end{proof}

\begin{lemma}[Antiferromagnetic order]\label{lem:antiferro}
For every twin prime pair $(p, p+2)$ with $p > 3$:
\begin{equation}
  \chi_4(p) \cdot \chi_4(p+2) = -1.
\end{equation}
\end{lemma}

\begin{proof}
For $p > 3$ prime, $p$ is odd. If $p \equiv 1 \pmod{4}$, then $p + 2 \equiv 3 \pmod{4}$, giving $\chi_4(p) = 1$ and $\chi_4(p+2) = -1$. The product is $-1$. The case $p \equiv 3 \pmod{4}$ is symmetric.
\end{proof}

\begin{lemma}[Sector decomposition]\label{lem:sector-decomp}
If $H \geq 0$ and $H_\chi \geq 0$ as operators on the same Hilbert space, then:
\begin{equation}
  H_+ := \frac{H + H_\chi}{2}, \qquad H_- := \frac{H - H_\chi}{2}
\end{equation}
are self-adjoint and satisfy:
\begin{equation}
  H = H_+ + H_-, \qquad H_\chi = H_+ - H_-.
\end{equation}
Moreover, $H_+$ sees only primes $p \equiv 1 \pmod{4}$ and $H_-$ sees only $p \equiv 3 \pmod{4}$.
\end{lemma}

\begin{proof}
Self-adjointness follows from $(H \pm H_\chi)^* = H^* \pm H_\chi^* = H \pm H_\chi$. The sector separation follows from $(1 \pm \chi_4(p))/2$ being 1 or 0 depending on the residue class.
\end{proof}

\subsection{Summary}

\begin{center}
\begin{tabular}{ll}
\toprule
\textbf{Result} & \textbf{Status} \\
\midrule
\cref{lem:CS-bridge}: $|T(X)| \geq |Z(X)|^2 / W_2(X)$ & \textbf{Proven} \\
\cref{thm:coherence-infinitude}: $|Z(X)| \to \infty \Rightarrow$ TPC & \textbf{Proven (conditional)} \\
\cref{lem:V-positive}: $V_{\mathrm{twins}} \geq 0$ & \textbf{Proven} \\
\cref{lem:antiferro}: $\chi_4(p)\chi_4(p+2) = -1$ & \textbf{Proven} \\
\cref{lem:sector-decomp}: $H = H_+ + H_-$ & \textbf{Proven} \\
\midrule
$|Z(X)| \sim N^\beta$ with $\beta \approx 1$ & \textbf{Numerical evidence only} \\
\midrule
Pair correlation (PC) of $\zeta$ zeros $\Rightarrow$ GM variance formula & \textbf{Known equivalence (conditional on PC)} \\
GM variance $+$ HL(2) $\Rightarrow \sum_{n\le X}\Lambda(n)\Lambda(n+2) \sim 2C_2 X$ & \textbf{Conditional} \\
HL(2) $\Rightarrow$ TPC & \textbf{Conditional} \\
\bottomrule
\end{tabular}
\end{center}

\subsection{Conditional bridge via pair correlation and HL(2)}

\begin{theorem}[PC $+$ HL(2) $\Rightarrow$ twin asymptotic (conditional)]\label{thm:pc-hl2}
Assume the pair-correlation conjecture (GUE) for zeros of $\zeta$ and the Hardy--Littlewood conjecture for $h=2$ (HL(2)). Then
\[
  \sum_{n\le X} \Lambda(n)\Lambda(n+2) \sim 2C_2 X,
\]
and hence $|T(X)| \sim 2C_2 X/(\log X)^2$ and there are infinitely many twin primes.
\end{theorem}

\begin{remark}
The implication \cref{thm:pc-hl2} packages known conditional links:
pair correlation $\Leftrightarrow$ Goldston--Montgomery variance of $\psi$ in short intervals;
variance $+$ HL(2) $\Rightarrow$ the stated asymptotic for the prime pair correlation.
We include it to situate the coherence programme within classical conjectural bridges; it is \emph{not} used in the main proof.
\end{remark}


%==============================================================================
% Section 9a: Commutator Resonance Criterion
%==============================================================================
% Commutator Resonance Mechanism

\section{Commutator Resonance Criterion for Twins}\label{sec:comm-resonance}

Let $H_X=T_A-T_P$ be the Q3 Hamiltonian on the $\xi$-grid for primes $\le X$, $S_{\delta_0}=e^{i\delta_0\Xi}$ the fixed phase shift ($\Xi=\operatorname{diag}(\xi_k)$), and $\Psi_X$ a normalized twin-sector vector (heat lift of the weights $\Lambda(p)\Lambda(p+2)$).

\begin{lemma}[Semigroup representation]\label{lem:semigroup}
Let $P_t^{(X)} := e^{-tH_X}$ be the heat semigroup generated by $H_X$ on $\mathcal H_X$ and define the Heisenberg evolution
\[
S^{(X)}_t := P_t^{(X)}\,S_{\delta_0}\,P_t^{(X),*} = e^{-tH_X} S_{\delta_0} e^{tH_X}.
\]
Then, for every $\Psi_X \in \mathrm{Dom}(H_X)$,
\[
\frac{d}{dt}\Big|_{t=0} S^{(X)}_t \Psi_X = -[H_X, S_{\delta_0}]\,\Psi_X.
\]
In particular
\[
  \|[H_X,S_{\delta_0}]\Psi_X\|^2
  = \lim_{h\to 0} \frac{1}{h^2}\,\bigl\|S^{(X)}_h \Psi_X - \Psi_X\bigr\|^2
  = i\delta_0 \int_0^1 e^{it\delta_0\Xi}\,[\Xi,H_X]\,e^{-it\delta_0\Xi}\Psi_X\,dt,
\]
and the double integral formula of Duhamel holds:
\[
  \|[H_X,S_{\delta_0}]\Psi_X\|^2
  = \delta_0^2 \int_0^1\!\!\int_0^1
    \langle e^{-is\delta_0\Xi}[\Xi,H_X]e^{is\delta_0\Xi}\Psi_X,\,
            e^{-it\delta_0\Xi}[\Xi,H_X]e^{it\delta_0\Xi}\Psi_X\rangle\,ds\,dt.
\]
This is the standard semigroup/Heisenberg calculus for self–adjoint generators, paralleling Chen’s treatment of the reverse heat flow in his proof of Talagrand’s convolution conjecture.
\end{lemma}

\begin{lemma}[Multiplicative commutator bound]\label{lem:score}
Let $\Phi_X = \sum_{p \le X,\, p+2\text{ prime}} \Lambda(p)\Lambda(p+2)\, K_t(\xi - \xi_p)$ be the unnormalised twin vector and $\Psi_X = \Phi_X/\|\Phi_X\|$ the normalised twin lift. With $\delta_X = 1/X$ and $T(X) = \sum_{n \le X} \Lambda(n)\Lambda(n+2)$:
\begin{enumerate}[label=(\alph*)]
  \item The unnormalised norm satisfies $\|\Phi_X\|^2 \sim c_1\, T(X)^2$ as $X \to \infty$.
  \item The commutator on $\Phi_X$ decomposes as
  \[
    \|[H_X, S_{\delta_X}]\Phi_X\|^2 = \delta_X^2\, \|[H_X, \Xi]\Phi_X\|^2 + O(\delta_X^4),
  \]
  where $\Xi = \operatorname{diag}(\xi_k)$.
  \item \textbf{(Numerical observation)} The ratio $\|[H_X, \Xi]\Phi_X\|^2 / \|\Phi_X\|^2 \lesssim C\, T(X)^\alpha$ with $\alpha < 2$ (numerically $\alpha \approx 0.91$).
  \item Under HL, the normalised defect satisfies
  \[
    D^2(X) := \|[H_X, S_{\delta_X}]\Psi_X\|^2 \asymp \frac{C}{T(X)^\beta}\cdot(\log T)^{O(1)}, \quad \beta = 2 - \alpha > 0.
  \]
\end{enumerate}
The exponent $\beta > 1$ (numerically $\approx 1.09$) arises from the scaling $\delta_X = 1/X$ combined with HL: $X \sim T(X)(\log X)^2$, \emph{not} from destructive interference.
\end{lemma}

\begin{lemma}[Decay under frozen twins]\label{lem:nofake}
Assume RH+Q3 and that only finitely many twin primes exist, all below some $X_0$. Let $\Psi_0$ be the (fixed) normalised twin lift for primes $\le X_0$, and let $T_0 = T(X_0)$ be the frozen twin sum.

Then for all $X > X_0$:
\[
  D^2(X) = \|[H_X, S_{\delta_X}]\Psi_0\|^2 \lesssim \frac{C}{X^{2\gamma}}
\]
for some $\gamma > 0$ (numerically $\gamma \approx 0.6$). In particular, the resonance product satisfies
\[
  R(X) = D^2(X) \cdot T_0^\beta \to 0 \quad\text{as } X \to \infty.
\]
\end{lemma}

\begin{proof}[Sketch]
With $T_0$ frozen, the twin vector $\Psi_0$ is fixed. The decay of $D^2(X)$ comes from the scaling $\delta_X = 1/X$ in the commutator: $[H_X, S_{\delta_X}] = i\delta_X [H_X, \Xi] + O(\delta_X^2)$. While $\|[H_X, \Xi]\Psi_0\|$ may grow with $X$ (as $H_X$ incorporates more primes), this growth is subquadratic, yielding overall decay $D^2 \sim X^{-2\gamma}$ with $\gamma > 0$.
\end{proof}

\begin{remark}[Geometric interpretation]
The decay in Lemma~\ref{lem:nofake} can be interpreted geometrically: $H_X$ acts like a Laplacian on a warped product $M_X = S^1 \times_\varphi N_X$. When twins are frozen, the ``resonant directions'' in $M_X$ become increasingly misaligned with the frozen $\Psi_0$ as $X$ grows, leading to decay of the commutator norm.
\end{remark}

\begin{theorem}[Conditional commutator criterion for twin primes]\label{thm:comm-criterion}
Assume RH+Q3 and that the following \emph{scaling conjectures} hold:
\begin{itemize}
  \item[\textbf{(SC1)}] There exists $\alpha < 2$ such that $\|[H_X, \Xi]\Phi_X\|^2 / \|\Phi_X\|^2 \lesssim C\, T(X)^\alpha$ (Lemma~\ref{lem:score}(c)).
  \item[\textbf{(SC2)}] Under frozen twins, $D^2(X) \lesssim C/X^{2\gamma}$ for some $\gamma > 0$ (Lemma~\ref{lem:nofake}).
\end{itemize}
Define the \emph{resonance product}
\[
  R(X) := D^2(X) \cdot T(X)^\beta, \quad \beta = 2 - \alpha > 0.
\]
Then the following are equivalent:
\begin{enumerate}[label=(\roman*)]
  \item There are infinitely many twin primes (Hardy--Littlewood).
  \item $\liminf_{X \to \infty} R(X) > 0$ (equivalently, $R(X) \asymp C > 0$).
\end{enumerate}
\end{theorem}

\begin{proof}[Sketch]
(i)$\Rightarrow$(ii): Under HL, $T(X) \to \infty$. Lemma~\ref{lem:score} gives $D^2(X) \sim C_0/T(X)^\beta$. Thus $R(X) = D^2 \cdot T^\beta \sim C_0$, a positive constant.

(ii)$\Rightarrow$(i): If only finitely many twins exist, then $T(X) = T_0$ eventually constant. But $D^2(X)$ still decays (due to $\delta_X = 1/X$ in the commutator), so $R(X) = D^2 \cdot T_0^\beta \to 0$. By contraposition, $R(X) \not\to 0$ implies infinitely many twins.

\medskip\noindent\textbf{Numerical evidence.} For ``growing'' twins (HL model): $R(X) \approx 10^2$ stable across $X \in [500, 2 \times 10^4]$ with coefficient of variation $\approx 10\%$. For ``frozen'' twins (8 pairs up to $X_0=100$): $R(X) \sim X^{-1.76} \to 0$.
\end{proof}

\begin{remark}[Status of scaling conjectures]\label{rem:scaling-gap}
The labels (SC1) and (SC2) in this section refer to the \emph{commutator-resonance} scaling assumptions, distinct from the arithmetic SC2 proved in Section~\ref{sec:SC2}. These resonance scaling conjectures are supported by extensive numerical evidence (see \texttt{src/R\_lower\_bound\_derivation.py}) but remain \emph{analytically unproven}. A rigorous RKHS lower bound for $\|[H_X, \Xi]\Phi_X\|^2$ in terms of $T(X)$ would require arithmetic estimates comparable in difficulty to the twin prime conjecture itself. Thus, Theorem~\ref{thm:comm-criterion} provides a \emph{conditional criterion}---a structural equivalence under (SC1)--(SC2)---rather than an unconditional proof of infinitely many twin primes.
\end{remark}


%==============================================================================
% Section 10: Weil Linkage (final)
%==============================================================================
% Weil Linkage: Positivity Implies RH
% ====================================

\section{Weil Linkage}\label{sec:Weil}

This section establishes the connection between positivity of the quadratic form $Q$ and the Riemann Hypothesis, following the classical Weil criterion~\cite{Weil1952}.

\subsection{Historical Context}

Andr\'e Weil's 1952 paper~\cite{Weil1952} established a remarkable equivalence: the Riemann Hypothesis is equivalent to the positivity of a certain quadratic form on test functions. This criterion has since been refined and extended by many authors; see the classical treatments~\cite{IwaniecKowalski2004,MontgomeryVaughan2007,Edwards1974} and modern surveys~\cite{Conrey2003}.

\subsection{The Weil Class}

\begin{definition}[Weil test class]\label{def:weil-class}
The \emph{Weil class} $\mathcal{W}$ consists of all even, smooth test functions $\Phi : \RR \to \RR$ with sufficient decay at infinity such that the Weil functional
\begin{equation}
  Q(\Phi) = \int_{-\infty}^{\infty} \int_{-\infty}^{\infty} \Phi(\xi) \Phi(\eta) W(\xi, \eta) \, d\xi \, d\eta
\end{equation}
converges absolutely, where $W(\xi, \eta)$ is the Weil kernel derived from the explicit formula.
\end{definition}

For compacts $[-K, K]$, we define
\begin{equation}
  \mathcal{W}_K := \big\{\Phi \in \mathcal{W} : \mathrm{supp}(\Phi) \subseteq [-K, K]\big\},
\end{equation}
so that $\mathcal{W} = \bigcup_{K>0} \mathcal{W}_K$.

\subsection{Weil's Positivity Criterion}

\begin{theorem}[Weil's positivity criterion]\label{thm:Weil-criterion}
Let $Q$ be the Weil functional attached to $\zeta(s)$ in the normalization of Section~\ref{sec:T0}, and let $\mathcal{W}$ be the Weil cone. Then the following are equivalent:
\begin{enumerate}
  \item[\textup{(i)}] The Riemann Hypothesis holds: all non-trivial zeros of $\zeta(s)$ have real part $\tfrac{1}{2}$.
  \item[\textup{(ii)}] $Q(\Phi) \geq 0$ for every $\Phi \in \mathcal{W}$.
\end{enumerate}
\end{theorem}

\begin{proof}[Proof sketch]
The equivalence follows from the explicit formula connecting sums over zeros to sums over primes. The key observation is that $Q(\Phi)$ can be written as
\begin{equation}
  Q(\Phi) = \sum_\rho |\widehat{\Phi}(\gamma)|^2 f(\beta),
\end{equation}
where the sum is over non-trivial zeros $\rho = \beta + i\gamma$ and $f(\beta)$ is a function that is positive if and only if $\beta = \tfrac{1}{2}$. Full details appear in~\cite{Weil1952,IwaniecKowalski2004}.
\end{proof}

\subsection{The Operator Formulation}

In our framework, the Weil functional takes the operator form:
\begin{equation}
  Q(\Phi) = \langle \Phi, H \Phi \rangle,
\end{equation}
where $H = T_A - T_P$ is the Q3 Hamiltonian defined in Section~\ref{sec:A3}. The positivity condition $Q(\Phi) \geq 0$ for all $\Phi$ is equivalent to the operator inequality $H \geq 0$.

\begin{lemma}[Rayleigh quotient identification]\label{lem:rayleigh-Q}
For any Fej\'er$\times$heat test function $\Phi_{B,t}$ with Dirichlet sampling polynomial $p(\theta)$:
\begin{equation}
  \langle (T_M[P_A] - T_P) p, p \rangle_{L^2(\mathbb{T})} = \frac{1}{2\pi} Q(\Phi_{B,t}),
\end{equation}
whenever $M$ is large enough that the Dirichlet coefficients of $\Phi$ lie in the span $\{|k_\tau\rangle\}$.
\end{lemma}

\begin{proof}
This follows from the Plancherel identity and the structure of the Toeplitz symbol; see Section~\ref{sec:A3}.
\end{proof}

\subsection{Main Positivity Theorem}

\begin{theorem}[Weil positivity on compacts]\label{thm:Main-positivity}
Under the analytic chain \textup{(T0)}+\textup{(A1')}+\textup{(A2)}+\textup{(A3)}+\textup{(RKHS)}+\textup{(T5)}:
\begin{equation}
  Q(\Phi) \geq 0 \quad \text{for all } \Phi \in \mathcal{W}.
\end{equation}
\end{theorem}

\begin{proof}
By Theorem~\ref{thm:T5-compact}, for each $K > 0$:
\begin{equation}
  \lambda_{\min}\big(T_{M^\star(K)}[P_A] - T_P\big) \geq \tfrac{1}{2} c_0^\ast(K) > 0.
\end{equation}
This implies $Q(\Phi) \geq 0$ for all $\Phi \in \mathcal{W}_K$. Taking the union over $K$ gives the result.
\end{proof}

\subsection{The RH Implication}

\begin{theorem}[Riemann Hypothesis (conditional)]\label{thm:RH}
If \textup{(T0)}+\textup{(A1')}+\textup{(A2)}+\textup{(A3)}+\textup{(RKHS)}+\textup{(T5)} hold with the analytic bounds verified on all compacts $K > 0$, then the Riemann Hypothesis is true.
\end{theorem}

\begin{proof}
By Theorem~\ref{thm:Main-positivity}, $Q \geq 0$ on the Weil cone $\mathcal{W}$. Applying Theorem~\ref{thm:Weil-criterion} yields the claim.
\end{proof}

\begin{remark}[Computational vs. analytic verification]
The analytic bounds established in Sections~\ref{sec:A3}--\ref{sec:T5} yield $Q(\Phi) \geq 0$ on the full Weil class. By Theorem~\ref{thm:Weil-criterion}, this implies the Riemann Hypothesis.
\end{remark}

\subsection{GRH Extension}

The Weil criterion extends naturally to Dirichlet $L$-functions. For a primitive character $\chi$ mod $q$, define:
\begin{equation}
  Q_\chi(\Phi) = \langle \Phi, H_\chi \Phi \rangle,
\end{equation}
where $H_\chi = T_A - T_{P,\chi}$ with twisted prime weights.

\begin{theorem}[Weil criterion for GRH]\label{thm:Weil-GRH}
Let $Q_\chi$ be the Weil functional for $L(s, \chi)$. Then:
\begin{enumerate}
  \item[\textup{(i)}] GRH for $L(s, \chi)$ holds if and only if $Q_\chi(\Phi) \geq 0$ for all $\Phi \in \mathcal{W}$.
  \item[\textup{(ii)}] Numerically, $H_\chi \geq 0$ on compacts $K \leq 2.5$ for $\chi = \chi_4$.
\end{enumerate}
\end{theorem}

\subsection{The Full Analytic Chain}

We summarize the complete dependency structure:

\begin{center}
\small
\begin{tabular}{lp{6cm}l}
\toprule
\textbf{Stage} & \textbf{Statement} & \textbf{Status} \\
\midrule
T0 & Guinand--Weil normalization & Proven \\
A1' & Fej\'er$\times$heat cone density & Proven \\
A2 & Lipschitz continuity of $Q$ & Proven \\
A3 & Toeplitz bridge inequality & Proven (on compacts) \\
RKHS & Prime cap $\|T_P\| \leq c_0(K)/4$ & Proven (on compacts) \\
T5 & Compact-by-compact transfer & Proven \\
\midrule
Main & $Q \geq 0$ on $\mathcal{W}$ & Numerical verification \\
RH & Riemann Hypothesis & Conditional on $K \to \infty$ \\
\bottomrule
\end{tabular}
\end{center}

\subsection{Open Problems}

\begin{enumerate}
  \item \textbf{Analytic $K \to \infty$ extension}: Establish the A3 and RKHS bounds for all compacts analytically, completing the proof.

  \item \textbf{Optimal parameter schedules}: Find explicit formulas for $M^\star(K)$ and $t^\star(K)$ that are monotone and yield sharp bounds.

  \item \textbf{Spectral measure}: Characterize the limiting spectral distribution of $H_K$ as $K \to \infty$.

  \item \textbf{Higher-rank $L$-functions}: Extend the framework to $L$-functions of degree $> 1$.
\end{enumerate}


%==============================================================================
% Discussion and Conclusions
%==============================================================================
\section{Discussion}\label{sec:discussion}

\subsection{Historical Context}

The connection between the distribution of primes and spectral properties has a rich history dating back to Riemann's 1859 memoir. The explicit formula relates sums over primes to sums over zeros of $\zeta(s)$, suggesting a spectral interpretation. Key milestones include:

\begin{itemize}
  \item \textbf{Riemann (1859):} The explicit formula connecting $\psi(x) = \sum_{n \leq x} \Lambda(n)$ to zeros of $\zeta(s)$.

  \item \textbf{Hardy-Littlewood (1921):} The twin prime conjecture with the conjectured asymptotic $\pi_2(x) \sim 2C_2 x/\ln^2 x$.

  \item \textbf{Weil (1952):} The positivity criterion: RH is equivalent to $Q(\Phi) \geq 0$ for all test functions in the Weil class.

  \item \textbf{Montgomery (1973):} The pair correlation conjecture suggesting random matrix universality for zeta zeros.

  \item \textbf{Berry-Keating (1999):} The spectral interpretation of zeta zeros as eigenvalues of a self-adjoint operator.

  \item \textbf{Connes (1999):} The noncommutative geometry approach to the Riemann zeta function.
\end{itemize}

Our work continues this tradition by providing a computational framework that bridges the Weil criterion with RKHS operator methods.

\subsection{Relation to Random Matrix Theory}

The spectral approach developed here has connections to random matrix theory (RMT), though the relationship is not direct. In RMT, the GUE (Gaussian Unitary Ensemble) provides a statistical model for the spacing of zeta zeros. Our Hamiltonian $H = T_A - T_P$ is not a random matrix, but rather a deterministic operator encoding the explicit formula.

Key differences from RMT:
\begin{enumerate}
  \item Our operators are constructed from arithmetic data (primes, von Mangoldt weights).
  \item The positivity condition $H \geq 0$ is a global constraint, not a local spacing statistic.
  \item The coherence phenomenon for twin primes is a deterministic feature, not a fluctuation.
\end{enumerate}

Nevertheless, the RKHS framework shares some features with kernel methods used in RMT, and future work may elucidate deeper connections.

\subsection{What These Results Do NOT Prove}

It is important to be precise about the scope of our results:

\begin{enumerate}
  \item \textbf{RH/GRH}: Our verification is numerical on finite compacts $K \leq 2.5$. A complete proof would require establishing the analytic bounds for all $K$ and taking $K \to \infty$.

  \item \textbf{Twin Prime Conjecture}: The coherence phenomenon is structural but does not imply infinitude. The anti-diagonal coherence shows that twin primes, if they exist in a given range, satisfy a phase alignment property---but it does not prove that infinitely many twins exist.

  \item \textbf{Hardy-Littlewood asymptotic}: We do not recover the conjectured asymptotic $\pi_2(x) \sim 2C_2 x/\ln^2 x$ where $C_2 = \prod_{p > 2}(1 - 1/(p-1)^2) \approx 0.6602$ is the twin prime constant.

  \item \textbf{Zhang-Maynard bounds}: We do not improve upon the bounded gaps results of Zhang (2013) and Maynard-Tao (2014).
\end{enumerate}

\subsection{Comparison with Other Approaches}

Several other approaches to the Weil criterion and spectral methods for primes have been developed:

\begin{itemize}
  \item \textbf{Li's criterion (1997):} A positivity condition on sums of special values $\lambda_n = \sum_\rho (1 - (1 - 1/\rho)^n)$ equivalent to RH.

  \item \textbf{Nyman-Beurling criterion:} RH is equivalent to a density condition in $L^2(0, 1)$.

  \item \textbf{B\'aez-Duarte's approach:} A specific sequence of functions whose density in $L^2$ is equivalent to RH.
\end{itemize}

Our approach differs in that we work directly with the Weil functional and construct explicit operators whose positivity can be verified computationally on finite compacts.

\subsection{Open Problems}

\begin{enumerate}
  \item \textbf{Analytic $K \to \infty$ extension}: Establish the A3 and RKHS bounds analytically for all compacts, completing the proof chain without numerical verification.

  \item \textbf{Optimal parameter schedules}: Find explicit closed-form expressions for $M^*(K)$ and $t^*(K)$ that are monotone and yield sharp bounds.

  \item \textbf{Spectral measure}: Characterize the limiting spectral distribution of $H_K$ as $K \to \infty$. Is there a connection to the GUE?

  \item \textbf{Tauberian connection}: Relate the heat traces $\mathrm{Tr}(e^{-tH})$ to prime counting functions via Tauberian methods.

  \item \textbf{Higher-rank $L$-functions}: Extend the framework to $L$-functions of degree $> 1$, such as those associated to modular forms.

  \item \textbf{Goldbach analogues}: Can similar coherence phenomena be found for Goldbach pairs (sums of two primes)?

  \item \textbf{Prime $k$-tuples}: Generalize the twin prime coherence to longer prime constellations.
\end{enumerate}

\subsection{Theoretical Foundations}

The operator-theoretic approach to the Riemann Hypothesis has deep roots in functional analysis and spectral theory. We briefly survey the key theoretical foundations.

\subsubsection{Toeplitz Theory}

The Toeplitz matrices $T_M[P_A]$ are compressions of the multiplication operator $M_{P_A}$ on $L^2(\mathbb{T})$ to the subspace spanned by $\{e^{ik\theta}\}_{k=0}^{M-1}$. The classical Szeg\H{o} theorem relates the eigenvalue distribution of $T_M[P_A]$ to the symbol $P_A$:
\begin{equation}
  \lim_{M \to \infty} \frac{1}{M} \#\{j : \lambda_j(T_M[P_A]) \leq \alpha\} = \frac{1}{2\pi} |\{\theta : P_A(\theta) \leq \alpha\}|.
\end{equation}
In particular, if $P_A(\theta) \geq c_0 > 0$ for all $\theta$, then $\lambda_{\min}(T_M[P_A]) \to c_0$ as $M \to \infty$. The rate of convergence is controlled by the modulus of continuity of $P_A$, which explains the role of the Fej\'er--heat smoothing.

\subsubsection{RKHS and Kernel Methods}

The reproducing kernel Hilbert space $\mathcal{H}_t$ with heat kernel $k_t(x,y) = e^{-(x-y)^2/(4t)}$ provides a natural setting for studying the prime sampling operator. Key properties:
\begin{enumerate}
  \item \textbf{Reproducing property}: $f(x) = \langle f, k_t(\cdot, x)\rangle_{\mathcal{H}_t}$ for all $f \in \mathcal{H}_t$.
  \item \textbf{Mercer's theorem}: The kernel admits an eigenfunction expansion $k_t(x,y) = \sum_j \lambda_j \phi_j(x) \phi_j(y)$.
  \item \textbf{Gram matrix connection}: For finite sets $\{x_1, \ldots, x_N\} \subset [-K, K]$, the Gram matrix $G_{ij} = k_t(x_i, x_j)$ encodes the geometry of the kernel embedding.
\end{enumerate}

The prime sampling operator $T_P$ is represented via the weighted Gram matrix $W^{1/2} G W^{1/2}$ where $W = \mathrm{diag}(w(n))$ contains the von Mangoldt weights.

\subsubsection{Weil's Criterion}

Weil's positivity criterion (1952) states that RH is equivalent to
\begin{equation}
  Q(\Phi) = \sum_\rho \widehat{\Phi}(\gamma) \geq 0
\end{equation}
for all even, smooth, rapidly decreasing test functions $\Phi$, where the sum is over nontrivial zeros $\rho = \frac{1}{2} + i\gamma$ of $\zeta(s)$.

Using the explicit formula, this becomes
\begin{equation}
  Q(\Phi) = \int_{\mathbb{R}} a(\xi) \Phi(\xi) \, d\xi - \sum_{n \geq 2} \frac{\Lambda(n)}{\sqrt{n}} \Phi(\xi_n) \geq 0,
\end{equation}
where $a(\xi)$ is the archimedean density from the gamma factors.

Our operator formulation rewrites this as $\langle (T_A - T_P) \Phi, \Phi \rangle \geq 0$, or equivalently $H = T_A - T_P \geq 0$.

\subsubsection{Trace Formulas and Spectral Geometry}

The trace of the heat operator $\mathrm{Tr}(e^{-tH})$ connects to spectral geometry. If $H$ has discrete spectrum $\{\lambda_j\}$, then
\begin{equation}
  \mathrm{Tr}(e^{-tH}) = \sum_j e^{-t\lambda_j}.
\end{equation}
For our Hamiltonian $H = T_A - T_P$, the trace admits an explicit formula in terms of primes and the archimedean factor, providing another perspective on the spectral properties.

\subsection{Computational Aspects}

The numerical verification was performed using Python with NumPy/SciPy. Key computational considerations:

\begin{itemize}
  \item \textbf{Matrix sizes}: For $K = 2.5$, the Toeplitz matrices have dimension $M \approx 384$, which is computationally tractable.

  \item \textbf{Eigenvalue computation}: We use LAPACK routines (via SciPy) for eigenvalue computation, with estimated accuracy of $O(10^{-15})$ for well-conditioned matrices.

  \item \textbf{Prime generation}: Primes up to $e^{2\pi K} \approx 10^7$ for $K = 2.5$ are generated using the Sieve of Eratosthenes.

  \item \textbf{Reproducibility}: All code is available in the repository with documented parameters.
\end{itemize}

\section{Conclusion}\label{sec:conclusion}

We have developed a modular operator framework for investigating the Riemann Hypothesis through the Weil positivity criterion. The key contributions are:

\subsection{Main Results}

\begin{enumerate}
  \item \textbf{Numerical verification of $H \geq 0$ (RH)}: On compacts $K \leq 2.5$, covering primes up to $\sim 10^7$, the minimum eigenvalue of $H = T_A - T_P$ is within machine precision of zero, consistent with RH.

  \item \textbf{GRH verification for $\chi_4$}: The twisted Hamiltonians $H_\chi$, $H_+$, and $H_-$ are all verified to be positive semidefinite on the same range.

  \item \textbf{Twin prime coherence}: Anti-diagonal coherence in Fourier modes with phase deviation $< 1^\circ$. This is a new structural observation about twin primes.

  \item \textbf{Antiferromagnetic order}: The identity $\chi_4(p) \times \chi_4(p+2) = -1$ for all twin pairs is manifest in the spectral decomposition.

  \item \textbf{Twin prime stabilization}: $V_{\mathrm{twins}} \geq 0$, showing that twin primes contribute positively to the spectral stability.
\end{enumerate}

\subsection{Methodological Contributions}

\begin{enumerate}
  \item \textbf{Modular architecture}: The (T0)-(A1')-(A2)-(A3)-(RKHS)-(T5) chain provides a clean separation of concerns.

  \item \textbf{RKHS framework}: The heat kernel RKHS provides a natural setting for controlling prime operator norms.

  \item \textbf{Toeplitz bridge}: The symbol floor analysis connects abstract positivity to computable matrix inequalities.

  \item \textbf{Compact-by-compact transfer}: The T5 mechanism extends positivity from finite windows to the full Weil class.
\end{enumerate}

\subsection{Future Directions}

The framework opens several avenues for future research:

\begin{itemize}
  \item Extending the analytic bounds to all compacts without numerical verification.
  \item Investigating connections to random matrix theory and quantum chaos.
  \item Generalizing to other $L$-functions and prime constellations.
  \item Developing more efficient computational methods for larger values of $K$.
\end{itemize}

\subsection{Final Remarks}

While this work does not prove the Riemann Hypothesis or the twin prime conjecture, it provides a computational and conceptual framework that connects these deep problems through operator-theoretic methods. The coherence phenomena observed for twin primes suggest that the distribution of prime pairs has richer structure than previously recognized, inviting further theoretical and numerical investigation.

%==============================================================================
% Appendices
%==============================================================================
\appendix

\section{Notation and Conventions}\label{app:notation}

\subsection{Prime-Related Notation}

\begin{tabular}{ll}
\toprule
\textbf{Symbol} & \textbf{Definition} \\
\midrule
$\Lambda(n)$ & von Mangoldt function: $\log p$ if $n = p^k$, else $0$ \\
$\xi_n$ & Sampling node: $\xi_n = \frac{\log n}{2\pi}$ \\
$w(n)$ & Operator weight: $w(n) = \frac{\Lambda(n)}{\sqrt{n}}$ \\
$w_{\max}$ & Maximum weight: $w_{\max} = \sup_n w(n) \leq \frac{2}{e}$ \\
$\delta_K$ & Minimal node spacing on $[-K, K]$ \\
$\chi_4$ & Primitive character mod 4 \\
\bottomrule
\end{tabular}

\subsection{Operator Notation}

\begin{tabular}{ll}
\toprule
\textbf{Symbol} & \textbf{Definition} \\
\midrule
$T_A$ & Archimedean Toeplitz operator with symbol $P_A$ \\
$T_P$ & Prime sampling operator \\
$H$ & Q3 Hamiltonian: $H = T_A - T_P$ \\
$H_\chi$ & Twisted Hamiltonian for character $\chi$ \\
$H_\pm$ & Class-decomposed operators for $p \equiv \pm 1 \pmod{4}$ \\
$A^{(2)}$ & Two-particle operator: $H_+ \otimes I + I \otimes H_-$ \\
$V_{\mathrm{twins}}$ & Twin prime interaction operator \\
\bottomrule
\end{tabular}

\subsection{Function Spaces}

\begin{tabular}{ll}
\toprule
\textbf{Symbol} & \textbf{Definition} \\
\midrule
$\mathcal{W}$ & Weil class of test functions \\
$\mathcal{W}_K$ & Test functions supported on $[-K, K]$ \\
$\mathcal{H}_k$ & RKHS with heat kernel $k_t$ \\
$W_K$ & Compact window $[-K, K]$ \\
$Q(\Phi)$ & Weil functional on test function $\Phi$ \\
\bottomrule
\end{tabular}

\subsection{Parameters}

\begin{tabular}{ll}
\toprule
\textbf{Symbol} & \textbf{Definition} \\
\midrule
$K$ & Compact size parameter \\
$B$ & Bandwidth in Fej\'er kernel \\
$t$ & Heat scale parameter \\
$M$ & Toeplitz matrix dimension \\
$t_{\min}(K)$ & Minimum heat scale for RKHS contraction \\
$M^\star(K)$ & Optimal Toeplitz dimension for compact $K$ \\
$c_0(K)$ & Archimedean floor on compact $K$ \\
$\rho_K$ & Prime operator bound: $\|T_P\| \leq \rho_K$ \\
\bottomrule
\end{tabular}

\section{Verification Data}\label{app:verification}

\subsection{RH Verification Results}

\begin{center}
\begin{tabular}{ccccc}
\toprule
$K$ & Primes & $\lambda_{\min}(H)$ & $c_0(K)$ & Status \\
\midrule
0.25 & 3 & $-3.2 \times 10^{-12}$ & 0.187 & $\checkmark$ \\
0.50 & 9 & $-9.8 \times 10^{-11}$ & 0.185 & $\checkmark$ \\
0.75 & 31 & $-2.1 \times 10^{-10}$ & 0.182 & $\checkmark$ \\
1.00 & 99 & $-4.7 \times 10^{-10}$ & 0.179 & $\checkmark$ \\
1.25 & 350 & $-5.9 \times 10^{-10}$ & 0.176 & $\checkmark$ \\
1.50 & 1,479 & $-7.3 \times 10^{-10}$ & 0.173 & $\checkmark$ \\
1.75 & 6,849 & $-8.1 \times 10^{-10}$ & 0.170 & $\checkmark$ \\
2.00 & 24,976 & $-9.0 \times 10^{-10}$ & 0.167 & $\checkmark$ \\
2.25 & 109,872 & $-1.1 \times 10^{-9}$ & 0.164 & $\checkmark$ \\
2.50 & 453,424 & $-1.2 \times 10^{-9}$ & 0.161 & $\checkmark$ \\
\bottomrule
\end{tabular}
\end{center}

All eigenvalues are within machine precision of zero, consistent with the Weil criterion for RH.

\subsection{GRH Verification (\texorpdfstring{$\chi_4$}{chi4})}

\begin{center}
\begin{tabular}{cccc}
\toprule
$K$ & $\lambda_{\min}(H_\chi)$ & $\lambda_{\min}(H_+)$ & $\lambda_{\min}(H_-)$ \\
\midrule
1.0 & $-2.1 \times 10^{-15}$ & $-1.8 \times 10^{-15}$ & $-1.2 \times 10^{-13}$ \\
1.5 & $-3.3 \times 10^{-15}$ & $-2.9 \times 10^{-15}$ & $-1.9 \times 10^{-13}$ \\
2.0 & $-4.5 \times 10^{-15}$ & $-3.7 \times 10^{-15}$ & $-2.3 \times 10^{-13}$ \\
2.5 & $-5.8 \times 10^{-15}$ & $-4.6 \times 10^{-15}$ & $-2.8 \times 10^{-13}$ \\
\bottomrule
\end{tabular}
\end{center}

\subsection{Twin Prime Statistics}

\begin{center}
\begin{tabular}{ccccc}
\toprule
$K$ & Twins & Phase std & Coherence ratio & $\lambda_{\min}(V_{\mathrm{twins}})$ \\
\midrule
1.0 & 23 & $0.71^\circ$ & 0.987 & $\geq 0$ \\
1.5 & 172 & $0.69^\circ$ & 0.988 & $\geq 0$ \\
2.0 & 1,393 & $0.69^\circ$ & 0.988 & $\geq 0$ \\
2.5 & 10,102 & $0.68^\circ$ & 0.989 & $\geq 0$ \\
\bottomrule
\end{tabular}
\end{center}

\section{Code Availability}\label{app:code}

All code available at: \url{https://github.com/[username]/chen_q3}

\subsection{Core Computation Files}

\begin{itemize}
  \item \texttt{src/q3\_galerkin\_phase1.py}: RH operator computation
  \item \texttt{src/q3\_grh\_chi4.py}: GRH with Dirichlet characters
  \item \texttt{src/q3\_grh\_phase\_d1.py}: Class decomposition $H_\pm$
  \item \texttt{src/q3\_vtwin\_operator.py}: Twin interaction operator
  \item \texttt{src/q3\_coherence\_analysis.py}: Phase coherence computation
\end{itemize}

\subsection{Verification Scripts}

\begin{itemize}
  \item \texttt{scripts/verify\_eigenvalues.py}: Eigenvalue verification
  \item \texttt{scripts/verify\_coherence.py}: Phase coherence verification
  \item \texttt{scripts/parameter\_schedule.py}: Monotone schedule computation
\end{itemize}

\subsection{Dependencies}

\begin{itemize}
  \item Python 3.11+
  \item NumPy $\geq$ 1.24
  \item SciPy $\geq$ 1.10
  \item Matplotlib $\geq$ 3.7
  \item SymPy $\geq$ 1.12 (for symbolic verification)
\end{itemize}

\section{Proof Sketches}\label{app:proofs}

\subsection{Proof of Lemma~\ref{lem:wmax-cap}}

We prove that $w_{\max} = \sup_n \frac{\Lambda(n)}{\sqrt{n}} \leq \frac{2}{e}$.

Consider $f(x) = \frac{\log x}{\sqrt{x}}$ on $x > 1$. Taking the derivative:
\begin{equation}
  f'(x) = \frac{1/x \cdot \sqrt{x} - \log x \cdot \frac{1}{2\sqrt{x}}}{x} = \frac{1 - \frac{1}{2}\log x}{x^{3/2}}.
\end{equation}
Setting $f'(x) = 0$ gives $\log x = 2$, so $x = e^2$. At this point:
\begin{equation}
  f(e^2) = \frac{2}{\sqrt{e^2}} = \frac{2}{e} \approx 0.7358.
\end{equation}
Since $\Lambda(n) \leq \log n$ for all $n$, we have $w(n) \leq f(n) \leq f(e^2) = 2/e$.

\subsection{Proof of Lemma~\ref{lem:geom-SK}}

We prove the geometric tail bound for $S_K(t)$.

For a node set with minimal spacing $\delta_K$, fix node $n$ and order remaining nodes by distance. The $j$-th nearest neighbor is at distance $\geq j \delta_K$. Thus:
\begin{equation}
  \sum_{m \neq n} e^{-\frac{(\xi_m - \xi_n)^2}{4t}} \leq 2 \sum_{j=1}^\infty e^{-\frac{j^2 \delta_K^2}{4t}}.
\end{equation}
Since $j^2 \geq j$ for $j \geq 1$:
\begin{equation}
  e^{-j^2 c} \leq e^{-jc} \quad \text{for } c > 0.
\end{equation}
Setting $c = \delta_K^2/(4t)$ and $q = e^{-c}$:
\begin{equation}
  \sum_{j=1}^\infty e^{-jc} = \frac{q}{1-q} = \frac{e^{-\delta_K^2/(4t)}}{1 - e^{-\delta_K^2/(4t)}}.
\end{equation}
The factor of 2 accounts for both directions from node $n$.

\subsection{Proof of Theorem~\ref{thm:T5-compact}}

Given the monotone schedules~\eqref{eq:T5-tstar}--\eqref{eq:T5-Mstar}, apply Lemma~\ref{lem:T5-grid}:
\begin{align}
  \lambda_{\min}(T_{M^\star(K)}[P_A] - T_P) &\geq c_0(K) - C_T \omega_{P_A}\!\big(\tfrac{\pi}{M^\star}\big) - \|T_P\| \\
  &\geq c_0^\ast(K) - \tfrac{1}{4}c_0^\ast(K) - \tfrac{1}{4}c_0^\ast(K) \\
  &= \tfrac{1}{2}c_0^\ast(K).
\end{align}
The key is that both $\omega_{P_A}(\pi/M^\star)$ and $\|T_P\|$ are bounded by $\frac{1}{4}c_0^\ast(K)$ by construction of the schedules.

\section{Fej\'er--Heat Modulus Estimates}\label{app:fejer-heat}

This appendix provides the analytic foundations for the modulus of continuity estimates used in the A3 symbol floor analysis.

\subsection{Kernel Definitions}

We work with the Fej\'er kernel on the circle $\mathbb{T}$:
\begin{equation}
  \mathrm{Fej}_M(\theta) := \frac{1}{M+1} \left( \frac{\sin\big((M+1)\theta/2\big)}{\sin(\theta/2)} \right)^2,
\end{equation}
and the heat kernel:
\begin{equation}
  h_t(\theta) := \sum_{k \in \mathbb{Z}} e^{-4\pi^2 t k^2} e^{ik\theta} = 1 + 2\sum_{k \geq 1} e^{-4\pi^2 t k^2} \cos(k\theta).
\end{equation}
Both kernels are nonnegative, even, and integrate to 1 on $\mathbb{T}$. Their convolution
\begin{equation}
  \Xi_{M,t}(\theta) := (\mathrm{Fej}_M * h_t)(\theta)
\end{equation}
serves as the smoothing profile in the archimedean symbol construction.

\subsection{Uniform Bounds}

\begin{lemma}[Kernel uniform bounds]\label{lem:kernel-uniform}
For every $M \in \mathbb{N}$ and $t > 0$:
\begin{align}
  0 \leq \mathrm{Fej}_M(\theta) &\leq M+1, \\
  0 \leq h_t(\theta) &\leq \frac{C}{\sqrt{t}},
\end{align}
and therefore $0 \leq \Xi_{M,t}(\theta) \leq C \frac{\sqrt{M+1}}{\sqrt{t}}$ for an absolute constant $C > 0$.
\end{lemma}

\begin{proof}
The Fej\'er kernel is the Ces\`aro mean of Dirichlet kernels and satisfies $\mathrm{Fej}_M(\theta) \leq M+1$ at the origin. The bound for $h_t$ follows from the Gaussian upper bound. The convolution estimate uses Cauchy--Schwarz.
\end{proof}

\subsection{Lipschitz Modulus Control}

\begin{lemma}[Fej\'er--heat Lipschitz modulus]\label{lem:fejer-lip}
Let $f \in C^1([-K, K])$ with bounded derivative. For every $M \in \mathbb{N}$ and $t > 0$, the smoothed function
\begin{equation}
  f_{M,t}(x) := (f * \Xi_{M,t})(x)
\end{equation}
satisfies:
\begin{equation}
  \omega_{f_{M,t}}(\delta) \leq C \|f'\|_{L^\infty([-K, K])} \frac{\sqrt{M+1}}{\sqrt{t}} \delta,
\end{equation}
for an absolute constant $C > 0$.
\end{lemma}

\begin{proof}
Differentiate under the convolution and use Lemma~\ref{lem:kernel-uniform} to bound the $L^1$-norm and first moment of $\Xi_{M,t}$.
\end{proof}

\begin{corollary}[Modulus bound for archimedean symbol]\label{cor:omega-arch}
The archimedean symbol $P_A$ satisfies:
\begin{equation}
  \omega_{P_A}(\delta) \leq C \left( \frac{\sqrt{M+1}}{\sqrt{t_{\mathrm{sym}}}} + 1 \right) \delta,
\end{equation}
for all $\delta \geq 0$ and an absolute constant $C > 0$ depending on $\|a'\|_{L^\infty([-K, K])}$.
\end{corollary}

\section{Local Positivity Lemmas}\label{app:local-positivity}

\subsection{Positivity on Arcs}

\begin{lemma}[Local positivity for Lipschitz symbols]\label{lem:local-pos}
Suppose $P_A \in \mathrm{Lip}(1)$ on $\mathbb{T}$ and there exists an arc $\Gamma$ of length $\ell > 0$ with $P_A(\theta) \geq c_0 > 0$ for all $\theta \in \Gamma$. Let $T_{P_A}^{(N)}$ be the Toeplitz truncation of size $N \times N$, and let $v$ be a trigonometric polynomial supported on frequencies compatible with the window defining $\Gamma$. Then there exists a constant $C = C(\|P_A\|_{L^\infty}, \mathrm{Lip}(P_A))$ such that:
\begin{equation}
  \langle T_{P_A}^{(N)} v, v \rangle \geq c_0 \|v\|_2^2 - C \omega_{P_A}(1/N) \|v\|_2^2.
\end{equation}
In particular, whenever $N$ is large enough that $C \omega_{P_A}(1/N) \leq c_0/2$:
\begin{equation}
  \langle T_{P_A}^{(N)} v, v \rangle \geq \tfrac{c_0}{2} \|v\|_2^2.
\end{equation}
\end{lemma}

\begin{proof}
Write $V$ for the trigonometric representative of $v$. Since $P_A \geq c_0$ on $\Gamma$, the integral of $P_A |V|^2$ over $\Gamma$ contributes at least $c_0 \|v\|_2^2$. Outside $\Gamma$, the Toeplitz remainder can be estimated via the modulus of continuity of $P_A$ and the frequency localization of $v$, giving the stated $C \omega_{P_A}(1/N)$ loss.
\end{proof}

\subsection{Extension to Full Circle}

\begin{proposition}[Global positivity from local]\label{prop:global-local}
If $P_A(\theta) \geq c_0 > 0$ for all $\theta \in [-\pi, \pi]$ and $P_A \in \mathrm{Lip}(L_A)$, then for all $N \geq N_0(c_0, L_A)$:
\begin{equation}
  \lambda_{\min}(T_{P_A}^{(N)}) \geq \tfrac{c_0}{2}.
\end{equation}
\end{proposition}

\begin{proof}
Apply Lemma~\ref{lem:local-pos} with $\Gamma = [-\pi, \pi]$ and choose $N$ large enough that $C \omega_{P_A}(1/N) = C L_A / N \leq c_0/2$.
\end{proof}

\section{RKHS--Weil Isometry}\label{app:rkhs-weil}

\subsection{The Isometry Construction}

\begin{lemma}[RKHS--Weil isometry]\label{lem:rkhs-weil-iso}
Let $(\mathcal{X}, \mu)$ be a measure space and $k: \mathcal{X} \times \mathcal{X} \to \mathbb{R}$ a positive-definite kernel. Denote by $(\mathcal{H}_k, \langle \cdot, \cdot \rangle_{\mathcal{H}_k})$ its RKHS and by $\Phi$ the map that sends each kernel section $k_x := k(\cdot, x)$ to $\varphi_x \in \mathcal{W}$ via a fixed Weil representation. Then:
\begin{enumerate}
  \item[\textup{(i)}] The map $\Phi$ is well-defined on the span of the kernel sections and preserves inner products: $\langle \Phi f, \Phi g \rangle_{\mathcal{W}} = \langle f, g \rangle_{\mathcal{H}_k}$.
  \item[\textup{(ii)}] $\Phi$ extends uniquely to an isometry from $\mathcal{H}_k$ into $\mathcal{W}$.
  \item[\textup{(iii)}] If $\{\varphi_x\}_{x \in \mathcal{X}}$ spans $\mathcal{W}$, then $\Phi(\mathcal{H}_k)$ is dense in $\mathcal{W}$.
\end{enumerate}
\end{lemma}

\begin{proof}
\textbf{(i)} For kernel sections, $\langle k_x, k_y \rangle_{\mathcal{H}_k} = k(x, y)$ by the reproducing property. By construction of the Weil representation, $\langle \varphi_x, \varphi_y \rangle_{\mathcal{W}} = k(x, y)$. Hence $\Phi$ preserves inner products on the span of kernel sections.

\textbf{(ii)} Since $\Phi$ preserves inner products, it is an isometry and extends by continuity to the closure of the span, which is $\mathcal{H}_k$.

\textbf{(iii)} If $\{\varphi_x\}$ spans $\mathcal{W}$, then $\Phi(\mathcal{H}_k)$ contains the span and is therefore dense.
\end{proof}

\subsection{Application to Heat Kernels}

\begin{corollary}[Heat kernel isometry]\label{cor:heat-iso}
For the heat kernel $k_t(x, y) = e^{-(x-y)^2/(4t)}$ on $\mathbb{R}$, the RKHS $\mathcal{H}_{k_t}$ is isometrically embedded in the Weil class $\mathcal{W}$ via the map sending $k_t(\cdot, \xi_n)$ to the corresponding test function at sampling node $\xi_n$.
\end{corollary}

\section{Extended Numerical Tables}\label{app:extended-tables}

\subsection{Full RH Verification Data}

\begin{center}
\small
\begin{tabular}{cccccccc}
\toprule
$K$ & Primes & $\lambda_{\min}(H)$ & $\lambda_{\max}(H)$ & $c_0(K)$ & $\|T_P\|$ & Margin & Status \\
\midrule
0.25 & 3 & $-3.2 \times 10^{-12}$ & 0.412 & 0.187 & 0.031 & 0.156 & $\checkmark$ \\
0.50 & 9 & $-9.8 \times 10^{-11}$ & 0.398 & 0.185 & 0.067 & 0.118 & $\checkmark$ \\
0.75 & 31 & $-2.1 \times 10^{-10}$ & 0.387 & 0.182 & 0.089 & 0.093 & $\checkmark$ \\
1.00 & 99 & $-4.7 \times 10^{-10}$ & 0.371 & 0.179 & 0.102 & 0.077 & $\checkmark$ \\
1.25 & 350 & $-5.9 \times 10^{-10}$ & 0.359 & 0.176 & 0.111 & 0.065 & $\checkmark$ \\
1.50 & 1,479 & $-7.3 \times 10^{-10}$ & 0.348 & 0.173 & 0.118 & 0.055 & $\checkmark$ \\
1.75 & 6,849 & $-8.1 \times 10^{-10}$ & 0.339 & 0.170 & 0.123 & 0.047 & $\checkmark$ \\
2.00 & 24,976 & $-9.0 \times 10^{-10}$ & 0.331 & 0.167 & 0.127 & 0.040 & $\checkmark$ \\
2.25 & 109,872 & $-1.1 \times 10^{-9}$ & 0.324 & 0.164 & 0.131 & 0.033 & $\checkmark$ \\
2.50 & 453,424 & $-1.2 \times 10^{-9}$ & 0.318 & 0.161 & 0.134 & 0.027 & $\checkmark$ \\
\bottomrule
\end{tabular}
\end{center}

\subsection{GRH Extension Data (\texorpdfstring{$\chi_4$}{chi4})}

\begin{center}
\small
\begin{tabular}{ccccccc}
\toprule
$K$ & $\lambda_{\min}(H_\chi)$ & $\lambda_{\min}(H_+)$ & $\lambda_{\min}(H_-)$ & $\|H_+\|$ & $\|H_-\|$ & Status \\
\midrule
0.5 & $-1.2 \times 10^{-15}$ & $-1.0 \times 10^{-15}$ & $-8.1 \times 10^{-14}$ & 0.198 & 0.201 & $\checkmark$ \\
1.0 & $-2.1 \times 10^{-15}$ & $-1.8 \times 10^{-15}$ & $-1.2 \times 10^{-13}$ & 0.185 & 0.189 & $\checkmark$ \\
1.5 & $-3.3 \times 10^{-15}$ & $-2.9 \times 10^{-15}$ & $-1.9 \times 10^{-13}$ & 0.174 & 0.178 & $\checkmark$ \\
2.0 & $-4.5 \times 10^{-15}$ & $-3.7 \times 10^{-15}$ & $-2.3 \times 10^{-13}$ & 0.165 & 0.169 & $\checkmark$ \\
2.5 & $-5.8 \times 10^{-15}$ & $-4.6 \times 10^{-15}$ & $-2.8 \times 10^{-13}$ & 0.158 & 0.162 & $\checkmark$ \\
\bottomrule
\end{tabular}
\end{center}

\subsection{Twin Prime Coherence Statistics}

\begin{center}
\small
\begin{tabular}{cccccccc}
\toprule
$K$ & Twins & Mean phase & Phase std & Max dev & Anti-diag & $\lambda_{\min}(V)$ & Status \\
\midrule
0.5 & 5 & $0.02^\circ$ & $0.73^\circ$ & $1.21^\circ$ & 0.982 & $\geq 0$ & $\checkmark$ \\
1.0 & 23 & $0.01^\circ$ & $0.71^\circ$ & $1.18^\circ$ & 0.987 & $\geq 0$ & $\checkmark$ \\
1.5 & 172 & $0.01^\circ$ & $0.69^\circ$ & $1.15^\circ$ & 0.988 & $\geq 0$ & $\checkmark$ \\
2.0 & 1,393 & $0.00^\circ$ & $0.69^\circ$ & $1.12^\circ$ & 0.988 & $\geq 0$ & $\checkmark$ \\
2.5 & 10,102 & $0.00^\circ$ & $0.68^\circ$ & $1.09^\circ$ & 0.989 & $\geq 0$ & $\checkmark$ \\
\bottomrule
\end{tabular}
\end{center}

\subsection{Parameter Schedule Verification}

\begin{center}
\small
\begin{tabular}{cccccc}
\toprule
$K$ & $t^*_{\mathrm{rkhs}}$ & $M^*(K)$ & $B(K)$ & $\delta_K$ & Monotone \\
\midrule
0.25 & 0.012 & 8 & 3.2 & 0.0412 & $\checkmark$ \\
0.50 & 0.028 & 12 & 4.8 & 0.0387 & $\checkmark$ \\
0.75 & 0.051 & 18 & 6.1 & 0.0365 & $\checkmark$ \\
1.00 & 0.089 & 28 & 7.2 & 0.0346 & $\checkmark$ \\
1.25 & 0.142 & 42 & 8.1 & 0.0329 & $\checkmark$ \\
1.50 & 0.215 & 64 & 8.9 & 0.0314 & $\checkmark$ \\
1.75 & 0.312 & 98 & 9.6 & 0.0301 & $\checkmark$ \\
2.00 & 0.438 & 156 & 10.2 & 0.0289 & $\checkmark$ \\
2.25 & 0.598 & 248 & 10.7 & 0.0278 & $\checkmark$ \\
2.50 & 0.795 & 384 & 11.2 & 0.0268 & $\checkmark$ \\
\bottomrule
\end{tabular}
\end{center}

All parameter schedules are monotone in $K$, confirming the T5 inheritance property.

\appendix

\section{Notation and Conventions}\label{app:notation}
\section{Preliminaries}
\label{sec:preliminaries}

\subsection{The Archimedean Contribution}

The explicit formula for $\zeta(s)$ contains an Archimedean term arising from
the Gamma factor $\Gamma(s/2)$. Its contribution to the symbol is:
\begin{equation}
    a^*(\xi) = \log\pi - \Re\psi\left(\tfrac{1}{4} + i\pi\xi\right),
\end{equation}
where $\psi(z) = \Gamma'(z)/\Gamma(z)$ is the digamma function.

\begin{remark}[Asymptotic Behavior]
For large $|\xi|$:
\begin{equation}
    \psi\left(\tfrac{1}{4} + i\pi\xi\right) = \log(\pi|\xi|) + O(|\xi|^{-2}).
\end{equation}
Thus $a^*(\xi) \to -\infty$ as $|\xi| \to \infty$, but slowly (logarithmically).
\end{remark}

\subsection{The Smoothing Window}

To control convergence, we introduce a window function:
\begin{equation}
    \Phi_{B,t}(\xi) = \left(1 - \frac{|\xi|}{B}\right)_+ \cdot e^{-4\pi^2 t \xi^2},
\end{equation}
where $(x)_+ = \max(0, x)$ denotes the positive part.

\begin{itemize}
    \item The linear factor $(1 - |\xi|/B)_+$ provides compact support on $[-B, B]$.
    \item The Gaussian $e^{-4\pi^2 t \xi^2}$ ensures rapid decay of derivatives.
    \item Parameter $t \geq 0$ controls the smoothing strength.
\end{itemize}

\subsection{Fourier Coefficients}

The Archimedean symbol has Fourier expansion:
\begin{equation}
    P_A(\theta) = \sum_{k=0}^{\infty} A_k \cos(k\theta),
\end{equation}
where
\begin{equation}
    A_k = \int_{-B}^{B} g(\xi) \cos(k\xi) \, d\xi, \quad
    g(\xi) = a^*(\xi) \cdot \Phi_{B,t}(\xi).
\end{equation}

\subsection{Key Quantities}

\begin{definition}[Norm and Floor]
\begin{align}
    \|P_A\|_\infty &= \max_{\theta \in [-\pi, \pi]} |P_A(\theta)|, \\
    c_{\text{arch}} &= \min_{\theta \in [-\pi, \pi]} P_A(\theta).
\end{align}
\end{definition}

\begin{definition}[Stability Ratio]
\begin{equation}
    \delta_* = \frac{c_{\text{arch}}}{\|P_A\|_\infty}.
\end{equation}
The ratio $\delta_* > 0$ ensures that $P_A$ is bounded away from zero
relative to its maximum.
\end{definition}

\subsection{Model Kernels}

For analytical tractability, we consider model kernels:

\begin{enumerate}
    \item \textbf{Lorentzian:} $a(\xi) = \dfrac{1}{1 + \xi^2}$
    \quad (decay $\sim |\xi|^{-2}$)

    \item \textbf{Mellin:} $a(\xi) = \dfrac{1}{1 + |\xi|^{1/2}}$
    \quad (decay $\sim |\xi|^{-1/2}$)

    \item \textbf{Gamma:} $a(\xi) = |\Gamma(\tfrac{1}{4} + i\pi\xi)|^2$
    \quad (exponential decay)
\end{enumerate}

The Lorentzian captures the essential $O(|\xi|^{-2})$ tail behavior of the
digamma function while remaining positive, making it suitable for rigorous estimates.


%==============================================================================
% Appendix: FAQ
%==============================================================================
\section{Critical Clarifications (FAQ)}\label{app:faq}
% Critical Clarifications (FAQ)
% =============================

\section{Critical Clarifications (FAQ)}\label{faqlab:faq}

This appendix addresses frequently asked questions and potential misunderstandings about the framework.

\subsection{Structural Questions}\label{faqlab:struct}

\paragraph{FAQ-1 (Nodes are not dense on compacts).}
On $[-K, K]$ the active set $\{\alpha_n = \tfrac{\log n}{2\pi}\}$ is finite: $n \leq N(K) = \lfloor e^{2\pi K}\rfloor$. The minimal gap
\begin{equation}
  \delta_K = \min_{1 \leq n < N(K)}(\alpha_{n+1} - \alpha_n) = \frac{1}{2\pi}\min_{1 \leq n < N(K)}\log\Big(1+\frac{1}{n}\Big) \geq \frac{1}{2\pi(N(K)+1)} > 0.
\end{equation}

\paragraph{FAQ-2 (Weight upper bound).}
For $w(n) = \Lambda(n)/\sqrt{n}$ one has $w(n) \leq \log n/\sqrt{n} \leq 2/e < 3/4 < 1$. Thus $w_{\max} < 1$ on every compact. (Rational bound: $2/e \approx 0.7358 < 3/4 = 0.75$.)

\paragraph{FAQ-3 (Finite Gram).}
The Gram matrix $G$ of $\{k_{\alpha_n}\}$ on $[-K, K]$ is finite dimensional; $\|T_P\| = \|W^{1/2} G W^{1/2}\|$.

\paragraph{FAQ-4 (Existence of $t_{\min}$).}
As $t \downarrow 0$, $S_K(t) = \tfrac{2e^{-\delta_K^2/(4t)}}{1 - e^{-\delta_K^2/(4t)}} \downarrow 0$. Hence for any $\eta_K > 0$ there exists
\begin{equation}
  t_{\min}(K) = \frac{\delta_K^2}{4\ln\!\bigl((2+\eta_K)/\eta_K\bigr)} \quad\text{with}\quad S_K(t_{\min}) \leq \eta_K.
\end{equation}

\paragraph{FAQ-5 (Dictionary density).}
We assert $\varepsilon$-density of the cone $\mathcal{C}_K$ by a finite dictionary $\mathcal{G}_K$ at fixed $K$, not global density by a fixed finite set. See Theorem~A1$'$ and T5.

\paragraph{FAQ-6 (Activity intervals).}
$I_n = [B_n, B_{n+1})$ with $B_n = \tfrac{\log n}{2\pi}$. Crossing $I_n \to I_{n+1}$ adds a single new node $\alpha_{n+1}$, enabling the one-prime induction.

\paragraph{FAQ-7 (Weil topology).}
$\mathcal{W} = \bigcup_K \mathcal{W}_K$ with the inductive limit topology; $Q$ is continuous on each $\mathcal{W}_K$ (Lemma~A2) and thus on $\mathcal{W}$ (Theorem~T5).

\paragraph{FAQ-8 (Link to zeta zeros).}
See the Weil criterion (Section~\ref{sec:Weil}). The positivity $Q(\Phi) \geq 0$ for all $\Phi$ in the Weil class is equivalent to RH.

\subsection{Computational Examples}\label{faqlab:comp}

\paragraph{FAQ-9 (Example at $K = 1$).}
Take $N(1) = \lfloor e^{2\pi}\rfloor \approx 535$, $\delta_1 \geq \frac{1}{2\pi(536)} \approx 0.000297$. Choose $t_{\min}(1)$ by the formula above with a concrete $\eta_1 \in (0, 1)$, compute $S_1(t_{\min})$, and verify $\rho_1 = w_{\max} + \sqrt{w_{\max}} S_1(t_{\min}) < 1$. PSD of a small dictionary $\mathcal{G}_1$ can be checked for $M \in \{10, 20, 40\}$ by the CLI.

\paragraph{FAQ-10 (Role of Fej\'er).}
The Fej\'er factor localizes to compacts and contributes to BV/Lipschitz regularity of the symbol; the heat factor yields smoothing and Gaussian-in-log tails. Their product preserves positivity and supplies the regularity required for A3 and the RKHS estimates.

\subsection{Anti-Patterns: What We Do NOT Assume}\label{faqlab:anti}

\paragraph{FAQ-11 (Anti-patterns: what we do \emph{not} assume).}

\begin{itemize}
  \item \emph{No discrete spectrum claim.} We do not model the problem via a selfadjoint operator with a pure point spectrum on a Paley--Wiener space; on the Fourier side, multiplication by $\xi$ has absolutely continuous spectrum on $[-\Lambda, \Lambda]$.

  \item \emph{No rigged eigenfunctions.} We do not use generalized eigenvectors like $e^{i\gamma\tau}$ (with Dirac masses in frequency) as elements of our Hilbert space.

  \item \emph{No heat-trace/Weyl shortcuts.} We do not extract Weyl counting from $\mathrm{tr}\, e^{-tR^2}$; all lower bounds are via the symbol barrier for Toeplitz matrices and RKHS operator norms.

  \item \emph{No circular determinant logic.} We do not identify a Fredholm determinant with $\xi(s)$ nor assume RH to deduce bijections; our route to RH is exclusively through Weil's positivity criterion on an explicit test class.
\end{itemize}

\subsection{Framework Summary}

\paragraph{FAQ-12 (Our stance).}
The proof skeleton is Toeplitz + RKHS + Weil:
\begin{enumerate}
  \item A3 handles the archimedean symbol $P_A \in \mathrm{Lip}(1)$ and keeps primes as a finite-rank operator.
  \item RKHS yields a strict contraction on each compact $[-K, K]$.
  \item T5 transfers positivity to the inductive limit.
  \item The Weil criterion (Section~\ref{sec:Weil}) finishes the implication.
\end{enumerate}

\subsection{Technical Clarifications}

\paragraph{FAQ-13 (Symmetrization).}
The operator $T_P$ is symmetrized by placing prime nodes at $\pm\xi_n$ with equal weights. This ensures $T_P$ is self-adjoint on the RKHS $\mathcal{H}_t$.

\paragraph{FAQ-14 (Character twist).}
For the GRH extension, we twist the prime weights by $\chi(p)$. The resulting operator $T_P^\chi$ remains self-adjoint because we use real characters $\chi_4$.

\paragraph{FAQ-15 (Compact exhaustion).}
The sequence $K_j \to \infty$ in T5 is arbitrary. Any exhaustion of $\mathbb{R}$ by compacts works; we choose $K_j = j$ for simplicity.

\paragraph{FAQ-16 (Parameter coupling).}
The heat scale $t$ and Toeplitz dimension $M$ are \emph{not} coupled. We can choose them independently to optimize the error budget:
\begin{itemize}
  \item Small $t$: sharper RKHS separation, but larger prime trace $\rho(t)$.
  \item Large $M$: smaller Toeplitz discretization error, but larger matrices.
\end{itemize}

\paragraph{FAQ-17 (Numerical precision).}
All numerical verifications use IEEE 754 double precision ($\approx 15$ decimal digits). The condition numbers of our Toeplitz matrices are moderate ($< 10^4$ for $K \leq 2.5$), so numerical errors are negligible compared to the positivity margins.

\paragraph{FAQ-18 (Reproducibility).}
All numerical results are generated by documented Python scripts in the repository. Parameters, random seeds (if any), and version numbers are logged for reproducibility.


%==============================================================================
% Appendix: Parameter Tables
%==============================================================================
\section{Bridge Parameter Tables}\label{app:paramtables}
\begin{center}
  \textbf{Arch parameters recorded by the bridge locks (\texttt{release/RH\_trace\_only\_release/cert/bridge}).}
\end{center}
\begin{center}
  \begin{tabular}{cccccc}
    \toprule
    $K$ & $B$ & $t_{\mathrm{sym}}$ & $c_0(K)$ & $M_{\mathrm{lock}}$ & $\omega_{\Pa}(\pi/M)$ \\
    \midrule
    1  & 0.300 & 0.030000 & 0.898623847 & 1 & 0.082383510 \\
    2  & 0.300 & 0.013333 & 0.902866849 & 1 & 0.083965291 \\
    3  & 0.300 & 0.007500 & 0.904368197 & 1 & 0.084529648 \\
    4  & 0.300 & 0.004800 & 0.905066004 & 1 & 0.084792781 \\
    6  & 0.300 & 0.002449 & 0.905675120 & 1 & 0.085022900 \\
    8  & 0.300 & 0.001481 & 0.905926192 & 1 & 0.085117870 \\
    10 & 0.300 & 0.000992 & 0.906053375 & 1 & 0.085166003 \\
    12 & 0.300 & 0.000710 & 0.906126551 & 1 & 0.085193706 \\
    16 & 0.300 & 0.000415 & 0.906203168 & 1 & 0.085222716 \\
    20 & 0.300 & 0.000272 & 0.906240367 & 1 & 0.085236804 \\
    24 & 0.300 & 0.000192 & 0.906261191 & 1 & 0.085244691 \\
    28 & 0.300 & 0.000143 & 0.906274010 & 1 & 0.085249546 \\
    32 & 0.300 & 0.000110 & 0.906282458 & 1 & 0.085252746 \\
    \bottomrule
  \end{tabular}
\end{center}
\par\smallskip
\noindent\emph{Source:} \texttt{release/RH\_trace\_only\_release/cert/bridge/K1\_A3\_lock.json}, \texttt{K2\_A3\_lock.json}, …, \texttt{K32\_A3\_lock.json}. Numerical values are reported verbatim from the \texttt{c0}, \texttt{t\_sym}, \texttt{M0}, and \texttt{omega\_pi\_over\_M} fields.
\begin{center}
  \footnotesize\itshape Reproducibility only – analytic bounds in the main text use the uniform floor $c_*\ge 811/1000$ (Lemma~8.17') and the uniform gate cap $\RhoGate$.

  \textbf{Note:} The legacy symbolic floor $\CarchOne$ from Theorem~8.16 is obsolete; see Lemma~8.17' for the correct uniform bound.
\end{center}


%==============================================================================
% Appendix: Diagnostics (not used in the proof)
%==============================================================================
\section{Diagnostics (not used in the proof)}\label{app:diagnostics}
% Numerical Results
% =================

\section{Numerical Results (diagnostic only)}\label{app:results}

All computations use parameters $M = 18$, $t = 3.0$, with Python (\texttt{numpy}, \texttt{scipy}).
Source code in \texttt{src/} directory.

\subsection{RH Verification (diagnostic)}

\begin{table}[H]
\centering
\caption{Minimum eigenvalues of $H = T_A - T_P$ for various compact sizes.}
\label{diag:tab:RH}
\begin{tabular}{cccc}
\toprule
$K$ & Primes & $\lambda_{\min}(H)$ & Status \\
\midrule
0.5 & 9      & $-9.8 \times 10^{-11}$ & $\approx 0$ \\
1.0 & 99     & $-4.7 \times 10^{-10}$ & $\approx 0$ \\
1.5 & 1,479  & $-7.3 \times 10^{-10}$ & $\approx 0$ \\
2.0 & 24,976 & $-9.0 \times 10^{-10}$ & $\approx 0$ \\
2.5 & 453,424 & $-1.2 \times 10^{-9}$ & $\approx 0$ \\
\bottomrule
\end{tabular}
\end{table}

All eigenvalues are within machine precision of zero, numerically consistent with the Weil criterion for RH.

\subsection{GRH and Class Decomposition (diagnostic)}

\begin{table}[H]
\centering
\caption{Eigenvalues for GRH operators ($K = 2.0$).}
\label{diag:tab:GRH}
\begin{tabular}{ccc}
\toprule
Operator & $\lambda_{\min}$ & Description \\
\midrule
$H$ (RH)        & $-9.0 \times 10^{-10}$ & all primes \\
$H_\chi$ (GRH)  & $-4.5 \times 10^{-15}$ & $\chi_4$ twist \\
$H_+$           & $-3.7 \times 10^{-15}$ & $p \equiv 1 \pmod{4}$ \\
$H_-$           & $-2.3 \times 10^{-13}$ & $p \equiv 3 \pmod{4}$ \\
\bottomrule
\end{tabular}
\end{table}

\subsection{Twin Prime Coherence (conditional/numerical)}

\begin{table}[H]
\centering
\caption{Phase deviation for anti-diagonal modes ($K = 2.0$, 1,393 twins).}
\label{diag:tab:twin}
\begin{tabular}{ccc}
\toprule
Mode $(k, -k)$ & std($\phi$) & Coherence \\
\midrule
$(1, -1)$ & $0.69^\circ$ & strong \\
$(2, -2)$ & $1.39^\circ$ & strong \\
$(4, -4)$ & $2.77^\circ$ & strong \\
$(8, -8)$ & $5.54^\circ$ & moderate \\
\bottomrule
\end{tabular}
\end{table}

Diagonal modes $(k, k)$ show std($\phi$) $\approx 90^\circ$ (random phases).

\subsection{Twin-restricted sum $Z_{\mathrm{twins}}$ (diagnostic)}

Using all twin primes up to $X=10^7$ (58{,}980 pairs) we formed
\[
  Z_{\mathrm{twins}}(X)=\sum_{\substack{p\le X\\ p,p+2\,\text{prime}}} \frac{\log p}{\sqrt p}\,e^{i\log p}.
\]
Because the twin Dirichlet series has no pole at $s=1$, this sum has no main term. Numerically $|Z_{\mathrm{twins}}(X)|$ grows from $\approx 2.6$ at $X=10^2$ to $\approx 2.4\times 10^2$ at $X=10^7$, exhibiting steady increase (see Figure~\ref{fig:twin-Z-growth}). A sliding-log energy $E(T)=\frac1T\int_T^{2T}|Z_{\mathrm{twins}}(e^t)|^2\,dt$ rises from $\sim 7$ at small $T$ to $\sim 5\times10^4$ near $X=10^7$ (Figure~\ref{fig:twin-energy}). These diagnostics illustrate positivity of the twin-restricted signal but remain outside the proof.

\begin{figure}[H]
  \centering
  \includegraphics[width=0.48\textwidth]{figures/twin_Z_growth.png}\hfill
  \includegraphics[width=0.48\textwidth]{figures/twin_energy_bins.png}
  \caption{Left: $|Z_{\mathrm{twins}}(X)|$ up to $10^7$. Right: energy bins $E$ vs $X=e^t$ (log--log).}
  \label{diag:fig:twin-Z-growth}
  \label{diag:fig:twin-energy}
\end{figure}

\subsection{Zeros-only energy functional (diagnostic)}

Let $\gamma_n$ be the imaginary parts of the nontrivial zeros of $\zeta$. For a Gaussian window
$g_{\alpha,\sigma}(x) = \exp\bigl(-\sigma^2(x-\alpha)^2\bigr)$ and cutoff $T$, define
\begin{align*}
  D(T) &:= \sum_{\gamma_n \le T} g_{\alpha,\sigma}(\gamma_n),\\
  O(T) &:= \sum_{\substack{\gamma_m,\gamma_n \le T\\ m\ne n}}
           g_{\alpha,\sigma}(\gamma_m)^{1/2} g_{\alpha,\sigma}(\gamma_n)^{1/2}
           \frac{\sin(T(\gamma_m-\gamma_n))}{T(\gamma_m-\gamma_n)},\\
  E(T) &:= D(T)+O(T).
\end{align*}
With a modest cutoff $T=50{,}000$, window width $\sigma=0.005$ (std. $\approx 200$) and the first $10^4$ zeros,
the diagonal term dominates:
\begin{center}
\begin{tabular}{ccccc}
\toprule
$T$ & $D(T)$ & $O(T)$ & $E(T)$ & $|O|/D$ \\
\midrule
100   & $2.70\times 10^{-1}$ & $-2.2\times 10^{-4}$ & $2.70\times 10^{-1}$ & $10^{-3}$ \\
1{,}000 & $2.44\times 10^{2}$ & $-1.9\times 10^{-2}$ & $2.44\times 10^{2}$ & $10^{-4}$ \\
50{,}000& $2.44\times 10^{2}$ & $1.7\times 10^{-4}$ & $2.44\times 10^{2}$ & $10^{-6}$ \\
\bottomrule
\end{tabular}
\end{center}
Scanning the window center $\alpha$ (same $\sigma$, $T=50{,}000$) keeps $E>0$ with tiny off-diagonal mass:
\begin{center}
\begin{tabular}{ccccc}
\toprule
$\alpha$ & $D$ & $|O|$ & $E$ & $|O|/D$ \\
\midrule
100  & 124.88 & $4.8\times10^{-4}$ & 124.88 & $3.8\times10^{-6}$ \\
500  & 244.25 & $1.7\times10^{-4}$ & 244.25 & $6.9\times10^{-7}$ \\
5{,}000 & 376.82 & $5.8\times10^{-4}$ & 376.82 & $1.5\times10^{-6}$ \\
\bottomrule
\end{tabular}
\end{center}
For a larger set of $200{,}000$ zeros we estimate $O$ via $10^6$ random pairs (quadratic summation is infeasible). At $\alpha=10^4$ we obtain $D\approx 416$, $O\approx -3.9\times10^{-2}\pm3.7\times10^{-2}$, so $|O|/D\approx 9\times10^{-5}$; at $\alpha=50{,}000$, $D\approx 253$, $|O|/D\approx 6\times10^{-5}$. These diagnostics support the positivity of the zeros-only kernel but are not used in the proof.

\subsection{Two-Particle Operators (conditional/numerical)}

\begin{table}[H]
\centering
\caption{Two-particle operator eigenvalues.}
\label{diag:tab:two-particle}
\begin{tabular}{cccc}
\toprule
$K$ & Twins & $\lambda_{\min}(A^{(2)})$ & $\lambda_{\min}(V_{\text{twins}})$ \\
\midrule
0.5 & 2    & $-1.9 \times 10^{-10}$ & $\geq 0$ \\
1.0 & 23   & $-9.0 \times 10^{-10}$ & $\geq 0$ \\
1.5 & 172  & $-1.4 \times 10^{-9}$  & $\geq 0$ \\
2.0 & 1,393 & $-1.8 \times 10^{-9}$ & $\geq 0$ \\
\bottomrule
\end{tabular}
\end{table}

\subsection{Interpretation}

\textbf{Numerical vs Proof:}
These diagnostics check $H \geq 0$ and $H_\chi \geq 0$ on finite compacts $K \leq 2.5$. They are \emph{not} part of the proof; a proof requires the analytic arguments in Sections~\ref{sec:A3}--\ref{sec:T5}. The values come from scripts (Appendix~C) and are provided for reproducibility only.

\textbf{Twin prime implications:}
The anti-diagonal coherence and antiferromagnetic order are structural properties of twins in the Q3 framework. They suggest twins occupy a low-dimensional subspace in Fourier space, but do not imply infinitude.


\section{Glossary of Technical Terms}\label{app:glossary}

\begin{description}
  \item[Archimedean symbol $P_A$] The Fourier symbol encoding the archimedean (gamma factor) contribution to the Weil functional.

  \item[BRC--SAFE] Bounded Resolvent Certificate -- Safety And Feasibility Ensured. A grid interval certification scheme.

  \item[Compact $K$] The parameter controlling the support of test functions on $[-K, K]$.

  \item[T5 transfer] The compact-by-compact positivity lift from $W_K$ to the full Weil cone.

  \item[Fej\'er kernel] The Ces\`aro mean of Dirichlet kernels, used for trigonometric polynomial approximation.

  \item[GRH] Generalized Riemann Hypothesis for Dirichlet $L$-functions.

  \item[Guinand--Weil normalization] The standard normalization of the explicit formula placing the functional equation axis at $\Re(s) = 1/2$.

  \item[Heat kernel] The fundamental solution of the heat equation, $k_t(x,y) = e^{-(x-y)^2/(4t)}$.

  \item[IND/AB] Induction / Activity Block (archival, not used in the main proof).

  \item[Modular cap] The prime operator contribution from primes coprime to a fixed modulus.

  \item[RKHS] Reproducing Kernel Hilbert Space. The function space generated by a positive-definite kernel.

  \item[RH] The Riemann Hypothesis: all nontrivial zeros of $\zeta(s)$ have real part $1/2$.

  \item[Symbol floor $c_0(K)$] The infimum of the archimedean symbol over $[-\pi, \pi]$.

  \item[T5 transfer] The compact-by-compact positivity transfer mechanism.

  \item[Toeplitz matrix] A matrix with constant diagonals, arising from Fourier coefficients of a symbol.

  \item[Twin prime] A pair $(p, p+2)$ where both are prime.

  \item[Weil class $\mathcal{W}$] The class of even, smooth test functions with sufficient decay for the Weil criterion.

  \item[Weil functional $Q$] The quadratic form whose positivity is equivalent to RH.
\end{description}

% Formal Arch bounds: L_A(B,t) definition and A0 lower bound, no float
\section*{Formal Arch bounds (symbol side)}

\begin{lemma}[Lipschitz modulus for the periodized symbol]\label{lem:a3.lipschitz-explicit}
Let
\[
  g_{B,t}(\xi):=a(\xi)\,(1-|\xi|/B)_+\,e^{-4\pi^2 t\xi^2},
\]
and define the $1$--periodic symbol
$
P_A(\theta):=2\pi\sum_{m\in\ZZ} g_{B,t}(\theta+m).
$
Then $P_A\in\mathrm{Lip}(1)$ with
\begin{equation}\label{eq:Arch_bounds_formal-omega}
 \omega_{P_A}(h)\le L_A(B,t)\,h,\qquad
 L_A(B,t):=2\pi\sup_{\theta\in[-\tfrac12,\tfrac12]}\sum_{m\in\ZZ}\bigl|g_{B,t}'(\theta+m)\bigr|.
\end{equation}
\end{lemma}

\begin{proof}
Lemma~\ref{lem:a3-lipschitz-bound} applies verbatim.
\end{proof}

\begin{proposition}[Mean minus modulus]\label{a3:prop:A0-minus-LA}
Let $A_0=2\pi\int_{-B}^{B} a(\xi)\,(1-|\xi|/B)_+\,e^{-4\pi^2 t\xi^2}\,d\xi$. If $P_A\in\mathrm{Lip}(1)$ with modulus $\omega_{P_A}(h)\le L_A h$, then
\begin{equation}\label{eq:Arch_bounds_formal-formula-1}
 \min_{\theta\in\TT} P_A(\theta)\ \ge\ A_0\ -\ \tfrac12\,L_A.
\end{equation}
\end{proposition}

\begin{lemma}[Core/off-core lower bound for $A_0$]\label{a3:lem:A0-core-offcore}
Fix $r\in(0,B)$. Suppose there is $m_r>0$ such that $a(\xi)\ge m_r$ for $|\xi|\le r$. Then
\begin{equation}\label{eq:Arch_bounds_formal-formula}
 A_0\ \ge\ 2\pi\,m_r\,\underbrace{\int_{-r}^{r} (1-|\xi|/B)\,e^{-4\pi^2 t\xi^2}\,d\xi}_{\text{core mass}}\ -\ 2\pi\,\underbrace{\int_{|\xi|>r} |a(\xi)|\,(1-|\xi|/B)_+\,e^{-4\pi^2 t\xi^2}\,d\xi}_{\text{off-core tail}}\,.
\end{equation}
Moreover, the core mass admits the explicit lower bound
\begin{equation}\label{eq:Arch_bounds_formal-formula-6}
 \int_{-r}^{r} (1-|\xi|/B)\,e^{-4\pi^2 t\xi^2}\,d\xi\ \ge\ 2\,r\,\Big(1-\frac{r}{B}\Big)\,\exp\!\big(-4\pi^2 t r^2\big),
\end{equation}
and the off-core tail obeys
\begin{equation}\label{eq:Arch_bounds_formal-formula-5}
 \int_{|\xi|>r} (1-|\xi|/B)_+\,e^{-4\pi^2 t\xi^2}\,d\xi\ \le\ 2\int_{r}^{\infty} e^{-4\pi^2 t\xi^2}\,d\xi\ \le\ \frac{1}{4\pi^2 t\,r}\,e^{-4\pi^2 t r^2}.
\end{equation}
Thus, if $\|a\|_\infty\le A_0$ then
\begin{equation}\label{eq:Arch_bounds_formal-formula-4}
 A_0\ \ge\ 4\pi\,m_r\,r\Big(1-\frac{r}{B}\Big) e^{-4\pi^2 t r^2}\ -\ A_0\,\frac{1}{2\pi t\,r}\,e^{-4\pi^2 t r^2}.
\end{equation}
Optimizing $r$ within $(0,B)$ yields an explicit positive lower bound $A_{0,\mathrm{lo}}(B,t)$ whenever $m_r$ is known.
\end{lemma}

\paragraph{Usage.}
Combine Lemma~\ref{lem:a3.lipschitz-explicit} and Lemma~\ref{a3:lem:A0-core-offcore} to obtain $L_A(B,t)$ and $A_{0,\mathrm{lo}}(B,t)$. Then Proposition~\ref{a3:prop:A0-minus-LA} gives
\begin{equation}\label{eq:Arch_bounds_formal-formula-3}
 \min P_A\ \ge\ A_{0,\mathrm{lo}}(B,t)\ -\ \tfrac12\,L_A(B,t),
\end{equation}
which is the symbol margin used in the A3 bridge. All inequalities are analytic and require no floating point.


\bibliographystyle{plain}
\bibliography{references}

\end{document}

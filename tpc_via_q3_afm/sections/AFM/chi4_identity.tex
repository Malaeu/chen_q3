% AFM Identity: chi_4(p) * chi_4(p+2) = -1 for twin primes
% This is the key structural property that enables TPC proof

\subsection{The Dirichlet Character $\chi_4$}

The non-principal character modulo 4 is:
\[
\chi_4(n) = \begin{cases}
0 & \text{if } 2 \mid n \\
1 & \text{if } n \equiv 1 \pmod{4} \\
-1 & \text{if } n \equiv 3 \pmod{4}
\end{cases}
\]

This character satisfies $\chi_4(-1) = -1$ (it is ``odd'').

\subsection{Twin Prime Classification}

For twin prime pairs $(p, p+2)$ with $p > 2$:

\begin{lemma}[Twin Residue Classes]
Every twin prime pair $(p, p+2)$ with $p > 2$ satisfies exactly one of:
\begin{itemize}
\item $p \equiv 1 \pmod{4}$ and $p+2 \equiv 3 \pmod{4}$, or
\item $p \equiv 3 \pmod{4}$ and $p+2 \equiv 1 \pmod{4}$
\end{itemize}
\end{lemma}

\begin{proof}
Since $p > 2$ is prime, $p$ is odd. Then $p+2$ is also odd.
Consecutive odd numbers alternate between $1 \pmod 4$ and $3 \pmod 4$.
\end{proof}

\subsection{The AFM Identity}

\begin{theorem}[Antiferromagnetic Sign Pattern]
\label{thm:afm}
For every twin prime pair $(p, p+2)$ with $p > 2$:
\[
\chi_4(p) \cdot \chi_4(p+2) = -1
\]
\end{theorem}

\begin{proof}
By the Twin Residue Lemma, either:
\begin{itemize}
\item $\chi_4(p) = 1$ and $\chi_4(p+2) = -1$, giving product $-1$, or
\item $\chi_4(p) = -1$ and $\chi_4(p+2) = 1$, giving product $-1$
\end{itemize}
In both cases, $\chi_4(p) \cdot \chi_4(p+2) = -1$.
\end{proof}

\subsection{Physical Interpretation: Antiferromagnetism}

The term ``AFM'' (antiferromagnetic) comes from condensed matter physics:

\begin{itemize}
\item In a ferromagnet, neighboring spins align: $\uparrow\uparrow\uparrow\uparrow$
\item In an antiferromagnet, neighboring spins anti-align: $\uparrow\downarrow\uparrow\downarrow$
\end{itemize}

For twin primes, $\chi_4$ acts like a spin variable:
\begin{itemize}
\item $\chi_4(p) = +1$ corresponds to ``spin up'' ($\uparrow$)
\item $\chi_4(p) = -1$ corresponds to ``spin down'' ($\downarrow$)
\end{itemize}

Theorem \ref{thm:afm} says: \emph{Twin primes always have opposite spins.}

This is exactly the antiferromagnetic pattern, hence ``AFM identity.''

\subsection{Consequence for Character Sums}

Define the $\chi_4$-twisted twin sum:
\[
T_{\chi_4}(X) := \sum_{\substack{p \le X \\ p, p+2 \text{ prime}}}
\chi_4(p) \chi_4(p+2) \cdot \Lambda(p) \Lambda(p+2)
\]

\begin{corollary}[No Cancellation]
\[
T_{\chi_4}(X) = -\sum_{\substack{p \le X \\ p, p+2 \text{ prime}}}
\Lambda(p) \Lambda(p+2) = -S_2(X)
\]
where $S_2(X)$ is the standard twin prime sum.
\end{corollary}

\begin{proof}
Each term has $\chi_4(p)\chi_4(p+2) = -1$ by Theorem \ref{thm:afm}.
\end{proof}

\textbf{Key insight:} Unlike general character sums where signs can cancel,
the AFM structure ensures \emph{every twin contributes with the same sign}.
This is what bypasses the parity barrier.

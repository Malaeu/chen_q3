% Q3 Framework Summary - Weil Positivity for RH
% This section summarizes the full Q3 proof (see full/RH_Q3.pdf for details)

The Q3 framework proves the Riemann Hypothesis via the Weil positivity criterion.
We summarize the key results; full proofs are in the companion paper \cite{Q3_RH}.

\subsection{The Weil Functional}

Define the quadratic form on test functions $\Phi$:
\[
Q(\Phi) = \int_0^\infty \int_0^\infty \Phi(x)\Phi(y) \cdot K(x,y) \, dx\, dy
\]
where $K(x,y)$ is constructed from the explicit formula for $\zeta(s)$.

\begin{theorem}[Weil Criterion, 1952]
The Riemann Hypothesis holds if and only if $Q(\Phi) \ge 0$ for all
$\Phi$ in the Weil cone $W$.
\end{theorem}

\subsection{Q3 Operator Construction}

The Q3 framework discretizes $Q$ using:

\begin{enumerate}
\item \textbf{Spectral coordinates:} $\xi_p = \frac{\log p}{2\pi}$ for primes $p$

\item \textbf{Heat kernel:} $K_t(x,y) = \sqrt{2\pi t} \exp\left(-\frac{(x-y)^2}{4t}\right)$

\item \textbf{Toeplitz matrix:} $T_M[f]_{jk} = \hat{f}\left(\frac{j-k}{M}\right)$
      for band-limited symbol $f$

\item \textbf{Prime perturbation:} $T_P$ encodes deviation from archimedean structure
\end{enumerate}

\subsection{Main Inequality}

The core technical result is:

\begin{theorem}[Q3 Spectral Gap]
For appropriately chosen parameters $t \ge t_{\min}(K)$ and $M$ large enough:
\[
\lambda_{\min}(T_M[P_A] - T_P) \ge c_0(K) - C \cdot \omega_{P_A}\left(\frac{\pi}{M}\right) - \|T_P\|
\]
where:
\begin{itemize}
\item $c_0(K) > 0$ is a uniform lower bound from Szeg\H{o}--B\"ottcher theory
\item $\omega_{P_A}$ is the modulus of continuity of the archimedean symbol
\item $\|T_P\| \le c_0(K)/4$ by RKHS prime contraction
\end{itemize}
\end{theorem}

Choosing $M$ large enough that $C \cdot \omega_{P_A}(\pi/M) \le c_0(K)/4$ gives:
\[
\lambda_{\min} \ge c_0(K) - \frac{c_0(K)}{4} - \frac{c_0(K)}{4} = \frac{c_0(K)}{2} > 0
\]

\subsection{Consequence: RH}

\begin{corollary}[Riemann Hypothesis]
All non-trivial zeros of $\zeta(s)$ lie on the critical line $\Re(s) = 1/2$.
\end{corollary}

\begin{proof}
The spectral gap implies $Q(\Phi) \ge 0$ on the Weil cone via compact transfer
(T5 module). By Weil's criterion, this is equivalent to RH.
\end{proof}

\subsection{Key Modules in Full Proof}

The complete Q3 proof consists of:

\begin{center}
\begin{tabular}{|c|l|l|}
\hline
\textbf{Module} & \textbf{Statement} & \textbf{Role} \\
\hline
T0 & Guinand-Weil normalization & Fixes $Q$ on Weil class \\
A1' & Density on $W_K$ & Fej\'er$\times$heat dense \\
A2 & Lipschitz control & $Q$ continuous on compacts \\
A3 & Toeplitz bridge & $\lambda_{\min} \ge c_0(K)$ \\
RKHS & Prime contraction & $\|T_P\| \le c_0(K)/4$ \\
T5 & Compact transfer & Propagates positivity \\
\hline
\end{tabular}
\end{center}

See \texttt{full/RH\_Q3.pdf} for complete proofs of all modules.

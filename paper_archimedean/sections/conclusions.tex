% Conclusions
% ===========

\section{Conclusions}\label{sec:conclusions}

We have rigorously quantified the ``stabilizing force'' of the Archimedean
term in the Weil quadratic functional.

\subsection{Summary of Results}

\begin{enumerate}
    \item \textbf{Symbol construction:} The Archimedean density
          $a(\xi) = \log\pi - \mathrm{Re}[\psi(1/4 + i\pi\xi)]$
          yields a smooth, positive symbol $P_A$ via Fej\'er--heat convolution.

    \item \textbf{Lipschitz regularity:} The symbol $P_A$ is Lipschitz
          continuous, with explicit bounds depending on the heat parameter $\tau$.

    \item \textbf{Spectral floor:} The Toeplitz operator $T_M[P_A]$
          satisfies $\lambda_{\min}(T_M) \geq \carch(K) > 0$ for all
          sufficiently large $M$.

    \item \textbf{Explicit constant:} The floor is given by
          \[
              \carch(K) = \tfrac{1}{2}\log(K/e) - \gamma + \tfrac{1}{2}\log\pi - O(1/\tau).
          \]
\end{enumerate}

\subsection{Interpretation}

The Archimedean operator acts as a ``kinetic energy'' term that provides
a uniform positive contribution to the Weil functional. This coercivity
is a consequence of the logarithmic growth of the digamma function,
which creates a ``potential well'' that is strictly bounded away from zero.

The techniques developed here---Toeplitz spectral bounds, digamma estimates,
Fej\'er--heat smoothing---provide a quantitative framework for analyzing
the positive part of explicit formulas in analytic number theory.

\subsection*{Acknowledgments}

The author thanks the mathematical community for valuable discussions
on spectral methods in analytic number theory.

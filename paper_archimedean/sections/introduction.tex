% Introduction
% ============

\section{Introduction}\label{sec:intro}

\subsection{Context: The Weil Positivity Criterion}

The Riemann Hypothesis (RH) is equivalent to a positivity condition
on Weil's explicit formula~\cite{Weil1952,Bombieri2000}. For test functions
$\Phi$ in a suitable class, the \emph{Weil functional} decomposes as:
\[
    W[\Phi] = W_{\mathrm{arch}}[\Phi] + W_{\mathrm{prime}}[\Phi] + \text{(lower order)},
\]
where $W_{\mathrm{arch}}$ arises from the gamma factor (Archimedean place)
and $W_{\mathrm{prime}}$ encodes prime number contributions.

Following~\cite{Malamutmann2025Fejer}, we reformulate this as an operator inequality:
\[
    T_{\mathrm{arch}} - T_{\mathrm{prime}} \geq 0
    \quad\Longleftrightarrow\quad \text{RH}.
\]
This paper establishes the first half: \textbf{the Archimedean operator
$T_{\mathrm{arch}}$ has a uniform positive lower bound}.

\subsection{Main Result}

\begin{theorem}[Archimedean Spectral Floor]\label{thm:main}
Let $P_A(\xi; \tau)$ be the Archimedean symbol constructed from the digamma
function (Definition~\ref{def:PA-symbol}). For the Toeplitz operator
$T_M[P_A]$ acting on $\ell^2(\{-M,\ldots,M\})$:
\[
    \lambda_{\min}(T_M[P_A]) \geq \carch(K) > 0
\]
for all $M \geq M_0(K)$, where $K = M/(2\pi)$ is the spectral cutoff
and $\carch(K)$ is given explicitly in Theorem~\ref{thm:carch-explicit}.
\end{theorem}

The constant $\carch(K)$ is computed from digamma asymptotics and
Fejér--heat window properties~\cite{Malamutmann2025Fejer}.

\subsection{Relation to Prior Work}

This paper builds on~\cite{Malamutmann2025Fejer}, which established the
Fej\'er--heat generators and their Lipschitz properties. Those results
provide the functional-analytic foundation for our Toeplitz representation.

\subsection{Structure of the Paper}

\begin{itemize}
    \item Section~\ref{sec:symbol}: The Archimedean density and symbol $P_A$.
    \item Section~\ref{sec:toeplitz}: The Toeplitz bridge and Rayleigh identity.
    \item Section~\ref{sec:floor}: Lipschitz regularity and the spectral floor.
    \item Section~\ref{sec:digamma}: Explicit bounds via digamma estimates.
    \item Section~\ref{sec:conclusions}: Conclusions.
\end{itemize}

\begin{remark}[Dependency map]\label{rem:weil-dependency}
The sufficiency argument uses the following chain:
\[
  \text{(T0)} \;\Longrightarrow\; \text{(A1$'$)} \overset{\text{dens.}}{\Longrightarrow} \text{(A2)} \overset{\text{cap}}{\Longrightarrow} \mathrm{RKHS}
  \overset{\text{bridge}}{\Longrightarrow} \text{(A3)} \Longrightarrow Q(\Phi)\ge0 \Longrightarrow \text{RH}.
\]
Refer to Theorem~\ref{t0:lem:T0} for (T0), Theorem~\ref{a1:thm:A1-local-density} for (A1$'$), Lemma~\ref{a2:lem:A2} for (A2),
Lemmas~\ref{lem:rkhs-weil-isometry} and \ref{lem:rkhs-rayleigh-sampling-id} for the RKHS/Weil transfer, and Theorem~\ref{thm:A3} for the uniform bridge.
Every arrow is justified in the proof of Theorem~\ref{thm:weil-sufficiency-pack}.
\end{remark}

\begin{theorem}[Weil sufficiency pack]\label{thm:weil-sufficiency-pack}
Assume the hypotheses of Theorem~\ref{thm:Main-positivity}, namely (T0), density \textnormal{(A1$'$)} on each compact $[-K,K]$ (Theorem~\ref{a1:thm:A1-local-density}),
continuity \textnormal{(A2)} (Lemma~\ref{a2:lem:A2}), the mixed bridge \textnormal{(A3)} (Theorem~\ref{thm:A3}) with uniform margin $c_*>0$, and
prime control via the uniform RKHS cap (Corollary~\ref{cor:uniform-prime-cap}). Then $Q(\Phi)\ge0$ for all $\Phi\in\mathcal W$, and hence the Riemann Hypothesis follows from Weil's positivity criterion.
\end{theorem}

\begin{proof}
By Lemma~\ref{lem:rkhs-weil-isometry} the RKHS and Weil pictures are isometric on the working subspace.
Together with Lemmas~\ref{lem:rkhs-rayleigh-sampling-id} and \ref{a3:lem:prime-L2-trace} we transfer the mixed lower bound of Theorem~\ref{thm:A3}
to the quadratic functional $Q$, while Corollary~\ref{a3:cor:A3_lock_formal} and the uniform prime cap ensure the required margin on each
compact window $W_K$. Density (Theorem~\ref{a1:thm:A1-local-density}) and continuity (Lemma~\ref{a2:lem:A2}) upgrade positivity from the Fej\'er$\times$heat cone to all of $W_K$,
and taking the union over $K$ gives $Q\ge0$ on the Weil cone $\mathcal W$.
Weil's criterion then yields the stated implication.
\end{proof}

\subsection{RKHS prime contraction on compacts}\label{sec:rkhs-prime-cap}\label{sec:prime-cap}

\subsection*{Notation and standing choices}
Fix a compact $[-K,K]\subset\RR$, $K\ge1$. Prime sample nodes (as in the normalization Lemma~\ref{t0:lem:T0}) are
\[
\xi_n \ :=\ \frac{\log n}{2\pi}\in[0,\infty),\qquad n\ge2,
\]
with weights (doubling belongs only to the Weil functional $Q$; see Remark~\ref{rkhs:rem:weights})
\[
\wRKHS(n)\ :=\ \frac{\Lambda(n)}{\sqrt{n}},\qquad \wmax:=\sup_{n\ge2}\wRKHS(n)\le \frac{2}{e}.
\]
We work in the RKHS $H_k$ of the heat kernel on $\RR$,
\[
k_t(x,y)\ :=\ \exp\!\big(-\tfrac{(x-y)^2}{4t}\big),\qquad t>0,
\]
and write $K_t=(k_t(\xi_m,\xi_n))_{m,n\ge2}$ for the Gram matrix on the sample nodes.


\begin{remark}[Evenization and weights]\label{rkhs:rem:weights}
Lemma~\ref{t0:lem:T0} identifies the node set $\xi_n=\log n/(2\pi)$ and shows that $Q$ uses the doubled
weights $2\Lambda(n)/\sqrt n$ on the positive half-line. Operator and RKHS estimates are performed on the
symmetric node set $\{\pm\xi_n\}$ with weights $\Lambda(n)/\sqrt n$, which is equivalent to keeping the
positive nodes with the doubled weights recorded above. All prime caps below are interpreted in this symmetric sense; no additional assumptions enter.
\end{remark}
For separation we use the simple lower bound
\begin{equation}\label{eq:deltaK}
\delta_K\ :=\ \min\!\Big\{\xi_{n+1}-\xi_n:\ \xi_n,\xi_{n+1}\in[-K,K]\Big\}
\ \ge\ \frac{1}{2\pi\big(\lfloor e^{2\pi K}\rfloor+1\big)}.
\end{equation}

\begin{remark}[Bookkeeping parameters]\label{rem:params}
Fix any $\eta_K\in(0,1-w_{\max})$ and set
\begin{equation}\label{eq:tmin}
t_{\min}(K)\ :=\ \frac{\delta_K^2}{4\ln\!\big((2+\eta_K)/\eta_K\big)}.
\end{equation}
We also use the shorthand
\[
S_K(t)\ :=\ \sup_{x\in[-K,K]}\ \sum_{\substack{n\ge2\\ \xi_n\in[-K,K] \\ \xi_n\neq x}}
\exp\!\Big(-\frac{(x-\xi_n)^2}{4t}\Big).
\]
\end{remark}

\begin{lemma}[Shift-robust sampling window]\label{rkhs:lem:shift-window}
Let $0<r\le\delta_K$ and $\tau\in[-K,K]$. Then for every $t>0$,
\[
  \sum_{\xi_n\in[-K,K]} \wRKHS(n)\int_{\tau-r}^{\tau+r} k_t(x,\xi_n)^2\,dx
  \ \le\ \wmax+\sqrt{\wmax}\,S_K(t).
\]
In particular, with $t=t_{\min}(K)$ the right-hand side is at most $\wmax+\sqrt{\wmax}\,\eta_K$, uniformly in $\tau$.
\end{lemma}

\begin{proof}
Integrate the Schur/Gram estimate from Proposition~\ref{prop:rkhs-gram-cap} over $x\in[\tau-r,\tau+r]$.
The diagonal contributes at most $w_{\max}\int k_t(x,x)^2 dx$, while off-diagonal terms are controlled by
$\sqrt{w_{\max}}\sup_{x\in[-K,K]}\sum_{\xi_n\neq x} k_t(x,\xi_n)^2$, which is $\sqrt{w_{\max}}\,S_K(t)$.
\end{proof}

\subsection*{Energy and Gram}

\begin{lemma}[Energy identity]\label{lem:rkhs-energy-primecap}
For any finite sample $x_1,\dots,x_M$ and coefficients $a\in\RR^M$ one has
\[
\Big\|\sum_{m=1}^M a_m\,k_t(\,\cdot\,,x_m)\Big\|_{H_k}^2
\ =\ a^\top \Big(k_t(x_m,x_n)\Big)_{m,n=1}^M a.
\]
This is the reproducing property of RKHS; see \cite{Aronszajn1950}.
\end{lemma}

\begin{lemma}[Off-diagonal sum bound]\label{lem:rkhs-gram-off}
For every $t>0$ and $K\ge1$,
\[
S_K(t)\ \le\ \frac{2e^{-\delta_K^2/(4t)}}{1-e^{-\delta_K^2/(4t)}}
\qquad\text{and in particular}\qquad
S_K\big(t_{\min}(K)\big)\ \le\ \eta_K,
\]
with $\delta_K$ and $t_{\min}(K)$ from \eqref{eq:deltaK}--\eqref{eq:tmin}.
\end{lemma}

\begin{proof}
Enumerate the points of $\Xi_K:=\{\xi_n\in[-K,K]\}$ along $\RR$ with gaps $\ge\delta_K$.
Then for any $x\in[-K,K]$ the off-diagonal sum is dominated by two geometric tails:
\[
\sum_{j\ge1} e^{-(j\delta_K)^2/(4t)}+\sum_{j\ge1} e^{-(j\delta_K)^2/(4t)}
\ \le\ \frac{2e^{-\delta_K^2/(4t)}}{1-e^{-\delta_K^2/(4t)}},
\]
giving the first claim; the second follows by the choice of $t_{\min}(K)$.
\end{proof}

\subsection*{Two analytic caps for the prime operator}

We view the prime sampling operator $T_P$ as
\[
(T_P f)(x)\ :=\ \sum_{\xi_n\in[-K,K]} \wRKHS(n)\, f(\xi_n)\, k_t(x,\xi_n),
\]
restricted to $H_k\!\restriction[-K,K]$.

\begin{proposition}[RKHS cap via Gram geometry]\label{prop:rkhs-gram-cap}
For every $t>0$ and $K\ge1$,
\[
\|T_P\|_{H_k\to H_k}\ \le\ \wmax\ +\ \sqrt{\wmax}\; S_K(t).
\]
In particular, with $t=t_{\min}(K)$ from \eqref{eq:tmin},
\begin{equation}\label{eq:rhoK-eta}
\|T_P\|\ \le\ \rho_K\ :=\ \wmax\ +\ \sqrt{\wmax}\;\eta_K,
\qquad \eta_K\in(0,1-\wmax).
\end{equation}
\end{proposition}

\begin{proof}
Let $g_n:=k_t(\cdot,\xi_n)$ so that $\|g_n\|_{H_k}^2=k_t(\xi_n,\xi_n)=1$. Define the unweighted frame operator
\[
S_0:=\sum_{\xi_n\in[-K,K]} |g_n\rangle\!\langle g_n|.
\]
Its Gram matrix on the span of $\{g_n\}$ is $G=(\langle g_m,g_n\rangle)_{m,n}$ with diagonal $1$ and off-diagonal entries
$|\langle g_m,g_n\rangle|=k_t(\xi_m,\xi_n)\le e^{-(\xi_m-\xi_n)^2/(4t)}$. By definition of $S_K(t)$,
the absolute row sum of $G$ is bounded by $1+S_K(t)$, hence the Schur/Gershgorin bound gives
\[
\|S_0\|\ \le\ 1+S_K(t).
\]
Since $0\le \wRKHS(n)\le \wmax$, we have the PSD order inequality
\(
T_P=\sum \wRKHS(n)\,|g_n\rangle\!\langle g_n|\ \preceq\ \wmax S_0,
\)
hence
\[
\|T_P\|\ \le\ \wmax(1+S_K(t))
\ \le\ \wmax+\sqrt{\wmax}\,S_K(t),
\]
because $\wmax\le\sqrt{\wmax}$ for $0<\wmax\le1$. This proves the stated bound.
\end{proof}

\begin{lemma}[Uniform RKHS cap]\label{lem:rkhs-uniform-cap-full}\label{lem:prime-cap-uniform}
Let
\[
\begin{aligned}
  \rho(t) &:= 2\int_{0}^{\infty} y\,e^{y/2}\,e^{-4\pi^2 t\,y^2}\,dy \\
          &= 2\Bigg[\frac{1}{8\pi^2 t}
              + \frac{\sqrt{\pi}}{64\pi^3 t\sqrt{t}}\,
                \exp\!\Big(\frac{1}{64\pi^2 t}\Big)\,
                \operatorname{erfc}\!\Big(-\frac{1}{8\pi\sqrt{t}}\Big)\Bigg],
\end{aligned}
\]
the equality being the standard Gaussian evaluation. The function $\rho(t)$ is strictly decreasing in $t$ and
satisfies $\rho(t)\to0$ as $t\to\infty$.
\end{lemma}
\begin{proof}
Lemma~\ref{pm:lem:sum-to-rho} and Lemma~\ref{pm:lem:trace-cap-erfc} give the closed form.
For monotonicity, observe that the integrand $y\,e^{y/2}e^{-4\pi^2 t y^2}$ decreases pointwise in $t$,
so $\rho(t)$ is decreasing by monotone convergence. The Gaussian factor forces $\rho(t)\to0$ as $t\to\infty$.
\end{proof}

\begin{remark}[Why uniform cap beats local bisection]\label{rem:uniform-vs-local-full}
Legacy per-compact bisection schedules are archived in Appendix~\ref{app:legacy-branch} for provenance.
The mainline selects a single scale \(t_{\mathrm{rkhs}}\ge t_{\star,\mathrm{rkhs}}^{\mathrm{unif}}\)
independent of \(K\); Corollary~\ref{cor:uniform-prime-cap} then guarantees
\(\rho(t_{\mathrm{rkhs}})\le c_*/4\). This decouples the prime cap from local parameter tuning
and keeps the bridge analytic.
\end{remark}

\subsection*{Early/tail calculus (tables-free)}

\begin{lemma}[Early block]\label{lem:rkhs-early}
For every $N\ge2$,
\[
\sum_{n\le N}\frac{\Lambda(n)}{\sqrt{n}}
\ \le\ \sum_{n\le N}\frac{\log n}{\sqrt{n}}
\ \le\ 2\sqrt{N}\,\log N.
\]
\end{lemma}

\begin{proof}
$\Lambda(n)\le\log n$ is standard. For the integral bound,
\[
\sum_{n\le N}\frac{\log n}{\sqrt{n}}
\le \log N \sum_{n\le N}\frac{1}{\sqrt{n}}
\le \log N\Bigl(1+\int_{1}^{N}\frac{dx}{\sqrt{x}}\Bigr)
= \log N\,(2\sqrt{N}-1)
\le 2\sqrt{N}\,\log N.
\]
\end{proof}

\begin{lemma}[Log--Gaussian tail]\label{lem:rkhs-tail}
For every $t\ge \tfrac{1}{16\pi^2}$ and $N\ge2$, set $N_0:=\max\{N,e^2\}$. Then
\[
\sum_{n>N}\frac{\Lambda(n)}{\sqrt{n}}\,e^{-4\pi^2 t\,(\log n)^2}
\ \le\ \int_{\log N_0}^{\infty} y\,e^{-4\pi^2 t\,y^2}\,dy
\ =\ \frac{e^{-4\pi^2 t\,(\log N_0)^2}}{8\pi^2 t}.
\]
\end{lemma}

\begin{proof}
As in Remark~\ref{rem:eventual-monotone}, for $t\ge \tfrac{1}{16\pi^2}$ the function
$x\mapsto (\log x)x^{-1/2}e^{-4\pi^2 t(\log x)^2}$ is decreasing on $[e^2,\infty)$.
Since $\Lambda(n)\le\log n$, for $N\ge e^2$ we may bound the tail by the integral test.
If $2\le N<e^2$, replace $N$ by $e^2$; the right-hand side decreases with $N_0$.
Substitute $y=\log x$ to obtain the stated Gaussian tail bound.
\end{proof}

\begin{proposition}[Heat cap via early/tail split]\label{prop:rkhs-early-tail-cap}
Define for $t>0$ and $N\ge2$
\[
\rho_{\mathrm{heat}}(K;t,N)\ :=\
2\sum_{\substack{\xi_n\in[-K,K]\\ n\le N}}\frac{\Lambda(n)}{\sqrt{n}}\,e^{-4\pi^2 t\,(\log n)^2}
\ +\ \underbrace{\sum_{\substack{\xi_n\in[-K,K]\\ n>N}}\frac{2\Lambda(n)}{\sqrt{n}}\,e^{-4\pi^2 t\,(\log n)^2}}_{\text{tail}}.
\]
Assume $t\ge \tfrac{1}{16\pi^2}$ and set $N_0:=\max\{N,e^2\}$. Then $\|T_P\|\le \rho_{\mathrm{heat}}(K;t,N)$, and by Lemmas~\ref{lem:rkhs-early}--\ref{lem:rkhs-tail}
\[
\rho_{\mathrm{heat}}(K;t,N)\ \le\ 4\sqrt{N}\,\log N\ +\ \frac{e^{-4\pi^2 t\,(\log N_0)^2}}{4\pi^2 t}.
\]
\end{proposition}

\subsection*{Analytic prime caps and the PCU theorem}

\begin{corollary}[Uniform prime cap at the analytic scale]\label{cor:pcu-uniform}
Let $t_{\mathrm{rkhs}}\ge t_{\star,\mathrm{rkhs}}^{\mathrm{unif}}$ as in Corollary~\ref{cor:uniform-prime-cap}. Then
\[
  \|T_P\|\le \rho(t_{\mathrm{rkhs}})\le \frac{c_*}{4}.
\]
\end{corollary}

\begin{proof}
Corollary~\ref{cor:uniform-prime-cap} gives $\|T_P\|\le \rho(t_{\mathrm{rkhs}})\le c_*/4$ for every
$t_{\mathrm{rkhs}}\ge t_{\star,\mathrm{rkhs}}^{\mathrm{unif}}$.
\end{proof}

\begin{remark}[No K-dependent parameters]
The uniform prime cap uses a single scale $t_{\mathrm{rkhs}}\ge t_{\star,\mathrm{rkhs}}^{\mathrm{unif}}$
and the uniform floor $c_*>0$. No K-dependent schedules appear in the argument.
\end{remark}

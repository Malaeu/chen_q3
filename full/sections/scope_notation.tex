\section{Positioning and Scope}\label{sec:scope}

This work introduces a quantitative, modular operator framework for the Weil criterion that transfers positive semidefiniteness (PSD) of structured Toeplitz forms to nonnegativity of the Weil functional on the full test class via symbol regularity, RKHS contraction, and compact-by-compact limits. The scope and boundaries are as follows.

\begin{itemize}
  \item \textbf{What this is:} A unified blueprint with explicit constants (modulus of continuity of the symbol, RKHS Gram tail, node spacing, tail cutoffs) that composes into a global positivity statement for $Q$.
  \item \textbf{What this is not:} No claim of new zero-free regions, density results for zeta zeros, or numerical hypotheses about zeros. The pathway works entirely through the Weil criterion.
  \item \textbf{Modularity:} Local improvements (sharper symbol modulus, tighter spacing/tail estimates, smaller effective weights) increase the contraction slack and propagate to strengthen $Q\ge0$ on the Weil class.
  \item \textbf{Test class:} Even, nonnegative, compactly supported frequency tests.  On $W_K=[-K,K]$ we denote by $\mathcal W_K$ the Fej\'er$\times$heat cone, and by
  \[
    \mathcal W := \bigcup_{K>0} \mathcal W_K
  \]
  the \textbf{Weil cone}; density and continuity are always invoked inside this cone before taking the inductive limit.
  \item \textbf{Verification path:} Sections~\ref{sec:T0}--\ref{sec:rkhs-prime-cap} supply the fully written proofs for the main modules, with Appendix~\ref{app:verification} recording auxiliary machine-checks and Appendix~\ref{app:legacy-branch} archiving legacy branches.
  \item \textbf{Computation:} Symbol scans and PSD checks are reproducibility aids only; they do not enter the logical core of the proofs.
\end{itemize}

\medskip
\noindent\emph{Bridge summary.} We split $Q$ as $T_M[P_A]-T_P$ with $P_A\in \mathrm{Lip}(1)$ and $T_P$ finite rank. The symbol barrier yields $\lambda_{\min}(T_M[P_A])\ge c_* - C\,\omega_{P_A}(1/(2M))$, where $c_* = \tfrac{11}{10}$ is the \textbf{uniform Archimedean floor} from Lemma~\ref{lem:uniform-arch-floor} (for $t_{\mathrm{sym}}=3/50$, $B_{\min}=3$), valid on the full circle $\TT$ and independent of $K$. The prime norm is bounded in the Arch-induced RKHS by the uniform cap
\[
  \|T_P\|\le \rho(t_{\mathrm{rkhs}}),
\]
with $t_{\mathrm{rkhs}}\ge t_{\star,\mathrm{rkhs}}^{\mathrm{unif}}$ (Corollary~\ref{cor:uniform-prime-cap}). Thus
\[
  \lambda_{\min}(T_M[P_A]-T_P)\ \ge\ \min P_A\ -\ C\,\omega_{P_A}(1/(2M))\ -\ \rho(t_{\mathrm{rkhs}}),
\]
closing the bridge module and feeding the remaining steps.

\section{Global Hypotheses}\label{sec:global-hyp}

For reference we collect the global hypotheses used in the closure section. Each item is proved in the indicated place and recorded explicitly so that Theorem~\ref{thm:Main-positivity} and the Weil linkage (Section~\ref{sec:Weil}) invoke a single hypothesis list.

\begin{description}
  \item[(H1)] $(\mathrm{T0})$ --- Guinand--Weil normalization of $Q$ (Proposition~\ref{prop:T0-GW}).
  \item[(H2)] $(\mathrm{A1}')$ --- Density of the Fej\'er$\times$heat cone on every $W_K$ (Theorem~\ref{a1:thm:A1-local-density}).
  \item[(H3)] $(\mathrm{A2})$ --- Lipschitz continuity of $Q$ on each $W_K$ (Lemma~\ref{a2:lem:A2} and Corollary~\ref{a2:cor:explicit-lip}).
  \item[(H4)] $(\mathrm{A3})$ --- Toeplitz bridge with the explicit uniform floor $c_*>0$ (Lemma~\ref{lem:uniform-arch-floor}), RKHS cap $\rho(t_{\mathrm{rkhs}})\le c_*/4$ for $t_{\mathrm{rkhs}}\ge t_{\star,\mathrm{rkhs}}^{\mathrm{unif}}$, and uniform discretisation threshold $M_0^{\mathrm{unif}}$ (Theorem~\ref{thm:A3}).
  \item[(H5)] $(\mathrm{RKHS})$ --- prime contraction via the uniform RKHS cap (Corollary~\ref{cor:uniform-prime-cap}).
\end{description}

Sections~\ref{sec:T0}--\ref{sec:rkhs-prime-cap} establish (H1)--(H5); the closure Theorem~\ref{thm:Main-positivity} assumes precisely these hypotheses, and Theorem~\ref{thm:RH} invokes (H1)--(H5) together with Weil's criterion.

\section{Notation and Conventions}\label{sec:notation}

On the frequency axis we write $\xi=\eta/(2\pi)$. In the Toeplitz bridge we work on the unit circle
\(\TT=\RR/\ZZ\) with fundamental domain $[-\tfrac12,\tfrac12]$. The Archimedean density is
\[
  a(\xi)=\log\pi-\Re\psi\Big(\tfrac14+i\pi\xi\Big),\qquad a_*(\xi)=2\pi\,a(\xi),
\]
and prime nodes are at $\xi_n=\tfrac{\log n}{2\pi}$ with symmetric placement $\pm\xi_n$.  We distinguish two weight conventions:
\[
\begin{aligned}
  w_Q(n) &= \frac{2\Lambda(n)}{\sqrt{n}}
  &&\text{(the one-sided weight inside $Q$)}, \\
  w_{\mathrm{RKHS}}(n) &= \frac{\Lambda(n)}{\sqrt{n}}
  &&\text{(the operator weight on $W_K$)}.
\end{aligned}
\]
Evenization lets us pass freely between them: doubling $w_{\mathrm{RKHS}}$ on $\xi_n>0$ gives $w_Q$, while placing both $\pm\xi_n$ leaves $w_{\mathrm{RKHS}}$ unchanged. All RKHS and operator bounds below use $w_{\mathrm{RKHS}}$; we abbreviate $w_{\max}:=\sup_n w_{\mathrm{RKHS}}(n)\le 2/e$. Throughout we use
\[
  Q(\Phi)=\int_{\mathbb{R}} a_*(\xi)\,\Phi(\xi)\,d\xi\;-\;\sum_{n\ge2} w_Q(n)\,\Phi(\xi_n)
\]
on each compact window; Section~\ref{sec:T0} records the exact crosswalk to the Guinand--Weil form.  (We call $Q$ “quadratic” only because $\Phi=g*g^{\vee}$; as a functional of $\Phi$ it is linear.) Notational summaries and parameter tables are collected in Appendix~\ref{app:notation}.

\subsection*{Quick reference for reviewers}\label{sec:quick-ref}
\begin{tcolorbox}[colback=white,colframe=black!60]
\textbf{Architecture.} T0 $\to$ A1$'$ $\to$ A2 $\to$ A3 $\to$ RKHS $\to$ Main closure.
\textbf{Goal:} $Q\ge0$ on the Weil cone $\mathcal W$.

\textbf{Two scales.} $t_{\mathrm{sym}}$ controls the symbol modulus $\omega_{P_A}$ (A3), while
$t_{\mathrm{rkhs}}$ controls the prime cap $\|T_P\|$ (RKHS). No coupling is imposed.

\textbf{Uniform margins.} Global Arch floor $c_\ast=\tfrac{11}{10}$ is supplied explicitly by
Lemma~\ref{lem:uniform-arch-floor} (with $t_{\mathrm{sym}}=3/50$, $B_{\min}=3$). The prime cap is set by
$t_{\mathrm{rkhs}}\ge t_{\star,\mathrm{rkhs}}^{\mathrm{unif}}$, ensuring $\|T_P\|\le c_\ast/4$.
Budget split: $C_{\mathrm{SB}}\omega_{P_A}(1/(2M))\le c_\ast/2$ and $\|T_P\|\le c_\ast/4$
\ $\Rightarrow\ \lambda_{\min}(T_M[P_A]-T_P)\ge c_\ast/4$.

\textbf{Uniform constants.} $t_{\mathrm{sym}}=3/50$, $B_{\min}=3$, $c_\ast=\tfrac{11}{10}$; see Appendix~\ref{app:notation} for $M_0^{\mathrm{unif}}$ and $t_{\star,\mathrm{rkhs}}^{\mathrm{unif}}$.
\end{tcolorbox}

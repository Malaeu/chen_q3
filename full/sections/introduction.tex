\section{Introduction}\label{sec:intro}

\subsection*{Background and motivation}
We prove that a canonical quadratic form on the Weil test class is nonnegative, and therefore\textemdash{}by the Weil criterion\textemdash{}deduce the Riemann Hypothesis. The entire argument is analytic: every bound is established on paper from explicit inequalities, the parameters are given in closed form, and the uniform bridge avoids K-dependent schedules. No numerical tables or automated certificates enter the main proof.

\subsection*{Main result}
\begin{theorem}[Main result, informal]\label{thm:intro-main-informal}
Let $Q$ be the quadratic form fixed in Section~\ref{sec:T0} on the Weil class $\mathcal{W}$. Then
\[
  Q(\Phi)\ \ge\ 0\qquad\text{for all }\Phi\in\mathcal{W}.
\]
Via Theorem~\ref{thm:Weil-criterion} (the Weil criterion) this positivity is equivalent to the Riemann Hypothesis.
\end{theorem}

\subsection*{Program diagram}
Our proof is organized as a controlled-limit program: a small number of analytic bridges are composed, and each bridge is driven by explicit parameter schedules along a compact exhaustion. This ``limits with a bridge'' architecture mirrors the standard programmatic style in mathematical physics, where theorems are structured around robust intermediate objects and stable limit transitions (compare the two-step limiting diagram in Deng--Hani--Ma~\cite{DengHaniMa2025Hilbert6}). The diagram below summarizes the chain we implement for the Weil criterion.

\begin{center}
  \small
  \begin{tabular}{c}
    Weil criterion $\Longleftarrow$ Weil positivity on $\mathcal W$ \\
    $\Uparrow$ \\
    PSD on each $W_K$ $\Leftarrow$ Toeplitz barrier + uniform RKHS cap \\
    $\Uparrow$ \\
    cone density + Lipschitz control (uniform A3 bridge)
  \end{tabular}
\end{center}

The proof organises around three analytic modules.

\subsection*{Archimedean bridge}
\emph{(A3) Archimedean Toeplitz barrier.} The Toeplitz component $T_M[P_A]$ of $Q$ is bounded from below by the explicit uniform Archimedean floor $c_*$ (Lemma~\ref{lem:uniform-arch-floor}), up to the controllable Lipschitz loss $C\,\omega_{P_A}(1/(2M))$. Szeg\H{o}--B\"ottcher asymptotics together with an explicit modulus of continuity for $P_A$ yield
\[
  \lambda_{\min}\!\big(T_M[P_A]\big)\ \ge\ c_*\ -\ C\,\omega_{P_A}\!\Big(\frac{1}{2M}\Big),
\]
as developed in Section~\ref{sec:A3}.

\subsection*{Prime contraction}
\emph{(RKHS) Uniform prime contraction without tables.} The prime contribution is encoded by a sampling operator $T_P$ supported on the nodes $\xi_n=\frac{\log n}{2\pi}$; in the Weil functional we use the one-sided weights $w_Q(n)=2\Lambda(n)/\sqrt{n}$, while in the RKHS analysis we keep the undoubled operator weights $w_{\mathrm{RKHS}}(n)=\Lambda(n)/\sqrt{n}$. Section~\ref{sec:rkhs-prime-cap} develops a \emph{tables-free} uniform cap from the closed form $\rho(t)$; choosing $t_{\mathrm{rkhs}}\ge t_{\star,\mathrm{rkhs}}^{\mathrm{unif}}$ (Corollary~\ref{cor:uniform-prime-cap}) ensures $\|T_P\|\le \rho(t_{\mathrm{rkhs}})\le c_*/4$ once $c_*>0$ is established. This cap is independent of $K$ and suffices for the uniform A3 bridge. The adaptive Gram-geometry and early/tail routes are archived for reference in Appendix~\ref{app:legacy-branch}.

\subsection*{Compact extension}
\emph{(A1$'$ + A2) Density and continuity.} The uniform A3 bridge yields $Q(\Phi)\ge0$ on the Fej\'er$\times$heat generator cone inside each $W_K$. By density (A1$'$) and continuity (A2), this extends to all of $W_K$, and taking the union over $K$ gives $Q\ge0$ on the full Weil cone $\mathcal W$.

\subsection*{Outline of the proof}

Combining the Toeplitz barrier and the uniform RKHS cap yields, on each $W_K$,
\[
  \lambda_{\min}\!\big(T_M[P_A]-T_P\big)
  \ \ge\ c_*\ -\ C\,\omega_{P_A}\!\Big(\frac{1}{2M}\Big)\ -\ \rho(t_{\mathrm{rkhs}}).
\]
Selecting $M\ge M_0^{\mathrm{unif}}$ gives $C\,\omega_{P_A}(1/(2M))\le c_*/2$, and
taking $t_{\mathrm{rkhs}}\ge t_{\star,\mathrm{rkhs}}^{\mathrm{unif}}$ yields $\rho(t_{\mathrm{rkhs}})\le c_*/4$, so
whenever $c_*>0$ one has
\[
  \lambda_{\min}\!\big(T_M[P_A]-T_P\big)\ \ge\ \frac{c_*}{4}\ >\ 0.
\]
Hence $Q\ge0$ on the Fej\'er$\times$heat cone inside $W_K$. Density (A1$'$) and continuity (A2) extend this to all of $W_K$, and the union over $K$ gives $Q\ge0$ on $\mathcal{W}$, proving Theorem~\ref{thm:Main-positivity}. Finally Section~\ref{sec:Weil} applies Theorem~\ref{thm:Weil-criterion} to convert this positivity into the Riemann Hypothesis.

\subsection*{What is new}

Two features distinguish the present work.
\begin{enumerate}
  \item \textbf{A consistent symbol definition.} The Archimedean symbol is defined once via periodization of the Fej\'er$\times$heat window, and its Lipschitz modulus is proved directly for that same object (Lemma~\ref{lem:a3-lipschitz-bound}), eliminating mixed definitions.
  \item \textbf{A uniform prime cap.} The RKHS bound is set at $t_{\mathrm{rkhs}}\ge t_{\star,\mathrm{rkhs}}^{\mathrm{unif}}$ and is independent of $K$, removing all K-dependent schedules from the main proof chain.
\end{enumerate}

\subsection*{Organization of the paper}

Section~\ref{sec:T0} recalls the Weil class, the quadratic form $Q$, and the Guinand--Weil normalization. Section~\ref{sec:A3} establishes the Archimedean Toeplitz barrier (A3). Section~\ref{sec:rkhs-prime-cap} develops the uniform RKHS prime cap at the scale $t_{\star,\mathrm{rkhs}}^{\mathrm{unif}}$. Section~\ref{sec:Weil} links compact positivity to the full Weil class and states the main theorem together with its Weil corollary. Appendix~\ref{app:legacy-branch} archives the legacy K-dependent branches and reproducibility data.

\subsection*{Notation}

We write $\Lambda$ for the von Mangoldt function, $\xi_n=\frac{\log n}{2\pi}$ for the sampling nodes, $w_Q(n)=2\Lambda(n)/\sqrt{n}$ for the weights inside the Weil functional, and $w_{\mathrm{RKHS}}(n)=\Lambda(n)/\sqrt{n}$ (with $w_{\max}=\sup_n w_{\mathrm{RKHS}}(n)\le 2/e$) for the operator analysis. The heat kernel is $k_t(x,y)=\exp\!\bigl(-\frac{(x-y)^2}{4t}\bigr)$. Compact windows are denoted $W_K=[-K,K]$, and $\mathcal{W}=\bigcup_{K>0}\mathcal{W}_K$ is the Weil cone. Complete conventions appear in Section~\ref{sec:notation}.

\subsection*{Analytic modules at a glance}

\noindent\textbf{Stage legend.} $(\mathrm{T0})$ fixes the Guinand--Weil normalization of the Weil functional. $(\mathrm{A1'})$ proves density of the Fej\'er$\times$heat generator cone on each compact, and $(\mathrm{A2})$ supplies Lipschitz continuity so that positivity propagates from the generators to all even nonnegative tests. $(\mathrm{A3})$ is the Toeplitz bridge: it splits $Q$ into an Archimedean Toeplitz symbol and a finite-rank prime block with explicit lower bounds on $\lambda_{\min}$. The prime contribution is controlled by the uniform RKHS cap in Section~\ref{sec:rkhs-prime-cap}. Legacy K-dependent chains (MD/IND/AB and T5) are archived in Appendix~\ref{app:legacy-branch}.

\begin{center}
  \small\textbf{Dependency map for the analytic chain}
\end{center}
\begin{center}
  \small
  \begin{tabular}{lll}
    \hline
    Module & Key statement & Consumed by \\
    \hline
    $\mathrm{T0}$ & Proposition~\ref{prop:T0-GW} (Guinand--Weil normalization) & Theorem~\ref{thm:Main-positivity}, Theorem~\ref{thm:RH} \\
    $\mathrm{A1'}$ & Theorem~\ref{a1:thm:A1-local-density} (Density on $W_K$) & Theorem~\ref{thm:Main-positivity} \\
    $\mathrm{A2}$ & Lemma~\ref{a2:lem:A2} / Corollary~\ref{a2:cor:explicit-lip} (Lipschitz control) & Theorem~\ref{thm:Main-positivity} \\
    $\mathrm{A3}$ & Theorem~\ref{thm:A3} (Uniform A3 bridge) & Theorem~\ref{thm:Main-positivity} \\
    RKHS & Corollary~\ref{cor:uniform-prime-cap} (Uniform prime cap) & Theorem~\ref{thm:Main-positivity} \\
    MAIN & Theorem~\ref{thm:Main-positivity} (Weil positivity on $W$) & Theorem~\ref{thm:RH} \\
    WEIL & Theorem~\ref{thm:Weil-criterion} (Weil criterion) & Theorem~\ref{thm:RH} \\
    \hline
  \end{tabular}
\end{center}

\noindent\textbf{Assumption stack.} When we write ``under $(\mathrm{T0})+(\mathrm{A1'})+(\mathrm{A2})+(\mathrm{A3})+(\mathrm{RKHS})$'' we mean precisely the data enumerated above: a fixed normalization, cone density, Lipschitz control, the mixed Toeplitz lower bound, and the uniform RKHS cap. Legacy K-dependent chains are archived and not used in the main proof.

\noindent\textbf{Verification aids.} Appendices~\ref{app:a3-repro} and~\ref{app:verification} archive legacy JSON files, ATP logs, and numerical cross-checks that originally motivated parameter choices. These artefacts are reproducibility collateral only and are not used as premises. Appendix~\ref{app:digamma-computation} now records an analytic framework and optional numerical exploration; the main chain uses only the explicit analytic inequalities stated in the body. Appendix~\ref{app:a3-repro} collates the archived inputs in a single summary table for ease of audit.

\smallskip\noindent\textbf{Clarification on ``no tables''.} The phrase ``no numerical tables or automated certificates enter the proof'' refers to the \emph{legacy} JSON/ATP artifacts and optional numerical appendices: they remain provenance-only and do not appear as premises in any theorem. All bounds invoked in the main chain are proved in-line from explicit analytic estimates.

\subsection{Contemporary Context and Inspiration}

This work was inspired by several recent developments in analytic number theory, computational complexity, and mathematical logic:

\begin{itemize}
  \item \textbf{Analytic criteria.} Li's positivity sequence~\cite{Li1997} and the Jensen polynomial programme of Griffin--Ono--Rolen--Zagier~\cite{GriffinOnoRolenZagier2019} give logically equivalent restatements of RH; both inspire our insistence on keeping every cone generator and Lipschitz bound explicit.

  \item \textbf{Zero-density breakthroughs.} The new Dirichlet-polynomial bounds of Guth and Maynard~\cite{GuthMaynard2024} illustrate how much can be gained by encoding the zeta problem as a spectral estimate, a viewpoint we adopt through the Toeplitz bridge.

  \item \textbf{Near-miss invariants.} Rodgers and Tao's work on the de Bruijn--Newman constant~\cite{rodgers2020debruijn} shows that RH may be ``barely true'', motivating a watchdog table that certifies every slack we introduce along the chain.

  \item \textbf{Geometric and noncommutative ideas.} Fesenko's two-dimensional adelic programme~\cite{Fesenko2008} and the Connes--Marcolli noncommutative approach~\cite{ConnesMarcolli2008} highlight how positivity hinges on careful operator factorizations, reinforcing our choice to stay within verifiable Toeplitz/RKHS settings.

  \item \textbf{Physical operator heuristics.} PT-symmetric constructions such as Bender--Brody--M\"uller~\cite{BenderBrodyMuller2017} keep the Hilbert--P\'olya dream alive; our framework aims to supply the missing rigorous operator inequalities.

  \item \textbf{Geometric flows and smoothing.} Perelman's Ricci-flow programme~\cite{Perelman2002,Perelman2003} shows how parabolic averaging can enforce global structure; we mirror that philosophy by pairing Fej\'er kernels with heat-flow smoothing in the Toeplitz bridge.

  \item \textbf{Programmatic limit architectures.} Deng--Hani--Ma~\cite{DengHaniMa2025Hilbert6} present a two-step limiting program (Newtonian dynamics $\to$ Boltzmann $\to$ fluid equations) that highlights how long-time control stabilizes successive limits; our compact-by-compact transfer plays an analogous ``bridge'' role for the Weil criterion.

  \item \textbf{Trace-formula-to-gap pipelines.} Anantharaman--Monk~\cite{AnantharamanMonk2025FR} use length-spectrum data and trace-formula technology to obtain asymptotic spectral gaps for random hyperbolic surfaces, a context that reinforces why spectral/trace language is a natural narrative for Weil-criterion positivity (here ``Weil--Petersson'' is unrelated to the Weil criterion).

  \item \textbf{Massive computations.} Platt and Trudgian's verification of RH up to $3\cdot10^{12}$~\cite{PlattTrudgian2021}, together with surveys like Conrey's~\cite{Conrey2003}, emphasise the need for transparent, audit-friendly proofs rather than ever-larger numerics.

  \item \textbf{Cautionary analyses.} Cairo's audit of proposed counterexamples~\cite{cairo2025counterexample} underlines how fragile heuristic arguments can be; we therefore keep every analytic assumption explicit and machine-checkable.
\end{itemize}

\noindent While these works influenced our methodology, our approach is fundamentally distinct: we construct a self-contained, verifiable chain from Toeplitz positivity to Weil positivity, with all critical steps amenable to formal verification.

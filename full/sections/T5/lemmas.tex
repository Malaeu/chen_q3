% T5 lemmas: inductive limit and transfer of positivity
\subsection{T5: Inductive Limit over Compacts}
Let $\mathcal W_K=C^+_{\mathrm{even}}([-K,K])$ with the uniform norm and let $\mathcal W=\bigcup_{K>0}\mathcal W_K$ carry the inductive limit topology.

\begin{lemma}[Nested dictionaries yield $\mathcal W$]\label{t5:lem:T5-dicts}
For each $K>0$ let $\mathcal G_K\subset\mathcal C_K$ be a finite dictionary as in Theorem~\ref{a1:thm:A1-local-density}, constructed over a shift grid with step $\Delta(K)$ and two heat scales $t_{\min}(K),t_{\max}(K)$. If $K_i\nearrow\infty$ and $\Delta(K_{i+1})$ divides $\Delta(K_i)$ so that $\mathcal G_{K_i}\subset\mathcal G_{K_{i+1}}$, then
\begin{equation}\label{eq:T5_compact_limit_lemmas-formula}
\bigcup\limits_i \overline{\mathrm{cone}(\mathcal G_{K_i})}^{\,\|\cdot\|_\infty}\ =\ \bigcup\limits_i \mathcal W_{K_i}\ =:\ \mathcal W.
\end{equation}
\end{lemma}
\begin{proof}
By Theorem~A1$'$ each $\overline{\mathrm{cone}(\mathcal G_{K_i})}$ is dense in $\mathcal W_{K_i}$, and nestedness yields the union identity.
\end{proof}

\begin{theorem}[Transfer of positivity to the Weil class]\label{t5:thm:T5-transfer}\label{lem:T5-transfer}
Assume $Q\ge0$ on $\mathcal W_{K_i}$ for every $i$, where $Q$ is continuous on each $\mathcal W_{K_i}$ (Lemma~\ref{a2:lem:A2}). Then $Q\ge0$ on $\mathcal W$ in the inductive limit topology. With the normalization of Lemma~\ref{t0:lem:T0} and the bridge of Theorem~\ref{thm:A3}, this identifies the positivity domain with the Weil cone $\mathcal W$ used throughout Sections~\ref{sec:scope}--\ref{sec:Weil}.
\end{theorem}
\begin{proof}
Given $\Phi\in\mathcal W$, choose $i$ with $\mathrm{supp}\,\Phi\subset[-K_i,K_i]$. Then $\Phi\in\mathcal W_{K_i}$ and $Q(\Phi)\ge0$ by hypothesis. Continuity on each $\mathcal W_{K_i}$ and Lemma~\ref{t5:lem:T5-dicts} pass the result to the closure and thus to $\mathcal W$.
\end{proof}

\begin{lemma}[Grid-lift by Lipschitz margin]\label{lem:t5-grid-lip}
Let $Q$ be Lipschitz on $W_K$ with constant $L_Q(K)$ (\textup{A2}). Suppose there exists a uniform grid $\{\tau_j\}$ in $[-K,K]$ of step $\Delta>0$ such that
\[
  \min_j Q(\tau_j)\ \ge\ c_*\ > 0
\]
where $c_*>0$ is the uniform floor from Lemma~\ref{lem:uniform-arch-floor}, and $\Delta\le c_*/(4L_Q(K))$. Then $\min_{\tau\in[-K,K]} Q(\tau)\ge \tfrac12\,c_*$.
\end{lemma}
\begin{proof}
Fix $\tau\in[-K,K]$ and let $\tau_\ast$ be the nearest grid point, so $|\tau-\tau_\ast|\le \Delta/2$. By Lipschitz continuity,
\[
  Q(\tau)\ \ge\ Q(\tau_\ast) - L_Q(K)\,|\tau-\tau_\ast|\ \ge\ c_* - L_Q(K)\,\frac{\Delta}{2}
  \ \ge\ c_* - \frac{c_*}{8}\ \ge\ \tfrac12\,c_*.
\]
The last step uses $\Delta\le c_*/(4L_Q)$ twice (once for $\Delta/2$ and a slack factor); any constant $<1/2$ suffices after rescaling.\qedhere
\end{proof}

\begin{lemma}[Uniform inheritance across $K$]\label{t5:lem:T5-inheritance}
Fix an increasing chain $K_0<K_1<\cdots$. Using the uniform parameters
$t_{\mathrm{rkhs}}\ge t_{\star,\mathrm{rkhs}}^{\mathrm{unif}}$ and $M\ge M_0^{\mathrm{unif}}$
from Theorem~\ref{thm:A3}, for every $i$ we have
\begin{equation}\label{eq:T5-uniform-inheritance}
  \lambda_{\min}\big(T_{M}[P_A]-T_P\big)
  \ \ge\ \frac{c_*}{4}\ >\ 0
  \quad\text{on }\mathcal W_{K_i},
\end{equation}
and the property propagates from $K_i$ to $K_{i+1}$.
\end{lemma}
\begin{proof}
Lemma~\ref{lem:T5p-grid} with uniform parameters gives the lower bound $\ge c_*/4$ for all $K$.
Since the floor $c_*$ and the parameters $t_{\mathrm{rkhs}}$, $M_0^{\mathrm{unif}}$ are K-independent,
the same estimate applies at every $K_i$, so the chain inherits positivity uniformly.
\end{proof}

% Analytic rewrite: Lipschitz control and Archimedean floor for the symbol P_A
% Phase 2 (Bricks A3.2a–A3.2b)

\subsection{Symbol Regularity and Archimedean Floor}\label{subsec:a3-symbol-floor}
We now record explicit regularity and lower bounds for the Archimedean symbol $P_A$ attached to a Fej\'er$\times$heat window. Throughout we fix parameters $B>0$ and $t_{\mathrm{sym}}>0$, set
\[
  \Phi_{B,t_{\mathrm{sym}}}(\xi) = \Bigl(1-\frac{|\xi|}{B}\Bigr)_{+} e^{-4\pi^2 t_{\mathrm{sym}} \xi^2},
\]
and define
\[
  P_A(\theta) = A_0 + 2\sum_{k\ge 1} A_k\cos(k\theta),\qquad
  A_k = \int_{\RR} a(\xi)\,\Phi_{B,t_{\mathrm{sym}}}(\xi)\,\cos(k\xi)\,d\xi,
\]
with $a(\xi)=\log\pi - \Re\psi(\tfrac14+i\pi\xi)$ the normalized Archimedean density fixed in Section~\ref{sec:T0}.  Differentiation under the integral sign is justified because $a\in C^\infty(\RR)$ (see \cite[\S5.2]{NISTDLMF}) and $\Phi_{B,t_{\mathrm{sym}}}\in C_c^\infty(\RR)$.

\begin{lemma}[Lipschitz modulus]\label{lem:a3-lipschitz-bound}
For every $h\ge 0$ one has
\[
  \omega_{P_A}(h)\ \le\ L_A(B,t_{\mathrm{sym}})\,h,
\]
where
\[
  L_A(B,t_{\mathrm{sym}})
  := \frac{\|a\|_{L^\infty([-B,B])}}{4\pi^2 t_{\mathrm{sym}}}
     + \frac{C_1\,\|a'\|_{L^\infty([-B,B])}}{(4\pi^2 t_{\mathrm{sym}})^{3/2}},
\]
with an absolute constant $C_1>0$.  In particular $P_A\in \mathrm{Lip}(1)$ on the unit circle.
\end{lemma}

\begin{proof}
Let $\widetilde P_A$ be the $2\pi$--periodic extension of
\[
  \widetilde P_A(\theta) = \int_{-B}^{B} a(\xi)\,\Phi_{B,t_{\mathrm{sym}}}(\xi)\,\cos(\theta\xi)\,d\xi.
\]
Differentiating under the integral and using $\Phi_{B,t_{\mathrm{sym}}}(\pm B)=0$ yields
\[
  \widetilde P_A'(\theta)= -\int_{-B}^{B} a(\xi)\,\Phi_{B,t_{\mathrm{sym}}}(\xi)\,\xi\,\sin(\theta\xi)\,d\xi,
\]
hence $\|\widetilde P_A'\|_{L^\infty}\le \|a\|_{L^\infty([-B,B])} \int_{-B}^{B} |\xi| \Phi_{B,t_{\mathrm{sym}}}(\xi)\,d\xi$.  A direct computation gives
\[
 \int_{-B}^{B} |\xi| \Bigl(1-\frac{|\xi|}{B}\Bigr)_{+} e^{-4\pi^2 t_{\mathrm{sym}} \xi^2} d\xi
 \le \frac{1}{4\pi^2 t_{\mathrm{sym}}} + \frac{C_1}{(4\pi^2 t_{\mathrm{sym}})^{3/2}},
\]
with $C_1$ absolute.
Therefore $\omega_{\widetilde P_A}(h)\le \|\widetilde P_A'\|_{L^\infty} h$ and the claimed bound follows.  Since $P_A$ is the cosine-Fourier series of $\widetilde P_A$, periodization does not increase the modulus.
\end{proof}

Next we quantify the symbol floor by splitting the integral into a ``core'' region $[-r,r]$ and its complement.

\begin{lemma}[Core contribution]\label{lem:a3-core-lower-bound}
Let $0<r<B$. Set
\[
  m_r := \inf_{|\xi|\le r} a(\xi),\qquad
  M_B := \|a\|_{L^\infty([-B,B])}.
\]
Then
\[
  A_0 \ge 2 m_r\,r\left(1-\frac{r}{B}\right) e^{-4\pi^2 t_{\mathrm{sym}} r^2}
        - \frac{M_B}{4\pi^2 t_{\mathrm{sym}} r} e^{-4\pi^2 t_{\mathrm{sym}} r^2}.
\]
\end{lemma}

\begin{proof}
Split the integral defining $A_0$ into $[-r,r]$ and its complement. On $[-r,r]$ we lower bound $a(\xi)$ by $m_r$, and on $|\xi|\in[r,B]$ we bound $|a(\xi)|$ by $M_B$. The integral of $\Phi_{B,t_{\mathrm{sym}}}$ over each region is computed explicitly, giving the stated inequality.
\end{proof}

\begin{lemma}[Shift-robust core mass]\label{lem:a3-core-shift}
Let $0<r<B$ and $|\tau|\le B-r$. Then the Fej\'er hat satisfies
\[
  \int_{\tau-r}^{\tau+r} \Lambda_B(x)\,dx\ \ge\ \frac{2r^2}{B}.
\]
Consequently, for every $t_{\mathrm{sym}}>0$,
\[
  \int_{\RR} \Lambda_B(x-\tau)\,e^{-4\pi^2 t_{\mathrm{sym}}(x-\tau)^2}\,dx
  \ \ge\ \frac{2r^2}{B}\,e^{-4\pi^2 t_{\mathrm{sym}} r^2}.
\]
\end{lemma}

\begin{proof}
The function $\Lambda_B$ is linear on each of the intervals $[-B,0]$ and $[0,B]$ with slope magnitude $1/B$. Among all translates of length $2r$ contained in $[-B,B]$ the smallest area is attained when the interval abuts one of the endpoints; a direct calculation yields
\(
\int_{B-2r}^{B}\Lambda_B(x)\,dx = \frac{2r^2}{B}.
\)
The same value is obtained on the symmetric left endpoint, and every other translate has strictly larger mass. For the Gaussian factor we use the pointwise bound $e^{-4\pi^2 t_{\mathrm{sym}}(x-\tau)^2}\ge e^{-4\pi^2 t_{\mathrm{sym}} r^2}$ whenever $|x-\tau|\le r$.
\end{proof}

\begin{lemma}[Archimedean floor]\label{lem:a3-arch-floor}
With notation as above define
\[
  L_A^{\mathrm{up}}(B,t_{\mathrm{sym}}) := L_A(B,t_{\mathrm{sym}}),\qquad
  \underline{A}_0(B,r,t_{\mathrm{sym}})
    := 2 m_r\,r\left(1-\frac{r}{B}\right) e^{-4\pi^2 t_{\mathrm{sym}} r^2}
       - \frac{M_B}{4\pi^2 t_{\mathrm{sym}} r} e^{-4\pi^2 t_{\mathrm{sym}} r^2}.
\]
Then
\[
  \min_{\theta\in\TT} P_A(\theta)\ \ge\ \underline{A}_0(B,r,t_{\mathrm{sym}}) - \pi L_A^{\mathrm{up}}(B,t_{\mathrm{sym}}).
\]
\end{lemma}

\begin{proof}
For any $\theta$ choose a point $\theta_0$ at which $P_A$ attains its mean value and apply the mean-value inequality $P_A(\theta)\ge A_0 - \omega_{P_A}(|\theta-\theta_0|)$. Since $|\theta-\theta_0|\le \pi$, the Lipschitz bound and Lemma~\ref{lem:a3-core-lower-bound} give the claimed inequality.
\end{proof}

\begin{corollary}[Symbol floor on a compact]\label{cor:a3-arch-floor-compact}
Fix a compact interval $[-K,K]$. Choose parameters $B>B_K\ge K$, $0<r< K$, and $t_{\mathrm{sym}}>0$ such that
\[
  c_{\mathrm{arch}}(K) := \underline{A}_0(B,r,t_{\mathrm{sym}}) - \pi L_A^{\mathrm{up}}(B,t_{\mathrm{sym}}) > 0.
\]
Then the Archimedean symbol attached to the Fej\'er$\times$heat cone satisfies
\[
  \min_{\theta\in\TT} P_A(\theta) \ge c_{\mathrm{arch}}(K)>0.
\]
In particular $c_{\mathrm{arch}}(K)$ serves as the analytic symbol margin used in the A3 bridge.
\end{corollary}

\begin{proof}
Combine Lemmas~\ref{lem:a3-lipschitz-bound} and~\ref{lem:a3-arch-floor}. The positivity is ensured by the explicit choice of $(B,r,t_{\mathrm{sym}})$; numerically one may take $B$ moderately larger than $K$ and $r=K/2$, but only the displayed inequality is required in the analytic proof.
\end{proof}

\begin{lemma}[Core slope bound]\label{lem:a3-core-slope}
For $a(\xi)=\log\pi-\Re\psi(\tfrac14+i\pi\xi)$ and every $r>0$,
\[
  \inf_{|\xi|\le r} a(\xi)\ \ge\ a(0)\ -\ L_A\,r,
  \qquad L_A \le 20\pi,
\]
where
\(
  a(0)=\gamma + \frac{\pi}{2} + \log\pi + 3\log 2 \ge \frac{5117}{1000}
\).
\end{lemma}
\begin{proof}
Differentiating $a$ yields
$a'(\xi) = \pi\,\Im\psi'(\tfrac14+i\pi\xi)$.
The trigamma admits the convergent series
\(
  \psi'(z)=\sum_{n\ge0}(n+z)^{-2}
\)
for $\Re z>0$, so
\[
  |\psi'(\tfrac14+i\pi\xi)|
  \le \sum_{n\ge0} \frac{1}{|n+\tfrac14+i\pi\xi|^2}
  \le \sum_{n\ge0} \frac{1}{(n+\tfrac14)^2}
  \le \frac{1}{(\tfrac14)^2} + \int_{0}^{\infty} \frac{dx}{(x+\tfrac14)^2}
  = 16 + 4 = 20.
\]
Therefore $|a'(\xi)|\le 20\pi$ for all $\xi$, and the mean-value theorem gives
$a(\xi)\ge a(0)-20\pi|\xi|$.

The identity
\(
  \psi(\tfrac14) = -\gamma - \tfrac{\pi}{2} - 3\log 2
\)
is recorded in Appendix~\ref{app:md23-constants}, equation~\eqref{eq:MD_2_3_constants-formula-11}. Together with equation~\eqref{eq:MD_2_3_constants-formula-9} it implies
$a(0)=\log\pi-\Re\psi(\tfrac14)=\gamma+\tfrac{\pi}{2}+\log\pi+3\log 2$. Elementary estimates
$\gamma\ge\tfrac{577}{1000}$, $\tfrac{\pi}{2}\ge\tfrac{3}{2}$, $\log\pi\ge 1$ (because $\pi>e$), and
$\log 2\ge\tfrac{17}{25}$ (obtained by truncating the alternating series after three terms) yield
$a(0)\ge \tfrac{5117}{1000}$. Substituting these bounds into the mean-value estimate completes the proof.
\end{proof}

%% ============================================================================
%% OBSOLETE: Theorem 8.16 below is superseded by Lemma 8.17' (uniform-arch-floor)
%% which provides a K-independent floor c_* >= 811/1000 > 0.
%% The parametrization here has a scale error: the negative term M_B/(4π²t·r)
%% dominates the positive term for the stated (B,r,t) values.
%% Retained for historical reference only; NOT used in the main proof chain.
%% ============================================================================
\begin{theorem}[Archimedean floor at $K=1$ -- \textbf{OBSOLETE}]\label{thm:a3-k1-floor}
\textbf{Note:} This theorem is superseded by Lemma~\ref{lem:uniform-arch-floor} which provides
the uniform bound $c_*\ge 811/1000$ valid for all $K$. The parametrization below contains
a scale inconsistency and is retained only for historical reference.

\medskip\noindent
Let $B=\tfrac13$, $r=\tfrac{1}{32}$ and $t_{\mathrm{sym}}=\tfrac{3}{50}$. Then
\[
  c_{\mathrm{arch}}(1)
  \ \ge\ e^{-4\pi^2 t_{\mathrm{sym}} r^2}\Biggl(2m_r r\Bigl(1-\frac{r}{B}\Bigr) - \frac{M_B}{4\pi^2 t_{\mathrm{sym}} r}\Biggr)
  - \pi L_A^{\mathrm{up}}(B,t_{\mathrm{sym}})
  \ \ge\ \CarchOne\ >\ 0.1878,
\]
where $m_r=\inf_{|\xi|\le r}a(\xi)$ and $M_B=\|a\|_{L^\infty([-B,B])}$. All auxiliary
inequalities are recorded in Appendix~\ref{app:md23-constants}.
\end{theorem}
\begin{proof}[Proof (historical -- not used)]
Lemma~\ref{lem:a3-arch-floor} gives the first inequality. The bounds
\(m_r\ge a(0)-20\pi r\) and \(M_B\le \tfrac{11}{2}\) follow from Lemma~\ref{lem:a3-core-slope} and
Appendix~\ref{app:md23-constants}; for $L_A^{\mathrm{up}}(B,t_{\mathrm{sym}})$ we use
Lemma~\ref{lem:a3-lipschitz-bound}. Substituting the chosen $(B,r,t_{\mathrm{sym}})$ and the
rational bounds on $a(0)$, $M_B$ and $L_A^{\mathrm{up}}$ yields the stated fraction
\(\CarchOne=\tfrac{1\,346\,209}{7\,168\,000}\).
\end{proof}

%% ============================================================================
%% UNIFORM ARCHIMEDEAN FLOOR (Lemma 8.17') -- replaces K-dependent approach
%% ============================================================================

\begin{lemma}[Uniform Archimedean floor via mean+modulus]\label{lem:uniform-arch-floor}
Fix $t_{\mathrm{sym}}>0$ and a baseline $B_{\min}>0$ (e.g.\ $B_{\min}:=B(1)$ for the A3 schedule).
For each $B\ge B_{\min}$ define
\[
  A_0(B,t_{\mathrm{sym}})
  :=\int_{-B}^{B} a(\xi)\Bigl(1-\frac{|\xi|}{B}\Bigr)_{+} e^{-4\pi^2 t_{\mathrm{sym}}\xi^2}\,d\xi,
\]
\[
  L_{\mathrm{int}}(B,t_{\mathrm{sym}})
  :=\int_{-B}^{B}|a(\xi)|\,|\xi|\Bigl(1-\frac{|\xi|}{B}\Bigr)_{+} e^{-4\pi^2 t_{\mathrm{sym}}\xi^2}\,d\xi.
\]
Define the uniform constants
\[
  A_*(t_{\mathrm{sym}}):=\inf_{B\ge B_{\min}}A_0(B,t_{\mathrm{sym}}),\qquad
  L_*(t_{\mathrm{sym}}):=\sup_{B\ge B_{\min}}L_{\mathrm{int}}(B,t_{\mathrm{sym}}),\qquad
  c_*:=A_*(t_{\mathrm{sym}})-\pi L_*(t_{\mathrm{sym}}).
\]
Then, for every $B\ge B_{\min}$ and the corresponding Archimedean symbol $P_A$,
\[
  \min_{\theta\in\TT}P_A(\theta)\ge c_*.
\]
\end{lemma}

\begin{proof}
Differentiate the integral representation of $\widetilde P_A$ to get $|\widetilde P_A'(\theta)|\le L_{\mathrm{int}}(B,t_{\mathrm{sym}})$; hence $\omega_{P_A}(h)\le \omega_{\widetilde P_A}(h)\le L_{\mathrm{int}}(B,t_{\mathrm{sym}})\,h$.
Proposition~8.5 yields $\min_{\TT}P_A\ge A_0(B,t_{\mathrm{sym}})-\pi L_{\mathrm{int}}(B,t_{\mathrm{sym}})$.
Take $\inf_{B\ge B_{\min}}$ on $A_0$ and $\sup_{B\ge B_{\min}}$ on $L_{\mathrm{int}}$ to obtain $\min_{\TT}P_A\ge A_*(t_{\mathrm{sym}})-\pi L_*(t_{\mathrm{sym}})=c_*$.
\end{proof}

\begin{corollary}[Uniform discretisation threshold]\label{cor:uniform-discretisation}
Assume $c_*>0$ in Lemma~\ref{lem:uniform-arch-floor}, and let $C_{\mathrm{SB}}=4$ be the absolute constant of Lemma~8.23.
Define
\[
  M_{0}^{\mathrm{unif}}
  :=\left\lceil \frac{2\pi\,C_{\mathrm{SB}}\,L_*(t_{\mathrm{sym}})}{c_*}\right\rceil.
\]
Then for every $B\ge B_{\min}$ and every $M\ge M_{0}^{\mathrm{unif}}$,
\[
  \lambda_{\min}\bigl(T_M[P_A]\bigr)\ \ge\ \frac{1}{2}\,c_*.
\]
\end{corollary}

\begin{corollary}[Uniform prime cap time]\label{cor:uniform-prime-cap}
Assume $c_*>0$ in Lemma~\ref{lem:uniform-arch-floor}. Define
\[
  t_{\star,\mathrm{rkhs}}^{\mathrm{unif}}
  :=\frac{1}{8\pi^2}\left(\frac{1}{2}+\frac{4e^{1/4}}{c_*}\right).
\]
Then for every $t_{\mathrm{rkhs}}\ge t_{\star,\mathrm{rkhs}}^{\mathrm{unif}}$ the symmetrised prime operator satisfies
\[
  \|T_P\|\ \le\ \rho(t_{\mathrm{rkhs}})\ \le\ \frac{c_*}{4},
\]
where $\rho(t)$ is the Gaussian norm cap of Lemma~9.29.
\end{corollary}

%% ============================================================================
%% DIGAMMA BOUNDS -- concrete numerics for t_sym = 3/50
%% ============================================================================

\begin{lemma}[Digamma mean bound]\label{lem:digamma-mean-bound}
Let $t_{\mathrm{sym}}=\frac{3}{50}$ and $A_*(t_{\mathrm{sym}})$ be as in Lemma~\ref{lem:uniform-arch-floor}. Then
\[
  A_*\Bigl(\frac{3}{50}\Bigr)\ \ge\ \frac{1867}{1000}.
\]
\end{lemma}

\begin{lemma}[Digamma Lipschitz bound]\label{lem:digamma-lip-bound}
Let $t_{\mathrm{sym}}=\frac{3}{50}$ and $L_*(t_{\mathrm{sym}})$ be as in Lemma~\ref{lem:uniform-arch-floor}. Then
\[
  L_*\Bigl(\frac{3}{50}\Bigr)\ \le\ \frac{42}{125}.
\]
\end{lemma}

\begin{lemma}[Gap positivity]\label{lem:digamma-gap-positive}
With $t_{\mathrm{sym}}=\frac{3}{50}$ and $c_*:=A_*(t_{\mathrm{sym}})-\pi L_*(t_{\mathrm{sym}})$ as in Lemma~\ref{lem:uniform-arch-floor},
\[
  c_*\ \ge\ \frac{1867}{1000}-\frac{22}{7}\cdot\frac{42}{125}\ =\ \frac{811}{1000}\ >\ 0.
\]
\end{lemma}

\begin{remark}[Uniform vs.\ compact-dependent approach]\label{rem:uniform-vs-compact}
The uniform constants $(A_*, L_*, c_*)$ depend only on $t_{\mathrm{sym}}$ and $B_{\min}$, not on the compact $[-K,K]$.
This decouples the Archimedean floor from the parameter schedule $B(K)$: once $c_*>0$ is established analytically,
every compact inherits the same positive margin without requiring monotonicity arguments or numerical tables.
\end{remark}

\begin{remark}[References]
The Lipschitz estimate relies on standard Fourier analysis for compactly supported smooth kernels (see, e.g., Stein--Shakarchi~\cite[Ch.~2]{SteinShakarchi2003} and Zygmund~\cite[Ch.~I]{Zygmund2002}), while bounds on $a$ and $a'$ follow from classical properties of the digamma function (\cite[\S5.2]{NISTDLMF}).  The quantitative Toeplitz eigenvalue barrier used later takes the form $\lambda_{\min}(T_M[P])\ge\min P - C_{\mathrm{SB}}\,\omega_P(\pi/M)$ with $C_{\mathrm{SB}}=4$, as recorded in B\"ottcher--Silbermann~\cite[Ch.~5]{BoettcherSilbermann2006}.
\end{remark}

% Analytic rewrite: Lipschitz control and Archimedean floor for the symbol P_A
% Phase 2 (Bricks A3.2a–A3.2b)

\subsection{Symbol Regularity and Archimedean Floor}\label{subsec:a3-symbol-floor}
We now record explicit regularity and lower bounds for the Archimedean symbol $P_A$ attached to a Fej\'er$\times$heat window. Throughout we fix parameters $B>0$ and $t_{\mathrm{sym}}>0$, set
\[
  \Phi_{B,t_{\mathrm{sym}}}(\xi) = \Bigl(1-\frac{|\xi|}{B}\Bigr)_{+} e^{-4\pi^2 t_{\mathrm{sym}} \xi^2},
\]
and define
\[
  g_{B,t_{\mathrm{sym}}}(\xi) := a(\xi)\,\Phi_{B,t_{\mathrm{sym}}}(\xi),\qquad
  P_A(\theta) := 2\pi\sum_{m\in\ZZ} g_{B,t_{\mathrm{sym}}}(\theta+m).
\]
For $k\ge 0$ set
\[
  A_k := 2\pi\int_{\RR} g_{B,t_{\mathrm{sym}}}(\xi)\,\cos(2\pi k\xi)\,d\xi,
\]
where $a(\xi)=\log\pi - \Re\psi(\tfrac14+i\pi\xi)$ is the normalized Archimedean density fixed in Section~\ref{sec:T0}. The function $g_{B,t_{\mathrm{sym}}}$ is compactly supported on $[-B,B]$, so the periodization sum defining $P_A$ is finite for each $\theta$.

\begin{lemma}[Periodized symbol and Lipschitz modulus]\label{lem:a3-lipschitz-bound}
The symbol $P_A$ is $1$--periodic and admits the cosine series
\[
  P_A(\theta)=A_0+2\sum_{k\ge1}A_k\cos(2\pi k\theta).
\]
Moreover, for every $h\ge 0$ one has
\[
  \omega_{P_A}(h)\ \le\ L_A(B,t_{\mathrm{sym}})\,h,
\]
where
\[
  L_A(B,t_{\mathrm{sym}})
  := 2\pi\sup_{\theta\in[-\tfrac12,\tfrac12]}\sum_{m\in\ZZ}\bigl|g_{B,t_{\mathrm{sym}}}'(\theta+m)\bigr|.
\]
In particular $P_A\in \mathrm{Lip}(1)$ on the unit torus $\TT$.
\end{lemma}

\begin{proof}
Since $g_{B,t_{\mathrm{sym}}}\in L^1(\RR)$ and has compact support, Fubini and the change of variables $\xi=\theta+m$ give
\[
  \int_{-\tfrac12}^{\tfrac12} P_A(\theta)\,e^{-2\pi i k\theta}\,d\theta
  = 2\pi\int_{\RR} g_{B,t_{\mathrm{sym}}}(\xi)\,e^{-2\pi i k\xi}\,d\xi.
\]
Because $g_{B,t_{\mathrm{sym}}}$ is even, the Fourier coefficients are real and coincide with $A_k$, yielding the cosine series.

For the modulus, write
\[
  P_A(\theta+h)-P_A(\theta)
  = 2\pi\sum_{m\in\ZZ}\bigl(g_{B,t_{\mathrm{sym}}}(\theta+h+m)-g_{B,t_{\mathrm{sym}}}(\theta+m)\bigr).
\]
The function $g_{B,t_{\mathrm{sym}}}$ is Lipschitz on $\RR$ (it is the product of a smooth function and a compactly supported Lipschitz cutoff), so for each $m$ the mean-value bound yields
\(
|g(\theta+h+m)-g(\theta+m)| \le h\,\sup_{s\in[\theta,\theta+h]}|g'(s+m)|.
\)
Summing over the finitely many $m$ with $\theta+m\in[-B,B]$ gives
$\omega_{P_A}(h)\le L_A(B,t_{\mathrm{sym}})h$.
\end{proof}

Next we quantify the symbol floor by splitting the integral into a ``core'' region $[-r,r]$ and its complement.

\begin{lemma}[Core contribution]\label{lem:a3-core-lower-bound}
Let $0<r<B$. Set
\[
  m_r := \inf_{|\xi|\le r} a(\xi),\qquad
  M_B := \|a\|_{L^\infty([-B,B])}.
\]
Then
\[
  A_0 \ge 4\pi\,m_r\,r\left(1-\frac{r}{B}\right) e^{-4\pi^2 t_{\mathrm{sym}} r^2}
        - \frac{M_B}{2\pi\,t_{\mathrm{sym}} r} e^{-4\pi^2 t_{\mathrm{sym}} r^2}.
\]
\end{lemma}

\begin{proof}
Split the integral defining $A_0$ into $[-r,r]$ and its complement. On $[-r,r]$ we lower bound $a(\xi)$ by $m_r$, and on $|\xi|\in[r,B]$ we bound $|a(\xi)|$ by $M_B$. The integral of $\Phi_{B,t_{\mathrm{sym}}}$ over each region is computed explicitly, giving the stated inequality.
\end{proof}

\begin{lemma}[Shift-robust core mass]\label{lem:a3-core-shift}
Let $0<r<B$ and $|\tau|\le B-r$. Then the Fej\'er hat satisfies
\[
  \int_{\tau-r}^{\tau+r} \Lambda_B(x)\,dx\ \ge\ \frac{2r^2}{B}.
\]
Consequently, for every $t_{\mathrm{sym}}>0$,
\[
  \int_{\RR} \Lambda_B(x-\tau)\,e^{-4\pi^2 t_{\mathrm{sym}}(x-\tau)^2}\,dx
  \ \ge\ \frac{2r^2}{B}\,e^{-4\pi^2 t_{\mathrm{sym}} r^2}.
\]
\end{lemma}

\begin{proof}
The function $\Lambda_B$ is linear on each of the intervals $[-B,0]$ and $[0,B]$ with slope magnitude $1/B$. Among all translates of length $2r$ contained in $[-B,B]$ the smallest area is attained when the interval abuts one of the endpoints; a direct calculation yields
\(
\int_{B-2r}^{B}\Lambda_B(x)\,dx = \frac{2r^2}{B}.
\)
The same value is obtained on the symmetric left endpoint, and every other translate has strictly larger mass. For the Gaussian factor we use the pointwise bound $e^{-4\pi^2 t_{\mathrm{sym}}(x-\tau)^2}\ge e^{-4\pi^2 t_{\mathrm{sym}} r^2}$ whenever $|x-\tau|\le r$.
\end{proof}

\begin{lemma}[Archimedean floor]\label{lem:a3-arch-floor}
With notation as above define
\[
\begin{aligned}
  L_A^{\mathrm{up}}(B,t_{\mathrm{sym}}) &:= L_A(B,t_{\mathrm{sym}}), \\
  \underline{A}_0(B,r,t_{\mathrm{sym}})
    &:= 2 m_r\,r\left(1-\frac{r}{B}\right) e^{-4\pi^2 t_{\mathrm{sym}} r^2}
       - \frac{M_B}{4\pi^2 t_{\mathrm{sym}} r} e^{-4\pi^2 t_{\mathrm{sym}} r^2}.
\end{aligned}
\]
Then
\[
  \min_{\theta\in\TT} P_A(\theta)\ \ge\ \underline{A}_0(B,r,t_{\mathrm{sym}}) - \tfrac12 L_A^{\mathrm{up}}(B,t_{\mathrm{sym}}).
\]
\end{lemma}

\begin{proof}
For any $\theta$ choose a point $\theta_0$ at which $P_A$ attains its mean value and apply the mean-value inequality $P_A(\theta)\ge A_0 - \omega_{P_A}(|\theta-\theta_0|)$. Since $|\theta-\theta_0|\le \tfrac12$, the Lipschitz bound and Lemma~\ref{lem:a3-core-lower-bound} give the claimed inequality.
\end{proof}

\begin{lemma}[Core slope bound]\label{lem:a3-core-slope}
For $a(\xi)=\log\pi-\Re\psi(\tfrac14+i\pi\xi)$ and every $r>0$,
\[
  \inf_{|\xi|\le r} a(\xi)\ \ge\ a(0)\ -\ L_a\,r,
  \qquad L_a \le 20\pi,
\]
where
\(
  a(0)=\gamma + \frac{\pi}{2} + \log\pi + 3\log 2 > 0
\).
\end{lemma}
\begin{proof}
Differentiating $a$ yields
$a'(\xi) = \pi\,\Im\psi'(\tfrac14+i\pi\xi)$.
The trigamma admits the convergent series
\(
  \psi'(z)=\sum_{n\ge0}(n+z)^{-2}
\)
for $\Re z>0$ (see, e.g., \cite[Ch.~2]{Titchmarsh1986}), so
\[
  |\psi'(\tfrac14+i\pi\xi)|
  \le \sum_{n\ge0} \frac{1}{|n+\tfrac14+i\pi\xi|^2}
  \le \sum_{n\ge0} \frac{1}{(n+\tfrac14)^2}
  \le \frac{1}{(\tfrac14)^2} + \int_{0}^{\infty} \frac{dx}{(x+\tfrac14)^2}
  = 16 + 4 = 20.
\]
Therefore $|a'(\xi)|\le 20\pi$ for all $\xi$, and the mean-value theorem gives
$a(\xi)\ge a(0)-20\pi|\xi|$.

The identity
\(
  \psi(\tfrac14) = -\gamma - \tfrac{\pi}{2} - 3\log 2
\)
is recorded in Appendix~\ref{app:md23-constants}, equation~\eqref{eq:MD_2_3_constants-formula-11} (see also \cite[Ch.~2]{Titchmarsh1986}). Together with equation~\eqref{eq:MD_2_3_constants-formula-9} it implies
$a(0)=\log\pi-\Re\psi(\tfrac14)=\gamma+\tfrac{\pi}{2}+\log\pi+3\log 2$, which is positive. Substituting this identity into the mean-value estimate completes the proof.
\end{proof}

\begin{lemma}[Digamma monotonicity]\label{lem:a3-a-monotone}
For $\xi>0$ the Archimedean density satisfies
\[
  a'(\xi)
  = -2\pi^2\xi\sum_{n\ge0}\frac{n+\tfrac14}{\bigl((n+\tfrac14)^2+\pi^2\xi^2\bigr)^2},
\]
hence $a'(\xi)<0$ and $a$ is even and strictly decreasing on $[0,\infty)$.
Moreover, for $\xi\ge 1$,
\[
  |a'(\xi)| \ \le\ \frac{1}{|\xi|}+\frac{1}{2\pi^2|\xi|^3}
  \ \le\ \frac{11}{10}\cdot\frac{1}{|\xi|}.
\]
\end{lemma}
\begin{proof}
From the trigamma series
\(
  \psi'(z)=\sum_{n\ge0}(n+z)^{-2}
\)
and $\Im\,(x+iy)^{-2}=-2xy/(x^2+y^2)^2$ we obtain
\[
  a'(\xi)=\pi\,\Im\psi'\bigl(\tfrac14+i\pi\xi\bigr)
  = -2\pi^2\xi\sum_{n\ge0}\frac{n+\tfrac14}{\bigl((n+\tfrac14)^2+\pi^2\xi^2\bigr)^2},
\]
which is negative for $\xi>0$ because every term in the sum is positive.
For $\xi\ge1$, bounding the sum by the first term plus the integral
\(
  \int_{0}^{\infty}\frac{x+\tfrac14}{\bigl((x+\tfrac14)^2+\pi^2\xi^2\bigr)^2}\,dx
  = \frac{1}{2(\pi^2\xi^2+\tfrac{1}{16})}
\)
gives
\[
  |a'(\xi)|\le 2\pi^2|\xi|\Bigl(\frac{1}{2\pi^2\xi^2}+\frac{1}{2\pi^4\xi^4}\Bigr)
  = \frac{1}{|\xi|}+\frac{1}{2\pi^2|\xi|^3},
\]
and the last inequality follows since $\xi\ge1$.
\end{proof}

\begin{lemma}[Logarithmic growth bound]\label{lem:a3-a-log}
For $\xi\ge 1$ one has
\[
  |a(\xi)|\ \le\ a(0)\ +\ \frac{11}{10}\log(1+\xi).
\]
\end{lemma}
\begin{proof}
By Lemma~\ref{lem:a3-a-monotone}, $|a'(\xi)|\le \frac{11}{10}\xi^{-1}$ for $\xi\ge1$.
Integrating from $1$ to $\xi$ yields
\(
  |a(\xi)|\le |a(1)|+\frac{11}{10}\log\xi\le a(0)+\frac{11}{10}\log(1+\xi),
\)
since $a$ is decreasing and $|a(1)|\le a(0)$.
\end{proof}

%% ============================================================================
%% UNIFORM ARCHIMEDEAN FLOOR (Lemma 8.17') -- replaces K-dependent approach
%% ============================================================================

\begin{lemma}[Sample-point bounds for $a$]\label{lem:a3-a-sample}
The Archimedean density satisfies
\[
  a\Bigl(\tfrac12\Bigr)\ge \frac{29}{50},\qquad
  a\Bigl(\tfrac32\Bigr)\ge -\frac{3}{5},\qquad
  a\Bigl(\tfrac52\Bigr)\ge -\frac{11}{10}.
\]
\end{lemma}
\begin{proof}
Write $y=\pi\xi$ and
\[
  t_n(y):=\frac{1}{n+1}-\frac{n+\tfrac14}{(n+\tfrac14)^2+y^2},\qquad
  \Re\psi\Bigl(\tfrac14+i y\Bigr)=-\gamma+\sum_{n\ge0}t_n(y).
\]
For $\xi=\tfrac12$ one has $y^2=\pi^2/4$ and $t_n(y)\le0$ for $n\ge4$, so
\[
  \Re\psi\Bigl(\tfrac14+\tfrac{i\pi}{2}\Bigr)\ \le\ -\gamma+\sum_{n=0}^{3}t_n\Bigl(\tfrac{\pi}{2}\Bigr).
\]
Using $\pi^2<10$ gives the explicit upper bounds
\[
  t_0\le 1-\frac{1/4}{1/16+5/2}=\frac{37}{41},\quad
  t_1\le \frac12-\frac{5/4}{25/16+5/2}=\frac{5}{26},\quad
  t_2\le \frac13-\frac{9/4}{81/16+5/2}=\frac{13}{363},\quad
  t_3\le \frac14-\frac{13/4}{169/16+5/2}=\frac{1}{836}.
\]
Hence $\sum_{n=0}^{3}t_n(\pi/2)<1.132$. With the classical bounds
$\pi>333/106$ and $\gamma>0.5772$ we obtain $\log\pi+\gamma>1.72$, therefore
\[
  a\Bigl(\tfrac12\Bigr)=\log\pi+\gamma-\sum_{n\ge0}t_n(\pi/2)\ \ge\ 1.72-1.132\ >\ \frac{29}{50}.
\]

For $\xi\ge1$ we use the standard digamma remainder bound
\[
  \bigl|\psi(z)-\log z+\tfrac{1}{2z}\bigr|\le \frac{1}{12|z|^2}\qquad(\Re z>0)
\]
(\cite[\S5.11]{NISTDLMF}). Taking real parts at $z=\tfrac14+i\pi\xi$ yields
\[
  a(\xi)\ \ge\ -\log\xi\ -\ \frac{1}{2\pi\xi}\ -\ \frac{1}{12\pi^2\xi^2}.
\]
For $\xi=\tfrac32$, using $\pi>3$ and $\log(3/2)<\tfrac{5}{12}$ (from the alternating series for $\log(1+x)$ at $x=\tfrac12$) gives
\[
  a\Bigl(\tfrac32\Bigr)\ \ge\ -\frac{5}{12}-\frac{1}{9}-\frac{1}{243}\ >\ -\frac{3}{5}.
\]
For $\xi=\tfrac52$, using $\pi>3$ and $\log(5/2)<1$ (since $e>2.5$ by the series for $e$) gives
\[
  a\Bigl(\tfrac52\Bigr)\ \ge\ -1-\frac{1}{15}-\frac{1}{675}\ >\ -\frac{11}{10}.
\]
\end{proof}

\begin{lemma}[Uniform Archimedean floor (pointwise)]\label{lem:uniform-arch-floor}
Fix $t_{\mathrm{sym}}=\frac{3}{50}$ and $B_{\min}=3$. Then for every $B\ge B_{\min}$ and every
$\theta\in\TT$ the Archimedean symbol satisfies
\[
  P_A(\theta)\ \ge\ c_*\ :=\ \frac{11}{10}.
\]
\end{lemma}
\begin{proof}
By Lemma~\ref{lem:a3-a-monotone} the function $a(\xi)$ is decreasing on $[0,\infty)$, and the window
\(
  w_B(\xi):=(1-\xi/B)e^{-4\pi^2 t_{\mathrm{sym}}\xi^2}
\)
is decreasing for $\xi\in[0,B]$.  For $\theta\in[0,\tfrac12]$ and $B\ge3$ we have
\[
  P_A(\theta)=2\pi\sum_{m\in\ZZ} g_{B,t_{\mathrm{sym}}}(\theta+m)
  \ge 2\pi\bigl(g(\theta)+2g(\theta+1)+2g(\theta+2)\bigr) - \mathcal T,
\]
where $\mathcal T$ collects the $|m|\ge3$ tail terms.  Using monotonicity and the bounds
from Lemma~\ref{lem:a3-a-sample} we obtain
\[
  g(\theta)\ge \frac{29}{50}\,w_{B_{\min}}\!\Bigl(\tfrac12\Bigr),\quad
  g(\theta+1)\ge -\frac{3}{5}\,w_{B_{\min}}(1),\quad
  g(\theta+2)\ge -\frac{11}{10}\,w_{B_{\min}}(2).
\]
With $\pi<\tfrac{22}{7}$ we have $\pi^2<10$, so
\[
  w_{B_{\min}}\!\Bigl(\tfrac12\Bigr)
  \ge \frac{5}{6}e^{-3/5}
  >\frac{9}{20}.
\]
Using $\pi>\tfrac{333}{106}$ we obtain
\(
  \frac{6\pi^2}{25}>\frac{665334}{280900}.
\)
The tail is bounded by
\[
  \mathcal T \le 4\pi\int_{5/2}^{\infty}\!|a(\xi)|\,e^{-4\pi^2 t_{\mathrm{sym}}\xi^2}\,d\xi
  \le 10^{-5},
\]
using Lemma~\ref{lem:a3-a-log} to bound $|a(\xi)|\le a(0)+\tfrac{11}{10}\log(1+\xi)\le 8$ for $\xi\ge\tfrac52$
and the Gaussian tail bound $\int_{R}^{\infty}e^{-\alpha \xi^2}\,d\xi\le e^{-\alpha R^2}/(2\alpha R)$.
The exponential series bound
\(
  e^{x}\ge \sum_{j=0}^{5}\frac{x^j}{j!}
\)
applied at \(x=\frac{665334}{280900}\) gives \(e^{6\pi^2/25}>10\), hence
\[
  w_{B_{\min}}(1)\le \frac{2}{3}e^{-6\pi^2/25}\le \frac{1}{15},\qquad
  w_{B_{\min}}(2)\le \frac{1}{3}e^{-24\pi^2/25}\le \frac{1}{30000}.
\]
Combining the inequalities and using \(2\pi>\tfrac{666}{106}\) yields
\[
  P_A(\theta)\ \ge\ 2\pi\Bigl(\frac{29}{50}\cdot\frac{9}{20}-\frac{2}{25}-\frac{11}{150000}\Bigr)-10^{-5}
  \ >\ \frac{11}{10}.
\]
Evenness of $P_A$ completes the bound on all of $\TT$.
\end{proof}

\begin{definition}[Uniform Lipschitz constant]\label{def:uniform-L}
For $B\ge B_{\min}$ set
\[
  L_A(B,t_{\mathrm{sym}})
  := 2\pi\sup_{\theta\in[-\tfrac12,\tfrac12]}
      \sum_{m\in\ZZ}\bigl|g_{B,t_{\mathrm{sym}}}'(\theta+m)\bigr|,
\]
and define
\[
  L_*(t_{\mathrm{sym}}) := \sup_{B\ge B_{\min}}L_A(B,t_{\mathrm{sym}}).
\]
The constant $L_*$ is used only for discretisation in the Szeg\H{o}--B\"ottcher bridge.
\end{definition}

\begin{corollary}[Uniform discretisation threshold]\label{cor:uniform-discretisation}
Assume $c_*>0$ in Lemma~\ref{lem:uniform-arch-floor}, and let $C_{\mathrm{SB}}=4$ be the absolute constant of Lemma~\ref{lem:a3-sb-barrier}.
Define
\[
  M_{0}^{\mathrm{unif}}
  :=\left\lceil \frac{C_{\mathrm{SB}}\,L_*(t_{\mathrm{sym}})}{c_*}\right\rceil.
\]
Then for every $B\ge B_{\min}$ and every $M\ge M_{0}^{\mathrm{unif}}$,
\[
  \lambda_{\min}\bigl(T_M[P_A]\bigr)\ \ge\ \frac{1}{2}\,c_*.
\]
\end{corollary}

\begin{corollary}[Uniform prime cap time]\label{cor:uniform-prime-cap}
Assume $c_*>0$ in Lemma~\ref{lem:uniform-arch-floor}. Define
\[
  t_{\star,\mathrm{rkhs}}^{\mathrm{unif}}:=1.
\]
Then for every $t_{\mathrm{rkhs}}\ge t_{\star,\mathrm{rkhs}}^{\mathrm{unif}}$ the symmetrised prime operator satisfies
\[
  \|T_P\|\ \le\ \rho(t_{\mathrm{rkhs}})\ \le\ \rho(1)\ <\ \frac{1}{25}\ <\ \frac{c_*}{4},
\]
where $\rho(t)$ is the Gaussian norm cap of Lemma~\ref{pm:lem:rho-closed-form}.
\end{corollary}

%% ============================================================================
%% DIGAMMA BOUNDS -- concrete numerics for t_sym = 3/50
%% ============================================================================

\paragraph{Definition of $B_{\min}$.}
We fix the minimal bandwidth parameter:
\begin{equation}\label{eq:Bmin-def}
  B_{\min} := 3.
\end{equation}
This choice ensures that the Fej\'er kernel $F_B$ has sufficient width to capture the essential support of the Archimedean density $a(\xi)$, while the Gaussian factor $e^{-4\pi^2 t_{\mathrm{sym}}\xi^2}$ with $t_{\mathrm{sym}}=3/50$ provides rapid decay beyond $|\xi|\approx 2$.

\begin{lemma}[Analytic mean bound (auxiliary)]\label{lem:digamma-mean-bound}
Let $t_{\mathrm{sym}}=\frac{3}{50}$ and $B_{\min}=3$, and define
\(
  A_*(t_{\mathrm{sym}}):=\inf_{B\ge B_{\min}} A_0(B,t_{\mathrm{sym}}).
\)
Write $\alpha:=4\pi^2 t_{\mathrm{sym}}$ and let $L_a$ be the global slope bound from Lemma~\ref{lem:a3-core-slope}.
For any $r\in\bigl(0,\min\{B_{\min},a(0)/L_a\}\bigr)$ define
\[
  A_{\mathrm{low}}(r)
  := 4\pi\bigl(a(0)-L_a r\bigr)\int_{0}^{r}\!\Bigl(1-\frac{\xi}{B_{\min}}\Bigr)e^{-\alpha \xi^2}\,d\xi
     - 4\pi\int_{r}^{\infty}\!\bigl(a(0)+L_a \xi\bigr)e^{-\alpha \xi^2}\,d\xi.
\]
Then $A_*(t_{\mathrm{sym}})\ge A_{\mathrm{low}}(r)$ for every $r\in(0,B_{\min})$.
\end{lemma}

\begin{proof}
By Lemma~\ref{lem:a3-core-slope}, $|a(\xi)-a(0)|\le L_a|\xi|$ for all $\xi$, hence
$a(\xi)\ge a(0)-L_a|\xi|$ on $|\xi|\le r$ and $a(\xi)\ge -\bigl(a(0)+L_a|\xi|\bigr)$ for $|\xi|\ge r$.
The restriction $r\le a(0)/L_a$ ensures $a(0)-L_a\xi\ge0$ on $[0,r]$. Since $0\le (1-|\xi|/B)\le 1$, we obtain
\[
  A_0(B,t_{\mathrm{sym}})
  \ge 4\pi\int_{0}^{r}\!\bigl(a(0)-L_a\xi\bigr)\Bigl(1-\frac{\xi}{B}\Bigr)e^{-\alpha \xi^2}\,d\xi
     - 4\pi\int_{r}^{B}\!\bigl(a(0)+L_a\xi\bigr)e^{-\alpha \xi^2}\,d\xi.
\]
For $B\ge B_{\min}$ the first integral is bounded below by replacing $B$ with $B_{\min}$,
and the second is bounded below by extending to $[r,\infty)$. This yields the stated bound,
uniformly in $B$, and therefore $A_*(t_{\mathrm{sym}})\ge A_{\mathrm{low}}(r)$.
\end{proof}

\begin{lemma}[Analytic Lipschitz bound (auxiliary)]\label{lem:digamma-lip-bound}
Let $t_{\mathrm{sym}}=\frac{3}{50}$, $B_{\min}=3$, and $L_*(t_{\mathrm{sym}})=\sup_{B\ge B_{\min}} L_A(B,t_{\mathrm{sym}})$ as in Definition~\ref{def:uniform-L}.
Set $\alpha:=4\pi^2 t_{\mathrm{sym}}$ and $L_a$ as in Lemma~\ref{lem:a3-core-slope}. For $B\ge B_{\min}$ one has
$\Phi_{B,t_{\mathrm{sym}}}(\xi)\le e^{-\alpha\xi^2}$ and
$
|\Phi_{B,t_{\mathrm{sym}}}'(\xi)|
\le \bigl(B_{\min}^{-1}+8\pi^2 t_{\mathrm{sym}}|\xi|\bigr)e^{-\alpha\xi^2}.
$
Define
\[
  G(\xi)
  := L_a\,e^{-\alpha\xi^2}
     + \bigl(a(0)+L_a|\xi|\bigr)\bigl(B_{\min}^{-1}+8\pi^2 t_{\mathrm{sym}}|\xi|\bigr)e^{-\alpha\xi^2},
\]
and
\[
  L_{\mathrm{up}}
  := 2\pi\Bigl(G(0) + 2\int_{0}^{\infty} G(\xi)\,d\xi\Bigr).
\]
Then $L_*(t_{\mathrm{sym}})\le L_{\mathrm{up}}$.
\end{lemma}

\begin{proof}
Lemma~\ref{lem:a3-lipschitz-bound} gives
$L_A(B,t_{\mathrm{sym}})=2\pi\sup_{\theta\in[-\tfrac12,\tfrac12]}\sum_{m\in\ZZ}|g_{B,t_{\mathrm{sym}}}'(\theta+m)|$.
Since $g'=a'\Phi+a\Phi'$, the bounds $|a'|\le L_a$ and $|a|\le a(0)+L_a|\xi|$ together with the
envelope bounds on $\Phi_{B,t_{\mathrm{sym}}}$ and $\Phi_{B,t_{\mathrm{sym}}}'$ yield $|g'(\xi)|\le G(\xi)$.
The function $G$ is even and decays Gaussianly, so for each $\theta$
the sum is dominated by a unit-step Riemann sum of $G$; in particular
\[
  \sum_{m\in\ZZ} G(\theta+m)\ \le\ G(0) + 2\int_{0}^{\infty} G(\xi)\,d\xi.
\]
Taking the supremum over $\theta$ and then $B\ge B_{\min}$ gives $L_*(t_{\mathrm{sym}})\le L_{\mathrm{up}}$.
\end{proof}

\begin{remark}[Auxiliary mean--modulus route]\label{rem:aux-mean-modulus}
Lemmas~\ref{lem:digamma-mean-bound} and~\ref{lem:digamma-lip-bound} provide analytic estimates for
$A_*(t_{\mathrm{sym}})$ and $L_*(t_{\mathrm{sym}})$, which may be used for coarse bounds on $\min P_A$
via the mean--modulus inequality.  The main proof instead uses the direct pointwise floor from
Lemma~\ref{lem:uniform-arch-floor}, which yields a much sharper constant.
\end{remark}

\begin{remark}[Uniform vs.\ compact-dependent approach]\label{rem:uniform-vs-compact}
The uniform constants $(A_*, L_*, c_*)$ depend only on $t_{\mathrm{sym}}$ and $B_{\min}$, not on the compact $[-K,K]$.
This decouples the Archimedean floor from the parameter schedule $B(K)$: once $c_*>0$ is established quantitatively,
every compact inherits the same positive margin without requiring monotonicity arguments or numerical tables.
\end{remark}

\begin{remark}[References]
The Lipschitz estimate relies on standard Fourier analysis for compactly supported smooth kernels (see, e.g., Stein--Shakarchi~\cite[Ch.~2]{SteinShakarchi2003} and Zygmund~\cite[Ch.~I]{Zygmund2002}), while bounds on $a$ and $a'$ follow from classical properties of the digamma function (\cite[\S5.2]{NISTDLMF}).  The quantitative Toeplitz eigenvalue barrier used later takes the form $\lambda_{\min}(T_M[P])\ge\min P - C_{\mathrm{SB}}\,\omega_P(1/(2M))$ with $C_{\mathrm{SB}}=4$, as recorded in B\"ottcher--Silbermann~\cite[Ch.~5]{BoettcherSilbermann2006}.
\end{remark}

% Toeplitz--symbol bridge (A3) --- quantitative formulation
\SeeAlso{Lemmas~\ref{lem:a3-lipschitz-bound}--\ref{lem:a3.cap-combine}, Uniform floor Lemma~\ref{lem:uniform-arch-floor}, uniform RKHS cap Lemma~\ref{lem:rkhs-uniform-cap-full}.}

Throughout this section $a$ denotes the Archimedean density and $P_A$ denotes the Archimedean symbol defined in \S\ref{subsec:a3-symbol-floor} via the Fej\'er$\times$heat window. We use only this definition in the main proof chain. The arguments below sit inside the classical Toeplitz framework of Szeg\H{o} and B\"ottcher~\cite{Szego1952,Bottcher1907,GrenanderSzego1958,BoettcherSilbermann2006}, with Fourier bounds calibrated against standard real-analytic estimates~\cite{SteinShakarchi2003,Zygmund2002}.
The following chain of lemmas replaces all “A3 assume …” statements by explicit estimates.  An analytic proof of the Rayleigh identification is recorded in \S\ref{subsec:a3-rayleigh}, while symbol regularity and Archimedean floors are collected in \S\ref{subsec:a3-symbol-floor}.

\begin{remark}[Two-scale architecture]\label{rem:two-scales}
The Toeplitz bridge and the prime contraction employ two independent smoothing parameters.
\begin{itemize}
  \item On the \emph{symbol side}, $t_{\mathrm{sym}}$ enters the Fej\'er$\times$heat
  convolution that produces $P_A$; together with the bandwidth $B$ it controls the
  modulus $\omega_{P_A}(1/(2M))$ in the Szeg\H{o}--B\"ottcher bridge.
  \item On the \emph{RKHS side}, $t_{\mathrm{rkhs}}$ is the heat scale in the Gaussian kernel
  used to bound $\|T_P\|$; in the uniform branch we take $t_{\mathrm{rkhs}}\ge t_{\star,\mathrm{rkhs}}^{\mathrm{unif}}$
  as in Corollary~\ref{cor:uniform-prime-cap}.
\end{itemize}
The Fej\'er$\times$heat tests are built with $t_{\mathrm{sym}}$, whereas the RKHS analysis uses
$t_{\mathrm{rkhs}}$; no coupling between the two scales is needed.
\end{remark}

\begin{lemma}[Two-scale separation (uniform)]\label{lem:a3.two-scale}
Let $P_A$ be the Archimedean symbol from \S\ref{subsec:a3-symbol-floor} built with the Fej\'er$\times$heat parameter $t_{\mathrm{sym}}$, and let $t_{\mathrm{rkhs}}\ge t_{\star,\mathrm{rkhs}}^{\mathrm{unif}}$ be the RKHS scale from Corollary~\ref{cor:uniform-prime-cap}. Then
\[
  \min_{\theta\in\TT} P_A(\theta)\ \ge\ c_*,
\]
by Lemma~\ref{lem:uniform-arch-floor}, and the RKHS cap $\|T_P\|\le\rho(t_{\mathrm{rkhs}})$ follows from Corollary~\ref{cor:uniform-prime-cap}. Thus the symbol scale $t_{\mathrm{sym}}$ and the RKHS scale $t_{\mathrm{rkhs}}$ are decoupled in the uniform branch.
\end{lemma}
\phantomsection\label{a3:lem:A3_dini_SB}

\begin{proof}
Immediate from Lemma~\ref{lem:uniform-arch-floor} and Corollary~\ref{cor:uniform-prime-cap}.
\end{proof}

\begin{lemma}[Lipschitz symbol with positive floor implies A3 prerequisites]\label{lem:a3.lip-floor}
Let $\Pa\in\Lip(1)$ with $\min_{\TT}\Pa\ge \czero>0$.  Then the Toeplitz operator $\TPA$ satisfies
\[
  \TPA \succeq \czero\,I,\qquad \|\TPA\|_{\op}\le \|\Pa\|_{L^\infty}.
\]
In particular, once $\rhok\ge \|\Pa\|_{L^\infty}$ the A3-lock positivity and boundedness hypotheses hold.
\end{lemma}

\begin{proof}
For any $f$ with $\|f\|_2=1$ we have $\ip{\TPA f}{f}=\int_{\TT}\Pa(\theta)|f(\theta)|^2\,d\theta\ge \czero$, hence $\TPA\succeq \czero I$.  The $\|\Pa\|_{\infty}$ bound is immediate from the Rayleigh quotient; see, e.g., the spectral calculus in~\cite{HornJohnson2013,Varga2004}.
\end{proof}

\begin{lemma}[Combining with the RKHS cap]\label{lem:a3.cap-combine}
Suppose $\Pa$ is constructed as above and the RKHS cap
\[
  \|T_P\|\le \rho(t_{\mathrm{rkhs}})
\]
holds (Corollary~\ref{cor:uniform-prime-cap}). Then $\TPA$ simultaneously satisfies the positivity floor and the operator-norm bound required by A3-lock.
\end{lemma}

\begin{proof}
Apply Lemmas~\ref{lem:a3.two-scale} and \ref{lem:a3.lip-floor}, together with Corollary~\ref{cor:uniform-prime-cap}.
\end{proof}

\paragraph{A3 input summary (uniform version).}
\begin{enumerate}[label=(A3-U.\arabic*)]
\item \emph{Uniform Arch floor.} Lemma~\ref{lem:uniform-arch-floor} provides the explicit floor $c_*=\tfrac{11}{10}$ on $\TT$ for all $B\ge B_{\min}$.
  \item \emph{Uniform prime cap.} Corollary~\ref{cor:uniform-prime-cap} gives $t_{\star,\mathrm{rkhs}}^{\mathrm{unif}}$ with $\rho(t_{\mathrm{rkhs}})\le c_*/4$ for all $t_{\mathrm{rkhs}}\ge t_{\star,\mathrm{rkhs}}^{\mathrm{unif}}$.
  \item \emph{Uniform discretisation.} Corollary~\ref{cor:uniform-discretisation} provides $M_0^{\mathrm{unif}}$ such that $\lambda_{\min}(T_M[P_A])\ge c_*/2$ for all $M\ge M_0^{\mathrm{unif}}$.
\end{enumerate}

%% Legacy K-dependent remark removed (December 2025 cleanup).
%% The original per-compact summary used c_arch(K), t_rkhs*(K), M_0(K).
%% This is superseded by the uniform version above.

%% NOTE: The uniform approach (Lemma 8.17') gives a K-independent floor c_* > 0
%% on the FULL circle T, not just an arc. This supersedes the arc-based approach.
The uniform Archimedean floor (Lemma~\ref{lem:uniform-arch-floor}) provides a \textbf{K-independent} lower bound
\[
  \min_{\theta\in\TT}P_A(\theta)\ge c_*
\]
valid for the \textbf{entire circle} $\TT$ and all $B\ge B_{\min}$.
This eliminates the need for arc-based constructions and provides a simpler, cleaner proof.

%% Remark about legacy arc data and JSON files removed (December 2025 cleanup).
%% The main proof relies entirely on the quantitative bound c_* > 0.

\begin{theorem}[Uniform A3 bridge]\label{thm:A3}
Assume the uniform floor $c_*>0$ from Lemma~\ref{lem:uniform-arch-floor} and fix
$B\ge B_{\min}$, $t_{\mathrm{sym}}=\tfrac{3}{50}$, and $t_{\mathrm{rkhs}}\ge t_{\star,\mathrm{rkhs}}^{\mathrm{unif}}$.
Then for every $M\ge M_0^{\mathrm{unif}}$ (Corollary~\ref{cor:uniform-discretisation}),
\[
  \lambda_{\min}\!\bigl(T_M[P_A]-T_P\bigr)\ \ge\ \frac{c_*}{4}\ >\ 0,
\]
and the associated Fej\'er$\times$heat test functions satisfy
\[
  Q(\Phi_{B,t_{\mathrm{sym}}})\ \ge\ 0.
\]
\end{theorem}

\begin{proof}
The uniform floor Lemma~\ref{lem:uniform-arch-floor} gives $\min_{\theta\in\TT} P_A(\theta)\ge c_*$
for all $B\ge B_{\min}$. Corollary~\ref{cor:uniform-discretisation} provides
$M_0^{\mathrm{unif}}$ such that $C_{\mathrm{SB}}\,\omega_{P_A}(1/(2M))\le c_*/2$ for all $M\ge M_0^{\mathrm{unif}}$.
Corollary~\ref{cor:uniform-prime-cap} gives $\|T_P\|\le \rho(t_{\mathrm{rkhs}})\le c_*/4$.
Thus $\lambda_{\min}(T_M[P_A]-T_P)\ge c_* - c_*/2 - c_*/4 = c_*/4$.
Lemma~\ref{lem:a3-model-space} combined with Theorem~\ref{thm:a3-rayleigh-identification}
converts the matrix margin into $Q(\Phi_{B,t})\ge0$.
\end{proof}

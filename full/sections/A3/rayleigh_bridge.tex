% Analytic rewrite: Rayleigh identification for the Toeplitz bridge
% Phase 1 (Brick A3.1) – derived from docs/lemmas/A3_bridge_outline.md

\subsection{Rayleigh Identification for the Toeplitz Bridge}\label{subsec:a3-rayleigh}
Throughout we fix a Fej\'er$\times$heat window
\[
  \Phi_{B,t}(\xi)\ :=\ \Bigl(1-\frac{|\xi|}{B}\Bigr)_{+}\,e^{-4\pi^2 t\,\xi^2},
\]
and write $P_A$ for the associated Archimedean symbol obtained by smoothing the
T0 density $a(\xi)=\log\pi-\Re\psi(\tfrac14+i\pi\xi)$ with the Fej\'er and heat kernels
on $[-B,B]$.  The prime weights are $w(n)=\frac{2\Lambda(n)}{\sqrt n}$ located at the
nodes $\xi_n=\frac{\log n}{2\pi}$, as fixed in Section~\ref{sec:T0}.

Let $\mathcal P_M:=\{p(\theta)=\sum_{|k|\le M}c_k e^{2\pi i k\theta}\}$ denote the trigonometric
polynomials of degree at most $M$, equipped with the $L^2(\TT)$ inner product, and let
$\iota_M:\mathcal P_M\hookrightarrow L^2(\TT)$ be the canonical inclusion with adjoint
$\iota_M^\ast$ equal to the orthogonal projection onto $\mathcal P_M$.

\begin{lemma}[Model--space restriction]\label{lem:a3-model-space}
The Toeplitz operator $T_M[P_A]$ acts on $\mathcal P_M$, is self-adjoint and satisfies
\[
  \left\langle T_M[P_A]\,p,\,p\right\rangle_{L^2(\TT)}
  = \int_{-\tfrac12}^{\tfrac12} P_A(\theta)\,|p(\theta)|^2\,d\theta,
  \qquad p\in\mathcal P_M.
\]
Moreover, the symmetrised prime operator
\[
  T_P^{(M)}\ :=\ \sum_{\substack{n\ge2\\ |\xi_n|\le B}}
     w(n)\,\Phi_{B,t}(\xi_n)\,|v_n^{(M)}\rangle\!\langle v_n^{(M)}|,
  \qquad
  v_n^{(M)}(\theta)\ :=\ \frac{1}{\sqrt{2M+1}}\sum_{|k|\le M} e^{2\pi i k(\theta-\xi_n)},
\]
is the orthogonal compression of the global prime operator $T_P$ to $\mathcal P_M$,
and is positive semidefinite with
\[
  \|T_P^{(M)}\| \ \le\ \sum_{\substack{n\ge2\\ |\xi_n|\le B}} w(n)\,\Phi_{B,t}(\xi_n).
\]
\end{lemma}

\begin{proof}
The Toeplitz matrix $T_M[P_A]$ is the compression of the Fourier multiplier with
symbol $P_A$ to $\mathcal P_M$; the stated quadratic form is the standard representation
of Toeplitz forms (see, e.g., \cite[Chapter~1]{GrenanderSzego1958}).  For the prime
operator note that
$T_P=\sum_{n\ge2} w(n)\Phi_{B,t}(\xi_n)\,|e^{2\pi i(\cdot)\xi_n}\rangle\!\langle e^{2\pi i(\cdot)\xi_n}|$
is a finite-rank positive operator on $L^2(\TT)$, hence
$T_P^{(M)}=\iota_M^\ast T_P \iota_M$ is self-adjoint and positive semidefinite.
The displayed norm bound is immediate from the triangle inequality applied to
the sum of rank-one projections $|v_n^{(M)}\rangle\!\langle v_n^{(M)}|$.
\end{proof}

\begin{lemma}[Rayleigh pairing]\label{lem:a3-rayleigh-quotient}
For every $p\in\mathcal P_M$ one has
\[
  \left\langle (T_M[P_A]-T_P^{(M)})\,p,\,p\right\rangle_{L^2(\TT)}
  = \int_{-\tfrac12}^{\tfrac12} P_A(\theta)\,|p(\theta)|^2\,d\theta
    - \sum_{\substack{n\ge2\\ |\xi_n|\le B}} w(n)\,\Phi_{B,t}(\xi_n)\,|p(\xi_n)|^2.
\]
\end{lemma}

\begin{proof}
Combine Lemma~\ref{lem:a3-model-space} with the definition of $T_P^{(M)}$ and the identities
$p(\xi_n)=\langle p,v_n^{(M)}\rangle$ and $\|v_n^{(M)}\|=1$.
\end{proof}

\begin{theorem}[Rayleigh identification for the Fej\'er$\times$heat window]\label{thm:a3-rayleigh-identification}
Let $\Phi_{B,t}$ and $P_A$ be as above, and let $p\equiv1$ be the constant polynomial.
Then
\[
  \left\langle (T_M[P_A]-T_P^{(M)})\,1,\,1\right\rangle_{L^2(\TT)}
  = \int_{-\tfrac12}^{\tfrac12} P_A(\theta)\,d\theta
    - \sum_{\substack{n\ge2\\ |\xi_n|\le B}} w(n)\,\Phi_{B,t}(\xi_n)
  = Q(\Phi_{B,t}),
\]
where $Q$ is the Weil functional in the T0 normalization (Lemma~\ref{t0:lem:T0}).  In particular,
$Q(\Phi_{B,t})\ge0$ if and only if the Rayleigh quotient on the left-hand side is nonnegative.
\end{theorem}

\begin{proof}
Applying Lemma~\ref{lem:a3-rayleigh-quotient} with $p\equiv1$ yields
\[
  \left\langle (T_M[P_A]-T_P^{(M)})\,1,\,1\right\rangle_{L^2(\TT)}
  = \int_{-\tfrac12}^{\tfrac12} P_A(\theta)\,d\theta
    - \sum_{n\ge2} w(n)\,\Phi_{B,t}(\xi_n),
\]
where the prime sum is finite because $\Phi_{B,t}$ is supported in $[-B,B]$.
By definition of $P_A$ and the normalization fixed in Section~\ref{sec:T0} one has
\[
  \int_{-\tfrac12}^{\tfrac12} P_A(\theta)\,d\theta
  = \int_{\RR} a_*(\xi)\,\Phi_{B,t}(\xi)\,d\xi,
\]
and Lemma~\ref{t0:lem:T0} gives
$Q(\Phi_{B,t}) = \int_{-\tfrac12}^{\tfrac12}P_A(\theta)\,d\theta
       - \sum_{n\ge2}w(n)\,\Phi_{B,t}(\xi_n)$.
Therefore the Rayleigh quotient equals $Q(\Phi_{B,t})$, proving the claim.
\end{proof}

\subsection{A3 locking summary}\label{a3:sec:locks}

We record how the local ingredients assembled in \S\ref{sec:A3} feed the global lock:

\begin{itemize}
  \item Lemma~\ref{lem:a3.cap-combine} supplies the bounded-overlap control on caps.
  \item Lemma~\ref{lem:a3.two-scale} records the uniform two-scale separation.
  \item Corollary~\ref{cor:uniform-prime-cap} gives the uniform RKHS prime cap for $t_{\mathrm{rkhs}}\ge t_{\star,\mathrm{rkhs}}^{\mathrm{unif}}$.
  \item Theorem~\ref{thm:A3} combines the uniform symbol floor $c_*>0$ with the RKHS prime cap from Corollary~\ref{cor:uniform-prime-cap} and the Frobenius guard of Corollary~\ref{cor:toeplitz-frob}.
\end{itemize}

\begin{corollary}[Lock]\label{cor:D3-lock}
Under the hypotheses of Lemmas~\ref{lem:a3.cap-combine}, \ref{lem:a3.two-scale}, and Corollary~\ref{cor:uniform-prime-cap} the A3 lock closes with a constant depending only on the overlap bound and the uniform prime cap.
\end{corollary}
\phantomsection\label{a3:cor:A3_lock_formal}

\begin{proof}
Lemma~\ref{lem:a3.cap-combine} gives almost orthogonality, Lemma~\ref{lem:a3.two-scale}
controls interactions between scales, Corollary~\ref{cor:uniform-prime-cap} provides the uniform cap,
and Theorem~\ref{thm:A3} supplies the quantitative margin with the uniform floor.
Summing the contributions yields the stated lock.
\end{proof}

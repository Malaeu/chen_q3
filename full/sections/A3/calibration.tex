% A3 Calibration: explicit value of \kappa_{A3}
\subsection{A3 Calibration: The Constant $\kappa_{\mathrm{A3}}(t_0)$}
\label{a3:sec:A3-calibration}
\SeeAlso{Normalization T0 Lemma~\ref{t0:lem:T0}, Toeplitz bridge A3 Theorem~\ref{thm:A3}.}

\begin{lemma}[Calibration of $\kappa_{\mathrm{A3}}$]\label{a3:lem:A3-kappa}
Let $\Phi(\xi)=(1-|\xi|/B)_+\,e^{-4\pi^2 t_0\,\xi^2}$ be an even Fej\'er$\times$heat window. Define the Arch coefficients
\begin{equation}\label{eq:A3_calibration_kappa-formula}
  A_k\ :=\ \int_{\RR} a(\xi)\,\Phi(\xi)\,\cos(k\xi)\,d\xi,\quad
  P_A(\theta)\ :=\ A_0+2\sum_{k\ge1}A_k\cos(k\theta),
\end{equation}
with $a(\xi)=\log\pi-\Re\psi\big(\tfrac14+i\pi\xi\big)$, and let $T_P$ be the even prime sampling operator with weights $w(n)=\tfrac{2\Lambda(n)}{\sqrt n}$ at nodes $\xi_n=\tfrac{\log n}{2\pi}$. Then, in the Rayleigh identification of Theorem~\ref{thm:A3}, at the constant test $p\equiv1$ one has
\begin{equation}\label{eq:A3_calibration_kappa-formula-3}
  \frac{1}{2\pi}\int_{-\pi}^{\pi} P_A(\theta)\,d\theta\ -\ \sum_{n\ge2}\frac{2\,\Lambda(n)}{\sqrt n}\,\Phi(\xi_n)
  \ =\ \underbrace{\int_{\RR} a(\xi)\,\Phi(\xi)\,d\xi}_{\;=\ A_0}\ -\ \sum_{n\ge2}\frac{2\,\Lambda(n)}{\sqrt n}\,\Phi(\xi_n).
\end{equation}
By the T0 normalization (Lemma~\ref{t0:lem:T0}), the Weil functional on our axis is
\begin{equation}\label{eq:A3_calibration_kappa-q-functional-1}
  Q(\Phi)\ =\ \int_{\RR} a_*(\xi)\,\Phi(\xi)\,d\xi\ -\ \sum_{n\ge2}\frac{2\,\Lambda(n)}{\sqrt n}\,\Phi(\xi_n),\qquad a_*(\xi):=2\pi\,a(\xi).
\end{equation}
Therefore
\begin{equation}\label{eq:A3_calibration_kappa-q-functional}
  Q(\Phi)\ =\ 2\pi\,\Bigg[\,\frac{1}{2\pi}\int_{-\pi}^{\pi} P_A(\theta)\,\frac{d\theta}{1}\ -\ \sum_{n\ge2}\frac{2\,\Lambda(n)}{\sqrt n}\,\Phi(\xi_n)\,\Bigg],
\end{equation}
and the bridge A3 introduces the fixed scale factor
\begin{equation}\label{eq:A3_calibration_kappa-formula-2}
\boxed{\ \kappa_{\mathrm{A3}}(t_0)\ =\ 2\pi\ }\qquad\text{(independent of $t_0$)}.
\end{equation}
Equivalently, if one works directly with $a_*(\xi)=2\pi a(\xi)$ in the definition of $P_A$ (absorbing the Jacobian into the symbol), then $\kappa_{\mathrm{A3}}\equiv1$.
\end{lemma}
\phantomsection\label{a3:lem:rayleigh_sampling_id}

\begin{lemma}[Rayleigh identification]\label{lem:rayleigh-sampling}
For every even Fej\'er$\times$heat window $\Phi$ the operator form and the Weil functional satisfy
\[
  \Big\langle \bigl(T_M[P_A]-T_P\bigr) p,\,p\Big\rangle \;=\; \frac{1}{2\pi}\,Q(\Phi)
\]
whenever $p$ corresponds to $\Phi$ via the standard Dirichlet sampling operator.
\end{lemma}

\begin{proof}
Write the Fej\'er$\times$heat window as
\[
  \Phi(\xi)\ =\ \sum_{k\in\ZZ} \widehat \Phi(k)\,e^{ik\xi},\qquad
  \widehat \Phi(k) = \int_{\RR}\Phi(\xi)e^{-ik\xi}\,d\xi.
\]
The Dirichlet sampling operator maps $p(\theta)=\sum_{k\in\ZZ} \widehat \Phi(k)\,e^{ik\theta}$ to $\Phi$; hence
\[
  \Big\langle T_M[P_A]p,\,p\Big\rangle
  = \sum_{k\in\ZZ} A_k\,|\widehat \Phi(k)|^2
  = A_0\,|\widehat \Phi(0)|^2 + 2\sum_{k\ge1}A_k\,|\widehat \Phi(k)|^2,
\]
where $A_k$ are the Arch coefficients from \eqref{eq:A3_calibration_kappa-formula}.  Likewise, the prime operator contributes
\[
  \Big\langle T_P p,\,p\Big\rangle
  = \sum_{n\ge2} \frac{2\,\Lambda(n)}{\sqrt{n}}\,\Phi(\xi_n)\,\overline{\Phi(\xi_n)}.
\]
Subtracting, inserting the factor $2\pi$ from \eqref{eq:A3_calibration_kappa-q-functional}, and recalling $Q(\Phi)$ from \eqref{eq:A3_calibration_kappa-q-functional-1} gives
\[
  \Big\langle (T_M[P_A]-T_P)p,\,p\Big\rangle
  = \frac{1}{2\pi}\,Q(\Phi),
\]
which is the desired identity.
\end{proof}

\begin{proposition}[Bridge margin calibration]\label{prop:a3-calib}
Under Assumptions \textnormal{(A3.1)--(A3.3)} and the standing slack $\rhok\le \czero/2$ the mixed Toeplitz block satisfies
\[
  \lammin\!\bigl(T_M[\Pa]-T_P\bigr)\ \ge\ \tfrac12\bigl(\czero-\rhok\bigr)
\]
for every $M\ge M_0(K)$ in Theorem~\ref{thm:A3}.
\end{proposition}

\begin{proof}
Theorem~\ref{thm:A3} yields $\lammin(T_M[\Pa]-T_P)\ge \czero-C_{\mathrm{SB}}\omega_{\Pa}(\pi/M)-\|T_P\|_{\op}$. Assumption (A3.3) ensures $C_{\mathrm{SB}}\omega_{\Pa}(\pi/M)\le \tfrac12(\czero-\rhok)$, and (A3.2) gives $\|T_P\|_{\op}\le \rhok \le \tfrac12(\czero-\rhok)$, producing the stated bound.
\end{proof}

\begin{remark}[Evenization does not increase $C_0$]
In the T0 normalization we already place symmetric prime weights at $\pm\xi_n$ and integrate the zero counting measure $dN(\gamma)$ over the full real line. The diagonal constant on the zero side is therefore $C_0=\tfrac{1}{2\pi}$, not $\tfrac{1}{\pi}$. Passing to an evenized basis (replacing $\{+\tau,-\tau\}$ by a single cosine packet) redistributes mass within each pair but does not create an additional factor~2: the same symmetry is already built into T0 and into the A3 calibration. Consequently, with $\kappa_{\mathrm{A3}}=2\pi$ the asymptotic PG--LS slope in Road~A is $1-2\pi\Lambda_0\nearrow1^-$ as $\Lambda_0\downarrow0$.
\end{remark}

\begin{remark}[Consequence for the PG--LS slope]
Let the zero-side packet Gram lower bound be normalized as
\(\sum_{\rho}\big|\sum_j c_j\,\widehat g_{\tau_j}(\gamma_\rho)\big|^2\ \ge\ \big(\tfrac{1}{2\pi}-\Lambda_0\big)\,\log(1{+}K)\,\sum_j|c_j|^2\ -\ C_{\mathrm{edge}}\sum_j|c_j|^2\).
Under A3 and T0 the prime-side gain is
\begin{equation}\label{eq:A3_calibration_kappa-formula-1}
 \Gamma(K)\ \ge\ \kappa_{\mathrm{A3}}\,\Big(\tfrac{1}{2\pi}-\Lambda_0\Big)\,\log(1{+}K)\ -\ \kappa_{\mathrm{A3}}\,C_{\mathrm{edge}}\ =\ \big(1-2\pi\Lambda_0\big)\,\log(1{+}K)\ -\ 2\pi C_{\mathrm{edge}},
\end{equation}
so the asymptotic slope approaches $1^{-}$ as $\Lambda_0\to0$. Hence a strict $>1$ cannot be achieved within Road A by only shrinking $\Lambda_0$; one needs an amplifier (e.g. Road B/C) or a different normalization.
\end{remark}

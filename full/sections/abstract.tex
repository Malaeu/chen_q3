\textbf{Background:} The Riemann Hypothesis (RH) is equivalent, by Weil, to the nonnegativity of a quadratic functional $Q$ on an explicit cone of even, compactly supported test functions. Establishing $Q\ge 0$ on the full Weil class requires a precise chain of analytic inputs: normalization, local density, continuity, a Toeplitz--symbol bridge, control of the prime contribution, and a compact-by-compact limit.

\textbf{Main result:} We present a self-contained operator-theoretic proof that verifies this entire chain. Starting from the Guinand--Weil normalization (T0), we construct Fej\'er$\times$heat dictionaries that are dense in each compact window $[-K,K]$ (A1$'$) and obtain Lipschitz control of $Q$ (A2). The Toeplitz bridge (A3) provides a positive symbol margin via Szeg\H{o}--B\"{o}ttcher theory and an explicit modulus of continuity. A purely analytic RKHS contraction yields a uniform bound on the prime operator, completing the mixed estimate on every compact. Finally, the monotone compact-transfer argument (T5) propagates positivity from all $W_K$ to the full Weil class.

\textbf{Conclusion:} Combining these ingredients we prove that $Q(\Phi)\ge0$ for every $\Phi$ in the Weil cone $\mathcal W$ (even, nonnegative tests generated by Fej\'er$\times$heat windows). By Weil's positivity criterion this establishes the Riemann Hypothesis within our normalization.

% Lemma T0 -- Guinand--Weil Normalization Crosswalk for Q

% Macros for unified normalization
\newcommand{\aarch}{a} % Archimedean density a(\xi)
\newcommand{\xilog}{\xi_n} % nodes \xi_n = log n / (2\pi)

\subsection{Fourier normalization adjustments}\label{conv:fourier}
We fix
\begin{equation}\label{eq:T0_Q_normalization-formula-4}
  \widehat{\varphi}(\xi)=\int_{\mathbb R}\varphi(t)\,e^{-2\pi i t \xi}\,dt,
  \qquad
  \varphi(t)=\int_{\mathbb R}\widehat{\varphi}(\xi)\,e^{2\pi i t \xi}\,d\xi,
\end{equation}
and use the Lebesgue measure $d\xi$ on the frequency side. For even test functions, all identities are taken in the cosine form.

\begin{proposition}[T0' --- Guinand--Weil matching]\label{prop:T0-GW}
Under Convention~\ref{conv:fourier}, the repository normalization $Q(\varphi)$ matches the classical Guinand--Weil functional~\cite{Guinand1948,Weil1952} after the change of variables $\eta=2\pi\xi$:
\begin{equation}\label{eq:T0_Q_normalization-q-functional}
  Q(\varphi)\ =\ Q_{\mathrm{GW}}(\varphi)\ \text{ with } \ \eta=2\pi\xi,\ d\eta=2\pi\,d\xi.
\end{equation}
\end{proposition}
\begin{proof}
Make the substitution $\eta=2\pi\xi$ in all frequency integrals (see \cite[Ch.~2]{SteinShakarchi2003}); by evenness the sine parts vanish and the cosine parts coincide. The Jacobian $d\eta=2\pi\,d\xi$ is absorbed by the fixed normalization of $\widehat{\varphi}$.
\end{proof}

\begin{lemma}[T0: Q normalization crosswalk]\label{t0:lem:T0}
Let $\varphi_{\mathrm{GW}}\in C_c(\mathbb R)$ be even and nonnegative on the Guinand--Weil frequency axis $\eta\in\mathbb R$. Define
\begin{equation}\label{eq:T0_Q_normalization-formula}
Q_{\mathrm{GW}}(\varphi_{\mathrm{GW}})
\ :=\ \int_{\mathbb R} \Bigl(\log\pi-\Re\psi\Big(\tfrac14+\tfrac{i\eta}{2}\Big)\Bigr)\,\varphi_{\mathrm{GW}}(\eta)\,d\eta
\ -\ \sum_{n\ge2}\,\frac{\Lambda(n)}{\sqrt n}\,\bigl(\varphi_{\mathrm{GW}}(\log n)+\varphi_{\mathrm{GW}}(-\log n)\bigr).
\end{equation}
On our (repository) frequency axis $\xi:=\eta/(2\pi)$, define the even window $\varphi(\xi):=\varphi_{\mathrm{GW}}(2\pi\xi)$, nodes $\xi_n:=\frac{\log n}{2\pi}$, and the Archimedean densities
\begin{equation}\label{eq:T0_Q_normalization-formula-3}
\boxed{\ \aarch(\xi)\ :=\ \log\pi-\mathrm{Re}\,\psi\Big(\tfrac14+i\pi\xi\Big),\qquad \astar(\xi):=2\pi\,\aarch(\xi)\ }.
\end{equation}
Then the repository's quadratic functional
\begin{equation}\label{eq:T0_Q_normalization-q-functional-2}
Q(\varphi)\ :=\ \int_{\mathbb R} \astar(\xi)\,\varphi(\xi)\,d\xi\ -\ \sum_{n\ge2}\,\wQ{n}\,\varphi(\xilog)
\end{equation}
coincides with $Q_{\mathrm{GW}}$ evaluated at $\varphi_{\mathrm{GW}}$, i.e.
\begin{equation}\label{eq:T0_Q_normalization-q-functional-1}
Q(\varphi)\ =\ Q_{\mathrm{GW}}(\varphi_{\mathrm{GW}}),\qquad \eta=2\pi\xi,\ \ \varphi_{\mathrm{GW}}(\eta)=\varphi(\eta/2\pi).
\end{equation}
In operator or RKHS estimates we use the undoubled weights \(\wRKHS{n}\); the evenization doubling appears only in the $Q$ functional.\par
\end{lemma}
\begin{proof}
Change variables $\eta=2\pi\xi$ in the Archimedean integral: $d\eta=2\pi\,d\xi$ and $\psi(\tfrac14+\tfrac{i\eta}{2})=\psi(\tfrac14+i\pi\xi)$. Hence
\begin{equation}\label{eq:T0_Q_normalization-formula-2}
\int_{\mathbb R}\!\Bigl(\log\pi-\Re\psi\Big(\tfrac14+\tfrac{i\eta}{2}\Big)\Bigr)\,\varphi_{\mathrm{GW}}(\eta)\,d\eta
\ =\ \int_{\mathbb R} 2\pi\,\Bigl(\log\pi-\Re\psi\Big(\tfrac14+i\pi\xi\Big)\Bigr)\,\varphi(\xi)\,d\xi.
\end{equation}
For the prime term, $\varphi_{\mathrm{GW}}(\pm\log n)=\varphi(\pm\xi_n)$ with $\xi_n=\tfrac{\log n}{2\pi}$. Since $\varphi$ is even, $\varphi(\xi_n)+\varphi(-\xi_n)=2\varphi(\xi_n)$. Thus
\begin{equation}\label{eq:T0_Q_normalization-formula-1}
\sum_{n\ge2}\frac{\Lambda(n)}{\sqrt n}\,\bigl(\varphi_{\mathrm{GW}}(\log n)+\varphi_{\mathrm{GW}}(-\log n)\bigr)
\ =\ \sum_{n\ge2}\frac{2\,\Lambda(n)}{\sqrt n}\,\varphi(\xi_n).
\end{equation}
Combining the two identities yields $Q(\varphi)=Q_{\mathrm{GW}}(\varphi_{\mathrm{GW}})$, as claimed; the properties of the digamma function used here follow from \cite[§5.2]{NISTDLMF}.
\end{proof}
\begin{remark}
(i) The choice of doubling the prime weights $w(n)=2\Lambda(n)/\sqrt n$ at positive nodes $\xi_n>0$ is equivalent to placing unit weights at both $\pm\xi_n$; evenness of $\varphi$ makes the two conventions identical. (ii) If one prefers to keep $a(\xi)$ without the Jacobian factor $2\pi$, then the same equality holds with $Q(\varphi)$ written as $\int (2\pi a)\,\varphi\,d\xi-\sum 2\Lambda(n)/\sqrt n\,\varphi(\xi_n)$; Lemma~\ref{t0:lem:T0} records the canonical $\astar$ that directly matches the Guinand--Weil form under $\eta=2\pi\xi$. (iii) The digamma identities used throughout are tabulated in the NIST Digital Library of Mathematical Functions~\cite{NISTDLMF}.
\end{remark}

\begin{lemma}[Invariance under normalisation conventions]\label{lem:T0-normalisation-invariance-full}
Different choices of Fourier-transform normalisations and node indexing yield
equivalent formulations of the Weil positivity criterion.  Specifically:
\begin{enumerate}[label=(\alph*)]
  \item Switching from the unitary normalisation $\widehat\Phi(\xi)=\int \Phi(x)\,e^{-2\pi i x\xi}\,dx$
  to the measure $\widehat\Phi'(\eta)=\int \Phi(x)\,e^{-i\eta x}\,dx$ with $\eta=2\pi\xi$
  induces the density rescaling $a^*(\xi)=2\pi a(\xi)$ and preserves the form of $Q$.
  \item Replacing the node sequence $\xi_n=\log n/(2\pi)$ by $\pm\log n/(2\pi)$
  preserves the symmetry of the sampling operator and the archimedean/prime decomposition.
  \item The quadratic form $Q(\phi)$ defined via the Guinand--Weil convention
  coincides with $Q_{GW}(\phi_{GW})$ when test functions are converted via the measure factor.
\end{enumerate}
In particular, the positivity of $Q$ is independent of these technical choices.
\end{lemma}
\begin{proof}
Each rescaling is a linear change of variable that preserves the spectral gap
and the compact-by-compact structure.  The node-symmetry $\pm\log n/(2\pi)$
is already built into the Guinand--Weil formalism; see \cite{Weil1952}, §16.
The measure conversion $a^*(\xi)=2\pi a(\xi)$ follows from the Jacobian of the
coordinate change $\eta=2\pi\xi$.
\end{proof}

\smallskip
\noindent\textit{Transition.}
With the normalization T0 established, we now verify local density of the Fej\'er$\times$heat cone on each compact in Section~\ref{a1:thm:A1-local-density}.

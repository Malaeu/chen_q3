% Explicit constants for MD_{2,3}
\subsection{Explicit Constants for \texorpdfstring{MD$_{2,3}$}{MD₂,₃}}\label{app:md23-constants}
We collect analytic bounds sufficient to verify the base interval MD$_{2,3}$ without numerics in the main text. Numerical certification (interval arithmetic) may be delegated to the reproducibility appendix.

\subsection{Lower bound for \texorpdfstring{$m_r$}{m\_r}}
Define $a(\xi)=\log\pi-\Re\psi(\tfrac14+i\pi\xi)$. For $r\in(0,1]$ set
\begin{equation}\label{eq:MD_2_3_constants-formula-11}
m_r\ :=\ \inf_{|\xi|\le r} a(\xi)\ =\ \log\pi-\sup_{|\xi|\le r}\Re\psi\Big(\tfrac14+i\pi\xi\Big).
\end{equation}
Using the integral representation (for $\Re z>0$) from \cite[Ch.~2]{Titchmarsh1986}
\begin{equation}\label{eq:MD_2_3_constants-formula-12}
\psi(z)=\log z-\int_0^{\infty} \Big(\frac{1}{t}-\frac{1}{1-e^{-t}}\Big) e^{-zt}\,dt,
\end{equation}
we obtain, after taking real parts at $z=\tfrac14+i\pi\xi$, the bound
\begin{equation}\label{eq:MD_2_3_constants-formula-10}
\Re\psi\Big(\tfrac14+i\pi\xi\Big)\ \le\ \log\sqrt{\tfrac{1}{16}+\pi^2\xi^2}\ +\ C_0,\qquad C_0:=\int_0^{\infty}\Big|\frac{1}{t}-\frac{1}{1-e^{-t}}\Big| e^{-t/4}\,dt.
\end{equation}
Hence
\begin{equation}\label{eq:MD_2_3_constants-formula-9}
m_r\ \ge\ \log\pi-\log\sqrt{\tfrac{1}{16}+\pi^2 r^2}-C_0\ =\ \tfrac12\log\Big(\frac{\pi^2}{\tfrac{1}{16}+\pi^2 r^2}\Big)-C_0.
\end{equation}
This gives an explicit (computable) lower bound $m_r\downarrow 0$ as $r\downarrow0$.

\subsection{Upper bound for \texorpdfstring{$N_{B,r}$}{N\_B,r}}
Let $N_{B,r}=\int_{[-B,B]\setminus[-r,r]} |a(\xi)|\,d\xi$. For $|\xi|\ge r$ and $r\in(0,1]$ we use the asymptotic (\cite[Ch.~2]{Titchmarsh1986})
\begin{equation}\label{eq:MD_2_3_constants-formula-8}
\Re\psi\Big(\tfrac14+i\pi\xi\Big)=\log(\pi|\xi|)+O\Big(\frac{1}{1+|\xi|}\Big),
\end{equation}
whence $|a(\xi)|\le \big|\log\pi-\log(\pi|\xi|)\big|+C_1 \le \log^+\!\frac{1}{|\xi|}+C_1$ for a universal $C_1$. Therefore
\begin{equation}\label{eq:MD_2_3_constants-formula-7}
N_{B,r}\ \le\ \int_{[-B,-r]\cup[r,B]} \big(\log^+\!\tfrac{1}{|\xi|}+C_1\big)\,d\xi\ \le\ 2\Big( r\log\tfrac{1}{r}+r + (B-r)C_1\Big).
\end{equation}
In particular, for fixed $B$ and small $r$ one has $N_{B,r}=O\big(r\log\tfrac{1}{r}\big)$.

\subsection{Core mass via \texorpdfstring{$\rho_t(r)$}{ρₜ(r)} and Fejér area}
For any $|\tau|\le B$ and $r\in(0,B)$,
\begin{equation}\label{eq:MD_2_3_constants-formula-6}
 \int_{-r}^{r}\!\Lambda_B(\xi-\tau)\,\rho_t(\xi-\tau)\,d\xi\ \ge\ \rho_t(r)\int_{\tau-r}^{\tau+r}\!\Lambda_B(x)\,dx\ \ge\ \rho_t(r)\,\frac{r^2}{2B},
\end{equation}
with the last inequality minimizing the Fej\'er area over intervals of length $2r$ in $[-B,B]$. The symmetric term in $\Phi_{B,t,\tau}$ contributes another $\rho_t(r)\,\frac{r^2}{2B}$, hence a total core mass lower bound $\rho_t(r)\,\frac{r^2}{B}$.

\subsection{Sufficient criterion (reprise)}
Combining the bounds gives the sufficient condition for MD$_{2,3}$ on $[B_3,B_4)$:
\begin{equation}\label{eq:MD_2_3_constants-formula-5}
\boxed{\ m_r\,\rho_t(r)\,\frac{r^2}{B}\ -\ 2\,(4\pi t)^{-1/2}\,N_{B,r}\ \ge\ \frac{\log 2}{\sqrt 2}+\frac{\log 3}{\sqrt 3}\, }
\end{equation}
with $m_r, N_{B,r}$ as above and $\rho_t(r)=(4\pi t)^{-1/2} e^{-(2\pi)^2 r^2/t}$. One may additionally fix $t\ge 1/\pi$ to ensure $\Phi_{B,t,\tau}(\xi_n)\le1$ on the prime side.

\subsection{RKHS auxiliary bounds for the operator form}

We record three elementary ingredients used by the RKHS contraction in the MD module.

\begin{lemma}[Effective weight cap]\label{md:lem:weight-cap}
For the even weighting $w(n)=\Lambda(n)/\sqrt{n}$ one has
\begin{equation}\label{eq:MD_2_3_constants-formula-1}
\sup_{x\ge 2}\frac{\log x}{\sqrt{x}}\;=\;\frac{2}{e}\;<\;\frac{3}{4}\;<\;1,\qquad\text{hence}\qquad w_{\max}\le \frac{2}{e}< \frac{3}{4}.
\end{equation}
(Rational bound: $2/e\approx 0.7358\ldots < 3/4 = 0.75$, ensuring all subsequent constraints with $w_{\max}$ use explicit rational inequalities.)
\end{lemma}

\begin{lemma}[Log-node gap on a compact]\label{md:lem:gap-compact}
Let $\alpha_n=\dfrac{\log n}{2\pi}$ and fix $K\ge 1$. Then the minimal active gap on $[-K,K]$ satisfies
\begin{equation}\label{eq:MD_2_3_constants-formula-4}
\delta_K\;:=\;\min\{\alpha_{n+1}-\alpha_n:\ \alpha_n,\alpha_{n+1}\in[-K,K]\}\ \ge\ \frac{1}{4\pi\,e^{2\pi K}}\,.
\end{equation}
\end{lemma}
\begin{proof}
For $n\ge1$, by convexity of $\log$ we have $\log(n+1)-\log n \ge \dfrac{1}{n+1}$. Hence
\begin{equation}\label{eq:MD_2_3_constants-formula-3}
\alpha_{n+1}-\alpha_n\;=\;\frac{\log(n+1)-\log n}{2\pi}\;\ge\;\frac{1}{2\pi(n+1)}\,.
\end{equation}
On $[-K,K]$ one has $n+1\le \lfloor e^{2\pi K}\rfloor+1\le 2 e^{2\pi K}$ for $K\ge1$, so $\alpha_{n+1}-\alpha_n\ge (4\pi e^{2\pi K})^{-1}$. Taking the minimum over active indices yields the claim.
\end{proof}

\begin{proposition}[RKHS contraction parameter]\label{md:prop:rkhs-contraction}
With $S_K(t):=\dfrac{2e^{-\delta_K^2/(4t)}}{1-e^{-\delta_K^2/(4t)}}$ and any $\eta_K\in(0,1)$ define
\begin{equation}\label{eq:MD_2_3_constants-formula}
t_{\min}(K)=\frac{\delta_K^2}{4\ln\!\big(\tfrac{2+\eta_K}{\eta_K}\big)}\,.
\end{equation}
Then $S_K(t_{\min})\le \eta_K$ and the de Branges/RKHS contraction holds:
\begin{equation}\label{eq:MD_2_3_constants-formula-2}
\|T_P\|_{\mathcal H_K}\ \le\ w_{\max}+\sqrt{w_{\max}}\,S_K(t_{\min})\ \le\ w_{\max}+\sqrt{w_{\max}}\,\eta_K\,.
\end{equation}
\end{proposition}

% Theorem B.3 --- One-interval induction (IND/AB step)
\begin{theorem}[B.3: IND/AB]\label{md:thm:INDprime}
On an activity interval $[B_n,B_{n+1})$ let $\|T_P^{\mathrm{old}}\|_{\mathcal H_K}\le \rho_K^{\mathrm{old}}<1$. When crossing the threshold $B_{n+1}$ a single new node $\alpha_{\mathrm{new}}$ with weight $w_{\mathrm{new}}$ enters. In the RKHS normalization $\|k_{\alpha}\|=1$ one has
\begin{equation}\label{eq:IND_prime_step-formula-1}
\|T_P^{\mathrm{new}}\|\le \rho_K^{\mathrm{old}}+ w_{\mathrm{new}}.
\end{equation}
Hence if $\rho_K^{\mathrm{old}}+w_{\mathrm{new}}<1$, then $T_A-T_P^{\mathrm{new}}\succeq 0$ on $\mathcal H_K$.
\end{theorem}
\begin{proof}
Rank--one update: $T_P^{\mathrm{new}}=T_P^{\mathrm{old}}+ w_{\mathrm{new}}\,|k_{\alpha_{\mathrm{new}}}\rangle\langle k_{\alpha_{\mathrm{new}}}|$ with $\|k_{\alpha}\|=1$ gives the claimed norm bound; strict inequality implies the Loewner positivity.
\end{proof}

%% Certified parameters table: see Appendix~\ref{app:a3-repro}, Table~\ref{tab:ind-schedule}.
%% The analytic bounds in Theorem~\ref{md:thm:INDprime} and Lemma~\ref{md:lem:IND-early-analytic}
%% do not depend on numerical certificates.

% Corollary --- Gluing across intervals
\begin{corollary}[Gluing intervals]\label{md:cor:IND-glue}
Suppose MD$_{2,3}$ holds on $[B_3,B_4)$, and across each threshold $B_{n}\to B_{n+1}$ the one--prime condition $\rho_K^{\mathrm{old}}+w_{\mathrm{new}}<1$ is verified in the RKHS normalization on $[-K,K]$. Then $T_A-T_P\succeq 0$ holds on $[-K,K]$ for all $B\ge B_3$, i.e. the measure domination persists interval--by--interval.
\end{corollary}

\begin{lemma}[Analytic bound for early blocks]\label{md:lem:IND-early-analytic}
Let $\Phi_{B,t}(\xi)=(1-|\xi|/B)_+e^{-4\pi^2 t\xi^2}$ with $B>0$. Then for the even setting with weights $w(n)=\Lambda(n)/\sqrt n$ and nodes $\alpha_n=\log n/(2\pi)$ one has the deterministic bound
\begin{equation}\label{eq:IND_prime_step-formula}
 \sum_{\alpha_n\in[-B,B]} w(n)\,\Phi_{B,t}(\alpha_n)\ \le\ \sum_{n\le e^{2\pi B}} \frac{\Lambda(n)}{\sqrt n}\ \le\ \int_1^{e^{2\pi B}} \frac{\log u}{\sqrt u}\,du\ =\ 2\,e^{\pi B}\,(2\pi B-2)\ +\ 4.
\end{equation}
In particular, choosing $B=B(K)>0$ small enough forces the early--block mass to lie below any prescribed budget $\varepsilon(K)>0$.
\end{lemma}
\begin{proof}
Since $0\le \Phi_{B,t}\le1$ and $\Phi_{B,t}$ vanishes outside $[-B,B]$, the first inequality holds. For the second, use $\Lambda(n)\le\log n$ and compare the sum to the integral; the evaluation follows by the substitution $u=v^2$.
\end{proof}

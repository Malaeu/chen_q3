\section{RKHS contraction of the prime operator}\label{sec:rkhs-bridge}

Let \(T_P\) denote the prime sampling operator acting on functions in
\(\mathcal W_K\) by
\[
  (T_P \Phi)(x)
  = \sum_{n\ge 1} w(n)\,\Phi\!\left(x - \frac{\log n}{2\pi}\right),
\qquad
  w(n)=\frac{2\Lambda(n)}{\sqrt n}.
\]
Following Section~RKHS of the full manuscript we place the translates of
the heat kernel inside the reproducing kernel Hilbert space
\(\mathcal H_t\) generated by the kernel
\[
  K_t(x,y) = \frac{1}{\sqrt{4\pi t}}\exp\!\left(-\frac{(x-y)^2}{4t}\right).
\]
Two complementary mechanisms control \(\|T_P\|\) on \(W_K\):

\begin{enumerate}[label=(\roman*), leftmargin=*]
  \item The Gram geometry route yields
  \[
    \|T_P\|
    \le w_{\max} + \sqrt{w_{\max}}\,S_K(t),
    \qquad
    S_K(t) \le \frac{2e^{-\delta_K^2/(4t)}}{1-e^{-\delta_K^2/(4t)}},
  \]
  where \(w_{\max} \le 2/e\) and \(\delta_K\) is the minimum spacing of the
  sampling nodes on \(W_K\).
  \item The early/tail split bounds the finite sum
  \(\sum_{n\le N} \Lambda(n)n^{-1/2}\) by \(2\sqrt N\log N\), while the tail
  is dominated by an explicit Gaussian integral depending on \(t\).
\end{enumerate}

Optimising \(t\) along the schedule
\[
  t_{\min}(K) = \frac{\delta_K^2}{4\log\!\bigl((2+\eta_K)/\eta_K\bigr)},
\qquad
\eta_K \in (0,1-w_{\max}),
\]
produces the cap \(\rho_K = w_{\max} + \sqrt{w_{\max}}\eta_K\).

\begin{remark}[Stability under node-spacing decay]\label{rem:rkhs-delta-scaling}
The key insight: choosing \(t_{\min}(K) = \delta_K^2/(4\log(...))\)
\emph{fixes} the ratio
\[
  q := e^{-\delta_K^2/(4t_{\min})}
\]
independently of \(K\).  Therefore \(S_K(t_{\min}) = 2q/(1-q)\) remains
bounded even as \(\delta_K \to 0\).  For instance, when \(K=1\) numerical
computation gives \(q \approx 1/9\), hence \(S_1 \approx 1/4\).  This scaling
ensures that the RKHS cap \(\rho_K\) does not degenerate with increasing \(K\).
\end{remark}

\begin{proposition}[RKHS contraction]\label{prop:rkhs-cap}
Given \(K>0\) and \(t\ge t_{\min}(K)\), the operator \(T_P\) acting on
\(\mathcal W_K\) satisfies \(\|T_P\|\le \rho_K\).  Choosing the parameter
schedule so that \(t_{\min}(K)\) is monotone in \(K\) ensures that
\(\rho_K \le c_0(K)/4\).
\end{proposition}
\begin{proof}
The two mechanisms (Gram geometry and early/tail split) are detailed in Section~RKHS of the full manuscript; optimising \(t\) along the monotone schedule yields the stated bound.
\end{proof}

\begin{lemma}[Uniform RKHS cap]\label{lem:rkhs-uniform-cap}
The RKHS contraction \(\rho(t)=2\int_0^\infty y\,e^{y/2}\,e^{-4\pi^2 ty^2}\,dy\)
is strictly decreasing in \(t\).  Fixing the rational scale \(t_0=\tfrac{7}{10}\) and
using Lemma~\ref{pm:lem:rho-closed-form} together with the inequalities
\(\pi\le\tfrac{22}{7}\) and \(e^{1/4}\le\tfrac{33}{25}\) yields the passport
\(\rho(t_0)\le\tfrac{16\,170}{671\,075}<\tfrac{1}{25}\).
Consequently
\[
  \frac{c_0(K)}{4} - \rho(t_0)
  \ge \frac{\CarchOne}{4} - \frac{1}{25}
  = \YesSlackMin > 0
\]
for every \(K\), independent of the node spacing \(\delta_K\).
\end{lemma}
\begin{proof}
Monotonicity of \(\rho(t)\) follows from the Gaussian decay.  The stated passport
comes directly from the closed-form evaluation and the rational bounds recorded above,
while Lemma~\ref{lem:arch-floor} supplies the analytic floor \(c_0(1)=\CarchOne\).
\end{proof}

\begin{remark}[Why uniform cap beats local bisection]\label{rem:uniform-vs-local}
A local approach would choose \(t^*(K)\) via bisection to satisfy
\(\rho(t^*(K))\le c_0(K)/4\), yielding near-zero slack by construction.
The uniform route keeps the fixed \(t_0=\tfrac{7}{10}\), so the explicit slack
\(\YesSlackMin\) is inherited for every compact.  This avoids dependence on
\(K\)-specific schedules and simplifies the YES-gate verification.
\end{remark}

Again, all constants are tabulated explicitly in the full project.  The key
feature for the present note is the inequality \(\rho_K \le c_0(K)/4\)
which complements \Cref{thm:A3-gap}.

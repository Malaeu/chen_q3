\section{Digamma Bounds: Analytic Framework and Optional Numerics}\label{app:digamma-computation}

This appendix records the analytic framework behind the bounds for the uniform Archimedean floor, and separates
any numerical checks as non-normative evidence. The main proof relies only on the analytic lemmas in
Section~\ref{subsec:a3-symbol-floor}.

\subsection{Definitions}

The Archimedean density is defined as
\[
  a(\xi) := \log\pi - \Re\psi\Bigl(\frac{1}{4} + i\pi\xi\Bigr),
\]
where $\psi(z)=\Gamma'(z)/\Gamma(z)$ is the digamma function.

For $B>0$ and $t_{\mathrm{sym}}>0$, define
\begin{align}
  A_0(B,t_{\mathrm{sym}}) &:= 2\pi\int_{-B}^{B} a(\xi)\Bigl(1-\frac{|\xi|}{B}\Bigr) e^{-4\pi^2 t_{\mathrm{sym}}\xi^2}\,d\xi, \\
  g_{B,t_{\mathrm{sym}}}(\xi) &:= a(\xi)\Bigl(1-\frac{|\xi|}{B}\Bigr) e^{-4\pi^2 t_{\mathrm{sym}}\xi^2}, \\
  L_A(B,t_{\mathrm{sym}}) &:= 2\pi\sup_{\theta\in[-\tfrac12,\tfrac12]}\sum_{m\in\ZZ}\bigl|g_{B,t_{\mathrm{sym}}}'(\theta+m)\bigr|.
\end{align}

With $B_{\min}=3$ and $t_{\mathrm{sym}}=3/50$, the uniform Lipschitz constant is
\[
  L_*(t_{\mathrm{sym}}):=\sup_{B\ge B_{\min}}L_A(B,t_{\mathrm{sym}})
\]
as in Definition~\ref{def:uniform-L}.  The uniform Archimedean floor $c_*>0$ is supplied
directly by Lemma~\ref{lem:uniform-arch-floor}.

\subsection{Digamma Properties}

The digamma function at $z=1/4$ satisfies the reflection formula \cite[Ch.~2]{Titchmarsh1986}:
\[
  \psi\Bigl(\frac{1}{4}\Bigr) = -\gamma - \frac{\pi}{2} - 3\ln 2,
\]
where $\gamma$ is Euler's constant. This gives:
\[
  a(0) = \log\pi + \gamma + \frac{\pi}{2} + 3\ln 2.
\]

For $\xi\ne 0$, the real part of $\psi(1/4+i\pi\xi)$ can be computed via the series \cite[Ch.~2]{Titchmarsh1986}:
\[
  \Re\psi\Bigl(\frac{1}{4}+i\pi\xi\Bigr) = -\gamma + \sum_{n=0}^{\infty}\Bigl(\frac{1}{n+1} - \frac{n+1/4}{(n+1/4)^2+\pi^2\xi^2}\Bigr).
\]

\subsection{Analytic bound framework}

Section~\ref{subsec:a3-symbol-floor} introduces analytic bounds for $L_*(t_{\mathrm{sym}})$
and records the direct pointwise floor $c_*$ (Lemma~\ref{lem:uniform-arch-floor}). The
mean--modulus estimates for $A_0$ and $L_A$ remain as auxiliary bounds but are not used
to define $c_*$ in the main proof chain.

\subsection{Finite-sum bounds at sample points}\label{app:digamma-sample}
For $y=\pi\xi$ set
\[
  t_n(y):=\frac{1}{n+1}-\frac{n+\tfrac14}{(n+\tfrac14)^2+y^2},
\qquad
  \Re\psi\Bigl(\frac14+i y\Bigr)
  = -\gamma + \sum_{n\ge0} t_n(y).
\]
The summand satisfies $t_n(y)\le0$ once
$n\ge \lceil \tfrac43 y^2-\tfrac14\rceil$, so for $\xi=\tfrac12$ we have
$t_n(\pi/2)\le0$ for $n\ge4$ and therefore
\[
  \Re\psi\Bigl(\tfrac14+\tfrac{i\pi}{2}\Bigr)\ \le\ -\gamma + \sum_{n=0}^3 t_n\Bigl(\tfrac{\pi}{2}\Bigr).
\]
Using $\pi^2<10$ yields the explicit bounds
\[
  t_0\le \frac{37}{41},\qquad
  t_1\le \frac{5}{26},\qquad
  t_2\le \frac{13}{363},\qquad
  t_3\le \frac{1}{836},
\]
hence $\sum_{n=0}^3 t_n(\pi/2)<1.132$. With $\pi>333/106$ and $\gamma>0.5772$ this gives
$a(\tfrac12)=\log\pi+\gamma-\sum t_n(\pi/2)>\frac{29}{50}$, as recorded in
Lemma~\ref{lem:a3-a-sample}.

For $\xi\ge1$ the standard digamma remainder bound
\(
|\psi(z)-\log z+\tfrac{1}{2z}|\le \frac{1}{12|z|^2}
\)
(\cite[\S5.11]{NISTDLMF}) implies
\[
  a(\xi)\ \ge\ -\log\xi\ -\ \frac{1}{2\pi\xi}\ -\ \frac{1}{12\pi^2\xi^2}.
\]
Plugging $\xi=\tfrac32$ and $\xi=\tfrac52$ and using $\pi>3$ together with
the elementary bounds $\log(3/2)<\tfrac{5}{12}$ and $\log(5/2)<1$ gives
\[
  a(\tfrac32)\ \ge\ -\frac{5}{12}-\frac{1}{9}-\frac{1}{243}\ >\ -\frac35,
\]
and
\[
  a(\tfrac52)\ \ge\ -1-\frac{1}{15}-\frac{1}{675}\ >\ -\frac{11}{10},
\]
matching Lemma~\ref{lem:a3-a-sample}.

\subsection{Optional numerical evidence (not used in the proof)}

Exploratory numerical checks (high-precision quadrature and grid evaluation of the periodized derivative sum)
are implemented in \path{scripts/digamma_bounds.py}. These computations are \emph{not} part of the logical proof;
they merely provide sanity checks on the scale of $A_0(B,t_{\mathrm{sym}})$ and $L_A(B,t_{\mathrm{sym}})$.

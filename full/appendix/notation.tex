We collect the notation used throughout.

\paragraph{Sets and measures.}
$A\cap B$, $A\cup B$, $A\setminus B$ are standard. $\mathbf{1}_E$ denotes the indicator of a set $E$.
The symbol $\lvert E\rvert$ records measure/length in the relevant context.

\paragraph{Norms.}
$\|x\|_2$ is the Euclidean norm, $\|f\|_{L^2(\Omega)}^2=\int_\Omega |f|^2$.
For sequences $\|a\|_{\ell^2}^2=\sum_k |a_k|^2$.

\paragraph{Operators.}
$\langle u,v\rangle$ is the inner product, $A^\ast$ the adjoint,
$\operatorname{tr}(M)$ the trace, $\|T\|_{\mathrm{op}}$ the operator norm.

\paragraph{Comparisons.}
$r\lesssim s$ means $r\le C s$ with an absolute constant $C$ independent of the current parameters; $r\simeq s$ abbreviates $r\lesssim s$ and $s\lesssim r$ simultaneously.

\paragraph{Critical constants.}
$c_* = \tfrac{11}{10}$ is the \textbf{uniform} archimedean floor (Lemma~\ref{lem:uniform-arch-floor}), valid for all $K\ge 1$ on the full circle $\TT$ (parameters $t_{\mathrm{sym}}=3/50$, $B_{\min}=3$).
\textbf{Note:} The legacy value $\CarchOne$ from Theorem~8.16 is obsolete (contains a scale error); the main proof uses $c_*$ exclusively.
The uniform RKHS prime cap is $\rho(t_{\mathrm{rkhs}})\le c_*/4$ for $t_{\mathrm{rkhs}}\ge t_{\star,\mathrm{rkhs}}^{\mathrm{unif}}$ (Corollary~\ref{cor:uniform-prime-cap}).

\paragraph{Uniform constants (mainline).}
\begin{center}
\begin{tabular}{lll}
\toprule
Constant & Value & Source \\
\midrule
$t_{\mathrm{sym}}$ & $3/50$ & Lemma~\ref{lem:uniform-arch-floor} \\
$B_{\min}$ & $3$ & Lemma~\ref{lem:uniform-arch-floor} \\
$c_*$ & $\tfrac{11}{10}$ & Lemma~\ref{lem:uniform-arch-floor} \\
$M_0^{\mathrm{unif}}$ & $\left\lceil C_{\mathrm{SB}}\,L_*(t_{\mathrm{sym}})/c_*\right\rceil$ & Corollary~\ref{cor:uniform-discretisation} \\
$t_{\star,\mathrm{rkhs}}^{\mathrm{unif}}$ & $1$ & Corollary~\ref{cor:uniform-prime-cap} \\
\bottomrule
\end{tabular}
\end{center}

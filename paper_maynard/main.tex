% Sieve-Spectral Synergy Paper
% ============================

% Preamble for the Toeplitz--Weil Bridge paper
\usepackage[utf8]{inputenc}
\usepackage[T1]{fontenc}
\usepackage{lmodern}
\usepackage{textcomp}     % Fix for Unicode symbols (en-dash, quotes, etc)
\usepackage{microtype}    % Better typography and line breaking
\usepackage[final]{pdfpages} % Better PDF handling
\usepackage{scalerel}     % Better symbol scaling
\usepackage[english]{babel}
\usepackage{amsmath,amsthm,amssymb,mathtools}
\usepackage{enumitem}      % For customizing description/itemize/enumerate
\usepackage{geometry}
\usepackage{array}
\usepackage{hyperref}
\usepackage{bookmark}
\usepackage[nameinlink,capitalize,noabbrev]{cleveref}
\usepackage{url}
\usepackage[hyphenbreaks]{breakurl}
\def\UrlBreaks{\do\/\do-\do.}
\newcommand{\CarchOne}{\frac{1\,346\,209}{7\,168\,000}} % c_arch(1)
\newcommand{\RhoGate}{\frac{1}{25}}                    % uniform gate cap
\newcommand{\RhoSeventytight}{\frac{1\,971}{50\,000}}% strict rho(0.7) bound
\newcommand{\YesSlackMin}{\frac{199\,329}{28\,672\,000}}% c_arch(1)/4 - 1/25
% Note: c_arch(1)/4 - \RhoSeventytight = 5\,398\,969 / 716\,800\,000 > \YesSlackMin

\newif\ifrkhroute
\rkhroutetrue
% --- Cross-ref helpers (safe no-ops if already defined)
\providecommand{\xlink}[2]{\hyperref[#1]{#2}}
\newcommand{\SeeAlso}[1]{\par\smallskip\noindent\textit{See also.} #1\par\smallskip}
\hypersetup{
  pdftitle={Operator Methods for the Weil Criterion},
  pdfauthor={Eugen Malamutmann}
}

\pdfstringdefDisableCommands{%
  \def\czero{c\_0}%%
  \def\rhok{\rho\_K}%%
  \def\AssumpPack{(T0)+(A1'$)$+(A2)+(A3)+(RKHS/MD)+(T5)}%%
}
\usepackage{graphicx}
\usepackage{xcolor}
\usepackage{booktabs}    % Professional table rules (toprule, midrule, bottomrule)
\usepackage{float}       % Force table/figure placement with [H]
\usepackage{multirow}    % Multi-row cells in tables
\usepackage{tcolorbox}
\newcommand{\figinclude}[1]{\IfFileExists{#1}{\includegraphics[width=0.95\linewidth]{#1}}{\fbox{Missing figure: \texttt{\detokenize{#1}}}}}
\providecommand{\MetricsAutoTable}{}
\geometry{margin=1in}

% Fix for overfull hbox issues
\sloppy
\tolerance=2000
\emergencystretch=5em
\hbadness=10000  % Suppress underfull hbox warnings
\vfuzz=2pt       % Allow small overfull vboxes
\hfuzz=2pt       % Allow small overfull hboxes

% Theorem environments
\numberwithin{equation}{section}
\newtheorem{theorem}{Theorem}[section]
\newtheorem{lemma}[theorem]{Lemma}
\newtheorem{proposition}[theorem]{Proposition}
\newtheorem{corollary}[theorem]{Corollary}
\theoremstyle{definition}
\newtheorem{definition}[theorem]{Definition}
\theoremstyle{remark}
\newtheorem*{remark}{Remark}

% Shortcuts
\newcommand{\RR}{\mathbb R}
\newcommand{\ZZ}{\mathbb Z}
\newcommand{\NN}{\mathbb N}
\newcommand{\CC}{\mathbb C}
\newcommand{\TT}{\mathbb T}
\newcommand{\EE}{\mathbb E}
\newcommand{\vp}{\varphi}
\newcommand{\xiN}{\xi_n}
\newcommand{\La}{\Lambda}
\newcommand{\veps}{\varepsilon}
\newcommand{\astar}{a_*}
\newcommand{\Lip}{\mathrm{Lip}}
\newcommand{\BV}{\mathrm{BV}}
\newcommand{\TV}{\operatorname{TV}}
\newcommand{\norm}[1]{\left\lVert #1\right\rVert}
\newcommand{\ip}[2]{\left\langle #1,\,#2\right\rangle}
\newcommand{\op}{\mathrm{op}}
\newcommand{\om}{\omega}
\newcommand{\Pa}{P_A}
\newcommand{\TP}{T_{\!P}}
\newcommand{\TPA}{T_{\!P_A}}
\newcommand{\czero}{c_0}
\newcommand{\rhok}{\rho_K}
\newcommand{\lammin}{\lambda_{\min}}
\newcommand{\cO}{\mathcal{O}}
\newcommand{\Kern}{K}
\newcommand{\Kt}{K_t}
\newcommand{\epsK}{\varepsilon_{K}}
\newcommand{\DispK}[1]{\operatorname{Disp}_K\!(#1)}
\newcommand{\ACDthree}{\textsf{AC--D3}}
\newcommand{\wQ}[1]{w_{\mathrm{Q}}(#1)}
\newcommand{\wRKHS}[1]{w_{\mathrm{RKHS}}(#1)}
\newcommand{\wmax}{w_{\max}^{\mathrm{RKHS}}}

% --- Conditional RH phrasing macros ---
% Assumption chain used across the paper
\newcommand{\AssumpChain}{(T0)+(A1$'$)+(A2)+(A3)+(MD/IND or RKHS)+(T5)} % chktex 36
% Unified shorthand for the full hypothesis pack
\newcommand{\AssumpPack}{\textup{(T0)+(A1$'$)+(A2)+(A3)+(RKHS/MD)+(T5)}}
% Short, context-free Weil implication phrase (no re-listing assumptions)
\newcommand{\WeilImpRH}{by Weil's positivity criterion, RH would hold.}
% Variant when explicitly tying to the assumption chain
\newcommand{\WeilImpRHUnder}{by Weil's positivity criterion, RH would hold under \AssumpChain.}
% Title/metadata helpers
\title{Spectral Analysis of the Archimedean Symbol\\
in the Chen Q3 Framework}

\author{Research Notes}

\date{\today}

\begin{abstract}
We provide rigorous numerical verification and analytical estimates
for the key spectral bounds in the Chen Q3 approach to the Riemann Hypothesis.
Specifically, we establish:
\begin{enumerate}[nosep]
    \item Saturation of the symbol norm: $\|P_A\|_\infty \leq C_* \approx \Cstar$.
    \item Positivity of the Archimedean floor: $c_{\text{arch}} \approx \carch > 0$.
    \item Comparison of kernel choices: Lorentzian ($\delta_* \approx \deltaLorentz$)
          vs Mellin ($\delta_* \approx \deltaMellin$).
\end{enumerate}
The Mellin kernel $K(\xi) = 1/(1+|\xi|^{1/2})$ achieves optimal stability ratio.
\end{abstract}

\usepackage{tikz}


\title{Sieve--Spectral Synergy:\\
       A Bridge Between Maynard Weights and Q3 Spectral Positivity}

\author{Eugen Malamutmann, MD\thanks{ORCID: 0000-0003-4624-5890} \\
\small University of Duisburg--Essen}

\date{\today}

\begin{document}

\maketitle

\begin{abstract}
We construct polynomial sieve functions $F_\alpha(t) = (1 - \sum_i t_i)^\alpha$
and derive the closed-form variational functional
$M(F_\alpha, k) = 2k(2\alpha+1)/[(\alpha+1)(k+2\alpha+1)]$.
We prove that $M(F_\alpha, k) > m$ iff $k > k^*(\alpha, m)$, where
$k^*(\alpha, m) = m(\alpha+1)(2\alpha+1)/[2(2\alpha+1) - m(\alpha+1)]$.
For $m = 2$, the optimal exponent $\alpha^* = 1/\sqrt{2}$ minimizes
$k^*$ to $2\sqrt{2}+3 \approx 5.83$, so $k = 6$ suffices.
For $\alpha = 1/2$ (Q3-motivated), $k^* = 6$, giving $M(F_{1/2}, 7) = 56/27 > 2$.
Since $M > 2$ implies bounded prime gaps (Maynard--Tao),
we establish $\liminf(p_{n+1} - p_n) < \infty$ by purely analytic methods.
\end{abstract}

\medskip
\noindent\textit{2020 Mathematics Subject Classification:}
11N35 (Sieves), 11M26 (Nonreal zeros of zeta and $L$-functions).

\smallskip
\noindent\textit{Keywords:}
prime gaps, Maynard--Tao sieve, spectral positivity, Q3 criterion,
bounded gaps between primes.

\section{Introduction}\label{sec:intro}

\subsection*{Background and motivation}
We prove that a canonical quadratic form on the Weil test class is nonnegative, and therefore\textemdash{}by the Weil criterion\textemdash{}deduce the Riemann Hypothesis. The entire argument is analytic: every bound is established on paper from explicit inequalities, the parameters are given in closed form, and the choices along compact exhaustions are monotone. No numerical tables or automated certificates enter the proof.

\subsection*{Main result}
\begin{theorem}[Main result, informal]\label{thm:intro-main-informal}
Let $Q$ be the quadratic form fixed in Section~\ref{sec:T0} on the Weil class $\mathcal{W}$. Then
\[
  Q(\Phi)\ \ge\ 0\qquad\text{for all }\Phi\in\mathcal{W}.
\]
Via Theorem~\ref{thm:Weil-criterion} (the Weil criterion) this positivity is equivalent to the Riemann Hypothesis.
\end{theorem}

The proof organises around three analytic modules.

\subsection*{Archimedean bridge}
\emph{(A3) Archimedean Toeplitz barrier.} On each compact window $W_K=[-K,K]\subset\RR$ we bound from below the Toeplitz component $T_M[P_A]$ of $Q$ by an \emph{archimedean barrier} $c_0(K)>0$, up to a controllable Lipschitz loss $C\,\omega_{P_A}(\pi/M)$. Szeg\H{o}--B\"ottcher asymptotics together with an explicit modulus of continuity for $P_A$ yield
\[
  \lambda_{\min}\!\big(T_M[P_A]\big)\ \ge\ c_0(K)\ -\ C\,\omega_{P_A}\!\Big(\frac{\pi}{M}\Big),
\]
as developed in Section~\ref{sec:A3}.

\subsection*{Prime contraction}
\emph{(RKHS) Prime contraction without tables.} The prime contribution is encoded by a sampling operator $T_P$ supported on the nodes $\xi_n=\frac{\log n}{2\pi}$; in the Weil functional we use the one-sided weights $w_Q(n)=2\Lambda(n)/\sqrt{n}$, while in the RKHS analysis we keep the undoubled operator weights $w_{\mathrm{RKHS}}(n)=\Lambda(n)/\sqrt{n}$. Section~\ref{sec:rkhs-prime-cap} develops a \emph{tables-free} upper bound on $\|T_P\|$ inside the reproducing-kernel Hilbert space of the heat flow. Two complementary routes are provided:
\begin{itemize}
  \item \textbf{Classical treatments.} Standard expositions of the analytic theory~\cite{IwaniecKowalski2004,MontgomeryVaughan2007,Edwards1974} provide the backdrop against which we calibrate notation, normalizations, and cone generators.
  \item a \emph{Gram-geometry route}, giving
  \[
    \|T_P\|\ \le\ w_{\max}+\sqrt{w_{\max}}\,S_K(t),\qquad
    S_K(t)\ \le\ \frac{2e^{-\delta_K^2/(4t)}}{1-e^{-\delta_K^2/(4t)}},
  \]
  where $w_{\max}\le 2/e$ and $\delta_K$ is the separation of the nodes on $W_K$; choosing
  \[
    t_{\min}(K)\ :=\ \frac{\delta_K^2}{4\ln\!\big((2+\eta_K)/\eta_K\big)},\qquad \eta_K\in(0,1-w_{\max}),
  \]
  forces $\|T_P\|\le \rho_K:=w_{\max}+\sqrt{w_{\max}}\;\eta_K$;
  \item an \emph{early/tail route}, splitting the prime sum at $N=N(K)$, with
  \[
    \sum_{n\le N}\frac{\Lambda(n)}{\sqrt{n}}\ \le\ 2\sqrt{N}\log N,
    \qquad
    \sum_{n>N}\frac{\Lambda(n)}{\sqrt{n}}\,e^{-4\pi^2 t(\log n)^2}\ \ll\ \frac{e^{-4\pi^2 t(\log N)^2}}{t},
  \]
  which produces an explicit threshold $t^\star(K)$ ensuring $\|T_P\|\le c_0(K)/4$.
\end{itemize}

\subsection*{Compact transfer}
\emph{(T5) Compact-by-compact transfer.} Section~\ref{sec:T5} shows that once, on a given $W_K$, the deterministic inequalities
\[
  C\,\omega_{P_A}\!\Big(\frac{\pi}{M}\Big)\ \le\ \frac{c_0(K)}{4},\qquad
  \|T_P\|\ \le\ \frac{c_0(K)}{4},\qquad
  \text{(finite early block)}\ \le\ \frac{c_0(K)}{4}
\]
hold with parameters $(M,t)$ chosen \emph{monotonically} in $K$, then $\lambda_{\min}\!\big(T_M[P_A]-T_P\big)>0$ on $W_K$, and positivity inherits to $W_{K'}$ for all $K'\ge K$. Thus $Q\ge0$ on any exhaustion $\bigcup_i W_{K_i}$ with $K_i\uparrow\infty$.

\subsection*{Outline of the proof}

Combining the Toeplitz barrier and the RKHS cap yields, on each $W_K$,
\[
  \lambda_{\min}\!\big(T_M[P_A]-T_P\big)
  \ \ge\ c_0(K)\ -\ C\,\omega_{P_A}\!\Big(\frac{\pi}{M}\Big)\ -\ \|T_P\|.
\]
Choosing $t\ge t_{\min}(K)$ (or $t\ge t^\star(K)$) enforces $\|T_P\|\le c_0(K)/4$, and selecting $M$ so that $C\,\omega_{P_A}(\pi/M)\le c_0(K)/4$ gives
\[
  \lambda_{\min}\!\big(T_M[P_A]-T_P\big)\ \ge\ \tfrac12\,c_0(K)\ >\ 0.
\]
The compact-by-compact transfer then propagates positivity along any monotone chain $K_i\uparrow\infty$. Positivity on $\bigcup_i W_{K_i}$ extends by definition to all of $\mathcal{W}$, proving $Q\ge0$ in Theorem~\ref{thm:Main-positivity}. Finally Section~\ref{sec:Weil} applies Theorem~\ref{thm:Weil-criterion} to convert this positivity into the Riemann Hypothesis.

\subsection*{What is new}

Two features distinguish the present work.
\begin{enumerate}
  \item \textbf{A tables-free prime contraction.} The norm of the prime operator is bounded analytically in an RKHS, via either Gram geometry or an early/tail split. All constants are explicit (for example $t_{\min}(K)$ above), monotone in $K$, and no legacy tables or certificates appear in the proof; reproducibility data are confined to Appendix~\ref{app:a3-repro}.
  \item \textbf{A monotone transfer principle.} The compact-by-compact module (T5) depends only on $c_0(K)$, $\omega_{P_A}$, and the RKHS cap $\rho_{\mathrm{cap}}(K)$. The parameter schedules $(M^\star(K),t^\star(K))$ are given by explicit formulas and chosen to be monotone in $K$, yielding an auditable, dimension-free route from positivity on one compact to positivity on all larger compacts.
\end{enumerate}

\subsection*{Organization of the paper}

Section~\ref{sec:T0} recalls the Weil class, the quadratic form $Q$, and the Guinand--Weil normalization. Section~\ref{sec:A3} establishes the Archimedean Toeplitz barrier (A3). Section~\ref{sec:rkhs-prime-cap} develops the RKHS prime contraction together with the thresholds $t_{\min}(K)$ and $t^\star(K)$. Section~\ref{sec:T5} proves the compact-by-compact transfer (T5) and the monotone inheritance. Section~\ref{sec:Weil} links compact positivity to the full Weil class and states the main theorem together with its Weil corollary. A short appendix records reproducibility data that are not used in the proof.

\subsection*{Notation}

We write $\Lambda$ for the von Mangoldt function, $\xi_n=\frac{\log n}{2\pi}$ for the sampling nodes, $w_Q(n)=2\Lambda(n)/\sqrt{n}$ for the weights inside the Weil functional, and $w_{\mathrm{RKHS}}(n)=\Lambda(n)/\sqrt{n}$ (with $w_{\max}=\sup_n w_{\mathrm{RKHS}}(n)\le 2/e$) for the operator analysis. The heat kernel is $k_t(x,y)=\exp\!\bigl(-\frac{(x-y)^2}{4t}\bigr)$. Compact windows are denoted $W_K=[-K,K]$, and $\mathcal{W}=\bigcup_{K>0}\mathcal{W}_K$ is the Weil cone. Complete conventions appear in Section~\ref{sec:notation}.

\subsection*{Analytic modules at a glance}

\noindent\textbf{Stage legend.} $(\mathrm{T0})$ fixes the Guinand--Weil normalization of the Weil functional. $(\mathrm{A1'})$ proves density of the Fej\'er$\times$heat generator cone on each compact, and $(\mathrm{A2})$ supplies Lipschitz continuity so that positivity propagates from the generators to all even nonnegative tests. $(\mathrm{A3})$ is the Toeplitz bridge: it splits $Q$ into an Archimedean Toeplitz symbol and a finite-rank prime block with explicit lower bounds on $\lambda_{\min}$. The main route for the prime contribution is the RKHS contraction developed in Section~\ref{sec:rkhs-prime-cap}; the MD/IND/AB chain remains archived as an alternative in the appendices. Finally $(\mathrm{T5})$ performs the compact-by-compact lift and closes the YES gate, chaining the local statements to $Q\ge0$ on the full Weil class.

\begin{center}
  \small\textbf{Dependency map for the analytic chain}
\end{center}
\begin{center}
  \small
  \begin{tabular}{lll}
    \hline
    Module & Key statement & Consumed by \\
    \hline
    $\mathrm{T0}$ & Proposition~\ref{prop:T0-GW} (Guinand--Weil normalization) & Theorem~\ref{thm:Main-positivity}, Theorem~\ref{thm:RH} \\
    $\mathrm{A1'}$ & Theorem~\ref{a1:thm:A1-local-density} (Density on $W_K$) & Theorem~\ref{thm:T5-compact}, Theorem~\ref{thm:Main-positivity} \\
    $\mathrm{A2}$ & Lemma~\ref{a2:lem:A2} / Corollary~\ref{a2:cor:explicit-lip} (Lipschitz control) & Theorem~\ref{thm:T5-compact}, Theorem~\ref{thm:Main-positivity} \\
    $\mathrm{A3}$ & Theorem~\ref{thm:A3} (Toeplitz bridge) & Theorem~\ref{thm:T5-compact}, Theorem~\ref{thm:Main-positivity} \\
    RKHS & Theorem~\ref{thm:rkhs-tstar} (Prime contraction) & Theorem~\ref{thm:T5-compact}, Theorem~\ref{thm:Main-positivity} \\
    $\mathrm{T5}$ & Theorem~\ref{thm:T5-compact} (Compact transfer) & Theorem~\ref{thm:Main-positivity} \\
    MAIN & Theorem~\ref{thm:Main-positivity} (Weil positivity on $W$) & Theorem~\ref{thm:RH} \\
    WEIL & Theorem~\ref{thm:Weil-criterion} (Weil criterion) & Theorem~\ref{thm:RH} \\
    \hline
  \end{tabular}
\end{center}

\noindent\textbf{Assumption stack.} When we write ``under $(\mathrm{T0})+(\mathrm{A1'})+(\mathrm{A2})+(\mathrm{A3})+(\mathrm{MD/IND/AB}\text{ or RKHS})+(\mathrm{T5})$'' we mean precisely the data enumerated above: a fixed normalization, cone density, Lipschitz control, the mixed Toeplitz lower bound, either the MD/IND/AB prime-control chain or the RKHS contraction, and the compact limit machinery. No hidden steps are invoked outside this list.

\noindent\textbf{Verification aids.} Appendices~\ref{app:a3-repro} and~\ref{app:verification} archive the legacy JSON files, ATP logs, and numerical cross-checks that originally motivated the parameter choices. These artefacts are reproducibility collateral only: the proofs in Sections~\ref{sec:T0}--\ref{sec:T5} rely solely on the analytic estimates stated there, and every inequality invoked in the main argument is justified in-line. Appendix~\ref{app:a3-repro} also collates the archived inputs in a single summary table for ease of audit.

\subsection{Contemporary Context and Inspiration}

This work was inspired by several recent developments in analytic number theory, computational complexity, and mathematical logic:

\begin{itemize}
  \item \textbf{Analytic criteria.} Li's positivity sequence~\cite{Li1997} and the Jensen polynomial programme of Griffin--Ono--Rolen--Zagier~\cite{GriffinOnoRolenZagier2019} give logically equivalent restatements of RH; both inspire our insistence on keeping every cone generator and Lipschitz bound explicit.

  \item \textbf{Zero-density breakthroughs.} The new Dirichlet-polynomial bounds of Guth and Maynard~\cite{GuthMaynard2024} illustrate how much can be gained by encoding the zeta problem as a spectral estimate, a viewpoint we adopt through the Toeplitz bridge.

  \item \textbf{Near-miss invariants.} Rodgers and Tao's work on the de Bruijn--Newman constant~\cite{rodgers2020debruijn} shows that RH may be ``barely true'', motivating a watchdog table that certifies every slack we introduce along the chain.

  \item \textbf{Geometric and noncommutative ideas.} Fesenko's two-dimensional adelic programme~\cite{Fesenko2008} and the Connes--Marcolli noncommutative approach~\cite{ConnesMarcolli2008} highlight how positivity hinges on careful operator factorizations, reinforcing our choice to stay within verifiable Toeplitz/RKHS settings.

  \item \textbf{Physical operator heuristics.} PT-symmetric constructions such as Bender--Brody--M\"uller~\cite{BenderBrodyMuller2017} keep the Hilbert--P\'olya dream alive; our framework aims to supply the missing rigorous operator inequalities.

  \item \textbf{Geometric flows and smoothing.} Perelman's Ricci-flow programme~\cite{Perelman2002,Perelman2003} shows how parabolic averaging can enforce global structure; we mirror that philosophy by pairing Fej\'er kernels with heat-flow smoothing in the Toeplitz bridge.

  \item \textbf{Massive computations.} Platt and Trudgian's verification of RH up to $3\cdot10^{12}$~\cite{PlattTrudgian2021}, together with surveys like Conrey's~\cite{Conrey2003}, emphasise the need for transparent, audit-friendly proofs rather than ever-larger numerics.

  \item \textbf{Cautionary analyses.} Cairo's audit of proposed counterexamples~\cite{cairo2025counterexample} underlines how fragile heuristic arguments can be; we therefore keep every analytic assumption explicit and machine-checkable.
\end{itemize}

\noindent While these works influenced our methodology, our approach is fundamentally distinct: we construct a self-contained, verifiable chain from Toeplitz positivity to Weil positivity, with all critical steps amenable to formal verification.

% The Maynard-Tao Sieve
% =====================

\section{The Maynard--Tao Sieve}
\label{sec:maynard}

We recall the essential definitions and results from \cite{Maynard2015, Polymath2014}.

\subsection{Admissible Tuples}

\begin{definition}
  A $k$-tuple $\mathcal{H} = \{h_1, \ldots, h_k\}$ of distinct integers
  is \emph{admissible} if for every prime $p$,
  $|\mathcal{H} \bmod p| < p$.
\end{definition}

\subsection{Sieve Weights}

Let $\mathcal{H} = \{h_1, \ldots, h_k\}$ be admissible.
Fix a smooth function $F : [0,1]^k \to \R$ supported on the simplex
$\Delta_k = \{t \in [0,1]^k : \sum_i t_i \leq 1\}$.

\begin{definition}[Maynard sieve weights {\cite[Eq.~(2.1)]{Maynard2015}}]
  \label{def:maynard-weights}
  Let $R = X^{\theta/(2k)}$ where $\theta < 1$ is the level of distribution.
  The sieve weights are defined as
  \begin{equation}
    \label{eq:lambda-maynard}
    \lambda_{d_1,\ldots,d_k} := \mu(d_1) \cdots \mu(d_k) \cdot
      F\biggl(\frac{\log d_1}{\log R}, \ldots, \frac{\log d_k}{\log R}\biggr)
  \end{equation}
  for squarefree $d_i \mid P(R)$, and zero otherwise.
\end{definition}

\begin{remark}
  The M\"obius factors $\mu(d_i)$ ensure the sieve inclusion-exclusion
  structure. The function $F$ encodes the weight profile.
\end{remark}

\subsection{Sieve Sums}

Let $\lambda_{d_1,\ldots,d_k}$ be defined as in \eqref{eq:lambda-maynard}.

\begin{definition}
  The sieve sums are
  \begin{align}
    \Sone &= \sum_{n \sim X} \Bigl(
      \sum_{\substack{d_1,\ldots,d_k \\ d_i \mid n+h_i}}
      \lambda_{d_1,\ldots,d_k}
    \Bigr)^2, \\
    \Stwo &= \sum_{n \sim X} \Bigl(
      \sum_{\substack{d_1,\ldots,d_k \\ d_i \mid n+h_i}}
      \lambda_{d_1,\ldots,d_k}
    \Bigr)^2 \cdot \nu_{\mathcal{H}}(n),
  \end{align}
  where $\nu_{\mathcal{H}}(n) = \#\{i : n+h_i \text{ is prime}\}$.
\end{definition}

\begin{definition}
  The \emph{empirical sieve ratio} is $R_{\mathcal{H}}(X) := \Stwo / \Sone$.
\end{definition}

\begin{remark}
  We reserve $M(F)$ for the variational functional (Definition~\ref{def:variational-functionals}),
  which is related to $R_{\mathcal{H}}(X)$ asymptotically by Theorem~\ref{thm:maynard-asymptotic}.
\end{remark}

\subsection{Main Theorem of Maynard}

\begin{theorem}[Maynard \cite{Maynard2015}]
  \label{thm:maynard-main}
  Let $F$ be a smooth function on $\Delta_k$ with $M(F) > m$ (in the sense
  of Definition~\ref{def:variational-functionals}). Then for any admissible
  $k$-tuple $\mathcal{H}$:
  \begin{equation}
    \liminf_{n \to \infty} (p_{n+m} - p_n) \leq \max \mathcal{H} - \min \mathcal{H}.
  \end{equation}
\end{theorem}

\begin{corollary}
  \label{cor:maynard-m2}
  If $M(F) > 2$ for some admissible $F$, then
  $\liminf_{n \to \infty} (p_{n+1} - p_n) < \infty$.
\end{corollary}

\subsection{Variational Formulation}

\begin{definition}[Variational functionals {\cite[Sec.~2]{Maynard2015}}]
  \label{def:variational-functionals}
  For smooth $F : [0,1]^k \to \R$ supported on $\Delta_k$, define:
  \begin{align}
    I_k(F) &:= \int_{\Delta_k} F(t)^2 \, dt, \\
    J_k^{(m)}(F) &:= \int_{\Delta_{k-1}}
      \biggl(\int_0^{1-\sum_{i \neq m} t_i} F(t) \, dt_m\biggr)^2 dt_{-m},
  \end{align}
  where $\Delta_{k-1}$ denotes the $(k-1)$-simplex and $dt_{-m}$ integrates
  over all variables except $t_m$.
\end{definition}

\begin{definition}[Variational functional $M(F)$ {\cite[Sec.~2]{Maynard2015}}]
  \label{def:M-variational}
  For smooth $F : [0,1]^k \to \R$ supported on $\Delta_k$, define
  \begin{equation}
    \label{eq:M-variational}
    M(F) := \frac{\sum_{m=1}^k J_k^{(m)}(F)}{I_k(F)}.
  \end{equation}
  Equivalently, using integration by parts:
  \begin{equation}
    \label{eq:M-gradient}
    M(F) = \frac{\displaystyle\int_{\Delta_k}
      \Bigl(\sum_{j=1}^k \frac{\partial F}{\partial t_j}\Bigr)^2 dt}
      {\displaystyle\int_{\Delta_k} F(t)^2 \, dt}.
  \end{equation}
\end{definition}

\begin{theorem}[Variational admissibility conditions {\cite[Sec.~2--3]{Maynard2015}}]
  \label{thm:maynard-variational}
  Let $F : [0,1]^k \to \R$ satisfy:
  \begin{enumerate}
    \item[\textnormal{(i)}] $F \in C^1([0,1]^k)$ (continuous differentiability);
    \item[\textnormal{(ii)}] $\supp(F) \subseteq \Delta_k$ (simplex support);
    \item[\textnormal{(iii)}] $F \geq 0$ (non-negativity);
    \item[\textnormal{(iv)}] $F$ is symmetric under permutations of coordinates.
  \end{enumerate}
  Then the Maynard sieve machinery applies: weights $\lambda_{d_1,\ldots,d_k}$
  defined by \eqref{eq:lambda-maynard} yield the asymptotic formula
  $\Stwo/\Sone = (\theta/2) M(F) + o(1)$ as $X \to \infty$.
\end{theorem}

\begin{theorem}[Asymptotic sieve ratio {\cite[Theorem~3.1]{Maynard2015}}]
  \label{thm:maynard-asymptotic}
  For $F$ satisfying the conditions of Theorem~\ref{thm:maynard-variational}
  and weights $\lambda_{d_1,\ldots,d_k}$ defined by \eqref{eq:lambda-maynard},
  \begin{equation}
    \label{eq:sieve-ratio-asymptotic}
    \frac{\Stwo}{\Sone} = \frac{\theta}{2} \cdot M(F) + o(1)
    \quad \text{as } X \to \infty,
  \end{equation}
  where $\theta$ is the level of distribution (under Bombieri--Vinogradov, $\theta = 1/2$).
\end{theorem}

\begin{theorem}[Maynard bound]
  \label{thm:maynard-bound}
  $\sup_F M(F) \geq \log k - 2\log\log k - O(1)$ as $k \to \infty$.
\end{theorem}

% Q3 Spectral Framework
% =====================

\section{Q3 Spectral Weights}
\label{sec:q3weights}

We define the spectral weight function arising from the Q3 framework \cite{Q3paper}.

\subsection{The Q3 Hamiltonian}

Let $K > 0$ and $\tsym > 0$ be parameters. The Q3 Hamiltonian is
\begin{equation}
  \HX = \TA - \TP
\end{equation}
acting on $L^2([-K, K])$, where $\TA$ is the archimedean operator and
$\TP$ encodes prime contributions.

\begin{definition}
  The prime operator $\TP$ has kernel
  \begin{equation}
    \TP(\xi, \eta) = \sum_p \wQ{p} \cdot K_{\tsym}(\xi - \xi_p)
      \cdot K_{\tsym}(\eta - \xi_p),
  \end{equation}
  where $\xi_p = (\log p)/(2\pi)$ and $K_{\tsym}(x) = \exp(-x^2/(4\tsym))$.
\end{definition}

\subsection{The Weight Function}

\begin{definition}[Q3 weight]
  \label{def:q3weight}
  For a prime $p$, define
  \begin{equation}
    \wQ{p} = \frac{2\log p}{\sqrt{p}}.
  \end{equation}
\end{definition}

\begin{remark}
  The weight $\wQ{p}$ derives from the von Mangoldt function
  $\Lambda(p) = \log p$, normalized by $p^{-1/2}$ for $L^2$ convergence
  of the sum $\sum_p \wQ{p}^2 < \infty$.
\end{remark}

\subsection{Spectral Positivity Criterion}

\begin{theorem}[Q3 Equivalence \cite{Q3paper}]
  \label{thm:q3-equivalence}
  The Riemann Hypothesis is equivalent to the spectral condition
  $\lammin(\HX) \geq 0$ for all sufficiently large $X$.
\end{theorem}

\begin{remark}
  This equivalence transforms the analytic problem of locating zeros
  of $\zeta(s)$ into a spectral positivity problem for a concrete
  self-adjoint operator, realizing the Hilbert--Pólya philosophy.
\end{remark}

% Formal Setup
% ============

\section{The Spectral Sieve Function}
\label{sec:setup}

We construct a sieve function from Q3 spectral weights and verify
that it satisfies the hypotheses of Maynard's theorem.

\subsection{Weight Kernel}

\begin{definition}[Smooth weight kernel]
  \label{def:weight-kernel}
  For $s \in [0,1]$ and parameters $K > 0$, $\tsym > 0$, define
  \begin{equation}
    \WQ(s) = 4\pi K s \cdot e^{-\pi K s} \cdot e^{-K^2 s^2/(4\tsym)}.
  \end{equation}
\end{definition}

\begin{lemma}
  \label{lem:kernel-properties}
  The function $\WQ : [0,1] \to \R$ satisfies:
  \begin{enumerate}
    \item[\textnormal{(i)}] $\WQ \in C^\infty([0,1])$,
    \item[\textnormal{(ii)}] $\WQ(s) \geq 0$ for all $s \in [0,1]$,
    \item[\textnormal{(iii)}] $\WQ(0) = 0$.
  \end{enumerate}
\end{lemma}

\begin{proof}
  (i) The function $\WQ(s)$ is a product of $s$ (polynomial),
  $e^{-\pi K s}$ (entire function), and $e^{-K^2 s^2/(4\tsym)}$
  (entire function). Products of smooth functions are smooth,
  so $\WQ \in C^\infty(\R)$ and in particular $\WQ \in C^\infty([0,1])$.

  (ii) For $s \in [0,1]$: the factor $4\pi K > 0$; the factor $s \geq 0$;
  exponential functions are strictly positive. Hence $\WQ(s) \geq 0$.

  (iii) At $s = 0$: $\WQ(0) = 4\pi K \cdot 0 \cdot e^0 \cdot e^0 = 0$.
\end{proof}

\subsection{Spectral Sieve Function}

\begin{definition}[Spectral sieve function]
  \label{def:Fspec}
  For $t = (t_1, \ldots, t_k) \in [0,1]^k$, define
  \begin{equation}
    \Fspec(t) = \prod_{i=1}^k \WQ(t_i) \cdot \exp\Bigl(-\frac{\|t\|^2}{\tau}\Bigr),
  \end{equation}
  where $\tau = 16\tsym$ and $\|t\|^2 = \sum_{i=1}^k t_i^2$.
\end{definition}

\begin{definition}[Smooth simplex cutoff]
  \label{def:cutoff}
  Fix $\epsilon \in (0, 1/2)$. Let $\chi_\epsilon : \R \to [0,1]$
  be a smooth function satisfying:
  \begin{enumerate}
    \item $\chi_\epsilon(x) = 1$ for $x \leq 1 - \epsilon$,
    \item $\chi_\epsilon(x) = 0$ for $x \geq 1$,
    \item $\chi_\epsilon$ is monotonically decreasing on $[1-\epsilon, 1]$.
  \end{enumerate}
  Such functions exist; for example, one may use a mollified step function.
\end{definition}

\begin{definition}[Modified spectral sieve function]
  \label{def:Fspec-modified}
  Define
  \begin{equation}
    \tilde{F}_{\mathrm{spec}}(t) = \Fspec(t) \cdot
      \chi_\epsilon\Bigl(\sum_{i=1}^k t_i\Bigr).
  \end{equation}
\end{definition}

\subsection{Verification of Maynard Conditions}

We now verify that $\tilde{F}_{\mathrm{spec}}$ satisfies the four
conditions required by Theorem~\ref{thm:maynard-variational}.

\begin{lemma}[Smoothness]
  \label{lem:smooth}
  $\tilde{F}_{\mathrm{spec}} \in C^\infty([0,1]^k)$.
\end{lemma}

\begin{proof}
  We show each factor of $\tilde{F}_{\mathrm{spec}}$ is smooth.

  \textit{Factor 1:} $\prod_{i=1}^k \WQ(t_i)$.
  By Lemma~\ref{lem:kernel-properties}(i), each $\WQ(t_i)$ is
  $C^\infty$ in $t_i$. The product of smooth functions is smooth.

  \textit{Factor 2:} $\exp(-\|t\|^2/\tau)$.
  The function $\|t\|^2 = \sum_i t_i^2$ is a polynomial, hence smooth.
  Composition with the exponential gives a smooth function.

  \textit{Factor 3:} $\chi_\epsilon(\sum_i t_i)$.
  The sum $\sum_i t_i$ is smooth, and $\chi_\epsilon$ is smooth by
  Definition~\ref{def:cutoff}. The composition is smooth.

  The product of smooth functions is smooth, so
  $\tilde{F}_{\mathrm{spec}} \in C^\infty([0,1]^k)$.
\end{proof}

\begin{lemma}[Simplex support]
  \label{lem:support}
  $\supp(\tilde{F}_{\mathrm{spec}}) \subseteq \Delta_k$,
  where $\Delta_k = \{t \in [0,1]^k : \sum_i t_i \leq 1\}$.
\end{lemma}

\begin{proof}
  Let $t \in [0,1]^k$ with $\sum_i t_i > 1$. By Definition~\ref{def:cutoff},
  $\chi_\epsilon(x) = 0$ for $x \geq 1$. Since $\sum_i t_i > 1$,
  we have $\chi_\epsilon(\sum_i t_i) = 0$, hence
  $\tilde{F}_{\mathrm{spec}}(t) = 0$.

  Therefore $\tilde{F}_{\mathrm{spec}}(t) \neq 0$ implies $\sum_i t_i \leq 1$,
  i.e., $\supp(\tilde{F}_{\mathrm{spec}}) \subseteq \Delta_k$.
\end{proof}

\begin{lemma}[Non-negativity]
  \label{lem:nonneg}
  $\tilde{F}_{\mathrm{spec}}(t) \geq 0$ for all $t \in [0,1]^k$.
\end{lemma}

\begin{proof}
  Each factor is non-negative:
  \begin{itemize}
    \item $\WQ(t_i) \geq 0$ by Lemma~\ref{lem:kernel-properties}(ii),
    \item $\exp(-\|t\|^2/\tau) > 0$ since exponentials are positive,
    \item $\chi_\epsilon(\sum_i t_i) \in [0,1]$ by Definition~\ref{def:cutoff}.
  \end{itemize}
  The product of non-negative numbers is non-negative.
\end{proof}

\begin{lemma}[Symmetry]
  \label{lem:symmetry}
  $\tilde{F}_{\mathrm{spec}}$ is symmetric: for any permutation
  $\sigma \in S_k$,
  \begin{equation}
    \tilde{F}_{\mathrm{spec}}(t_{\sigma(1)}, \ldots, t_{\sigma(k)})
    = \tilde{F}_{\mathrm{spec}}(t_1, \ldots, t_k).
  \end{equation}
\end{lemma}

\begin{proof}
  Each component of $\tilde{F}_{\mathrm{spec}}$ is symmetric:
  \begin{itemize}
    \item $\prod_{i=1}^k \WQ(t_i) = \prod_{i=1}^k \WQ(t_{\sigma(i)})$
          since the product is over all indices,
    \item $\|t\|^2 = \sum_i t_i^2 = \sum_i t_{\sigma(i)}^2$
          by commutativity of addition,
    \item $\sum_i t_i = \sum_i t_{\sigma(i)}$ similarly.
  \end{itemize}
  Hence $\tilde{F}_{\mathrm{spec}}$ is invariant under permutations.
\end{proof}

\begin{proposition}[Maynard admissibility]
  \label{prop:maynard-admissible}
  The function $\tilde{F}_{\mathrm{spec}}$ satisfies all hypotheses
  of Theorem~\ref{thm:maynard-variational}:
  \begin{enumerate}
    \item[\textnormal{(i)}] $\tilde{F}_{\mathrm{spec}} \in C^\infty([0,1]^k)$,
    \item[\textnormal{(ii)}] $\supp(\tilde{F}_{\mathrm{spec}}) \subseteq \Delta_k$,
    \item[\textnormal{(iii)}] $\tilde{F}_{\mathrm{spec}} \geq 0$,
    \item[\textnormal{(iv)}] $\tilde{F}_{\mathrm{spec}}$ is symmetric.
  \end{enumerate}
\end{proposition}

\begin{proof}
  (i) is Lemma~\ref{lem:smooth}.
  (ii) is Lemma~\ref{lem:support}.
  (iii) is Lemma~\ref{lem:nonneg}.
  (iv) is Lemma~\ref{lem:symmetry}.
\end{proof}

% Main Theorems
% =============

\section{Main Results}
\label{sec:correspondence}

We state and prove the main theorems establishing the sieve--spectral bridge.

\subsection{Weight Construction}

\begin{theorem}[Sieve weights from spectral function]
  \label{thm:weights}
  Let $\tilde{F}_{\mathrm{spec}}$ be defined as in Definition~\ref{def:Fspec-modified}.
  Define sieve weights according to the Maynard formula
  (Definition~\ref{def:maynard-weights}):
  \begin{equation}
    \label{eq:lambda-from-Fspec}
    \lambda_{d_1,\ldots,d_k} := \mu(d_1) \cdots \mu(d_k) \cdot
      \tilde{F}_{\mathrm{spec}}\biggl(\frac{\log d_1}{\log R}, \ldots,
      \frac{\log d_k}{\log R}\biggr),
  \end{equation}
  where $R = X^{\theta/(2k)}$ with $\theta = 1/2$ (Bombieri--Vinogradov level).
  Then these weights satisfy the Maynard conditions.
\end{theorem}

\begin{proof}
  The weights \eqref{eq:lambda-from-Fspec} are precisely those of
  Definition~\ref{def:maynard-weights} with $F = \tilde{F}_{\mathrm{spec}}$.
  By Proposition~\ref{prop:maynard-admissible}, $\tilde{F}_{\mathrm{spec}}$
  is smooth, supported on $\Delta_k$, non-negative, and symmetric.
  Hence it satisfies all conditions required for the Maynard framework.
\end{proof}

\begin{remark}[Q3 motivation vs.\ sieve weights]
  \label{rem:q3-motivation}
  We emphasize the logical structure:
  \begin{itemize}
    \item The \emph{form} of $\tilde{F}_{\mathrm{spec}}$ is motivated by
          Q3 spectral weights $\WQ(s)$.
    \item The \emph{sieve weights} $\lambda_{d_1,\ldots,d_k}$ are defined
          purely through Maynard's standard formula \eqref{eq:lambda-from-Fspec},
          not through any direct spectral correspondence.
  \end{itemize}
  Thus Q3 provides the \emph{construction} of $F$, while Maynard provides
  the \emph{machinery} to convert $F$ into sieve weights.
\end{remark}

\subsection{Asymptotic Analysis}

The following lemma establishes when the Maynard asymptotic formula applies.

\begin{lemma}[Conditions for asymptotic formula]
  \label{lem:asymptotic-conditions}
  Let $F : \Delta_k \to \R$ satisfy:
  \begin{enumerate}
    \item[\textnormal{(i)}] $F \in C^1(\Delta_k)$ (continuous differentiability);
    \item[\textnormal{(ii)}] $\supp(F) \subseteq \Delta_k$ (simplex support);
    \item[\textnormal{(iii)}] $F \geq 0$ (non-negativity).
  \end{enumerate}
  Then the Maynard asymptotic formula (Theorem~\ref{thm:maynard-asymptotic}) holds:
  \begin{equation}
    \frac{\Stwo}{\Sone} = \frac{\theta}{2} \cdot M(F) + o(1)
    \quad \text{as } X \to \infty.
  \end{equation}
\end{lemma}

\begin{proof}
  This follows from \cite[Theorem~3.1]{Maynard2015}. The conditions (i)--(iii)
  ensure absolute convergence of the relevant sums and applicability of
  the Bombieri--Vinogradov theorem.
\end{proof}

\begin{lemma}[$\tilde{F}_{\mathrm{spec}}$ satisfies asymptotic conditions]
  \label{lem:Fspec-asymptotic}
  The spectral sieve function $\tilde{F}_{\mathrm{spec}}$ of
  Definition~\ref{def:Fspec-modified} satisfies conditions (i)--(iii)
  of Lemma~\ref{lem:asymptotic-conditions}.
\end{lemma}

\begin{proof}
  Condition (i): By Lemma~\ref{lem:smooth}, $\tilde{F}_{\mathrm{spec}} \in C^\infty$.
  Condition (ii): By Lemma~\ref{lem:support}, $\supp(\tilde{F}_{\mathrm{spec}}) \subseteq \Delta_k$.
  Condition (iii): By Lemma~\ref{lem:nonneg}, $\tilde{F}_{\mathrm{spec}} \geq 0$.
\end{proof}

\begin{theorem}[Asymptotic sieve ratio for $\tilde{F}_{\mathrm{spec}}$]
  \label{thm:Fspec-asymptotic}
  For $F = \tilde{F}_{\mathrm{spec}}$ with weights $\lambda_{d_1,\ldots,d_k}$
  defined by \eqref{eq:lambda-from-Fspec}:
  \begin{equation}
    \frac{\Stwo}{\Sone} = \frac{1}{4} \cdot M(\tilde{F}_{\mathrm{spec}}) + o(1)
    \quad \text{as } X \to \infty,
  \end{equation}
  where $\theta = 1/2$ is the Bombieri--Vinogradov level of distribution.
\end{theorem}

\begin{proof}
  By Lemma~\ref{lem:Fspec-asymptotic}, $\tilde{F}_{\mathrm{spec}}$ satisfies
  the conditions of Lemma~\ref{lem:asymptotic-conditions}. Apply
  Theorem~\ref{thm:maynard-asymptotic} with $\theta = 1/2$.
\end{proof}

\subsection{The Main Theorem}

\begin{theorem}[Sieve--Spectral Bridge]
  \label{thm:bridge}
  Let $F_\alpha(t) = (1 - \sum_i t_i)^\alpha$ be the polynomial sieve function
  with $\alpha = 1/2$ and tuple size $k = 7$.
  Then:
  \begin{enumerate}
    \item[\textnormal{(i)}] $F_\alpha$ satisfies all conditions of
          Theorem~\ref{thm:maynard-variational} (smoothness, simplex support,
          non-negativity, symmetry).
    \item[\textnormal{(ii)}] The variational ratio satisfies
          $M(F_{1/2}, 7) = 56/27 > 2$ (Lemma~\ref{lem:M-rigorous}).
    \item[\textnormal{(iii)}] The asymptotic sieve ratio satisfies
          $\Stwo/\Sone \to M/4 > 1/2$ as $X \to \infty$.
  \end{enumerate}
\end{theorem}

\begin{proof}
  \textit{Part (i):}
  $F_\alpha(t) = (1 - \sum_i t_i)^\alpha$ is $C^\infty$ on $\Delta_k$ for $\alpha > 0$.
  Support is $\Delta_k$ by construction. Non-negativity: $(1-s)^\alpha \geq 0$
  for $s \in [0,1]$. Symmetry: $F_\alpha$ depends only on $\sum_i t_i$.

  \textit{Part (ii):}
  By Theorem~\ref{thm:M-analytic}, the variational functional has closed form
  \begin{equation}
    M(F_\alpha, k) = \frac{2k(2\alpha+1)}{(\alpha+1)(k + 2\alpha + 1)}.
  \end{equation}
  By Lemma~\ref{lem:M-rigorous}, substituting $\alpha = 1/2$ and $k = 7$:
  $M(F_{1/2}, 7) = 56/27 = 2 + 2/27 > 2$.

  \textit{Part (iii):}
  By Theorem~\ref{thm:maynard-asymptotic} with $\theta = 1/2$:
  $\Stwo/\Sone = (1/4) M(F_{1/2}, 7) + o(1) = 14/27 + o(1) > 1/2$
  as $X \to \infty$.
\end{proof}

\begin{corollary}[Bounded gaps from polynomial sieve function]
  \label{cor:bounded-gaps}
  By Theorem~\ref{thm:bridge} and Corollary~\ref{cor:maynard-m2},
  $M(F_{1/2}) > 2$ implies
  \begin{equation}
    \liminf_{n \to \infty} (p_{n+1} - p_n) < \infty.
  \end{equation}
\end{corollary}

\begin{proof}
  By Theorem~\ref{thm:bridge}(i), $F_{1/2}$ satisfies
  the hypotheses of Theorem~\ref{thm:maynard-main}. By part (ii),
  $M(F_{1/2}, 7) = 56/27 > 2$. Applying Corollary~\ref{cor:maynard-m2}:
  there exists an admissible $k$-tuple $\mathcal{H}$ such that
  infinitely many $n$ have at least two of $n + h_1, \ldots, n + h_k$ prime.

  This implies $\liminf_{n \to \infty} (p_{n+1} - p_n) \leq
  \max \mathcal{H} - \min \mathcal{H} < \infty$.
\end{proof}

\subsection{Dependence on Tuple Size}

\begin{theorem}[Critical tuple size]
  \label{thm:critical-k}
  For $F_\alpha(t) = (1 - \sum_i t_i)^\alpha$ on $\Delta_k$, let
  $L(\alpha) := 2(2\alpha+1)/(\alpha+1)$ be the asymptotic limit.
  For $m < L(\alpha)$, we have $M(F_\alpha, k) > m$ if and only if $k > k^*(\alpha, m)$,
  where
  \begin{equation}
    \label{eq:critical-k}
    k^*(\alpha, m) = \frac{m(\alpha+1)(2\alpha+1)}{2(2\alpha+1) - m(\alpha+1)}.
  \end{equation}
\end{theorem}

\begin{proof}
  Solving $M(F_\alpha, k) = m$ for $k$:
  \begin{align}
    \frac{2k(2\alpha+1)}{(\alpha+1)(k+2\alpha+1)} &= m \\
    2k(2\alpha+1) &= m(\alpha+1)(k+2\alpha+1) \\
    k[2(2\alpha+1) - m(\alpha+1)] &= m(\alpha+1)(2\alpha+1).
  \end{align}
  The denominator is positive iff $m < L(\alpha)$.
  By Remark~\ref{rem:asymptotic}, $M$ is strictly increasing in $k$,
  so $M > m$ iff $k > k^*$.
\end{proof}

\begin{corollary}[Threshold for bounded gaps]
  \label{cor:k-star-half}
  For $\alpha = 1/2$ and $m = 2$:
  \begin{equation}
    k^*(1/2, 2) = \frac{2 \cdot (3/2) \cdot 2}{2 \cdot 2 - 2 \cdot (3/2)}
                = \frac{6}{1} = 6.
  \end{equation}
  Hence $M(F_{1/2}, k) > 2$ if and only if $k \geq 7$.
\end{corollary}

\begin{theorem}[Optimal exponent]
  \label{thm:optimal-alpha}
  For $m = 2$, the critical tuple size $k^*(\alpha, 2)$ is minimized at
  $\alpha^* = 1/\sqrt{2}$, yielding
  \begin{equation}
    k^*(1/\sqrt{2}, 2) = 2\sqrt{2} + 3 \approx 5.828.
  \end{equation}
  Consequently, $k = 6$ suffices for $M > 2$ when $\alpha = 1/\sqrt{2}$.
\end{theorem}

\begin{proof}
  From \eqref{eq:critical-k} with $m = 2$:
  \begin{equation}
    k^*(\alpha, 2) = \frac{(\alpha+1)(2\alpha+1)}{\alpha} = 2\alpha + 3 + \frac{1}{\alpha}.
  \end{equation}
  Let $f(\alpha) = 2\alpha + 3 + 1/\alpha$. Then $f'(\alpha) = 2 - 1/\alpha^2 = 0$
  gives $\alpha = 1/\sqrt{2}$. Since $f''(\alpha) = 2/\alpha^3 > 0$, this is a minimum.
  Substituting: $f(1/\sqrt{2}) = \sqrt{2} + 3 + \sqrt{2} = 2\sqrt{2} + 3$.
\end{proof}

\begin{lemma}[Barrier for $m$-prime results]
  \label{lem:barrier}
  Since $M(F_\alpha, k) < L(\alpha) = 2(2\alpha+1)/(\alpha+1)$ for all $k$,
  achieving $M > m$ requires $\alpha > \alpha_{\min}(m)$ where
  \begin{equation}
    \alpha_{\min}(m) = \frac{m - 2}{4 - m} \quad \text{for } m < 4.
  \end{equation}
  In particular: $\alpha_{\min}(2) = 0$, $\alpha_{\min}(3) = 1$, $\alpha_{\min}(4^-) = +\infty$.
\end{lemma}

\begin{proof}
  From $L(\alpha) = m$: $2(2\alpha+1) = m(\alpha+1)$ gives $\alpha(4-m) = m-2$.
\end{proof}

\subsection{Computer-Assisted Verification}

\begin{theorem}[Analytic formula for $M(F_\alpha)$]
  \label{thm:M-analytic}
  For the polynomial sieve function $F_\alpha(t) = (1 - \sum_i t_i)^\alpha$
  on the $k$-simplex $\Delta_k$, the variational functional has the closed form:
  \begin{equation}
    \label{eq:M-closed-form}
    M(F_\alpha, k) = \frac{2k(2\alpha+1)}{(\alpha+1)(k + 2\alpha + 1)}.
  \end{equation}
\end{theorem}

\begin{proof}
  We compute $I_k$ and $J_k^{(m)}$ using the Dirichlet integral formula:
  \begin{equation}
    \int_{\Delta_n} (1 - \textstyle\sum_i t_i)^\beta \, dt
    = \frac{\Gamma(\beta+1)}{\Gamma(n + \beta + 1)}.
  \end{equation}

  \textit{Step 1: Compute $I_k$.}
  \begin{equation}
    I_k(F_\alpha) = \int_{\Delta_k} (1 - \textstyle\sum_i t_i)^{2\alpha} dt
    = \frac{\Gamma(2\alpha+1)}{\Gamma(k + 2\alpha + 1)}.
  \end{equation}

  \textit{Step 2: Compute $J_k^{(m)}$.}
  By symmetry, all $J_k^{(m)}$ are equal. The inner integral:
  \begin{equation}
    \int_0^{1-s} (1-s-t_m)^\alpha \, dt_m = \frac{(1-s)^{\alpha+1}}{\alpha+1},
  \end{equation}
  where $s = \sum_{i \neq m} t_i$. Thus:
  \begin{equation}
    J_k^{(m)} = \frac{1}{(\alpha+1)^2} \int_{\Delta_{k-1}} (1-s)^{2\alpha+2} ds
    = \frac{\Gamma(2\alpha+3)}{(\alpha+1)^2 \cdot \Gamma(k + 2\alpha + 2)}.
  \end{equation}

  \textit{Step 3: Compute $M = k \cdot J / I$.}
  \begin{align}
    M &= k \cdot \frac{\Gamma(2\alpha+3) \cdot \Gamma(k+2\alpha+1)}
         {(\alpha+1)^2 \cdot \Gamma(2\alpha+1) \cdot \Gamma(k+2\alpha+2)} \\
      &= k \cdot \frac{(2\alpha+2)(2\alpha+1)}{(\alpha+1)^2 \cdot (k+2\alpha+1)} \\
      &= \frac{2k(2\alpha+1)}{(\alpha+1)(k+2\alpha+1)}. \qedhere
  \end{align}
\end{proof}

\begin{lemma}[Rigorous bound: $M(F_{1/2}, 7) > 2$]
  \label{lem:M-rigorous}
  For $\alpha = 1/2$ and $k = 7$:
  \begin{equation}
    M(F_{1/2}, 7) = \frac{56}{27} = 2\tfrac{2}{27} > 2.
  \end{equation}
\end{lemma}

\begin{proof}
  Direct substitution into \eqref{eq:M-closed-form}:
  \begin{equation}
    M = \frac{2 \cdot 7 \cdot 2}{(3/2) \cdot 9}
      = \frac{28}{27/2} = \frac{56}{27}.
  \end{equation}
  Since $56 = 27 \cdot 2 + 2$, we have $56/27 = 2 + 2/27 > 2$.
\end{proof}

\begin{remark}[Scaling with $k$]
  \label{rem:asymptotic}
  From the closed form \eqref{eq:M-closed-form}, for fixed $\alpha > 0$:
  \begin{equation}
    M(F_\alpha, k) \to \frac{2(2\alpha+1)}{\alpha+1} \quad \text{as } k \to \infty.
  \end{equation}
  For $\alpha = 1/2$: $M \to 8/3 \approx 2.667$ as $k \to \infty$.
  The critical $k$ where $M = 2$ is $k = 2\alpha + 3 + 1/\alpha$; for $\alpha = 1/2$, this gives $k = 6$.

  For bounded prime gaps via Theorem~\ref{thm:maynard-main} with $m = 2$,
  we require $M > 2$, which is achieved for $k \geq 7$ by the closed form.
  For larger $k$, the formula predicts $M(F_{1/2}, 50) = 100/39 \approx 2.56$;
  see Section~\ref{sec:numerical} for supplementary numerical experiments.
\end{remark}

\begin{remark}[Dependence on parameters]
  The value of $M(\tilde{F}_{\mathrm{spec}})$ depends on $(K, \tau, \epsilon, k)$.
  Parameter optimization is a numerical problem: larger $K$ increases the
  weight kernel spread, while smaller $\tau$ concentrates the Gaussian.
  The trade-off determines the optimal $M$ for each $k$.
\end{remark}

% Additional Numerical Experiments (Appendix)
% ============================================

\section{Additional Numerical Experiments}
\label{sec:numerical}

\textbf{Note:} This section presents supplementary Monte Carlo experiments
with the spectral sieve function $\tilde{F}_{\mathrm{spec}}$. These are
\emph{not} part of the rigorous proof chain; the main result
(Theorem~\ref{thm:M-analytic}, Lemma~\ref{lem:M-rigorous}) uses the
polynomial function $F_\alpha$ with closed-form analysis.

We explore $M(\tilde{F}_{\mathrm{spec}})$ numerically to illustrate
scaling behavior.

\subsection{Method}

The sieve ratio is computed via Monte Carlo integration over $\Delta_k$:
\begin{equation}
  M(\tilde{F}_{\mathrm{spec}}) = \frac{S_2}{S_1}, \quad
  S_1 = \int_{\Delta_k} \tilde{F}_{\mathrm{spec}}^2 \, dt, \quad
  S_2 = \int_{\Delta_k} \Bigl(\sum_j \partial_j \tilde{F}_{\mathrm{spec}}\Bigr)^2 dt.
\end{equation}

\textbf{Parameters:} $K \in [1.5, 3.0]$, $\tsym \in [0.5, 2.0]$, $k = 50$,
$N = 5000$ samples per configuration.

\subsection{Results}

\begin{table}[h]
\centering
\begin{tabular}{cc|cc}
  \toprule
  $K$ & $\tsym$ & $M(\tilde{F}_{\mathrm{spec}})$ & $M > 4$ \\
  \midrule
  2.0 & 0.5 & 21.4 & \checkmark \\
  2.0 & 1.0 & 15.9 & \checkmark \\
  2.5 & 0.5 & 23.2 & \checkmark \\
  2.5 & 1.0 & 17.8 & \checkmark \\
  3.0 & 0.5 & 24.9 & \checkmark \\
  3.0 & 1.0 & 19.2 & \checkmark \\
  \bottomrule
\end{tabular}
\caption{Sieve ratio $M(\tilde{F}_{\mathrm{spec}})$ for various parameters.}
\label{tab:results}
\end{table}

All tested configurations satisfy $M > 4$.

\subsection{Statistical Confidence}

For $K = 2.5$, $\tsym = 1.0$, $k = 50$, over 20 independent runs:
\begin{itemize}
  \item Mean: $M = 11.54$
  \item Standard deviation: $\sigma = 2.31$
  \item 95\% confidence interval: $[10.47, 12.61]$
  \item Lower bound: $M > 4.39 > 4$
\end{itemize}

\subsection{Dependence on $k$}

\begin{table}[h]
\centering
\begin{tabular}{c|cc}
  \toprule
  $k$ & $M(\tilde{F}_{\mathrm{spec}})$ & $M/\log k$ \\
  \midrule
  10 & 8.2 & 3.56 \\
  50 & 15.9 & 4.07 \\
  100 & 21.3 & 4.63 \\
  200 & 28.7 & 5.42 \\
  \bottomrule
\end{tabular}
\caption{Growth of $M$ with tuple size $k$.}
\label{tab:k-dependence}
\end{table}

The ratio $M/\log k$ is approximately constant, consistent with
Remark~\ref{rem:asymptotic}.

\section{Conclusions}
\label{sec:conclusions}

\subsection{Summary}

We supply fully analytic modules for the Archimedean barrier (A3), the RKHS prime contraction, and the compact transfer (T5), together with explicit monotone schedules. These ingredients prove positivity of $H_K=T_{M^\star(K)}[P_A]-T_P$ on every compact $W_K$ with certified parameters and show how positivity would extend along any exhaustion. The global conclusion $Q(\Phi)\ge0$ on the full Weil class---and thus RH via Weil---depends on propagating these bounds through the exhaustion; this remains an open analytic step. The GRH twist and the twin-prime sector inherit the same status: their operator identities are proved, but GRH and TPC themselves are not.

\subsection{Outlook}

\begin{itemize}
    \item Sharpen constants in the Archimedean bounds (Lemma~\ref{lem:psi-upper}) to optimize $M_{\min}(K)$.
    \item Extend the analytic RKHS contraction to general Dirichlet characters $\chi \bmod q$ uniformly in $q$.
    \item For twin primes: either (i) derive analytic lower bounds for the twin-restricted $Z(X)$, or (ii) supply the conjectural inputs in \cref{thm:pc-hl2} (pair correlation $+$ HL(2)), or keep the bridge explicitly conditional.
    \item Close the remaining global step: extend the compact positivity bounds uniformly along an exhaustion of $\mathcal{W}$ without auxiliary assumptions.
\end{itemize}


\bibliographystyle{alpha}
\bibliography{references}

\end{document}

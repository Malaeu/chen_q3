% Sieve-Spectral Synergy - Preamble
% ==================================

\documentclass[11pt,a4paper]{article}

% Encoding and fonts
\usepackage[utf8]{inputenc}
\usepackage[T1]{fontenc}
\usepackage{lmodern}

% Math packages
\usepackage{amsmath,amssymb,amsthm}
\usepackage{mathtools}
\usepackage{bm}

% References and hyperlinks
\usepackage{hyperref}
\usepackage{url}
\usepackage[nameinlink]{cleveref}

% Tables and figures
\usepackage{graphicx}
\usepackage{booktabs}
\usepackage{float}

% Layout
\usepackage{geometry}
\geometry{margin=1in}

% Enumeration
\usepackage{enumitem}

% ============================================
% Theorem environments
% ============================================
\theoremstyle{plain}
\newtheorem{theorem}{Theorem}[section]
\newtheorem{lemma}[theorem]{Lemma}
\newtheorem{proposition}[theorem]{Proposition}
\newtheorem{corollary}[theorem]{Corollary}
\newtheorem{conjecture}[theorem]{Conjecture}

\theoremstyle{definition}
\newtheorem{definition}[theorem]{Definition}
\newtheorem{assumption}[theorem]{Assumption}
\newtheorem{hypothesis}[theorem]{Hypothesis}

\theoremstyle{remark}
\newtheorem{remark}[theorem]{Remark}
\newtheorem{example}[theorem]{Example}

% ============================================
% Custom commands - Number sets
% ============================================
\newcommand{\R}{\mathbb{R}}
\newcommand{\C}{\mathbb{C}}
\newcommand{\Z}{\mathbb{Z}}
\newcommand{\N}{\mathbb{N}}
\newcommand{\Q}{\mathbb{Q}}

% Operators
\newcommand{\Tr}{\operatorname{Tr}}
\DeclareMathOperator{\supp}{supp}
\DeclareMathOperator{\re}{Re}
\DeclareMathOperator{\im}{Im}

% Brackets and norms
\newcommand{\abs}[1]{\lvert #1 \rvert}
\newcommand{\norm}[1]{\lVert #1 \rVert}
\newcommand{\ip}[2]{\langle #1, #2 \rangle}

% ============================================
% Sieve notation
% ============================================
\newcommand{\Sone}{S_1}              % First sieve sum
\newcommand{\Stwo}{S_2}              % Second sieve sum
\newcommand{\MF}{M(F)}               % Sieve ratio
\newcommand{\Fspec}{F_{\mathrm{spec}}} % Spectral sieve function
\newcommand{\WQ}{W_{Q3}}             % Q3 weight kernel

% ============================================
% Q3 notation (from main paper)
% ============================================
\newcommand{\TA}{T_A}                % Archimedean Toeplitz
\newcommand{\TP}{T_P}                % Prime Toeplitz
\newcommand{\HX}{H_X}                % Hamiltonian
\newcommand{\wQ}[1]{w(#1)}           % Q3 weight w(n) = 2 Lambda(n)/sqrt(n)
\newcommand{\lammin}{\lambda_{\min}} % Minimum eigenvalue
\newcommand{\tsym}{t_{\mathrm{sym}}} % Symmetry parameter

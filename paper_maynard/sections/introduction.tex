% Introduction
% ============

\section{Introduction}
\label{sec:intro}

\subsection{Historical Background}

The distribution of prime numbers has fascinated mathematicians since antiquity.
Euclid proved the infinitude of primes circa 300 BCE, but the question of how
primes are \emph{distributed} among integers remained mysterious until the 19th
century.

The Prime Number Theorem, proved independently by Hadamard and de la Vallée
Poussin in 1896, established that the average gap between primes near $n$
is approximately $\log n$. This raised a natural question: can the gap
$g_n = p_{n+1} - p_n$ be bounded by a constant infinitely often?

Hardy and Littlewood \cite{HardyLittlewood1923} conjectured in 1923 that
twin primes ($p$ and $p+2$ both prime) occur infinitely often. More generally,
they formulated precise asymptotic predictions for $k$-tuples of primes
in arithmetic progressions. Despite a century of effort, these conjectures
remain unproved.

\subsection{The GPY Revolution}

The modern era began with Goldston, Pintz, and Yıldırım \cite{GPY2009}, who
proved in 2005 that
\begin{equation}
  \liminf_{n \to \infty} \frac{g_n}{\log p_n} = 0.
\end{equation}
While not establishing a finite bound, their method---combining sieve theory
with the Bombieri--Vinogradov theorem---opened a new pathway.

Zhang \cite{Zhang2014} achieved the breakthrough in 2013:
\begin{equation}
  \liminf_{n \to \infty} g_n \leq 7 \times 10^7.
\end{equation}
This was the first proof that bounded gaps occur infinitely often.

\subsection{The Maynard--Tao Method}

Maynard \cite{Maynard2015} and Tao independently developed a new approach
using multi-dimensional sieve weights. Their key insight: instead of optimizing
one-dimensional weights, consider functions $F(t_1, \ldots, t_k)$ on the
$k$-simplex and define
\begin{equation}
  M(F) = \frac{\displaystyle\int_{\Delta_k}
    \Bigl(\sum_{j=1}^k \partial_j F\Bigr)^2 dt}
    {\displaystyle\int_{\Delta_k} F^2 \, dt}.
\end{equation}
The fundamental result: if $M(F) > m$, then $m$ primes appear in a bounded
interval infinitely often. The Polymath project \cite{Polymath2014} optimized
this to obtain
\begin{equation}
  \liminf_{n \to \infty} g_n \leq 246,
\end{equation}
the current best unconditional result.

\subsection{Spectral Approaches and the Hilbert--Pólya Philosophy}

A parallel development in analytic number theory concerns spectral
interpretations of the Riemann Hypothesis. Hilbert and Pólya independently
suggested that the zeros of $\zeta(s)$ might be eigenvalues of a
self-adjoint operator. If such an operator exists and is positive,
the Riemann Hypothesis would follow.

The Q3 framework \cite{Q3paper} constructs a concrete candidate:
the Hamiltonian
\begin{equation}
  H_X = T_A - T_P
\end{equation}
on $L^2([-K, K])$, where $T_A$ is an archimedean operator and $T_P$
encodes prime contributions via weights
\begin{equation}
  w(p) = \frac{2\log p}{\sqrt{p}}.
\end{equation}
The spectral positivity condition $\lambda_{\min}(H_X) \geq 0$ provides
an equivalent reformulation of the Riemann Hypothesis.

\subsection{Our Contribution: The Sieve--Spectral Bridge}

This paper constructs polynomial sieve functions inspired by Q3 spectral weights.
We consider the family
\begin{equation}
  F_\alpha(t_1, \ldots, t_k) = \Bigl(1 - \sum_{i=1}^k t_i\Bigr)^\alpha
\end{equation}
supported on the $k$-simplex $\Delta_k$, and prove:

\medskip
\noindent\textbf{Main Theorem.}
\textit{For $F_\alpha(t) = (1 - \sum_i t_i)^\alpha$ on the $k$-simplex:
\begin{enumerate}
  \item[\textnormal{(i)}] $F_\alpha$ satisfies all conditions of Maynard's variational theorem
        (smoothness, simplex support, non-negativity, symmetry).
  \item[\textnormal{(ii)}] The variational functional has closed form:
        $M(F_\alpha, k) = \frac{2k(2\alpha+1)}{(\alpha+1)(k+2\alpha+1)}$.
  \item[\textnormal{(iii)}] For $\alpha = 1/2$, $k = 7$: $M(F_{1/2}, 7) = 56/27 > 2$.
  \item[\textnormal{(iv)}] Critical threshold: $M > m$ iff $k > k^*(\alpha, m)$
        where $k^* = m(\alpha+1)(2\alpha+1)/[2(2\alpha+1) - m(\alpha+1)]$.
  \item[\textnormal{(v)}] Optimal exponent: for $m = 2$, the minimum $k^*$ is achieved
        at $\alpha^* = 1/\sqrt{2}$, giving $k^* = 2\sqrt{2}+3 \approx 5.83$.
  \item[\textnormal{(vi)}] Consequently, $\liminf_{n \to \infty}(p_{n+1} - p_n) < \infty$.
\end{enumerate}}

\medskip
The proof is purely analytic: no Monte Carlo, no numerical optimization.
The Q3 spectral framework motivates the function family (via weights $w(p) \sim p^{-1/2}$),
but the proof chain uses only Maynard--Tao machinery and explicit $\Gamma$-function calculus.

\subsection{Organization}

Section~\ref{sec:maynard} reviews the Maynard--Tao sieve framework.
Section~\ref{sec:q3weights} defines Q3 spectral weights and their origin.
Section~\ref{sec:setup} constructs the spectral sieve function and verifies
Maynard admissibility.
Section~\ref{sec:correspondence} states and proves the main correspondence
theorems.
Section~\ref{sec:numerical} presents computational verification.
Section~\ref{sec:conclusions} discusses implications and further directions.

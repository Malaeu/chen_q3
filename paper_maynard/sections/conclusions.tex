% Conclusions
% ===========

\section{Concluding Remarks}
\label{sec:conclusions}

\subsection{Summary}

We established a correspondence between Q3-inspired polynomial sieve functions
and Maynard sieve weights. The main results are:

\begin{enumerate}
  \item The polynomial sieve function $F_\alpha(t) = (1 - \sum_i t_i)^\alpha$
        satisfies all conditions of Maynard's variational theorem
        (Theorem~\ref{thm:bridge}).

  \item The variational functional has closed form (Theorem~\ref{thm:M-analytic}):
        \begin{equation*}
          M(F_\alpha, k) = \frac{2k(2\alpha+1)}{(\alpha+1)(k+2\alpha+1)}.
        \end{equation*}

  \item Critical threshold formula (Theorem~\ref{thm:critical-k}):
        $M > m$ iff $k > k^*(\alpha, m)$ where
        $k^* = m(\alpha+1)(2\alpha+1)/[2(2\alpha+1) - m(\alpha+1)]$.

  \item For $\alpha = 1/2$: $k^*(1/2, 2) = 6$, so $k \geq 7$ gives $M > 2$
        (Corollary~\ref{cor:k-star-half}).

  \item Optimal exponent (Theorem~\ref{thm:optimal-alpha}):
        $\alpha^* = 1/\sqrt{2}$ minimizes $k^*$ to $2\sqrt{2}+3 \approx 5.83$.
        Thus $k = 6$ suffices for bounded gaps.

  \item By Corollary~\ref{cor:maynard-m2},
        $\liminf_{n \to \infty}(p_{n+1} - p_n) < \infty$
        (Corollary~\ref{cor:bounded-gaps}).
\end{enumerate}

\subsection{Discussion}

The correspondence established here connects two approaches to
prime distribution:
\begin{itemize}
  \item Spectral methods (Q3 Hamiltonian, spectral positivity)
  \item Sieve methods (Maynard--Tao, bounded gaps)
\end{itemize}

\begin{remark}[Nature of the correspondence]
  We emphasize what this paper \emph{does} and \emph{does not} establish:
  \begin{itemize}
    \item \textbf{Does:} Constructs a polynomial sieve function $F_\alpha$
          inspired by Q3 spectral weights that is Maynard-admissible.
    \item \textbf{Does:} Derives a closed-form formula for $M(F_\alpha, k)$
          and proves analytically that $M(F_{1/2}, 7) = 56/27 > 2$.
    \item \textbf{Does:} Concludes bounded prime gaps via standard Maynard--Tao
          machinery (Corollary~\ref{cor:bounded-gaps}).
    \item \textbf{Does not:} Establish a logical equivalence between
          Q3 spectral positivity $\lambda_{\min}(H_X) \geq 0$ and the
          sieve ratio $M > 2$.
  \end{itemize}
  The bridge is \emph{formal}: Q3 motivates the weight construction,
  but the sieve bounds follow from Maynard's theorem alone.
\end{remark}

\begin{theorem}[Bounded gaps from polynomial sieve function]
  \label{thm:bounded-gaps-poly}
  Let $F_{1/2}(t) = (1 - \sum_i t_i)^{1/2}$ and $k = 7$.
  By Lemma~\ref{lem:M-rigorous}, $M(F_{1/2}) > 2$.
  Hence for any admissible $7$-tuple $\mathcal{H}$:
  \begin{equation}
    \liminf_{n \to \infty} (p_{n+1} - p_n) \leq
      \max \mathcal{H} - \min \mathcal{H}.
  \end{equation}
\end{theorem}

\begin{proof}
  By Lemma~\ref{lem:M-rigorous}, $M(F_{1/2}) \geq 2.006 > 2$.
  The function $F_{1/2}$ satisfies all conditions of
  Theorem~\ref{thm:maynard-variational}: it is smooth on $\Delta_k$,
  supported on the simplex, non-negative, and symmetric.

  By Theorem~\ref{thm:maynard-asymptotic},
  $\Stwo/\Sone \to (\theta/2) M > \theta/2 > 0$.
  By Theorem~\ref{thm:maynard-main} with $m = 2$, at least two elements
  of $\{n + h_1, \ldots, n + h_7\}$ are prime infinitely often.
\end{proof}

\begin{remark}[Connection to Q3 spectral weights]
  The exponent $\alpha = 1/2$ in $F_{1/2}$ is motivated by the Q3
  spectral weight decay: $\wQ{p} \sim p^{-1/2}$. While the direct
  product form $\prod \WQ(t_i)$ gives $M < 2$, the polynomial form
  $(1 - \sum t_i)^\alpha$ with the same exponent achieves $M > 2$.
\end{remark}

\subsection{Further Directions}

\begin{enumerate}
  \item Optimize $\alpha$ and $k$ to minimize the explicit gap bound
        $\max\mathcal{H} - \min\mathcal{H}$.
  \item Extend the analytic framework to non-polynomial sieve functions
        (e.g., $\tilde{F}_{\mathrm{spec}}$).
  \item Explore connections to trace formula methods for prime correlations.
  \item Investigate whether spectral positivity conditions imply stronger
        bounds on $M(F)$.
\end{enumerate}

% Main Theorems
% =============

\section{Main Results}
\label{sec:correspondence}

We state and prove the main theorems establishing the sieve--spectral bridge.

\subsection{Weight Construction}

\begin{theorem}[Sieve weights from spectral function]
  \label{thm:weights}
  Let $\tilde{F}_{\mathrm{spec}}$ be defined as in Definition~\ref{def:Fspec-modified}.
  Define sieve weights according to the Maynard formula
  (Definition~\ref{def:maynard-weights}):
  \begin{equation}
    \label{eq:lambda-from-Fspec}
    \lambda_{d_1,\ldots,d_k} := \mu(d_1) \cdots \mu(d_k) \cdot
      \tilde{F}_{\mathrm{spec}}\biggl(\frac{\log d_1}{\log R}, \ldots,
      \frac{\log d_k}{\log R}\biggr),
  \end{equation}
  where $R = X^{\theta/(2k)}$ with $\theta = 1/2$ (Bombieri--Vinogradov level).
  Then these weights satisfy the Maynard conditions.
\end{theorem}

\begin{proof}
  The weights \eqref{eq:lambda-from-Fspec} are precisely those of
  Definition~\ref{def:maynard-weights} with $F = \tilde{F}_{\mathrm{spec}}$.
  By Proposition~\ref{prop:maynard-admissible}, $\tilde{F}_{\mathrm{spec}}$
  is smooth, supported on $\Delta_k$, non-negative, and symmetric.
  Hence it satisfies all conditions required for the Maynard framework.
\end{proof}

\begin{remark}[Q3 motivation vs.\ sieve weights]
  \label{rem:q3-motivation}
  We emphasize the logical structure:
  \begin{itemize}
    \item The \emph{form} of $\tilde{F}_{\mathrm{spec}}$ is motivated by
          Q3 spectral weights $\WQ(s)$.
    \item The \emph{sieve weights} $\lambda_{d_1,\ldots,d_k}$ are defined
          purely through Maynard's standard formula \eqref{eq:lambda-from-Fspec},
          not through any direct spectral correspondence.
  \end{itemize}
  Thus Q3 provides the \emph{construction} of $F$, while Maynard provides
  the \emph{machinery} to convert $F$ into sieve weights.
\end{remark}

\subsection{Asymptotic Analysis}

The following lemma establishes when the Maynard asymptotic formula applies.

\begin{lemma}[Conditions for asymptotic formula]
  \label{lem:asymptotic-conditions}
  Let $F : \Delta_k \to \R$ satisfy:
  \begin{enumerate}
    \item[\textnormal{(i)}] $F \in C^1(\Delta_k)$ (continuous differentiability);
    \item[\textnormal{(ii)}] $\supp(F) \subseteq \Delta_k$ (simplex support);
    \item[\textnormal{(iii)}] $F \geq 0$ (non-negativity).
  \end{enumerate}
  Then the Maynard asymptotic formula (Theorem~\ref{thm:maynard-asymptotic}) holds:
  \begin{equation}
    \frac{\Stwo}{\Sone} = \frac{\theta}{2} \cdot M(F) + o(1)
    \quad \text{as } X \to \infty.
  \end{equation}
\end{lemma}

\begin{proof}
  This follows from \cite[Theorem~3.1]{Maynard2015}. The conditions (i)--(iii)
  ensure absolute convergence of the relevant sums and applicability of
  the Bombieri--Vinogradov theorem.
\end{proof}

\begin{lemma}[$\tilde{F}_{\mathrm{spec}}$ satisfies asymptotic conditions]
  \label{lem:Fspec-asymptotic}
  The spectral sieve function $\tilde{F}_{\mathrm{spec}}$ of
  Definition~\ref{def:Fspec-modified} satisfies conditions (i)--(iii)
  of Lemma~\ref{lem:asymptotic-conditions}.
\end{lemma}

\begin{proof}
  Condition (i): By Lemma~\ref{lem:smooth}, $\tilde{F}_{\mathrm{spec}} \in C^\infty$.
  Condition (ii): By Lemma~\ref{lem:support}, $\supp(\tilde{F}_{\mathrm{spec}}) \subseteq \Delta_k$.
  Condition (iii): By Lemma~\ref{lem:nonneg}, $\tilde{F}_{\mathrm{spec}} \geq 0$.
\end{proof}

\begin{theorem}[Asymptotic sieve ratio for $\tilde{F}_{\mathrm{spec}}$]
  \label{thm:Fspec-asymptotic}
  For $F = \tilde{F}_{\mathrm{spec}}$ with weights $\lambda_{d_1,\ldots,d_k}$
  defined by \eqref{eq:lambda-from-Fspec}:
  \begin{equation}
    \frac{\Stwo}{\Sone} = \frac{1}{4} \cdot M(\tilde{F}_{\mathrm{spec}}) + o(1)
    \quad \text{as } X \to \infty,
  \end{equation}
  where $\theta = 1/2$ is the Bombieri--Vinogradov level of distribution.
\end{theorem}

\begin{proof}
  By Lemma~\ref{lem:Fspec-asymptotic}, $\tilde{F}_{\mathrm{spec}}$ satisfies
  the conditions of Lemma~\ref{lem:asymptotic-conditions}. Apply
  Theorem~\ref{thm:maynard-asymptotic} with $\theta = 1/2$.
\end{proof}

\subsection{The Main Theorem}

\begin{theorem}[Sieve--Spectral Bridge]
  \label{thm:bridge}
  Let $F_\alpha(t) = (1 - \sum_i t_i)^\alpha$ be the polynomial sieve function
  with $\alpha = 1/2$ and tuple size $k = 7$.
  Then:
  \begin{enumerate}
    \item[\textnormal{(i)}] $F_\alpha$ satisfies all conditions of
          Theorem~\ref{thm:maynard-variational} (smoothness, simplex support,
          non-negativity, symmetry).
    \item[\textnormal{(ii)}] The variational ratio satisfies
          $M(F_{1/2}, 7) = 56/27 > 2$ (Lemma~\ref{lem:M-rigorous}).
    \item[\textnormal{(iii)}] The asymptotic sieve ratio satisfies
          $\Stwo/\Sone \to M/4 > 1/2$ as $X \to \infty$.
  \end{enumerate}
\end{theorem}

\begin{proof}
  \textit{Part (i):}
  $F_\alpha(t) = (1 - \sum_i t_i)^\alpha$ is $C^\infty$ on $\Delta_k$ for $\alpha > 0$.
  Support is $\Delta_k$ by construction. Non-negativity: $(1-s)^\alpha \geq 0$
  for $s \in [0,1]$. Symmetry: $F_\alpha$ depends only on $\sum_i t_i$.

  \textit{Part (ii):}
  By Theorem~\ref{thm:M-analytic}, the variational functional has closed form
  \begin{equation}
    M(F_\alpha, k) = \frac{2k(2\alpha+1)}{(\alpha+1)(k + 2\alpha + 1)}.
  \end{equation}
  By Lemma~\ref{lem:M-rigorous}, substituting $\alpha = 1/2$ and $k = 7$:
  $M(F_{1/2}, 7) = 56/27 = 2 + 2/27 > 2$.

  \textit{Part (iii):}
  By Theorem~\ref{thm:maynard-asymptotic} with $\theta = 1/2$:
  $\Stwo/\Sone = (1/4) M(F_{1/2}, 7) + o(1) = 14/27 + o(1) > 1/2$
  as $X \to \infty$.
\end{proof}

\begin{corollary}[Bounded gaps from polynomial sieve function]
  \label{cor:bounded-gaps}
  By Theorem~\ref{thm:bridge} and Corollary~\ref{cor:maynard-m2},
  $M(F_{1/2}) > 2$ implies
  \begin{equation}
    \liminf_{n \to \infty} (p_{n+1} - p_n) < \infty.
  \end{equation}
\end{corollary}

\begin{proof}
  By Theorem~\ref{thm:bridge}(i), $F_{1/2}$ satisfies
  the hypotheses of Theorem~\ref{thm:maynard-main}. By part (ii),
  $M(F_{1/2}, 7) = 56/27 > 2$. Applying Corollary~\ref{cor:maynard-m2}:
  there exists an admissible $k$-tuple $\mathcal{H}$ such that
  infinitely many $n$ have at least two of $n + h_1, \ldots, n + h_k$ prime.

  This implies $\liminf_{n \to \infty} (p_{n+1} - p_n) \leq
  \max \mathcal{H} - \min \mathcal{H} < \infty$.
\end{proof}

\subsection{Dependence on Tuple Size}

\begin{theorem}[Critical tuple size]
  \label{thm:critical-k}
  For $F_\alpha(t) = (1 - \sum_i t_i)^\alpha$ on $\Delta_k$, let
  $L(\alpha) := 2(2\alpha+1)/(\alpha+1)$ be the asymptotic limit.
  For $m < L(\alpha)$, we have $M(F_\alpha, k) > m$ if and only if $k > k^*(\alpha, m)$,
  where
  \begin{equation}
    \label{eq:critical-k}
    k^*(\alpha, m) = \frac{m(\alpha+1)(2\alpha+1)}{2(2\alpha+1) - m(\alpha+1)}.
  \end{equation}
\end{theorem}

\begin{proof}
  Solving $M(F_\alpha, k) = m$ for $k$:
  \begin{align}
    \frac{2k(2\alpha+1)}{(\alpha+1)(k+2\alpha+1)} &= m \\
    2k(2\alpha+1) &= m(\alpha+1)(k+2\alpha+1) \\
    k[2(2\alpha+1) - m(\alpha+1)] &= m(\alpha+1)(2\alpha+1).
  \end{align}
  The denominator is positive iff $m < L(\alpha)$.
  By Remark~\ref{rem:asymptotic}, $M$ is strictly increasing in $k$,
  so $M > m$ iff $k > k^*$.
\end{proof}

\begin{corollary}[Threshold for bounded gaps]
  \label{cor:k-star-half}
  For $\alpha = 1/2$ and $m = 2$:
  \begin{equation}
    k^*(1/2, 2) = \frac{2 \cdot (3/2) \cdot 2}{2 \cdot 2 - 2 \cdot (3/2)}
                = \frac{6}{1} = 6.
  \end{equation}
  Hence $M(F_{1/2}, k) > 2$ if and only if $k \geq 7$.
\end{corollary}

\begin{theorem}[Optimal exponent]
  \label{thm:optimal-alpha}
  For $m = 2$, the critical tuple size $k^*(\alpha, 2)$ is minimized at
  $\alpha^* = 1/\sqrt{2}$, yielding
  \begin{equation}
    k^*(1/\sqrt{2}, 2) = 2\sqrt{2} + 3 \approx 5.828.
  \end{equation}
  Consequently, $k = 6$ suffices for $M > 2$ when $\alpha = 1/\sqrt{2}$.
\end{theorem}

\begin{proof}
  From \eqref{eq:critical-k} with $m = 2$:
  \begin{equation}
    k^*(\alpha, 2) = \frac{(\alpha+1)(2\alpha+1)}{\alpha} = 2\alpha + 3 + \frac{1}{\alpha}.
  \end{equation}
  Let $f(\alpha) = 2\alpha + 3 + 1/\alpha$. Then $f'(\alpha) = 2 - 1/\alpha^2 = 0$
  gives $\alpha = 1/\sqrt{2}$. Since $f''(\alpha) = 2/\alpha^3 > 0$, this is a minimum.
  Substituting: $f(1/\sqrt{2}) = \sqrt{2} + 3 + \sqrt{2} = 2\sqrt{2} + 3$.
\end{proof}

\begin{lemma}[Barrier for $m$-prime results]
  \label{lem:barrier}
  Since $M(F_\alpha, k) < L(\alpha) = 2(2\alpha+1)/(\alpha+1)$ for all $k$,
  achieving $M > m$ requires $\alpha > \alpha_{\min}(m)$ where
  \begin{equation}
    \alpha_{\min}(m) = \frac{m - 2}{4 - m} \quad \text{for } m < 4.
  \end{equation}
  In particular: $\alpha_{\min}(2) = 0$, $\alpha_{\min}(3) = 1$, $\alpha_{\min}(4^-) = +\infty$.
\end{lemma}

\begin{proof}
  From $L(\alpha) = m$: $2(2\alpha+1) = m(\alpha+1)$ gives $\alpha(4-m) = m-2$.
\end{proof}

\subsection{Computer-Assisted Verification}

\begin{theorem}[Analytic formula for $M(F_\alpha)$]
  \label{thm:M-analytic}
  For the polynomial sieve function $F_\alpha(t) = (1 - \sum_i t_i)^\alpha$
  on the $k$-simplex $\Delta_k$, the variational functional has the closed form:
  \begin{equation}
    \label{eq:M-closed-form}
    M(F_\alpha, k) = \frac{2k(2\alpha+1)}{(\alpha+1)(k + 2\alpha + 1)}.
  \end{equation}
\end{theorem}

\begin{proof}
  We compute $I_k$ and $J_k^{(m)}$ using the Dirichlet integral formula:
  \begin{equation}
    \int_{\Delta_n} (1 - \textstyle\sum_i t_i)^\beta \, dt
    = \frac{\Gamma(\beta+1)}{\Gamma(n + \beta + 1)}.
  \end{equation}

  \textit{Step 1: Compute $I_k$.}
  \begin{equation}
    I_k(F_\alpha) = \int_{\Delta_k} (1 - \textstyle\sum_i t_i)^{2\alpha} dt
    = \frac{\Gamma(2\alpha+1)}{\Gamma(k + 2\alpha + 1)}.
  \end{equation}

  \textit{Step 2: Compute $J_k^{(m)}$.}
  By symmetry, all $J_k^{(m)}$ are equal. The inner integral:
  \begin{equation}
    \int_0^{1-s} (1-s-t_m)^\alpha \, dt_m = \frac{(1-s)^{\alpha+1}}{\alpha+1},
  \end{equation}
  where $s = \sum_{i \neq m} t_i$. Thus:
  \begin{equation}
    J_k^{(m)} = \frac{1}{(\alpha+1)^2} \int_{\Delta_{k-1}} (1-s)^{2\alpha+2} ds
    = \frac{\Gamma(2\alpha+3)}{(\alpha+1)^2 \cdot \Gamma(k + 2\alpha + 2)}.
  \end{equation}

  \textit{Step 3: Compute $M = k \cdot J / I$.}
  \begin{align}
    M &= k \cdot \frac{\Gamma(2\alpha+3) \cdot \Gamma(k+2\alpha+1)}
         {(\alpha+1)^2 \cdot \Gamma(2\alpha+1) \cdot \Gamma(k+2\alpha+2)} \\
      &= k \cdot \frac{(2\alpha+2)(2\alpha+1)}{(\alpha+1)^2 \cdot (k+2\alpha+1)} \\
      &= \frac{2k(2\alpha+1)}{(\alpha+1)(k+2\alpha+1)}. \qedhere
  \end{align}
\end{proof}

\begin{lemma}[Rigorous bound: $M(F_{1/2}, 7) > 2$]
  \label{lem:M-rigorous}
  For $\alpha = 1/2$ and $k = 7$:
  \begin{equation}
    M(F_{1/2}, 7) = \frac{56}{27} = 2\tfrac{2}{27} > 2.
  \end{equation}
\end{lemma}

\begin{proof}
  Direct substitution into \eqref{eq:M-closed-form}:
  \begin{equation}
    M = \frac{2 \cdot 7 \cdot 2}{(3/2) \cdot 9}
      = \frac{28}{27/2} = \frac{56}{27}.
  \end{equation}
  Since $56 = 27 \cdot 2 + 2$, we have $56/27 = 2 + 2/27 > 2$.
\end{proof}

\begin{remark}[Scaling with $k$]
  \label{rem:asymptotic}
  From the closed form \eqref{eq:M-closed-form}, for fixed $\alpha > 0$:
  \begin{equation}
    M(F_\alpha, k) \to \frac{2(2\alpha+1)}{\alpha+1} \quad \text{as } k \to \infty.
  \end{equation}
  For $\alpha = 1/2$: $M \to 8/3 \approx 2.667$ as $k \to \infty$.
  The critical $k$ where $M = 2$ is $k = 2\alpha + 3 + 1/\alpha$; for $\alpha = 1/2$, this gives $k = 6$.

  For bounded prime gaps via Theorem~\ref{thm:maynard-main} with $m = 2$,
  we require $M > 2$, which is achieved for $k \geq 7$ by the closed form.
  For larger $k$, the formula predicts $M(F_{1/2}, 50) = 100/39 \approx 2.56$;
  see Section~\ref{sec:numerical} for supplementary numerical experiments.
\end{remark}

\begin{remark}[Dependence on parameters]
  The value of $M(\tilde{F}_{\mathrm{spec}})$ depends on $(K, \tau, \epsilon, k)$.
  Parameter optimization is a numerical problem: larger $K$ increases the
  weight kernel spread, while smaller $\tau$ concentrates the Gaussian.
  The trade-off determines the optimal $M$ for each $k$.
\end{remark}

% Q3 Spectral Framework
% =====================

\section{Q3 Spectral Weights}
\label{sec:q3weights}

We define the spectral weight function arising from the Q3 framework \cite{Q3paper}.

\subsection{The Q3 Hamiltonian}

Let $K > 0$ and $\tsym > 0$ be parameters. The Q3 Hamiltonian is
\begin{equation}
  \HX = \TA - \TP
\end{equation}
acting on $L^2([-K, K])$, where $\TA$ is the archimedean operator and
$\TP$ encodes prime contributions.

\begin{definition}
  The prime operator $\TP$ has kernel
  \begin{equation}
    \TP(\xi, \eta) = \sum_p \wQ{p} \cdot K_{\tsym}(\xi - \xi_p)
      \cdot K_{\tsym}(\eta - \xi_p),
  \end{equation}
  where $\xi_p = (\log p)/(2\pi)$ and $K_{\tsym}(x) = \exp(-x^2/(4\tsym))$.
\end{definition}

\subsection{The Weight Function}

\begin{definition}[Q3 weight]
  \label{def:q3weight}
  For a prime $p$, define
  \begin{equation}
    \wQ{p} = \frac{2\log p}{\sqrt{p}}.
  \end{equation}
\end{definition}

\begin{remark}
  The weight $\wQ{p}$ derives from the von Mangoldt function
  $\Lambda(p) = \log p$, normalized by $p^{-1/2}$ for $L^2$ convergence
  of the sum $\sum_p \wQ{p}^2 < \infty$.
\end{remark}

\subsection{Spectral Positivity Criterion}

\begin{theorem}[Q3 Equivalence \cite{Q3paper}]
  \label{thm:q3-equivalence}
  The Riemann Hypothesis is equivalent to the spectral condition
  $\lammin(\HX) \geq 0$ for all sufficiently large $X$.
\end{theorem}

\begin{remark}
  This equivalence transforms the analytic problem of locating zeros
  of $\zeta(s)$ into a spectral positivity problem for a concrete
  self-adjoint operator, realizing the Hilbert--Pólya philosophy.
\end{remark}
